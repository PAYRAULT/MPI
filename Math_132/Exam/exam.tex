\documentclass[]{book}

%These tell TeX which packages to use.
\usepackage{array,epsfig}
\usepackage{amsmath}
\usepackage{amsfonts}
\usepackage{amssymb}
\usepackage{amsxtra}
\usepackage{amsthm}
\usepackage{mathrsfs}
\usepackage{color}
\usepackage{pgfplots}
\usepackage{fancyhdr}

%Here I define some theorem styles and shortcut commands for symbols I use often
\theoremstyle{definition}
\newtheorem{defn}{Definition}
\newtheorem{thm}{Theorem}
\newtheorem{cor}{Corollary}
\newtheorem*{rmk}{Remark}
\newtheorem{lem}{Lemma}
\newtheorem*{joke}{Joke}
\newtheorem{ex}{Example}
\newtheorem*{soln}{Solution}
\newtheorem{prop}{Proposition}

\newcommand{\lra}{\longrightarrow}
\newcommand{\ra}{\rightarrow}
\newcommand{\surj}{\twoheadrightarrow}
\newcommand{\graph}{\mathrm{graph}}
\newcommand{\bb}[1]{\mathbb{#1}}
\newcommand{\Z}{\bb{Z}}
\newcommand{\Q}{\bb{Q}}
\newcommand{\R}{\bb{R}}
\newcommand{\C}{\bb{C}}
\newcommand{\N}{\bb{N}}
\newcommand{\M}{\mathbf{M}}
\newcommand{\m}{\mathbf{m}}
\newcommand{\MM}{\mathscr{M}}
\newcommand{\HH}{\mathscr{H}}
\newcommand{\Om}{\Omega}
\newcommand{\Ho}{\in\HH(\Om)}
\newcommand{\bd}{\partial}
\newcommand{\del}{\partial}
\newcommand{\bardel}{\overline\partial}
\newcommand{\textdf}[1]{\textbf{\textsf{#1}}\index{#1}}
\newcommand{\img}{\mathrm{img}}
\newcommand{\ip}[2]{\left\langle{#1},{#2}\right\rangle}
\newcommand{\inter}[1]{\mathrm{int}{#1}}
\newcommand{\exter}[1]{\mathrm{ext}{#1}}
\newcommand{\cl}[1]{\mathrm{cl}{#1}}
\newcommand{\ds}{\displaystyle}
\newcommand{\vol}{\mathrm{vol}}
\newcommand{\cnt}{\mathrm{ct}}
\newcommand{\osc}{\mathrm{osc}}
\newcommand{\LL}{\mathbf{L}}
\newcommand{\UU}{\mathbf{U}}
\newcommand{\support}{\mathrm{support}}
\newcommand{\AND}{\;\wedge\;}
\newcommand{\OR}{\;\vee\;}
\newcommand{\Oset}{\varnothing}
\newcommand{\st}{\ni}
\newcommand{\wh}{\widehat}

%Pagination stuff.
\setlength{\topmargin}{-.3 in}
\setlength{\oddsidemargin}{0in}
\setlength{\evensidemargin}{0in}
\setlength{\textheight}{9.in}
\setlength{\textwidth}{6.5in}
\pagestyle{fancy}
\fancyhf{}
\rhead{Math\_104}
\lhead{Exam}
\rfoot{Page \thepage}


\begin{document}

\subsection*{Rappel de cours}

M\'ethode de Newton

\begin{itemize}
\item Identification des racines d'une fonction. (ie. une racine est une valeur $r$ tel que $f(r)=0$.
\item La m\'ethode se fait par approximation \`a partir d'une valeur suppos\'ee proche de la racine
\item Developper la suite $x_{n+1} = x_n - \frac{f(x_n)}{f'(x_n)}$. Le plus loin on va dans la suite, le plus proche on est de la racine.
\end{itemize}


\subsection*{Exercice 1}

La suite $(x_n)_n$ est 
$$x_{n+1} = x_n - \frac{f(x_n)}{f'(x_n)}\; avec\; x_0 = 2$$

On a $f(x) = xe^{-x}$, donc $f'(x) = (1-x)e^{-x}$
$$x_{n+1} = x_n - \frac{x_ne^{-x_n}}{(1-x_n)e^{-x_n}} = x_n - \frac{x_n}{1-x_n} = x_n + \frac{x_n}{x_n-1}$$

On a $x-1 < x$, donc $\frac{x}{x-1} > 1$ lorsque $x > 1$. On a $x_0 \geq 2 > 1$, \`a chaque pas on ajoute une valeur positive donc $x_n>2$.\\

$$x_{n+1} - x_{n} = x_n + \frac{x_n}{x_n-1} - x_n = \frac{x_n}{x_n-1}$$
On a $x-1 < x$, donc $\frac{x}{x-1} > 1$ lorsque $x > 1$. On a $x_0 \geq 2 > 1$, donc $x_{n+1} - x_{n} > 1$. La suite est strictement croissante donc elle divergence quand $x \to \infty$.



\end{document}

