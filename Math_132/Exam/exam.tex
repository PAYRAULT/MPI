\documentclass[]{book}

%These tell TeX which packages to use.
\usepackage{array,epsfig}
\usepackage{amsmath}
\usepackage{amsfonts}
\usepackage{amssymb}
\usepackage{amsxtra}
\usepackage{amsthm}
\usepackage{mathrsfs}
\usepackage{color}
\usepackage{pgfplots}
\usepackage{fancyhdr}

%Here I define some theorem styles and shortcut commands for symbols I use often
\theoremstyle{definition}
\newtheorem{defn}{Definition}
\newtheorem{thm}{Theorem}
\newtheorem{cor}{Corollary}
\newtheorem*{rmk}{Remark}
\newtheorem{lem}{Lemma}
\newtheorem*{joke}{Joke}
\newtheorem{ex}{Example}
\newtheorem*{soln}{Solution}
\newtheorem{prop}{Proposition}

\newcommand{\lra}{\longrightarrow}
\newcommand{\ra}{\rightarrow}
\newcommand{\surj}{\twoheadrightarrow}
\newcommand{\graph}{\mathrm{graph}}
\newcommand{\bb}[1]{\mathbb{#1}}
\newcommand{\Z}{\bb{Z}}
\newcommand{\Q}{\bb{Q}}
\newcommand{\R}{\bb{R}}
\newcommand{\C}{\bb{C}}
\newcommand{\N}{\bb{N}}
\newcommand{\M}{\mathbf{M}}
\newcommand{\m}{\mathbf{m}}
\newcommand{\MM}{\mathscr{M}}
\newcommand{\HH}{\mathscr{H}}
\newcommand{\Om}{\Omega}
\newcommand{\Ho}{\in\HH(\Om)}
\newcommand{\bd}{\partial}
\newcommand{\del}{\partial}
\newcommand{\bardel}{\overline\partial}
\newcommand{\textdf}[1]{\textbf{\textsf{#1}}\index{#1}}
\newcommand{\img}{\mathrm{img}}
\newcommand{\ip}[2]{\left\langle{#1},{#2}\right\rangle}
\newcommand{\inter}[1]{\mathrm{int}{#1}}
\newcommand{\exter}[1]{\mathrm{ext}{#1}}
\newcommand{\cl}[1]{\mathrm{cl}{#1}}
\newcommand{\ds}{\displaystyle}
\newcommand{\vol}{\mathrm{vol}}
\newcommand{\cnt}{\mathrm{ct}}
\newcommand{\osc}{\mathrm{osc}}
\newcommand{\LL}{\mathbf{L}}
\newcommand{\UU}{\mathbf{U}}
\newcommand{\support}{\mathrm{support}}
\newcommand{\AND}{\;\wedge\;}
\newcommand{\OR}{\;\vee\;}
\newcommand{\Oset}{\varnothing}
\newcommand{\st}{\ni}
\newcommand{\wh}{\widehat}

%Pagination stuff.
\setlength{\topmargin}{-.3 in}
\setlength{\oddsidemargin}{0in}
\setlength{\evensidemargin}{0in}
\setlength{\textheight}{9.in}
\setlength{\textwidth}{6.5in}
\pagestyle{fancy}
\fancyhf{}
\rhead{Math\_132}
\lhead{Exam}
\rfoot{Page \thepage}


\begin{document}

\subsection*{Rappel de cours}

M\'ethode de Newton

\begin{itemize}
\item Identification des racines d'une fonction. (ie. une racine est une valeur $r$ tel que $f(r)=0$.
\item La m\'ethode se fait par approximation \`a partir d'une valeur suppos\'ee proche de la racine
\item Developper la suite $x_{n+1} = x_n - \frac{f(x_n)}{f'(x_n)}$. Le plus loin on va dans la suite, le plus proche on est de la racine.
\end{itemize}


\subsection*{Exercice 1}

La suite $(x_n)_n$ est 
$$x_{n+1} = x_n - \frac{f(x_n)}{f'(x_n)}\; avec\; x_0 = 2$$

On a $f(x) = xe^{-x}$, donc $f'(x) = (1-x)e^{-x}$
$$x_{n+1} = x_n - \frac{x_ne^{-x_n}}{(1-x_n)e^{-x_n}} = x_n - \frac{x_n}{1-x_n} = x_n + \frac{x_n}{x_n-1}$$

On a $x-1 < x$, donc $\frac{x}{x-1} > 1$ lorsque $x > 1$. On a $x_0 \geq 2 > 1$, \`a chaque pas on ajoute une valeur positive donc $x_n>2$.\\

$$x_{n+1} - x_{n} = x_n + \frac{x_n}{x_n-1} - x_n = \frac{x_n}{x_n-1}$$
On a $x-1 < x$, donc $\frac{x}{x-1} > 1$ lorsque $x > 1$. On a $x_0 \geq 2 > 1$, donc $x_{n+1} - x_{n} > 1$. La suite est strictement croissante donc elle divergence quand $x \to \infty$.

\newpage
\section*{S\'eance 2 - Densit\'e dans $\R$}

\subsection*{Exercice 1 -  Une d\'efinition \'equivalente}

1 - Si $\forall x \in \R, \forall \epsilon > 0, \exists d \in D \bigcap\; ]x-\epsilon,x+\epsilon[$ alors $x-\epsilon < d < x+\epsilon$, prenons $a=x-\epsilon$ et $b=x+\epsilon$ alors $a<d<b$\\
2 - $\forall a,b \in \R, \exists d \in D, a < d < b$ alors prenons $x=\frac{a+b}{2}$ et $\epsilon = b-x = \frac{b-a}{2}$ alors $x+\epsilon = x + b-x = b$ et $x-\epsilon = = \frac{a+b}{2} - \frac{b-a}{2} = a$ donc $x-\epsilon < d < x+\epsilon$.

\subsection*{Exercice 2 -  Les rationnels sont denses dans $\R$}
\subsubsection*{2.1}

Si $\beta > 0$, prenons l'entier $i=E(\beta)$ alors $\beta-1< i \leq \beta$ et $\beta - \alpha > 1$, donc  $\alpha < \beta-1$ donc $\alpha<i<\beta$.\\

Si $\alpha < 0$, prenons l'entier $i=E(\alpha)$ alors $\alpha \leq i<\alpha+1$ et $\beta - \alpha > 1$, donc  $\beta > \alpha+1$ donc $\alpha<i<\beta$.\\

\subsubsection*{2.2}
On a $x-1 < E(x) \leq x$.\\ 
On a $b-a > 0$. Si $q(b-a) > 1$ alors $q>\frac{1}{b-a}$. Prenons $q=E(\frac{1}{b-a}) + 1$. 
$$x - 1 < E(x) $$
$$x-1+1 < E(x)+1 $$
$$\frac{1}{b-a} < E(\frac{1}{b-a}) + 1$$
$$\frac{1}{b-a} < q $$
$$(b-a)\frac{1}{b-a} < q(b-a) $$
$$1 < q(b-a)$$

L'entier $q=E(\frac{1}{b-a}) + 1$.

\subsubsection*{2.3}
On cherche $p, q \in \N, \forall a, b \in \R, a < \frac{p}{q} < b$.\\ 
De la question 2.3, on a montr\'e que $\exists q \in \N^{*}, q(b-a)>1$ et $b-a>0$ donc 
$$b-a>\frac{1}{q}$$
$$b> a+\frac{1}{q}$$
$$b>\frac{aq + 1}{q}$$
Prenons $p=E(aq)$, donc $p \leq aq < p+1$.

$$b>\frac{aq + 1}{q} > \frac{p + 1}{q}$$
et
$$\frac{p}{q} \leq a < \frac{p+1}{q}$$
donc
$$a < \frac{p+1}{q} < \frac{aq + 1}{q} < b$$

Prenons $d=\frac{p+1}{q}$ alors $\forall a,b \in \R, \exists d \in \Q, a < d < b$. \\
Donc $\Q$ est dense dans $\R$.

\subsection*{Exercice 3 - Densit\'e et ensembles finis}
\subsubsection*{3.1}
Tout ensemble fini admet un majorant $M$.\\
Prenons $a, b \in \R, a>M\; et\; b>M$, $\nexists d \in X, a<d<b$.\\
Donc X n'est pas dense dans $\R$. 

\subsubsection*{3.2}
Comme $D$ est dense dans $\R$ alors $\forall a, b \in \R, \exists d \in D, a<d<b$. Si on consed\`ere $n+1$ sous-intervalles de $]a;b[$ alors la propri\'et\'e et \'egalement v\'erifi\'ee sur chaque sous-intervalle (car vrai pour tout r\'eel $a$ et $b$).\\
Donc, l'ensemble $D \backslash \{d_1,...,d_n\}$ contient au moins une valeur tel que $\forall a, b \in \R, \exists d \in D, a<d<b$
(choisir une valeur dans l'intervalle qui ne contient aucun $\{d_1,...,d_n\}$). \\
Donc $D \backslash \{d_1,...,d_n\}$ est dense.


\subsection*{Exercice 4 - Une dichotomie modifi\'ee}
\subsubsection*{4.1}

\subsubsection*{4.2}
Preuve par contradiction\\
Si $I$ est vide alors $\alpha=\beta$, donc $Inf\;X = Sup\; X$. L'ensemble $X$ contient 2 \'el\'ements distincts $x_1$ et $x_2$. On a $x_1 < x_2$ (car ils sont distincts)\\
et $Inf\; X \leq x_1$ car $Inf\; X$ est une borne inf\'erieure.\\
et $Sup\; X \geq x_2$ car $Sup\; X$ est une borne inf\'erieure.\\
Donc $Inf\; X \leq x_1 < x_2 \leq Sup\; X$\\

Ceci contredit l'hypoth\`ese $Inf\;X = Sup\; X$ donc $I$ n'est pas vide.



\subsection*{Exercice 5 - Rationnel ou irrationnel ?}
\subsubsection*{5.1}
Preuve par contradiction.\\

Si $\sqrt{2}$ est rationnel alors $\exists p, q \in \N^{*}, \frac{p}{q}=\sqrt{2}$ avec $p$,$q$ premiers entre eux.
$$\frac{p}{q}=\sqrt{2}$$
$$\frac{p^2}{q^2}=2$$
$$p^2=2q^2$$

Comme $2q^2$ est pair, alors $p^2$ est doit \^etre pair, donc $p$ est pair \'egalement.\\
Soit $p=2r$ (comme p est pair), donc $p^2=(2r)^2 = 4r^2 = 2q^2$, donc $q^2 = 2r^2$ alors $q^2$ est pair, donc $q$ est pair.\\
$p$ et $q$ sont tous les deux pairs, ils ne peuvent pas \^etre premier entre eux, $\to$ contradiction.\\
Donc $\sqrt(2)$ est irrationnel. 

\subsubsection*{5.2}
Preuve par contradiction.\\

Si $p^{\frac{1}{n}}$ est rationnel alors  $\exists a, b \in \N^{*}, \frac{a}{b}=p^{\frac{1}{n}}$ avec $a$,$b$ premiers entre eux.\\
On a 
$$p = \left(\frac{a}{b}\right)^{n}$$
$$p.b^{n} = a^{n}$$
$$p.b^{n} = a.a^{n-1}$$

$p$ est premier donc $a$ ne divise pas $p$.\\
$a$,$b$ sont premiers entre eux donc $a$ ne divise pas $b^n$.\\
Donc la proposition est fausse et $p^{\frac{1}{n}}$ est irrationnel.

\subsubsection*{5.3}
Preuve par contradiction.\\
Si $r^{\frac{1}{n}}$ est rationnel alors  $\exists a, b \in \N^{*}, \frac{a}{b}=r^{\frac{1}{n}}$ avec $a$,$b$ premiers entre eux.\\
On a 
$$r = \left(\frac{a}{b}\right)^{n}$$
$r$ est un nombre rationnel donc $r=\frac{c}{d}$.
$$\frac{c}{d} = \left(\frac{a}{b}\right)^{n}$$
$$\log\left(\frac{c}{d}\right) = n.\log\left(\frac{a}{b}\right)$$

$c$ et $d$ sont fix\'es par $r$, donc pour une certaine valeur $N$ tr\'es grande cette relation n'est pas v\'erifi\'ee car $\log\left(\frac{c}{d}\right) < N.\log\left(\frac{a}{b}\right)$.\\
Donc $r^{\frac{1}{n}}$ est irrationnel pour $n>N$.

\end{document}

