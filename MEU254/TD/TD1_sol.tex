\documentclass[]{book}

%These tell TeX which packages to use.
\usepackage{array,epsfig}
\usepackage{amsmath}
\usepackage{amsfonts}
\usepackage{amssymb}
\usepackage{amsxtra}
\usepackage{amsthm}
\usepackage{mathrsfs}
\usepackage{color}
\usepackage{pgfplots}
\usepackage{bbm}

%Here I define some theorem styles and shortcut commands for symbols I use often
\theoremstyle{definition}
\newtheorem{defn}{Definition}
\newtheorem{thm}{Theorem}
\newtheorem{cor}{Corollary}
\newtheorem*{rmk}{Remark}
\newtheorem{lem}{Lemma}
\newtheorem*{joke}{Joke}
\newtheorem{ex}{Example}
\newtheorem*{soln}{Solution}
\newtheorem{prop}{Proposition}

\newcommand{\lra}{\longrightarrow}
\newcommand{\ra}{\rightarrow}
\newcommand{\surj}{\twoheadrightarrow}
\newcommand{\graph}{\mathrm{graph}}
\newcommand{\bb}[1]{\mathbb{#1}}
\newcommand{\Z}{\bb{Z}}
\newcommand{\Q}{\bb{Q}}
\newcommand{\R}{\bb{R}}
\newcommand{\C}{\bb{C}}
\newcommand{\N}{\bb{N}}
\newcommand{\M}{\mathbf{M}}
\newcommand{\m}{\mathbf{m}}
\newcommand{\MM}{\mathscr{M}}
\newcommand{\HH}{\mathscr{H}}
\newcommand{\Om}{\Omega}
\newcommand{\Ho}{\in\HH(\Om)}
\newcommand{\bd}{\partial}
\newcommand{\del}{\partial}
\newcommand{\bardel}{\overline\partial}
\newcommand{\textdf}[1]{\textbf{\textsf{#1}}\index{#1}}
\newcommand{\img}{\mathrm{img}}
\newcommand{\ip}[2]{\left\langle{#1},{#2}\right\rangle}
\newcommand{\inter}[1]{\mathrm{int}{#1}}
\newcommand{\exter}[1]{\mathrm{ext}{#1}}
\newcommand{\cl}[1]{\mathrm{cl}{#1}}
\newcommand{\ds}{\displaystyle}
\newcommand{\vol}{\mathrm{vol}}
\newcommand{\cnt}{\mathrm{ct}}
\newcommand{\osc}{\mathrm{osc}}
\newcommand{\LL}{\mathbf{L}}
\newcommand{\UU}{\mathbf{U}}
\newcommand{\support}{\mathrm{support}}
\newcommand{\AND}{\;\wedge\;}
\newcommand{\OR}{\;\vee\;}
\newcommand{\Oset}{\varnothing}
\newcommand{\st}{\ni}
\newcommand{\wh}{\widehat}

%Pagination stuff.
\setlength{\topmargin}{-.3 in}
\setlength{\oddsidemargin}{0in}
\setlength{\evensidemargin}{0in}
\setlength{\textheight}{9.in}
\setlength{\textwidth}{6.5in}
\pagestyle{empty}



\begin{document}

\subsection*{Rappel de cours}

\begin{defn}
La fonction indicatrice $\mathbbm{1}_A$ sur l'ensemble $A$ est d\'efinie comme:
$$
\mathbbm{1}_A : A \to \{0,1\}, x \to \mathbbm{1}_A(x) = 
\left\{ 
\begin{array}{l}
0, \text{ si } x \notin A \\
1, \text{ si } x \in A \\
\end{array}
\right.
$$
\end{defn}



\newpage
\subsection*{Exercice 1}
On a $\mathbbm{1}_{A^c} = 1 - \mathbbm{1}_{A}$ car si $\mathbbm{1}_{A}(x) = 1$ alors $\mathbbm{1}_{A^c}(x) = 0$ car quand $x \in A$ alors $x \notin A^c$ (m\^eme raisonnement avec $1_A(x) = 0$), et $\mathbbm{1}_{A \cup B} = \mathbbm{1}_A + \mathbbm{1}_B - \mathbbm{1}_A.\mathbbm{1}_B$ donc
$$\mathbbm{1}_{A^c \cup B^c} = \mathbbm{1}_{A^c} + \mathbbm{1}_{B^c} - \mathbbm{1}_{A^c}.\mathbbm{1}_{B^c} = (1 - \mathbbm{1}_{A}) + (1 - \mathbbm{1}_{B}) - (1 - \mathbbm{1}_{A}).(1 - \mathbbm{1}_{B}) =  1 - \mathbbm{1}_{A} + 1 - \mathbbm{1}_{B} - (1 - \mathbbm{1}_{A} - \mathbbm{1}_{B} + \mathbbm{1}_{A}.\mathbbm{1}_{B}) = 1 - \mathbbm{1}_{A}.\mathbbm{1}_{B}$$ 


QED

\end{document}

