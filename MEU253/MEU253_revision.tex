\documentclass[]{book}

%These tell TeX which packages to use.
\usepackage{array,epsfig}
\usepackage{amsmath}
\usepackage{amsfonts}
\usepackage{amssymb}
\usepackage{amsxtra}
\usepackage{amsthm}
\usepackage{mathrsfs}
\usepackage{color}
\usepackage{tikz}
\usepackage{graphicx}

%Here I define some theorem styles and shortcut commands for symbols I use often
\theoremstyle{definition}
\newtheorem{defn}{Definition}
\newtheorem{thm}{Theorem}
\newtheorem{cor}{Corollary}
\newtheorem*{rmk}{Remark}
\newtheorem{lem}{Lemma}
\newtheorem*{joke}{Joke}
\newtheorem{ex}{Example}
\newtheorem*{soln}{Solution}
\newtheorem{prop}{Proposition}

\newcommand{\lra}{\longrightarrow}
\newcommand{\ra}{\rightarrow}
\newcommand{\surj}{\twoheadrightarrow}
\newcommand{\graph}{\mathrm{graph}}
\newcommand{\bb}[1]{\mathbb{#1}}
\newcommand{\Z}{\bb{Z}}
\newcommand{\Q}{\bb{Q}}
\newcommand{\R}{\bb{R}}
\newcommand{\C}{\bb{C}}
\newcommand{\N}{\bb{N}}
\newcommand{\M}{\mathbf{M}}
\newcommand{\m}{\mathbf{m}}
\newcommand{\MM}{\mathscr{M}}
\newcommand{\HH}{\mathscr{H}}
\newcommand{\Om}{\Omega}
\newcommand{\Ho}{\in\HH(\Om)}
\newcommand{\bd}{\partial}
\newcommand{\del}{\partial}
\newcommand{\bardel}{\overline\partial}
\newcommand{\textdf}[1]{\textbf{\textsf{#1}}\index{#1}}
\newcommand{\img}{\mathrm{img}}
\newcommand{\ip}[2]{\left\langle{#1},{#2}\right\rangle}
\newcommand{\inter}[1]{\mathrm{int}{#1}}
\newcommand{\exter}[1]{\mathrm{ext}{#1}}
\newcommand{\cl}[1]{\mathrm{cl}{#1}}
\newcommand{\ds}{\displaystyle}
\newcommand{\vol}{\mathrm{vol}}
\newcommand{\cnt}{\mathrm{ct}}
\newcommand{\osc}{\mathrm{osc}}
\newcommand{\LL}{\mathbf{L}}
\newcommand{\UU}{\mathbf{U}}
\newcommand{\support}{\mathrm{support}}
\newcommand{\AND}{\;\wedge\;}
\newcommand{\OR}{\;\vee\;}
\newcommand{\Oset}{\varnothing}
\newcommand{\st}{\ni}
\newcommand{\wh}{\widehat}

%Pagination stuff.
\setlength{\topmargin}{-.3 in}
\setlength{\oddsidemargin}{0in}
\setlength{\evensidemargin}{0in}
\setlength{\textheight}{9.in}
\setlength{\textwidth}{6.5in}
\pagestyle{empty}



\begin{document}

\subsection*{R\'evision}
\begin{defn}
Un \emph{n-espace Euclidien} $\R^n$ est un ensemble de $n$ r\'eels (appel\'e \emph{n-tuple}).
$$
\R^n = \{(p_1,\ldots,p_n), p_i \text{est un r\'eel pour } i = 1,\ldots,n\}
$$ 
Si $p = (p_1, \ldots, p_n)$ et $q = (q_1, \ldots, q_n)$, on d\'efinit $p+q = (p_1+q_1,\ldots,p_n+q_n)$ et $\lambda p = (\lambda p_1, \ldots, \lambda p_n)$.

\end{defn}


\begin{defn}
On d\'efinit le produit scalaire $ \cdot $ comme 
$$p \cdot q = \sum_{i=1}^{n} p_iq_i$$
La norme d'un vecteur comme
$$\lVert p \rVert = \sqrt{p \cdot p}$$
La distance entre 2 points:
$$distance(p,q) = \lVert p - q \rVert$$
\end{defn}


\begin{defn}
On a les propri\'et\'es suivantes: $p \cdot q = q \cdot p$, $(p+r) \cdot q = p \cdot q + r \cdot q$, $(\lambda p) \cdot q = \lambda(p \cdot q) = p \cdot (\lambda q)$ et $\lVert \lambda p \rVert = \lvert \lambda \rvert \lVert p \rVert$. \\
Les in\'egalit\'es suivantes: $\lvert p \cdot q \rvert \le \lVert p \rVert \lVert q \rVert$ (Cauchy-Schwarz) et $\lVert p+q \rVert \le \lVert p \rVert + \lVert q \rVert$ (triangulaire).   	  
\end{defn}

\begin{defn}
On d\'efinit un angle entre 2 vecteurs comme:
$$\cos \theta = \frac{p \cdot q}{\lVert p \rVert \lVert q \rVert}$$
\end{defn}

\begin{defn}
On d\'efinit une application lin\'eaire $A$ (linear map) de $\R^m \to \R^n$ comme:
$$A(\lambda_1 p + \lambda_2 q) = \lambda_1 Ap + \lambda_2 Aq$$
\end{defn}

\begin{defn}
On d\'efinit une application lin\'eaire $J$ (complex) de $\R^2 \to \R^2$ comme:
$$J(p,q) = (-q,p)$$
L'application $J$ a les propri\'et\'es suivantes: $J^2 = -1$, $(Jp) \cdot (Jq) = p \cdot q$ et $(Jp) \cdot p = 0$. 
\end{defn}


\begin{defn}
On d\'efinit une courbe param\'etr\'ee $\alpha(t): ]t_1,t_2[ \to \R^n$ comme:
$$\alpha(t) = (a_1(t), a_2(t), \ldots, a_n(t))$$
avec chaque $a_i(t)$ une fonction de $]t_1, t_2[ \to \R$. 
\end{defn}

Exemple de courbes param\'etr\'ees, 
\begin{itemize}
\item la droite passant par 2 points p et q, $\beta(t) = p +t(q-p) = (1-t)p + tq$
\item le cercle de centre $p = (p_1, p_2)$ et de rayon r, $\theta(t) = (p_1 + r\cos(t), p_2\sin(t))$ avec $0 \le t < 2\pi$.
\end{itemize}


\begin{defn}
Si $\alpha(t) = (a_1(t), a_2(t), \ldots, a_n(t))$ est une courbe param\'etr\'ed, on d\'efinit la vitesse (v\'elocut\'e)de $\alpha$  comme:
$$\alpha'(t) = (a_1'(t), a_2'(t), \ldots, a_n'(t))$$
\end{defn}

\begin{defn}
Une courbe $\alpha(t) = ]t_1, t_2[ \to \R^n$ est dite r\'eguliaire si chaque $a_i(t)$ est d\'erivable et que la v\'elocit\'e en tout point n'est pas nulle. Si $\forall t \in ]t_1, t_2[, \lVert \alpha'(t) \rVert = 1$ alors $\alpha$ est dite \emph{vitesse unit\'e}
\end{defn}


\begin{defn}
Soit 2 courbes param\'etrables diff\'erentiables $\alpha(t) = (a,b) \to \R^n$ et $\beta(t) = (a,b) \to \R^n$, on dit que:
\begin{itemize}
\item $\beta$ est une \emph{reparam\'etrage positif} de $\alpha$ si il existe une fonction d\'erivable $h: (c,d) \to (a,b)$ tel que $\forall t in ]c,d[, h'(t) > 0$ et $\beta(t) = (\alpha \circ h)(t)$. 
\item $\beta$ est une \emph{reparam\'etrage n\'egatif} de $\alpha$ si il existe une fonction d\'erivable $h: (c,d) \to (a,b)$ tel que $\forall t in ]c,d[, h'(t) < 0$ et $\beta(t) = (\alpha \circ h)(t)$. 
\item $\beta$ est une \emph{reparam\'etrage} de $\alpha$ si $\beta$ est soit un reparam\'etrage positif, soit un reparam\'etrage n\'egatif de $\alpha$. 
\end{itemize}
\end{defn}

 
\begin{defn}
Quelques propri\'et\'es $\beta'(t) = \alpha'(h(t))h'(t)$, $\lVert \beta'(t)\rVert = \lVert\alpha'(h(t))\rVert h'(t)$
\end{defn}
 

\begin{defn}
On d\'efinit la \emph{longueur} d'une courbe param\'etr\'ee $\alpha$ sur l'intervale $[a,b]$ comme:
$$length[\alpha] = \int_a^b \lVert \alpha'(t)\rVert dt$$
\end{defn}

\begin{defn}
Si $\beta$ est un reparam\'etrage de $\alpha$ alors
$$length[\beta] \ sur\ [c,d] = length[\alpha]  \ sur\ [h(c),h(d)]$$ 
\end{defn}
 

\begin{defn}
Prenons une chiffre $c$ avec $a < c < b$. La \emph{la fonction de longueur d'arc} $s_{\alpha}$ de la courbe param\'etr\'ee $\alpha:(a,b) \to \R$ \`a partir de $c$ comme:
$$
s_{\alpha}(t) = length[c,t][\alpha] = \int_c^t \lVert \alpha'(u) \rVert du
$$ 
pour $a \le t < b$.
\end{defn}
 
\begin{defn}
Soit $\alpha: (a,b) \to \R^2$, on d\'efinit la courbure $k2[\alpha]$ et $\alpha$ par:
$$
k2[\alpha](t) = \frac{\alpha''(t).J\alpha'(t)}{\lVert \alpha'(t) \rVert^3}
$$ 
pour $a \le t < b$.
\end{defn}


\begin{defn}
Le reparam\'etrage par longueur d'arcde la courbe param\'etr\'ee $\alpha$, (ou reparametrage selon le vecteur vitesse unit\'e) est :
$$
s_{\alpha}(t) = length[0,t][\alpha] = \int_0^t \lVert \alpha'(u) \rVert du
$$ 
\end{defn}


\subsection*{Exercice 1}

\subsubsection*{1.1}


QED


\end{document}

