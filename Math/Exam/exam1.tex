\documentclass[]{book}

%These tell TeX which packages to use.
\usepackage{array,epsfig}
\usepackage{amsmath}
\usepackage{amsfonts}
\usepackage{amssymb}
\usepackage{amsxtra}
\usepackage{amsthm}
\usepackage{mathrsfs}
\usepackage{color}

%Here I define some theorem styles and shortcut commands for symbols I use often
\theoremstyle{definition}
\newtheorem{defn}{Definition}
\newtheorem{thm}{Theorem}
\newtheorem{cor}{Corollary}
\newtheorem*{rmk}{Remark}
\newtheorem{lem}{Lemma}
\newtheorem*{joke}{Joke}
\newtheorem{ex}{Example}
\newtheorem*{soln}{Solution}
\newtheorem{prop}{Proposition}

\newcommand{\lra}{\longrightarrow}
\newcommand{\ra}{\rightarrow}
\newcommand{\surj}{\twoheadrightarrow}
\newcommand{\graph}{\mathrm{graph}}
\newcommand{\bb}[1]{\mathbb{#1}}
\newcommand{\Z}{\bb{Z}}
\newcommand{\Q}{\bb{Q}}
\newcommand{\R}{\bb{R}}
\newcommand{\C}{\bb{C}}
\newcommand{\N}{\bb{N}}
\newcommand{\M}{\mathbf{M}}
\newcommand{\m}{\mathbf{m}}
\newcommand{\MM}{\mathscr{M}}
\newcommand{\HH}{\mathscr{H}}
\newcommand{\Om}{\Omega}
\newcommand{\Ho}{\in\HH(\Om)}
\newcommand{\bd}{\partial}
\newcommand{\del}{\partial}
\newcommand{\bardel}{\overline\partial}
\newcommand{\textdf}[1]{\textbf{\textsf{#1}}\index{#1}}
\newcommand{\img}{\mathrm{img}}
\newcommand{\ip}[2]{\left\langle{#1},{#2}\right\rangle}
\newcommand{\inter}[1]{\mathrm{int}{#1}}
\newcommand{\exter}[1]{\mathrm{ext}{#1}}
\newcommand{\cl}[1]{\mathrm{cl}{#1}}
\newcommand{\ds}{\displaystyle}
\newcommand{\vol}{\mathrm{vol}}
\newcommand{\cnt}{\mathrm{ct}}
\newcommand{\osc}{\mathrm{osc}}
\newcommand{\LL}{\mathbf{L}}
\newcommand{\UU}{\mathbf{U}}
\newcommand{\support}{\mathrm{support}}
\newcommand{\AND}{\;\wedge\;}
\newcommand{\OR}{\;\vee\;}
\newcommand{\Oset}{\varnothing}
\newcommand{\st}{\ni}
\newcommand{\wh}{\widehat}

%Pagination stuff.
\setlength{\topmargin}{-.3 in}
\setlength{\oddsidemargin}{0in}
\setlength{\evensidemargin}{0in}
\setlength{\textheight}{9.in}
\setlength{\textwidth}{6.5in}
\pagestyle{empty}



\begin{document}

\subsection*{Exo 2}
La fonction $f(x) = \sqrt{(x^3-1)(x-1)}$ est d\'efinie si $(x^3-1)(x-1)$ est positif ou nul . Il y a 4 cas possibles:
\begin{center}
\begin{tabular}{c|c|c} 
     & $(x-1) \geq 0$ & $(x-1) \leq 0$ \\
      \hline
     $(x^3-1) \geq 0$  & $x \geq 1$ & $x =1$\\
      \hline
     $(x^3-1) \leq 0$ & $x =1$ & $x \leq 1$\\
\end{tabular}
\end{center}


La fonction $f$ est d\'efinie pour toute les valeurs $x \in \mathbb{R}$.
\\ 
Donc la proposition est vraie.


\subsection*{Exo 3}
La fonction $f$ est bijective si et seulement si elle est injective et surjective.
\\
V\'erifions la proposition $f$ est surjective.
$f$ est surjective si $\forall y \in \mathbb{R}, \exists x \in \mathbb{N}, f(x) = y$
\\
Prenons la valeur $y = 0.3$. Existe-t-il un $x$ tel que $f(x) = y$. 2 cas possibles.
\\
Cas 1, $x$ est pair. Donc $f(x) = \frac{x}{2}$.
$$\frac{x}{2} = 0.3$$
$$x = 0.6$$
$$x \notin \mathbb{N}$$
\\
Cas 2, $x$ est impair. Donc $f(x) = -\frac{x+1}{2}$.
$$-\frac{x+1}{2} = 0.3$$
$$x = -1.6$$
$$x \notin \mathbb{N}$$

Il n'existe pas de $x \in \mathbb{N}$ pour $y=0.3$.
\\ 
Donc la proposition est fausse.

\subsection*{Exo 4}
La fonction $f$ est bijective si et seulement si elle est injective et surjective.
\subsubsection*{$f$ Injective ?}
V\'erifions la proposition $f$ est injective.
$f$ est injective si $\forall x_1, x_2 \in \mathbb{N}, f(x_1) = f(x_2) \implies x_1 = x_2$. Il y a 4 cas possibles.

\begin{center}
\begin{tabular}{c|c|c} 
                                  & $x_1$ pair & $x_1$ impair\\
      \hline
     $x_2$ pair   & $\frac{x_1}{2} = \frac{x_2}{2}$ donc $x_1=x_2$ & $\frac{x_2}{2} = -\frac{x_1+1}{2}$ \\
      \hline
     $x_2$ impair  & $\frac{x_1}{2} = -\frac{x_2+1}{2}$ & $-\frac{x_1+1}{2} = -\frac{x_2+1}{2}$ donc $x_1=x_2$\\
\end{tabular}
\end{center}



$\frac{x_2}{2} = -\frac{x_1+1}{2}$, $x_2 = -(x_1+1)$, $f(x_1) = f(x_2)$ impossible car $x_1, x_2 \in \mathbb{N}$ 
\\
$\frac{x_1}{2} = -\frac{x_1+1}{2}$, $x_1 = -(x_2+1)$, $f(x_1) = f(x_2)$ impossible car $x_1, x_2 \in \mathbb{N}$ 
\\ 
Donc f est injective

\subsubsection*{$f$ Surjective ?}
V\'erifions la proposition $f$ est surjective. f est surjective
$f$ est surjective si $\forall y \in \mathbb{Z}, \exists x \in \mathbb{N}, f(x) = y$
\\
Existe-t-il un $x$ tel que $f(x) = y$. 2 cas possibles.
\\
Cas 1, $x$ est pair. Donc $f(x) = \frac{x}{2}$.
$$\frac{x}{2} = y$$
$$x = 2*y$$
$$x \in \mathbb{N} \; si\; y \geq 0$$
\\
Cas 2, $x$ est impair. Donc $f(x) = -\frac{x+1}{2}$.
$$-\frac{x+1}{2} = y$$
$$x = -2*y -1$$
$$x \in \mathbb{N} \; si \; -2*y -1 \geq 0 \; ou \; y \leq -1$$
\\
Donc, $\forall y \in \mathbb{Z}, \exists x \in \mathbb{N}, f(x) = y$, donc f est surjective.


$f$ est injective et surjective, donc $f$ est bijective.
\\ 
 Donc la proposition est vraie.


\subsection*{Exo 5}
$$(e^{n^{3}} - e^{n}) = (e^{3n} - e^{n})$$
$$= e^{3n} (1 - \frac{1}{e^{2n}})$$
On a 
$$\lim_{n \to + \infty} e^{3n}  = + \infty$$
$$\lim_{n \to + \infty} \frac{1}{e^{2n}} = 0$$
Donc
$$\lim_{n \to + \infty} (e^{n^{3}} - e^{n}) = + \infty$$
\\ 
Donc la proposition est fausse.


\subsection*{Exo 6}
$$\lim_{n \to + \infty} \frac{1}{n} = 0, \lim_{n \to + \infty} e^{\frac{1}{n}} = 1$$
Donc
$$\lim_{n \to + \infty} \frac{3*ln^2 \, n}{e^{\frac{1}{n}}+(1+ln \, n)^2} = \lim_{n \to + \infty} \frac{3*ln^2 \, n}{(1+ln \, n)^2}$$
$$=\lim_{n \to + \infty} \frac{3*ln^2 \, n}{1+2ln \, n+ln^2 \, n}$$
$$=\lim_{n \to + \infty} \frac{3*ln^2 \, n}{ln^2 \, n *(\frac {1}{ln^2 \, n}+\frac {2}{ln \, n} +1)}$$
$$=\lim_{n \to + \infty} \frac{3}{\frac {1}{ln^2 \, n}+\frac {2}{ln \, n} +1}$$
\\
On a 
$$\lim_{n \to + \infty} \frac {1}{ln^2 \, n} = 0$$
$$\lim_{n \to + \infty} \frac {2}{ln \, n} = 0$$
\\
Donc
$$\lim_{n \to + \infty} \frac{3}{\frac {1}{ln^2 \, n}+\frac {2}{ln \, n} +1} = 3$$
\\ 
Donc la proposition est vraie.


\subsection*{Exo 7}
Si il existe $N \in \mathbb{N}$ tel que $\forall n > N, \frac{n^2+n*ln(n)}{n^3} > \frac{1}{2}$ alors 
$$ \lim_{n \to + \infty} \frac{n^2+n*ln(n)}{n^3} > \frac{1}{2} $$
\\
$$ \frac{n^2+n*ln(n)}{n^3} = \frac{n^3(\frac{1}{n}+\frac {ln(n)}{n^2})}{n^3}$$
$$= \frac{1}{n}+\frac {ln(n)}{n^2}$$
\\
$$ \lim_{n \to + \infty} \frac{1}{n} = 0 $$
$$ \lim_{n \to + \infty} \frac {ln(n)}{n^2} = 0$$
\\
Donc
$$ \lim_{n \to + \infty} \frac{n^2+n*ln(n)}{n^3} = 0$$
Ce qui contredit l'hypothese
\\ 
Donc la proposition est fausse.



\subsection*{Exo 8}
On a $a = e^{ln\, a}$ pour tout $a$ positif.
Calculons la limite: $$\lim_{n \to \infty} ln((1+1/n)^n)$$
$$\lim_{n \to \infty} n\, ln((1+1/n))$$
On a $\lim_{n \to \infty} n = \infty$ et $\lim_{n \to \infty} ln((1+1/n) = 0$. 
Donc la limite ne peut pas \^etre calcul\'ee.
Utilisons la r\`egle de L'Hospital.
$$\lim_{n \to \infty} n\, ln((1+1/n)) = \lim_{n \to \infty} \frac{ln((1+1/n))}{\frac{1}{n}}
 = \lim_{n \to \infty} \frac{(ln((1+1/n)))'}{(\frac{1}{n})'}$$

On a:


 $(\frac{1}{n})' =  \frac{-1}{n^2}$ 


$(ln((1+1/n)))' = \frac{\frac{-1}{n^2}}{1+\frac{1}{n}} = \frac{\frac{-1}{n^2}}{\frac{n+1}{n}} = \frac{-1*n}{n^2*(n+1)} 
= \frac{-1}{n*(n+1)}$.


Donc
$$
 \lim_{n \to \infty} \frac{(ln((1+1/n)))'}{(\frac{1}{n})'} = 
 \lim_{n \to \infty} \frac{\frac{-1}{n*(n+1)}}{\frac{-1}{n^2}} = 
 \lim_{n \to \infty} \frac{n^2}{n*(n+1)} =
 \lim_{n \to \infty} \frac{n}{n+1} = 1$$
$$\lim_{n \to \infty} e^{ln((1+1/n)^n)} = e$$

Donc la proposition est fausse.


\subsection*{Exo 9}
La suite $u_n$ est croissante si $u_{n+1} - u_{n} > 0$.
$$u_{n+1} - u_{n} = \sum_{k=1}^{n+1} \frac{1}{k} - \sum_{k=1}^{n} \frac{1}{k}$$
$$ = \sum_{k=1}^{n} \frac{1}{k} + \frac{1}{n+1} - \sum_{k=1}^{n} \frac{1}{k}$$
$$ = \frac{1}{n+1} $$

$\frac{1}{n+1}$ est un nombre positif, par cons\'equent la suite $u_n$ est croissante.
\\ 
Donc la proposition est vraie.


\subsection*{Exo 10}
La suite $v_n$ est croissante si $v_{n+1} -v_n  \geq 0$.
$$v_{n+1} -v_n = \frac{u_1+ \ldots +u_{n+1}}{n+1} - \frac{u_1+ \ldots +u_{n}}{n}$$
$$= \frac{n*(u_1+ \ldots +u_{n+1}) - (n+1)*(u_1+ \ldots +u_{n})}  {n*(n+1)}$$
$$= \frac{n*(u_1+ \ldots +u_{n}+u_{n+1}) - (n+1)*(u_1+ \ldots +u_{n})}  {n*(n+1)}$$
$$= \frac{n*(u_1+ \ldots +u_n) + n*u_{n+1} - n*(u_1+ \ldots +u_{n}) - (u_1+ \ldots +u_{n})}  {n*(n+1)}$$
$$= \frac{ n*u_{n+1} - (u_1+ \ldots +u_{n})}  {n*(n+1)}$$

Comme la suite $u_n$ est croissante, $u_{n+1}$ est plus grand que $u_{1}, u_{2} \ldots n_{n}$. Donc $n*u_{n+1} > u_{1}+ u_{2} \ldots +n_{n}$. Par consequent,
$$\frac{ n*u_{n+1} - (u_1+ \ldots +u_{n})}  {n*(n+1)} > 0$$
$$v_{n+1} -v_n  > 0$$
\\ 
Donc la proposition est vraie.

\subsection*{Exo 11}
La fonction $e^{-n}$ pour $n \in \mathbb{N}^*$ est d\'ecroissante et tend vers $0$ quand $n=+\infty$. Donc la fonction $2-e^{-n}$ est croissante, toujours strictement inferieure \'a $2$ et tend vers $2$ quand $n=+\infty$. Donc, $2$ est une borne sup\'erieure de $2-e^{-n}$.
\\ 
Donc la proposition est vraie.

\subsection*{Exo 12}
Soit la suite $u_{n} = (-1)^n$.
$$\lim_{n \to +\infty} \frac{u_{n+1}}{u_{n}} = \lim_{n \to +\infty} \frac{(-1)^{n+1}}{(-1)^n}=-1$$
La suite $u_{n}$ n'est pas convergente. \\
Il existe une suite $u_n$ qui n'est pas convergente et qui v\'erifie l'hypothese. \\ 
Donc la proposition est fausse.

\subsection*{Exo 13}
$$\lim_{n \to +\infty} \frac{u_{n+1}}{u_{n}} = \frac{1}{2}$$ 
alors $\exists N_0 \in \mathbb{N}$ tel que $\forall n > N_{0}, \frac{u_{n+1}}{u_{n}} \approx \frac{1}{2}$. \\
Soit $a = u_{N_0}$ alors $u_{N_0+1} \approx \frac{a}{2}, u_{N_0+2} \approx \frac{a}{4}, \ldots , u_{N_0+m} \approx \frac{a}{2^{m}}$. \\
Par consequent, la suite $u_n$ converge vers $0$.
\\ 
Donc la proposition est vraie.


\subsection*{Exo 14}
La suite $u_n=sin(n)$ est une suite p\'eriodique de p\'eriode p=360. Soit la fonction $f(n) = n * 360$. \\
La fonction $f$ est strictement croissante et la suite extraite $v_n=(u_n)_{f}$ est une suite constante ($=0$). \\
La suite $(sin (n))_{n}$ admet une suite extraite qui est convergente: ($v_n$) . \\
\\ 
Donc la proposition est fausse.

\end{document}

