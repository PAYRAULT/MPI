\documentclass[]{book}

%These tell TeX which packages to use.
\usepackage{array,epsfig}
\usepackage{amsmath}
\usepackage{amsfonts}
\usepackage{amssymb}
\usepackage{amsxtra}
\usepackage{amsthm}
\usepackage{mathrsfs}
\usepackage{color}

%Here I define some theorem styles and shortcut commands for symbols I use often
\theoremstyle{definition}
\newtheorem{defn}{Definition}
\newtheorem{thm}{Theorem}
\newtheorem{cor}{Corollary}
\newtheorem*{rmk}{Remark}
\newtheorem{lem}{Lemma}
\newtheorem*{joke}{Joke}
\newtheorem{ex}{Example}
\newtheorem*{soln}{Solution}
\newtheorem{prop}{Proposition}

\newcommand{\lra}{\longrightarrow}
\newcommand{\ra}{\rightarrow}
\newcommand{\surj}{\twoheadrightarrow}
\newcommand{\graph}{\mathrm{graph}}
\newcommand{\bb}[1]{\mathbb{#1}}
\newcommand{\Z}{\bb{Z}}
\newcommand{\Q}{\bb{Q}}
\newcommand{\R}{\bb{R}}
\newcommand{\C}{\bb{C}}
\newcommand{\N}{\bb{N}}
\newcommand{\M}{\mathbf{M}}
\newcommand{\m}{\mathbf{m}}
\newcommand{\MM}{\mathscr{M}}
\newcommand{\HH}{\mathscr{H}}
\newcommand{\Om}{\Omega}
\newcommand{\Ho}{\in\HH(\Om)}
\newcommand{\bd}{\partial}
\newcommand{\del}{\partial}
\newcommand{\bardel}{\overline\partial}
\newcommand{\textdf}[1]{\textbf{\textsf{#1}}\index{#1}}
\newcommand{\img}{\mathrm{img}}
\newcommand{\ip}[2]{\left\langle{#1},{#2}\right\rangle}
\newcommand{\inter}[1]{\mathrm{int}{#1}}
\newcommand{\exter}[1]{\mathrm{ext}{#1}}
\newcommand{\cl}[1]{\mathrm{cl}{#1}}
\newcommand{\ds}{\displaystyle}
\newcommand{\vol}{\mathrm{vol}}
\newcommand{\cnt}{\mathrm{ct}}
\newcommand{\osc}{\mathrm{osc}}
\newcommand{\LL}{\mathbf{L}}
\newcommand{\UU}{\mathbf{U}}
\newcommand{\support}{\mathrm{support}}
\newcommand{\AND}{\;\wedge\;}
\newcommand{\OR}{\;\vee\;}
\newcommand{\Oset}{\varnothing}
\newcommand{\st}{\ni}
\newcommand{\wh}{\widehat}

%Pagination stuff.
\setlength{\topmargin}{-.3 in}
\setlength{\oddsidemargin}{0in}
\setlength{\evensidemargin}{0in}
\setlength{\textheight}{9.in}
\setlength{\textwidth}{6.5in}
\pagestyle{empty}



\begin{document}

\subsection*{Exercice 7}
(a) on a $tan\, x = \frac{sin\, x}{cos\, x}$, donc $tan\, x$ est d\'efinie lorsque $cos\, x \neq 0$, ou $x \neq \pi/2+n\,\pi$.

$$tan^2 x \leq 3$$
$$\sqrt{tan^2 x} \leq \sqrt{3}$$
$$|tan\, x| \leq \sqrt{3}$$
$$tan\, x \leq \sqrt{3} \; and \; -tan\, x \leq \sqrt{3}$$
$$x \leq \arctan\, \sqrt{3} \; and \; x \geq -\arctan\, \sqrt{3}$$
$$x \leq 1.249 \; and \; x \geq -1.249$$
Comme $\pi/2 > 1.249$ alors $x \in [-1.249, 1.249]$ et la fonction $tan$ est de p\'eriode $\pi$, l'in\'equation est vraie pour $x \in [-1.249+n.\pi, 1.249+n.\pi]$.



(b) La function $tan\, x$ est d'efini pour $x \neq \pi/2+n\,\pi$. Faisons le changement de variable $y=tan\, x$. L'in\'equation devient $\frac{y^2-2}{y^2-1}$ avec $y \neq |1|$.
$$\frac{y^2-2}{y^2-1} \leq \frac{1}{2}$$
$$2(y^2-2) \leq y^2-1$$
$$y^2 \leq 3$$
$$|y| \leq \sqrt(3)$$
$$y \leq \sqrt(3) \; and \; y \geq -3$$
$$tan\, x \leq \sqrt(3) \; and \; tan\, x \geq -3$$
$x \leq 1.249 \; and \; x \geq -1.249$ par (a) et $tan\, x \neq |1|$ (car $y \neq |1|$), donc $x \neq |0.7854|$. Par cons\'equent l'in\'equation est v\'erifi\'ee lorsque $x \in [-1.249,-0.7854[ \cup ]-0.7854,0.7854[ \cup ]0.7854,1.249]$ \`a la p\'eriode de $\pi$.
  
\subsection*{Exercice 8}
$$P(t) = (\lambda + 1)t^2 + 2at + \lambda -1$$
$$\Delta = (2a)^2 - 4.(\lambda + 1)(\lambda - 1)$$
$$\Delta = (2a)^2 - 4.(\lambda^2 - 1)$$
$$\Delta = 4(a^2 - \lambda^2 + 1)$$
$$t = \frac{-2a \pm \sqrt{4(a^2 - \lambda^2 + 1)}}{2.(\lambda + 1)}$$
$$t = \frac{-a \pm \sqrt{a^2 - \lambda^2 + 1}}{(\lambda + 1)}$$

(a) $a^2-\lambda^2+1=0$, 1 seule racine $t=\frac{-a}{\lambda+1}$ avec $\lambda+1 \neq 0$. On a $P(t)$ positif entre $]-\infty, \frac{-a}{\lambda+1}[$ et n\'egatif entre $]\frac{-a}{\lambda+1},+\infty]$ si $\frac{-a}{\lambda+1} > 0$ et l'inverse sinon.
  


(b) $a^2-\lambda^2+1>0$, 2 seule racine $t=\frac{-a \pm \sqrt{a^2 - \lambda^2 + 1}}{(\lambda + 1)}$, avec $\lambda+1 \neq 0$. On a $P(t)$ positif entre $]-\infty, \frac{-a - \sqrt{a^2 - \lambda^2 + 1}}{(\lambda + 1)}[ \cup ]\frac{-a + \sqrt{a^2 - \lambda^2 + 1}}{(\lambda + 1)}, +\infty[$ et n\'egatif entre $]\frac{-a - \sqrt{a^2 - \lambda^2 + 1}}{(\lambda + 1)},\frac{-a + \sqrt{a^2 - \lambda^2 + 1}}{(\lambda + 1)}[$ si $\frac{-a}{\lambda+1} > 0$ et l'inverse sinon.

  
\subsection*{Exercice 9}
\subsubsection*{P1}
$$acos(t+b) = a(cos(t)cos(b)-sin(t)sin(b))$$ 
$$a(cos(t)cos(b)-sin(t)sin(b)) = a.cos(b).cos(t) - a.sin(b).sin(t)$$
$$a.cos(b).cos(t) - a.sin(b).sin(t) = \alpha cos(t) + \beta sin(t)$$ 
donc $\alpha = a.cos(b)$ et $\beta=-a.sin(b)$. 

(a) $\frac{\beta}{\alpha} = \frac{-a.sin(b)}{a.cos(b)} = -tan(b)$ donc $b=tan^{-1}(\frac{-\beta}{\alpha})$


(b) $\alpha^2 + \beta^2 = a^2.cos^2(b) + a^2.sin^2(b) = a^2.(cos^2(b) + sin^2(b)) = a^2$ donc $a=\sqrt{\alpha^2 + \beta^2}$.

\subsubsection*{P2}
(a) $cos(t) + sin(t) = \lambda$. Donc, $\alpha = \beta = 1$, ce qui fait $a = \sqrt{2}$ et $b=tan^{-1}(-1)=-\pi/4$. On a $cos(t-\pi/4) = cos(t) + sin(t) = \lambda$, donc $t=cos^{-1}(\lambda)+\pi/4$.


(b) idem avec $\alpha = 1$ et $\beta = \sqrt{3}$. Donc $a=2$ et $b=tan^{-1}(-\sqrt{3})$.

(c) idem avec $\alpha = 1$ et $\beta = -1$. $a = \sqrt{2}$ et $b=tan^{-1}(-1)=\pi/4$. On a $cos(t+\pi/4) = cos(t) + sin(t) = \lambda$, donc $t=cos^{-1}(\lambda)-\pi/4$.

\subsection*{Exercice 14}
\subsubsection*{Fonction$f_1$}
(a)$f_1$ Bien d\'efinie?\\
Soit $x \in \R$, montrons que $f(x) \in \R$.
\begin{itemize}
\item Si $x < 0$, on a: 
$$ f_1(x) = x^2 \in \R_{+} \in \R$$
\item Si $x \ge 0$, on a:
$$f_1(x) = -\frac{x}{2} \in \R_{-} \in \R$$
\end{itemize}
$f_1$ est bien d\'efinie sur $\R \mapsto \R$.


(b) Injective?\\
$f$ est injective si $\forall x_1, x_2 \in \mathbb{N}, f(x_1) = f(x_2) \implies x_1 = x_2$. Il y a 4 cas possibles.

\begin{center}
\begin{tabular}{c|c|c} 
                 & $x_1 < 0$ & $x_1 \ge 0$ \\
      \hline
     $x_2 < 0$   & $x_1^2 = x_2^2$ donc $x_1=x_2$ quand $x_1 <0$ et $x_2 < 0$ & $x_2^2 = -\frac{x_1}{2}$ on a $x_1=-2x_2^2$  impossible car $x_1 \ge 0$\\
      \hline
     $x_2 \ge 0$ & $x_1^2 = -\frac{x_2}{2}$ on a $x_1=-2x_1^2$  impossible car $x_2 \ge 0$ & $-\frac{x_1}{2} = -\frac{x_2}{2}$ donc $x_1=x_2$\\
\end{tabular}
\end{center}
$f_1(x)$ est injective.\\

(c) Surjective?\\
$f$ est surjective si $\forall y \in \mathbb{R}, \exists x \in \mathbb{N}, f(x) = y$.\\
\begin{itemize}
\item Si $x < 0$, on a: 
$ f_1(x) = x^2 = y$, soit $x=|\sqrt{y}|$ donc $\forall y \in \R_{+}, \exists x = -\sqrt{y} < 0$
\item Si $x \ge 0$, on a:
$f_1(x) = -\frac{x}{2}=y$, soit $x= -2y$ donc $\forall y \in \R_{-}^{*}, \exists x = -2y \ge 0$
\end{itemize}
La fonction $f_1$ est surjective.

(d) Bijective?\\
La fonction$f_1(n)$ est bijective car elle est injective et surjective.\\

(e) Fonction inverse?\\
$$f_1^{-1}(x) = 
\left\{ 
\begin{array}{l l}
 -\sqrt{x} & x > 0\\
 -2x & x \le 0\\
\end{array}
\right. 
$$

Preuve:
$$(f_1 \circ f_1^{-1})(x) =
\left\{ 
\begin{array}{l l}
 (-\sqrt{x})^2 & x > 0\\
 -\frac{-2x}{2} & x \le 0\\
\end{array}
\right. 
$$

$$(f_1 \circ f_1^{-1})(x) =
\left\{ 
\begin{array}{l l}
 x & x > 0\\
 x & x \le 0\\
\end{array}
\right. 
$$


\subsubsection*{Fonction$f_2$}
(a)$f_2$ Bien d\'efinie?\\
Soit $x \in \R$, montrons que $f(x) \in \R$.
\begin{itemize}
\item Si $x \le 1$, on a: 
$$ f_2(x) = \frac{3}{2} - \frac{x}{2} \in \R$$
\item Si $x > 1$, on a:
$$f_2(x) = (x - 1) + 1 = x \in \R$$
\end{itemize}
$f_2$ est bien d\'efinie sur $\R \mapsto \R$.

(b) Injective?\\
$f$ est injective si $\forall x_1, x_2 \in \mathbb{N}, f(x_1) = f(x_2) \implies x_1 = x_2$. Il y a 4 cas possibles.

\begin{center}
\begin{tabular}{c|c|c} 
                 & $x_1 \le 1$ & $x_1 > 1$ \\
      \hline
     $x_2 \le 1$   & $\frac{3}{2} - \frac{x_1}{2} = \frac{3}{2} - \frac{x_2}{2}$, donc $x_1=x_2$ & $x_1 = \frac{3}{2} - \frac{x_2}{2}$ faux car $f_2(\frac{3}{2}) = f_2(0)$\\
      \hline
     $x_2 > 1$ & & \\
\end{tabular}
\end{center}
$f(x)$ n'est pas injective.\\

(c) Surjective?\\
$f$ est surjective si $\forall y \in \mathbb{R}, \exists x \in \mathbb{N}, f(x) = y$.\\
\begin{itemize}
\item Si $x \le 1$, on a: 
$ f_2(x) = \frac{3}{2} - \frac{x}{2} = y$, soit $x=3-2y$ donc $\forall y \ge 1, \exists x = 3 - 2y$
\item Si $x > 1$, on a:
$f_2(x) = y$, soit $x= y$ donc $\forall y >1, \exists x$
\end{itemize}
La fonction $f_2$ est surjective.

(d) Bijective?\\
La fonction$f_2(n)$ n'est pas bijective car elle n'est pas injective.

\subsubsection*{Fonction$f_3$}
(a)$f_3$ Bien d\'efinie?\\
Soit $x \in \R$, montrons que $f(x) \in \R$.
\begin{itemize}
\item Si $x \le 1$, on a: 
$$ f_3(x) = x^2 \in \R^{+} \in \R$$
\item Si $x > 1$, on a:
$$f_3(x) = x^3 \in \R^{+} \in \R$$
\end{itemize}
$f_3$ est bien d\'efinie sur $\R \mapsto \R$.

(b) Injective?\\
$f$ est injective si $\forall x_1, x_2 \in \mathbb{N}, f(x_1) = f(x_2) \implies x_1 = x_2$. Il y a 4 cas possibles.

\begin{center}
\begin{tabular}{c|c|c} 
                 & $x_1 \le 1$ & $x_1 > 1$ \\
      \hline
     $x_2 \le 1$ & $x_1^2 = x_2^2$, faux $(0.5)^2=(-0.5)^2$ & \\
      \hline
     $x_2 > 1$ & & \\
\end{tabular}
\end{center}
$f(x)$ n'est pas injective.\\

(c) Surjective?\\
$f$ est surjective si $\forall y \in \mathbb{R}, \exists x \in \mathbb{N}, f(x) = y$.\\
\begin{itemize}
\item Si $x \le 1$, on a: 
$ f_2(x) = \frac{3}{2} - \frac{x}{2} = y$, soit $x=3-2y$ donc $\forall y \ge 1, \exists x = 3 - 2y$
\item Si $x > 1$, on a:
$f_2(x) = y$, soit $x= y$ donc $\forall y >1, \exists x$
\end{itemize}
La fonction $f_3$ est surjective.

(d) Bijective?\\
La fonction$f_2(n)$ n'est pas bijective car elle n'est pas injective.

\subsubsection*{Fonction$f_4$}
(a)$f_4$ Bien d\'efinie?\\
Soit $x \in \R^{+}$, montrons que $f(x) \in \R^{+}$.
\begin{itemize}
\item Si $x \in [0,1]$, on a: 
$$ f_4(x) = x^2 \in [0,1] \in \R^{+}$$
\item Si $x > 1$, on a:
$$f_4(x) = x^3 \in \R^{+}$$
\end{itemize}
$f_4$ est bien d\'efinie  sur $\R^{+} \mapsto \R^{+}$.

(b) Injective?\\
$f$ est injective si $\forall x_1, x_2 \in \mathbb{N}, f(x_1) = f(x_2) \implies x_1 = x_2$. Il y a 4 cas possibles.

\begin{center}
\begin{tabular}{c|c|c} 
                 & $x_1 \in [0,1]$ & $x_1 > 1$ \\
      \hline
     $x_2 \in [0,1]$ & $x_1^2 = x_2^2$,  $x_1=x_2$ quand $x_1 \in [0,1]$ et $x_2 \in [0,1]$ &  $x_2^2 = x_1^3$ impossible quand $x_2 \in [0,1]$ et $x_1 >1 $ \\ 
      \hline
     $x_2 > 1$ & $x_1^2 = x_2^3$ impossible quand $x_1 \in [0,1]$ et $x_2 >1 $ & $x_1^3 = x_2^3$ $x_1=x_2$ quand $x_1 > 1$ et $x_2 >1 $\\
\end{tabular}
\end{center}
$f_4(x)$ est injective.\\



(c) Surjective?\\
$f$ est surjective si $\forall y \in \mathbb{R}_{+}, \exists x \in \mathbb{R}_{+}, f(x) = y$.\\
\begin{itemize}
\item Si $x \in [0,1]$, on a: 
$ f_4(x) = x^2 = y$, soit $x=\sqrt(y)$ donc $\forall y \in \mathbb{R}_{+}, \exists x = \sqrt(y)$
\item Si $x > 1$, on a:
$ f_4(x) = y^3$, soit $x= y^{\frac{1}{3}}$ donc $\forall y >1, \exists x = y^{\frac{1}{3}}$
\end{itemize}
La fonction $f_3$ est surjective.

(d) Bijective?\\
La fonction$f_4(n)$ est bijective car elle est injective et surjective.

(e) Fonction inverse?\\
$$f_4^{-1}(x) = 
\left\{L
\begin{array}{l l}
 \sqrt{x} & x \in [0,1] \\
 x^{\frac{1}{3}} & x > 1\\
\end{array}
\right. 
$$

\subsection*{Exercice 17}
\subsubsection*{P1}
La suite $u_n$ est stationnaire. En effet, $\forall n > 1, U_n = U_1$. Donc, la suite $U_n$ admet une limite: $2$.

\subsubsection*{P2}
\begin{itemize}
\item $|a|< 1$ $\forall \epsilon > 0, \exists N_{\epsilon} \in \N, \; tel\; que\; \forall n \in \N, n \ge N_{\epsilon} \implies | U_n - l| \le \epsilon$. Soit $N_{\epsilon}$ tel que $U_{N_{\epsilon}} = \epsilon$ et $l=0$, alors pour $n>N_{\epsilon}$ on a $| U_n - 0| \le \epsilon$. Vrai car $|a|<1 \implies |a^n|<|a^{n+1}$
\item $a = 1$, la suite est stationnaire, et elle converge vers $1$.
\item $a > 1$, $\forall A \in \R, \exists \N_{A} \in \N, \; tel\; que\; \forall n \in \N, n \ge N_{A} \implies u_n \ge A$. Soit $N_{A}\; tel\; que\; U_{N_{A}} = A$ alors $\forall n > 1, U_{N_{A+n}} = a^n.U_{N_{A}} > A$.
\item $a \leq -1$, la suite diverge. En effet $U_{2n-1}<U_{2n}>U_{2n+1}$.
\end{itemize}

\subsubsection*{P3}
La suite $U_n = \frac{n^n}{n!}$ tend vers $+\infty$. Soit $N_{A}\; tel\; que\; U_{N_{A}} = \frac{N_{A}^{N_{A}}}{N_{A}!} = A$ alors $U_{N_{A+1}} = \frac{(N_{A}+1)^{(N_{A}+1)}}{(N_{A}+1)!} = \frac{(N_{A}+1)^{N_{A}}.(N_{A}+1)}{N_{A}!.(N_{A}+1)} = \frac{(N_{A}+1)^{N_{A}}}{N_{A}!} > A$ car $(N_{A}+1)^{N_{A}} > N_{A}^{N_{A}}$  



La suite $V_n = \frac{n!}{2^n}$ tend vers $+\infty$. Soit $N_{A}\; tel\; que\; U_{N_{A}} = \frac{N_{A}!}{2^{N_{A}}} = A$ alors $U_{N_{A+1}} = \frac{(N_{A}+1)!}{2^{(N_{A}+1)}} = \frac{N_{A}!.(N_{A}+1)}{2^{N_{A}}.2} = A.\frac{(N_{A}+1)}{2} > A$ car $\frac{(N_{A}+1)}{2} > 1$.  



La suite $W_n = \frac{a^n}{n!}$ tend vers $0$. A finir.  



\subsection*{Exercice 20}
Montrer que:
$$ \forall n \in \N^{*}, 2 - \frac{1}{2n} \leq \sqrt{4+\frac{(-1)^n}{n}} \leq 2 + \frac{1}{\sqrt{n}}$$
$$ \forall n \in \N^{*}, (2 - \frac{1}{2n})^2 \leq 4+\frac{(-1)^n}{n} \leq (2 + \frac{1}{\sqrt{n}})^2$$
$$ \forall n \in \N^{*}, 4 - \frac{4}{2n} + \frac{1}{4n^2} \leq 4+\frac{(-1)^n}{n} \leq 4 + \frac{4}{\sqrt{n}} +\frac{1}{n}$$
$$ \forall n \in \N^{*}, - \frac{2}{n} + \frac{1}{4n^2} \leq \frac{(-1)^n}{n} \leq \frac{4}{\sqrt{n}} +\frac{1}{n}$$

\begin{itemize}
\item n est pair
$$ \forall n \in \N^{*}, - \frac{2}{n} + \frac{1}{4n^2} \leq \frac{1}{n} \leq \frac{4}{\sqrt{n}} +\frac{1}{n}$$
$$ \forall n \in \N^{*}, - \frac{2}{n} + \frac{1}{4n^2} - \frac{1}{n}\leq 0 \leq \frac{4}{\sqrt{n}}$$
$$ \forall n \in \N^{*}, - \frac{3}{n} + \frac{1}{4n^2} \leq 0 \leq \frac{4}{\sqrt{n}}$$
$$ \forall n \in \N^{*}, - \frac{12n}{4n^2} + \frac{1}{4n^2} \leq 0 \leq \frac{4}{\sqrt{n}}$$
$$ \forall n \in \N^{*}, \frac{1-12n}{4n^2} \leq 0 \leq \frac{4}{\sqrt{n}}$$
vraie

\item n est impair
$$ \forall n \in \N^{*}, - \frac{2}{n} + \frac{1}{4n^2} \leq \frac{-1}{n} \leq \frac{4}{\sqrt{n}} +\frac{1}{n}$$
$$ \forall n \in \N^{*}, - \frac{2}{n} + \frac{1}{4n^2} \leq \frac{-1}{n} \leq \frac{1+4\sqrt{n}}{n}$$
$$ \forall n \in \N^{*}, - \frac{2}{n} + \frac{1}{4n^2} + \frac{1}{n} \leq 0 \leq \frac{1+4\sqrt{n}}{n} + \frac{1}{n}$$
$$ \forall n \in \N^{*}, - \frac{1}{n} + \frac{1}{4n^2} \leq 0 \leq \frac{2+4\sqrt{n}}{n}$$
$$ \forall n \in \N^{*}, - \frac{4n}{4n^2} + \frac{1}{4n^2} \leq 0 \leq \frac{2+4\sqrt{n}}{n}$$
$$ \forall n \in \N^{*}, \frac{1-4n}{4n^2} \leq 0 \leq \frac{2+4\sqrt{n}}{n}$$
Vraie
\end{itemize}

La suite converge vers $2$.
$$\lim_{n \to \infty}2 - \frac{1}{2n} = 2$$
$$\lim_{n \to \infty}2 - \frac{1}{\sqrt{n}} = 2$$

\subsection*{Exercice 21}
A faire


\subsection*{Exercice 23}
A faire


\subsection*{Exercice 25}
\subsubsection*{P1}
$$\sum_{m=0}^{n}(n-m) = n + (n-1) + n-2 + \ldots + 1 + 0 = 0 + 1 + \ldots + (n-1) + n = \sum_{m=0}^{n}m=S_{1,n}$$
Et
$$2S_{1,n} = \sum_{m=0}^{n}(m)+\sum_{m=0}^{n}(n-m) = \sum_{m=0}^{n}m+(n-m) = \sum_{m=0}^{n}n = n+n+n \ldots +n = n(n+1)$$
$$S_{1,n} = \frac{n(n+1)}{2}$$

\subsubsection*{P2}
(a) $$(x+1)^3 = (x+1)(x+1)^2 = (x+1)(x^2+2x+1) = x^3+2x^2+x+x^2+2x+1=x^3+3x^2+3x+1$$


(b) trivial

(c) 
On a 
$$\sum_{m=0}^{n}(m+1)^3 = \sum_{m=0}^{n}(m^3+3m^2+3m+1)$$
$$\sum_{m=0}^{n}(m+1)^3 - \sum_{m=0}^{n}m^3 = \sum_{m=0}^{n}(m^3+3m^2+3m+1) - \sum_{m=0}^{n}m^3 $$
$$(1^3+2^3+3^3+\ldots+(n+1)^3)-(0^3+1^3+2^3+\ldots+n^3) = \sum_{m=0}^{n}(3m^2+3m+1)$$
$$(1^3+2^3+3^3+\ldots+(n+1)^3)-(0^3+1^3+2^3+\ldots+n^3) = \sum_{m=0}^{n}(3m^2+3m+1)$$
$$(n+1)^3 = \sum_{m=0}^{n}(3m^2+3m+1)$$
$$(n+1)^3 = \sum_{m=0}^{n}3m^2+\sum_{m=0}^{n}3m+\sum_{m=0}^{n}1)$$
$$(n+1)^3 = 3\sum_{m=0}^{n}m^2+3\frac{n(n+1)}{2}+n$$
$$3\sum_{m=0}^{n}m^2 = (n+1)^3 - 3\frac{n(n+1)}{2} -n$$
$$\sum_{m=0}^{n}m^2 = \frac{(n+1)^3}{3} - \frac{n(n+1)}{2} - \frac{n}{3}$$
$$\sum_{m=0}^{n}m^2 = \frac{2(n+1)^3 -3n(n+1) -2n}{6}$$

\subsection*{Exercice 26}
\subsubsection*{P1}

$$(a-b)(\sum_{j=0}^{n-1}a^{n-1-j}b^j)$$
$$= (a-b)(a^{n-1}b^{0} + a^{n-2}b^{1} + a^{n-3}b^{2} + \ldots + a^{0}b^{n-1})$$
$$= a(a^{n-1}b^{0} + a^{n-2}b^{1} + a^{n-3}b^{2} + \ldots + a^{0}b^{n-1}) - b(a^{n-1}b^{0} + a^{n-2}b^{1} + a^{n-3}b^{2} + \ldots + a^{0}b^{n-1})$$
$$= (a^{n}b^{0} + a^{n-1}b^{1} + a^{n-2}b^{2} + \ldots + a^{1}b^{n-1}) - (a^{n-1}b^{1} + a^{n-2}b^{2} + a^{n-3}b^{3} + \ldots + a^{0}b^{n})$$
$$= a^n - b^n$$

\subsubsection*{P2}


\subsection*{Exercice 29}
$M$ est un majorant de l'ensemble $A$, si $M \in \R, x \in A , x \le M$. \\
$M$ est un minorant de l'ensemble $A$, si $M \in \R, x \in A , x \ge M$. \\
$M$ est la borne sup\'erieure de l'ensemble $A$ (not\'e $M = sup\, A$) si $M_1$ est un majorant de $A$ alors $M \le M_1$.\\
$M$ est la borne inf\'erieure de l'ensemble (not\'e $M = inf\, A$) $A$ si $M_1$ est un minorant de $A$ alors $M \le M_1$.\\

\subsection*{P1}
(a) $sup [1,2] = 2$ et $inf [1,2] = 1$. En effet, admettons qu'il existe un majorant $M_1$ de l'ensemble $A$ inf\'erieur \`a 2. $M_1 < 2$. Mais $M_1$ n'est pas un majorant car $2 \in [1,2]$. Contradiction, donc $M_1$ n'existe pas et $2$ est la borne sup\'erieure. M\^eme raisonnement pour la borne inf\'erieure. \\

(b) $sup ]1,2[ = 2$ et $inf ]1,2[ = 1$. En effet, admettons qu'il existe un majorant $M_1$ de l'ensemble $A$ inf\'erieur \`a 2, $M_1 < 2$. Mais $M_1$ n'est pas un majorant car il existe toujours un r\'eel entre $M_1$ et $2$. Contradiction, donc $M_1$ n'existe pas et $2$ est la borne sup\'erieure. M\^eme raisonnement pour la borne inf\'erieure. \\
 
(c) $A = \{\frac{1}{n}| n \in \N\}$. $sup\, A = 1$ et $inf\, A = 0$. En effet $A = ]0,1]$ car la limite de $\frac{1}{n}$ pour $n \to \infty$ est $0$, $\frac{1}{1}=1$ et la fonction $f(x)=\frac{1}{x}$ est d\'ecroissante. \\

(d) $A = \{\frac{(-1)^n}{n}| n \in \N\}$. $sup\, A = \frac{1}{2}$ et $inf\, A = -1$.  Il faut montrer que $A = [-1, \frac{1}{2}[$. $n$ est impair, $A_1 = [-1,0[$, $n$ est pair, $A_p = ]0, \frac{1}{2}]$.


\subsection*{P2}


\subsection*{P3}
 


\subsection*{Exercice 32}
A faire


\subsection*{Exercice 33}
A faire


\subsection*{Exercice 37}
A faire


\subsection*{Exercice 38}
A faire


QED

\end{document}

