\documentclass[]{book}

%These tell TeX which packages to use.
\usepackage{array,epsfig}
\usepackage{amsmath}
\usepackage{amsfonts}
\usepackage{amssymb}
\usepackage{amsxtra}
\usepackage{amsthm}
\usepackage{mathrsfs}
\usepackage{color}

%Here I define some theorem styles and shortcut commands for symbols I use often
\theoremstyle{definition}
\newtheorem{defn}{Definition}
\newtheorem{thm}{Theorem}
\newtheorem{cor}{Corollary}
\newtheorem*{rmk}{Remark}
\newtheorem{lem}{Lemma}
\newtheorem*{joke}{Joke}
\newtheorem{ex}{Example}
\newtheorem*{soln}{Solution}
\newtheorem{prop}{Proposition}

\newcommand{\lra}{\longrightarrow}
\newcommand{\ra}{\rightarrow}
\newcommand{\surj}{\twoheadrightarrow}
\newcommand{\graph}{\mathrm{graph}}
\newcommand{\bb}[1]{\mathbb{#1}}
\newcommand{\Z}{\bb{Z}}
\newcommand{\Q}{\bb{Q}}
\newcommand{\R}{\bb{R}}
\newcommand{\C}{\bb{C}}
\newcommand{\N}{\bb{N}}
\newcommand{\M}{\mathbf{M}}
\newcommand{\m}{\mathbf{m}}
\newcommand{\MM}{\mathscr{M}}
\newcommand{\HH}{\mathscr{H}}
\newcommand{\Om}{\Omega}
\newcommand{\Ho}{\in\HH(\Om)}
\newcommand{\bd}{\partial}
\newcommand{\del}{\partial}
\newcommand{\bardel}{\overline\partial}
\newcommand{\textdf}[1]{\textbf{\textsf{#1}}\index{#1}}
\newcommand{\img}{\mathrm{img}}
\newcommand{\ip}[2]{\left\langle{#1},{#2}\right\rangle}
\newcommand{\inter}[1]{\mathrm{int}{#1}}
\newcommand{\exter}[1]{\mathrm{ext}{#1}}
\newcommand{\cl}[1]{\mathrm{cl}{#1}}
\newcommand{\ds}{\displaystyle}
\newcommand{\vol}{\mathrm{vol}}
\newcommand{\cnt}{\mathrm{ct}}
\newcommand{\osc}{\mathrm{osc}}
\newcommand{\LL}{\mathbf{L}}
\newcommand{\UU}{\mathbf{U}}
\newcommand{\support}{\mathrm{support}}
\newcommand{\AND}{\;\wedge\;}
\newcommand{\OR}{\;\vee\;}
\newcommand{\Oset}{\varnothing}
\newcommand{\st}{\ni}
\newcommand{\wh}{\widehat}

%Pagination stuff.
\setlength{\topmargin}{-.3 in}
\setlength{\oddsidemargin}{0in}
\setlength{\evensidemargin}{0in}
\setlength{\textheight}{9.in}
\setlength{\textwidth}{6.5in}
\pagestyle{empty}



\begin{document}

\subsection*{Exercice 7}
(a) on a $tan\, x = \frac{sin\, x}{cos\, x}$, donc $tan\, x$ est d\'efinie lorsque $cos\, x \neq 0$, ou $x \neq \pi/2+n\,\pi$.

$$tan^2 x \leq 3$$
$$\sqrt{tan^2 x} \leq \sqrt{3}$$
$$|tan\, x| \leq \sqrt{3}$$
$$tan\, x \leq \sqrt{3} \; and \; -tan\, x \leq \sqrt{3}$$
$$x \leq \arctan\, \sqrt{3} \; and \; x \geq -\arctan\, \sqrt{3}$$
$$x \leq 1.249 \; and \; x \geq -1.249$$
Comme $\pi/2 > 1.249$ alors $x \in [-1.249, 1.249]$ et la fonction $tan$ est de p\'eriode $\pi$, l'in\'equation est vraie pour $x \in [-1.249+n.\pi, 1.249+n.\pi]$.



(b) La function $tan\, x$ est d'efini pour $x \neq \pi/2+n\,\pi$. Faisons le changement de variable $y=tan\, x$. L'in\'equation devient $\frac{y^2-2}{y^2-1}$ avec $y \neq |1|$.
$$\frac{y^2-2}{y^2-1} \leq \frac{1}{2}$$
$$2(y^2-2) \leq y^2-1$$
$$y^2 \leq 3$$
$$|y| \leq \sqrt(3)$$
$$y \leq \sqrt(3) \; and \; y \geq -3$$
$$tan\, x \leq \sqrt(3) \; and \; tan\, x \geq -3$$
$x \leq 1.249 \; and \; x \geq -1.249$ par (a) et $tan\, x \neq |1|$ (car $y \neq |1|$), donc $x \neq |0.7854|$. Par cons\'equent l'in\'equation est v\'erifi\'ee lorsque $x \in [-1.249,-0.7854[ \cup ]-0.7854,0.7854[ \cup ]0.7854,1.249]$ \`a la p\'eriode de $\pi$.
  
\subsection*{Exercice 8}
$$P(t) = (\lambda + 1)t^2 + 2at + \lambda -1$$
$$\Delta = (2a)^2 - 4.(\lambda + 1)(\lambda - 1)$$
$$\Delta = (2a)^2 - 4.(\lambda^2 - 1)$$
$$\Delta = 4(a^2 - \lambda^2 + 1)$$
$$t = \frac{-2a \pm \sqrt{4(a^2 - \lambda^2 + 1)}}{2.(\lambda + 1)}$$
$$t = \frac{-a \pm \sqrt{a^2 - \lambda^2 + 1}}{(\lambda + 1)}$$

(a) $a^2-\lambda^2+1=0$, 1 seule racine $t=\frac{-a}{\lambda+1}$ avec $\lambda+1 \neq 0$. On a $P(t)$ positif entre $]-\infty, \frac{-a}{\lambda+1}[$ et n\'egatif entre $]\frac{-a}{\lambda+1},+\infty]$ si $\frac{-a}{\lambda+1} > 0$ et l'inverse sinon.
  


(b) $a^2-\lambda^2+1>0$, 2 seule racine $t=\frac{-a \pm \sqrt{a^2 - \lambda^2 + 1}}{(\lambda + 1)}$, avec $\lambda+1 \neq 0$. On a $P(t)$ positif entre $]-\infty, \frac{-a - \sqrt{a^2 - \lambda^2 + 1}}{(\lambda + 1)}[ \cup ]\frac{-a + \sqrt{a^2 - \lambda^2 + 1}}{(\lambda + 1)}, +\infty[$ et n\'egatif entre $]\frac{-a - \sqrt{a^2 - \lambda^2 + 1}}{(\lambda + 1)},\frac{-a + \sqrt{a^2 - \lambda^2 + 1}}{(\lambda + 1)}[$ si $\frac{-a}{\lambda+1} > 0$ et l'inverse sinon.

  
\subsection*{Exercice 9}
\subsubsection*{P1}
$$acos(t+b) = a(cos(t)cos(b)-sin(t)sin(b))$$ 
$$a(cos(t)cos(b)-sin(t)sin(b)) = a.cos(b).cos(t) - a.sin(b).sin(t)$$
$$a.cos(b).cos(t) - a.sin(b).sin(t) = \alpha cos(t) + \beta sin(t)$$ 
donc $\alpha = a.cos(b)$ et $\beta=-a.sin(b)$. 

(a) $\frac{\beta}{\alpha} = \frac{-a.sin(b)}{a.cos(b)} = -tan(b)$ donc $b=tan^{-1}(\frac{-\beta}{\alpha})$


(b) $\alpha^2 + \beta^2 = a^2.cos^2(b) + a^2.sin^2(b) = a^2.(cos^2(b) + sin^2(b)) = a^2$ donc $a=\sqrt{\alpha^2 + \beta^2}$.

\subsubsection*{P2}
(a) $cos(t) + sin(t) = \lambda$. Donc, $\alpha = \beta = 1$, ce qui fait $a = \sqrt{2}$ et $b=tan^{-1}(-1)=-\pi/4$. On a $cos(t-\pi/4) = cos(t) + sin(t) = \lambda$, donc $t=cos^{-1}(\lambda)+\pi/4$.


(b) idem avec $\alpha = 1$ et $\beta = \sqrt{3}$. Donc $a=2$ et $b=tan^{-1}(-\sqrt{3})$.

(c) idem avec $\alpha = 1$ et $\beta = -1$. $a = \sqrt{2}$ et $b=tan^{-1}(-1)=\pi/4$. On a $cos(t+\pi/4) = cos(t) + sin(t) = \lambda$, donc $t=cos^{-1}(\lambda)-\pi/4$.

\subsection*{Exercice 14}
A faire


\subsection*{Exercice 17}
A faire


\subsection*{Exercice 20}
A faire


\subsection*{Exercice 21}
A faire


\subsection*{Exercice 23}
A faire



\subsection*{Exercice 25}
A faire


\subsection*{Exercice 26}
A faire



\subsection*{Exercice 32}
A faire


\subsection*{Exercice 33}
A faire


\subsection*{Exercice 37}
A faire


\subsection*{Exercice 38}
A faire


QED

\end{document}

