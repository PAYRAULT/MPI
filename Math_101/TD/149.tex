\documentclass[]{book}

%These tell TeX which packages to use.
\usepackage{array,epsfig}
\usepackage{amsmath}
\usepackage{amsfonts}
\usepackage{amssymb}
\usepackage{amsxtra}
\usepackage{amsthm}
\usepackage{mathrsfs}
\usepackage{color}

%Here I define some theorem styles and shortcut commands for symbols I use often
\theoremstyle{definition}
\newtheorem{defn}{Definition}
\newtheorem{thm}{Theorem}
\newtheorem{cor}{Corollary}
\newtheorem*{rmk}{Remark}
\newtheorem{lem}{Lemma}
\newtheorem*{joke}{Joke}
\newtheorem{ex}{Example}
\newtheorem*{soln}{Solution}
\newtheorem{prop}{Proposition}

\newcommand{\lra}{\longrightarrow}
\newcommand{\ra}{\rightarrow}
\newcommand{\surj}{\twoheadrightarrow}
\newcommand{\graph}{\mathrm{graph}}
\newcommand{\bb}[1]{\mathbb{#1}}
\newcommand{\Z}{\bb{Z}}
\newcommand{\Q}{\bb{Q}}
\newcommand{\R}{\bb{R}}
\newcommand{\C}{\bb{C}}
\newcommand{\N}{\bb{N}}
\newcommand{\M}{\mathbf{M}}
\newcommand{\m}{\mathbf{m}}
\newcommand{\MM}{\mathscr{M}}
\newcommand{\HH}{\mathscr{H}}
\newcommand{\Om}{\Omega}
\newcommand{\Ho}{\in\HH(\Om)}
\newcommand{\bd}{\partial}
\newcommand{\del}{\partial}
\newcommand{\bardel}{\overline\partial}
\newcommand{\textdf}[1]{\textbf{\textsf{#1}}\index{#1}}
\newcommand{\img}{\mathrm{img}}
\newcommand{\ip}[2]{\left\langle{#1},{#2}\right\rangle}
\newcommand{\inter}[1]{\mathrm{int}{#1}}
\newcommand{\exter}[1]{\mathrm{ext}{#1}}
\newcommand{\cl}[1]{\mathrm{cl}{#1}}
\newcommand{\ds}{\displaystyle}
\newcommand{\vol}{\mathrm{vol}}
\newcommand{\cnt}{\mathrm{ct}}
\newcommand{\osc}{\mathrm{osc}}
\newcommand{\LL}{\mathbf{L}}
\newcommand{\UU}{\mathbf{U}}
\newcommand{\support}{\mathrm{support}}
\newcommand{\AND}{\;\wedge\;}
\newcommand{\OR}{\;\vee\;}
\newcommand{\Oset}{\varnothing}
\newcommand{\st}{\ni}
\newcommand{\wh}{\widehat}

%Pagination stuff.
\setlength{\topmargin}{-.3 in}
\setlength{\oddsidemargin}{0in}
\setlength{\evensidemargin}{0in}
\setlength{\textheight}{9.in}
\setlength{\textwidth}{6.5in}
\pagestyle{empty}



\begin{document}

\subsection*{Exercice 17}
Une suite r\'elle $u_n$ converge vers le r\'eel $l$ si
$$\forall \epsilon > 0, \exists N_{\epsilon} \in \N, \forall n \geq N_{\epsilon}  \implies  |u_n -l| < \epsilon \textrm{  [P1]}$$

Une suite r\'elle $u_n$ diverge vers $+\infty$ si
$$\forall A \in \R \exists N_{A} \in \N, \forall n \geq N_{A}  \implies  U_A \geq A \textrm{  [P2]}$$

Une suite r\'elle $u_n$ diverge vers $-\infty$ si
$$\forall B \in \R \exists N_{B} \in \N, \forall n \geq N_{B}  \implies  U_B \leq B \textrm{  [P3]}$$


\subsubsection*{Exercice 17.1}
Supposons que $l=2$.\\
Prenons un $\epsilon>0$, trouvons un $N_{\epsilon}$ tel que $|u_{N_{\epsilon}} - 2| < \epsilon $. Par exemple, $N_{\epsilon} = 4$, car $|u_4 - 2| = |2-2| = 0 < \epsilon$.\\ 
Maintenant, v\'erifions [P1] pour $l=2$.
$\forall \epsilon > 0, \forall n > 4, |u_n - 2| < \epsilon$, calculons $u_{n>4} = 2 = u_4$, la propri\'et\'e [P1] est v\'erif\'ee pour tous les $n > N_{\epsilon}$.

\subsubsection*{Exercice 17.2}
\begin{itemize}
\item $a = 1$, $\lim_{n\to\infty} u_n = 1$
\item $|a| < 1$, $\lim_{n\to\infty} u_n = 0$
\item $a \leq -1$, pas de limite
\item $a \geq 1$, $\lim_{n\to\infty} u_n = +\infty$
\end{itemize}

Pour le second cas, posons $l=0$, trouvons un $N_{\epsilon}$ tel que $|u_{N_{\epsilon}} - 0| < \epsilon $. Par exemple, $N_{\epsilon}, |a^{N_{\epsilon}}| < \epsilon$. $N_{\epsilon}$ existe car $|a|<1$. On a bien $|u_{N_{\epsilon}} - 0| = |a^{N_{\epsilon}}| < \epsilon$.\\ 
Maintenant, v\'erifions [P1] pour $l=0$. 
$\forall \epsilon > 0, \exists N_{\epsilon} \in \N, |u_{N_{\epsilon}} - 0| < \epsilon$. Calculons $u_{N_{\epsilon}+1} = a^{N_{\epsilon}+1} < a^{N_{\epsilon}}$, la propri\'et\'e [P1] est v\'erif\'ee pour tous les $n> N_{\epsilon}$.\\
M\^eme raisonnement pour les autres cas.


\subsubsection*{Exercice 17.3}
La suite diverge. Prenons un $A$, et calculons $N_A$ tel que $u_{N_A} > A$. Calculons $u_{N_A+1}$. \\
$u_{N_A+1} = \frac{(N_A+1)^{N_A+1}}{(N_A+1)!} = \frac{(N_A+1).\ldots.(N_A+1).(N_A+1)}{N_A!(N_A+1)} = \frac{(N_A+1).\ldots.(N_A+1)}{N_A!}$, Les nombres au num\'erateur sont toujours plus grand que ceux du num\'erateur pour $u_{N_A}$, donc la suite $u_{N_A+1} > u_{N_A} > A$.\\
La propri\'et\'e $[P2]$ est v\'erifi\'ee.\\

La suite diverge. Prenons un $A$, et calculons $N_A$ tel que $u_{N_A} > A$. Calculons $u_{N_A+1}$. \\
$u_{N_A+1} = \frac{(N_A+1)!}{2^{N_A+1}} = \frac{N_A!(N_A+1)}{2.2.2.\ldots.2} = U_{N_A}.\frac{(N_A+1)}{2}$. Donc la suite $u_{N_A+1} > u_{N_A} > A$.\\
La propri\'et\'e $[P2]$ est v\'erifi\'ee.\\

\subsection*{Exercice 18}
On a $\forall \epsilon > 0, \exists N_{\epsilon} \in \N, \forall n \geq N_{\epsilon}  \implies  |u_n -l| < \epsilon$

QED

\end{document}

