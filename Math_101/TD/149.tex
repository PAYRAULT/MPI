\documentclass[]{book}

%These tell TeX which packages to use.
\usepackage{array,epsfig}
\usepackage{amsmath}
\usepackage{amsfonts}
\usepackage{amssymb}
\usepackage{amsxtra}
\usepackage{amsthm}
\usepackage{mathrsfs}
\usepackage{color}

%Here I define some theorem styles and shortcut commands for symbols I use often
\theoremstyle{definition}
\newtheorem{defn}{Definition}
\newtheorem{thm}{Theorem}
\newtheorem{cor}{Corollary}
\newtheorem*{rmk}{Remark}
\newtheorem{lem}{Lemma}
\newtheorem*{joke}{Joke}
\newtheorem{ex}{Example}
\newtheorem*{soln}{Solution}
\newtheorem{prop}{Proposition}

\newcommand{\lra}{\longrightarrow}
\newcommand{\ra}{\rightarrow}
\newcommand{\surj}{\twoheadrightarrow}
\newcommand{\graph}{\mathrm{graph}}
\newcommand{\bb}[1]{\mathbb{#1}}
\newcommand{\Z}{\bb{Z}}
\newcommand{\Q}{\bb{Q}}
\newcommand{\R}{\bb{R}}
\newcommand{\C}{\bb{C}}
\newcommand{\N}{\bb{N}}
\newcommand{\M}{\mathbf{M}}
\newcommand{\m}{\mathbf{m}}
\newcommand{\MM}{\mathscr{M}}
\newcommand{\HH}{\mathscr{H}}
\newcommand{\Om}{\Omega}
\newcommand{\Ho}{\in\HH(\Om)}
\newcommand{\bd}{\partial}
\newcommand{\del}{\partial}
\newcommand{\bardel}{\overline\partial}
\newcommand{\textdf}[1]{\textbf{\textsf{#1}}\index{#1}}
\newcommand{\img}{\mathrm{img}}
\newcommand{\ip}[2]{\left\langle{#1},{#2}\right\rangle}
\newcommand{\inter}[1]{\mathrm{int}{#1}}
\newcommand{\exter}[1]{\mathrm{ext}{#1}}
\newcommand{\cl}[1]{\mathrm{cl}{#1}}
\newcommand{\ds}{\displaystyle}
\newcommand{\vol}{\mathrm{vol}}
\newcommand{\cnt}{\mathrm{ct}}
\newcommand{\osc}{\mathrm{osc}}
\newcommand{\LL}{\mathbf{L}}
\newcommand{\UU}{\mathbf{U}}
\newcommand{\support}{\mathrm{support}}
\newcommand{\AND}{\;\wedge\;}
\newcommand{\OR}{\;\vee\;}
\newcommand{\Oset}{\varnothing}
\newcommand{\st}{\ni}
\newcommand{\wh}{\widehat}

%Pagination stuff.
\setlength{\topmargin}{-.3 in}
\setlength{\oddsidemargin}{0in}
\setlength{\evensidemargin}{0in}
\setlength{\textheight}{9.in}
\setlength{\textwidth}{6.5in}
\pagestyle{empty}



\begin{document}

\subsection*{Exercice 17}
Une suite r\'elle $u_n$ converge vers le r\'eel $l$ si
$$\forall \epsilon > 0, \exists N_{\epsilon} \in \N, \forall n \geq N_{\epsilon}  \implies  |u_n -l| < \epsilon \textrm{  [P1]}$$

Une suite r\'elle $u_n$ diverge vers $+\infty$ si
$$\forall A \in \R \exists N_{A} \in \N, \forall n \geq N_{A}  \implies  U_A \geq A \textrm{  [P2]}$$

Une suite r\'elle $u_n$ diverge vers $-\infty$ si
$$\forall B \in \R \exists N_{B} \in \N, \forall n \geq N_{B}  \implies  U_B \leq B \textrm{  [P3]}$$


\subsubsection*{Exercice 17.1}
Supposons que $l=2$.\\
Prenons un $\epsilon>0$, trouvons un $N_{\epsilon}$ tel que $|u_{N_{\epsilon}} - 2| < \epsilon $. Par exemple, $N_{\epsilon} = 4$, car $|u_4 - 2| = |2-2| = 0 < \epsilon$.\\ 
Maintenant, v\'erifions [P1] pour $l=2$.
$\forall \epsilon > 0, \forall n > 4, |u_n - 2| < \epsilon$, calculons $u_{n>4} = 2 = u_4$, la propri\'et\'e [P1] est v\'erif\'ee pour tous les $n > N_{\epsilon}$.

\subsubsection*{Exercice 17.2}
\begin{itemize}
\item $a = 1$, $\lim_{n\to\infty} u_n = 1$
\item $|a| < 1$, $\lim_{n\to\infty} u_n = 0$
\item $a \leq -1$, pas de limite
\item $a \geq 1$, $\lim_{n\to\infty} u_n = +\infty$
\end{itemize}

Pour le second cas, posons $l=0$, trouvons un $N_{\epsilon}$ tel que $|u_{N_{\epsilon}} - 0| < \epsilon $. Par exemple, $N_{\epsilon}, |a^{N_{\epsilon}}| < \epsilon$. $N_{\epsilon}$ existe car $|a|<1$. On a bien $|u_{N_{\epsilon}} - 0| = |a^{N_{\epsilon}}| < \epsilon$.\\ 
Maintenant, v\'erifions [P1] pour $l=0$. 
$\forall \epsilon > 0, \exists N_{\epsilon} \in \N, |u_{N_{\epsilon}} - 0| < \epsilon$. Calculons $u_{N_{\epsilon}+1} = a^{N_{\epsilon}+1} < a^{N_{\epsilon}}$, la propri\'et\'e [P1] est v\'erif\'ee pour tous les $n> N_{\epsilon}$.\\
M\^eme raisonnement pour les autres cas.


\subsubsection*{Exercice 17.3}
La suite diverge. Prenons un $A$, et calculons $N_A$ tel que $u_{N_A} > A$. Calculons $u_{N_A+1}$. \\
$u_{N_A+1} = \frac{(N_A+1)^{N_A+1}}{(N_A+1)!} = \frac{(N_A+1).\ldots.(N_A+1).(N_A+1)}{N_A!(N_A+1)} = \frac{(N_A+1).\ldots.(N_A+1)}{N_A!}$, Les nombres au num\'erateur sont toujours plus grand que ceux du num\'erateur pour $u_{N_A}$, donc la suite $u_{N_A+1} > u_{N_A} > A$.\\
La propri\'et\'e $[P2]$ est v\'erifi\'ee.\\

La suite diverge. Prenons un $A$, et calculons $N_A$ tel que $u_{N_A} > A$. Calculons $u_{N_A+1}$. \\
$u_{N_A+1} = \frac{(N_A+1)!}{2^{N_A+1}} = \frac{N_A!(N_A+1)}{2.2.2.\ldots.2} = U_{N_A}.\frac{(N_A+1)}{2}$. Donc la suite $u_{N_A+1} > u_{N_A} > A$.\\
La propri\'et\'e $[P2]$ est v\'erifi\'ee.\\

\subsection*{Exercice 18}
On a $\forall \epsilon > 0, \exists N_{\epsilon} \in \N, \forall n \geq N_{\epsilon}  \implies  |u_n -l| < \epsilon$


\subsection*{Exercice 39}
\subsubsection*{Exercice 39.1}
$$\lim_{n \to +\infty}{u_{2n}} = (-1)^{2n} = 1 \textrm{  et    } \lim_{n \to +\infty}{u_{2n+1}} = (-1)^{2n+1} = -1$$

\subsubsection*{Exercice 39.2}
$$\lim_{n \to +\infty}{v_{4n}} = \sin(4n\frac{\pi}{2}) = 0 \textrm{  et    } \lim_{n \to +\infty}{v_{4n+1}} = \sin((4n+1)\frac{\pi}{2}) = 1$$
$$\lim_{n \to +\infty}{v_{4n+2}} = \sin((4n+2)\frac{\pi}{2}) = 0 \textrm{  et    } \lim_{n \to +\infty}{v_{4n+3}} = \sin((4n+3)\frac{\pi}{2}) = -1$$

\subsubsection*{Exercice 39.3}
$$\lim_{n \to +\infty}{w_{6n}} = \sin(6n\frac{\pi}{3}) = 0 \textrm{  et    } \lim_{n \to +\infty}{w_{6n+1}} = \sin((6n+1)\frac{\pi}{3}) = \frac{\sqrt{3}}{2}$$
$$\lim_{n \to +\infty}{w_{6n+2}} = \sin((6n+2)\frac{\pi}{3}) = \frac{\sqrt{3}}{2} \textrm{  et    } \lim_{n \to +\infty}{w_{6n+3}} = \sin((6n+3)\frac{\pi}{3}) = -1$$
$$\lim_{n \to +\infty}{w_{6n+4}} = \sin((6n+4)\frac{\pi}{3}) = -\frac{\sqrt{3}}{2} \textrm{  et    } \lim_{n \to +\infty}{w_{6n+5}} = \sin((6n+5)\frac{\pi}{3}) = -\frac{\sqrt{3}}{2}$$

\subsubsection*{Exercice 39.4}
$$\lim_{n \to +\infty}{x_{4n}} = \sin^{4n}(4n\frac{\pi}{2}) = 0 \textrm{  et    } \lim_{n \to +\infty}{x_{4n+1}} = \sin^{4n+1}((4n+1)\frac{\pi}{2}) = 1$$
$$\lim_{n \to +\infty}{x_{4n+2}} = \sin^{4n+2}((4n+2)\frac{\pi}{2}) = 0 \textrm{  et    } \lim_{n \to +\infty}{x_{4n+3}} = \sin^{4n+3}((4n+3)\frac{\pi}{2}) = -1$$

\subsubsection*{Exercice 39.5}
$$\lim_{n \to +\infty}{x_{4n}} = |\sin(4n\frac{\pi}{2})|^{\frac{4n}{2}} = 0 \textrm{  et    } \lim_{n \to +\infty}{x_{4n+1}} = |\sin((4n+1)\frac{\pi}{2})|^{\frac{4n+1}{2}} = 1$$
$$\lim_{n \to +\infty}{x_{4n+2}} = |\sin((4n+2)\frac{\pi}{2})|^{\frac{4n+2}{2}} = 0 \textrm{  et  } \lim_{n \to +\infty}{x_{4n+3}} = |\sin((4n+3)\frac{\pi}{2})|^{\frac{4n+3}{2}} = 1$$

\subsubsection*{Exercice 39.6}
$$\lim_{n \to +\infty}{v_{4n}} = 4n\sin(4n\frac{\pi}{2}) = 0 \textrm{  et    } \lim_{n \to +\infty}{v_{4n+1}} = (4n+1)\sin((4n+1)\frac{\pi}{2}) = 4n+1$$
$$\lim_{n \to +\infty}{v_{4n+2}} = \sin((4n+2)\frac{\pi}{2}) = 0 \textrm{  et    } \lim_{n \to +\infty}{v_{4n+3}} = (4n+3)\sin((4n+3)\frac{\pi}{2}) = -4n-3$$

\subsection*{Exercice 40}
\subsubsection*{Exercice 40.1}
$$\lim_{n \to +\infty}u_{2n} = \frac{1}{2^{2n}} = 0 \textrm{  et  } \lim_{n \to +\infty}u_{2n+1} = 2n+1 = +\infty$$ 

\subsubsection*{Exercice 40.2}
$$\lim_{n \to +\infty}u_{2n} = \frac{2n+4}{2^{2n}} = 0 \textrm{  et  } \lim_{n \to +\infty}u_{2n+1} = \frac{2n+7}{e^{2n+1}} = 0$$ 

\subsubsection*{Exercice 40.3}
Valeur $a=0$, $u_{2n}$ n'est pas d\'efinie.\\

Valeur $|a|<1$, $u_{2n}$\\
$$\lim_{n \to +\infty}u_{2n} = \frac{2n+4}{a^{2n}} = +\infty$$

Valeur $|a|=1$, $u_{2n}$\\
$$\lim_{n \to +\infty}u_{2n} = \frac{2n+4}{1^{2n}} = 2n+4$$

Valeur $|a|>1$, $u_{2n}$\\
$$\lim_{n \to +\infty}u_{2n} = \frac{2n+4}{a^{2n}} = 0$$

Et
$$ \lim_{n \to +\infty}u_{2n+1} = 2n+1 = +\infty$$ 

\subsubsection*{Exercice 40.4}
$$ \lim_{n \to +\infty}u_{3n} = \frac{3n+4}{3n} = 1+\frac{4}{3n} = 1$$ 
Et
$$ \lim_{n \to +\infty}u_{3n+1} = 3$$ 
Et
$$ \lim_{n \to +\infty}u_{3n+2} = \frac{(3n+2)^2+1}{3n^2} = \frac{9n^2+12n+5}{3n^2} = 3+\frac{4}{n}+\frac{5}{3n^2} = 3$$ 



\subsection*{Exercice 51}
\subsubsection*{Exercice 51.1}
$$\lim_{x\to1}x^2+1 = \lim_{x\to0} (x+1)^2+1 = \lim_{x\to0}x^2 + \lim_{x\to0}2x + \lim_{x\to0}1 + \lim_{x\to0}1 = 0+0+1+1 = 2$$

\subsubsection*{Exercice 51.2}
$$\lim_{x\to+\infty}-x-\ln(x) = -\lim_{x\to+\infty}x - \lim_{x\to+\infty} \ln(x) = -\infty -\infty = -\infty  $$

\subsubsection*{Exercice 51.3}
$$\lim_{x\to+\infty}x-\ln(x) = x.(1-\frac{\ln(x)}{x}) = \lim_{x\to+\infty}x.\lim_{x\to+\infty}(1-\frac{\ln(x)}{x}) = +\infty.(1-0) = +\infty$$
Car $\ln(x) << x$.

\subsubsection*{Exercice 51.4}
$$\lim_{x\to1^{+}}\frac{1}{x-1}+\ln(x-1) = \lim_{x\to0^{+}}\frac{1}{x}+\ln(x) = \lim_{x\to0^{+}}\frac{1+x\ln(x)}{x} = \lim_{x\to0^{+}}\frac{1}{x}.\lim_{x\to0^{+}}(1+x\ln(x))$$
$$=\frac{1}{0^+}.(1+0) = +\infty$$
Voir exercice 17.

\subsubsection*{Exercice 51.5}
$$\lim_{x\to-\infty}x^2-x = \lim_{x\to-\infty}x(x-1) = \lim_{x\to-\infty}x.\lim_{x\to-\infty}(x-1) = -\infty.-\infty = +\infty$$


\subsection*{Exercice 52}
On a $\lim_{x\to0}\frac{\sin(x)}{x} = 1$, calculons $\lim_{x\to0}\cos(x)$.\\
En utilisant la r\`egle de l'Hospital, on a $\lim_{x\to0}\frac{\sin(x)}{x} = \lim_{x\to0}\frac{\sin'(x)}{x'} = \lim_{x\to0}\frac{\cos(x)}{1} = \lim_{x\to0}\cos(x) = 1$.\\


\subsubsection*{Exercice 51.1}
$$\lim_{x\to0}\frac{\cos(x)}{x-\frac{\pi}{2}} = \frac{\lim_{x\to0}\cos(x)}{\lim_{x\to0}x-\frac{\pi}{2}}= \frac{1}{-\frac{\pi}{2}} = -\frac{2}{\pi}$$


\subsubsection*{Exercice 52.2}
On a $\tan(x) = \frac{\sin(x)}{\cos(x)}$. Calculons $\lim_{x\to0}\tan(x) = \lim_{x\to0}\frac{\sin(x)}{\cos(x)} = \lim_{x\to0}\frac{x\sin(x)}{x\cos(x)}$.
$$ \lim_{x\to0}\frac{sin(x)}{x}.\lim_{x\to0}\frac{x}{cos(x)} = 1 . \frac{\lim_{x\to0}x}{\lim_{x\to0}cos x} = 1.0 = 0$$

\subsubsection*{Exercice 52.3}
$$\lim_{x\to0}\frac{\cos(x)-1}{x^2} = \lim_{x\to0}\frac{\cos(x)-(\cos^2(x)+\sin^2(x))}{x^2} = \lim_{x\to0}\frac{\cos(x)}{x^2}-\frac{\cos^2(x)}{x^2}-\frac{\sin^2(x))}{x^2}$$
$$\lim_{x\to0}\left(\frac{\cos(x)}{x^2}-\frac{\cos^2(x)}{x^2}\right)-\lim_{x\to0}\frac{\sin^2(x)}{x^2} = \lim_{x\to0}\left(\frac{\cos(x)}{x^2}-\frac{\cos^2(x)}{x^2}\right)-\lim_{x\to0}\frac{\sin(x)}{x}.\lim_{x\to0}\frac{\sin(x)}{x}$$
$$\lim_{x\to0}\left(\frac{\cos(x)}{x^2}(1-\cos(x)\right)-1$$

\subsection*{Exercice 53}
\subsubsection*{Exercice 53.1}
$$\lim_{x\to1}\frac{1}{x-1}-\frac{2}{x2-1} = \lim_{x\to1}\frac{x+1}{(x+1)(x-1)}-\frac{2}{(x+1)(x-1)} = \lim_{x\to1}\frac{x-1}{(x+1)(x-1)} = \lim_{x\to1}\frac{1}{x+1} = \frac{1}{2}$$

\subsubsection*{Exercice 53.2}
$$\lim_{x\to0}\frac{\sqrt{1+x}-1}{x} = \lim_{x\to0}\frac{(\sqrt{1+x}-1)(\sqrt{1+x}+1)}{x(\sqrt{1+x}+1)} = \lim_{x\to0}\frac{1+x-1}{x(\sqrt{1+x}+1)} = \lim_{x\to0}\frac{1}{\sqrt{1+x}+1} = \frac{1}{\sqrt{1}+1} = \frac{1}{2}$$

\subsubsection*{Exercice 53.3}
Calculer $\lim_{x\to0}\frac{|x|}{x}$. 2 cas $\lim_{x\to0^+}\frac{|x|}{x}$ et $\lim_{x\to0^-}\frac{|x|}{x}$
$$\lim_{x\to0^+}\frac{|x|}{x} = \lim_{x\to0^+}\frac{x}{x} = 1$$
$$\lim_{x\to0^-}\frac{|x|}{x} = \lim_{x\to0^-}\frac{-x}{x} = -1$$

\subsubsection*{Exercice 53.4}
$$\lim_{x\to0}\frac{\sin(4x)}{\tan(5x)}$$
Utilisation des d\'eveloppements limit\'es de $sin(x)$ et $\tan(x)$
$$\lim_{x\to0}\frac{4x-\frac{(4x)^3}{3!}+\frac{(4x)^5}{5!}+\epsilon(x)}{5x+\frac{(5x)^3}{3}+\frac{2(5x)^5}{15}+\epsilon(x)} = \lim_{x\to0}\frac{4-\frac{(4x)^2}{3!}+\frac{(4x)^4}{5!}+\epsilon(x)}{5+\frac{(5x)^2}{3}+\frac{(25x)^4}{15}+\epsilon(x)} = \frac{\lim_{x\to0}4-\frac{(4x)^2}{3!}+\frac{(4x)^4}{5!}+\epsilon(x)}{\lim_{x\to0}5-\frac{(5x)^2}{3}+\frac{(25x)^4}{15}+\epsilon(x)} = \frac{4}{5}$$

\subsubsection*{Exercice 53.5}
Calculer $\lim_{x\to0}\frac{|\sin(4x)|}{\tan(5x)}$. 2 cas $\lim_{x\to0^+}\frac{|\sin(4x)|}{\tan(5x)}$ et $\lim_{x\to0^-}\frac{|\sin(4x)|}{\tan(5x)}$
Utilisation des d\'eveloppements limit\'es de $sin(x)$ et $\tan(x)$
$$\lim_{x\to0+}\frac{|4x-\frac{(4x)^3}{3!}+\frac{(4x)^5}{5!}+\epsilon(x)|}{5x+\frac{(5x)^3}{3}+\frac{2(5x)^5}{15}+\epsilon(x)} = \lim_{x\to0}\frac{4-\frac{(4x)^2}{3!}+\frac{(4x)^4}{5!}+\epsilon(x)}{5+\frac{(5x)^2}{3}+\frac{(25x)^4}{15}+\epsilon(x)} = \frac{\lim_{x\to0}4-\frac{(4x)^2}{3!}+\frac{(4x)^4}{5!}+\epsilon(x)}{\lim_{x\to0}5-\frac{(5x)^2}{3}+\frac{(25x)^4}{15}+\epsilon(x)} = \frac{4}{5}$$

On sait que la fonction $\sin$ est impaire donc $\sin(-x) = -\sin(x)$.
$$\lim_{x\to0-}\frac{|4x-\frac{(4x)^3}{3!}+\frac{(4x)^5}{5!}+\epsilon(x)|}{5x+\frac{(5x)^3}{3}+\frac{2(5x)^5}{15}+\epsilon(x)} = \frac{-4x+\frac{(4x)^3}{3!}-\frac{(4x)^5}{5!}+\epsilon(x)}{5x+\frac{(5x)^3}{3}+\frac{2(5x)^5}{15}+\epsilon(x)} = \lim_{x\to0}\frac{-4+\frac{(4x)^2}{3!}-\frac{(4x)^4}{5!}+\epsilon(x)}{5+\frac{(5x)^2}{3}+\frac{(25x)^4}{15}+\epsilon(x)} = -\frac{4}{5}$$

\subsubsection*{Exercice 53.6}
$$\lim_{x\to\frac{\pi}{2}}\frac{\cos(x)}{x-\frac{\pi}{2}} = \lim_{x\to0}\frac{\cos(x+\frac{\pi}{2})}{x} = \lim_{x\to0}-\frac{\sin(x)}{x} = -1$$

\subsection*{Exercice 54}
\subsubsection*{Exercice 54.1.1}
$$\lim_{x\to+\infty}\frac{2x^3-3x^2+1}{-4x^3+3x+1} = \lim_{x\to+\infty}\frac{2-3x^{-1}+x^{-3}}{-4+3x^{-2}+x^{-3}} = \frac{2}{-4} = -\frac{1}{2}$$

\subsubsection*{Exercice 54.1.2}
$$\lim_{x\to1}\frac{2x^3-3x^2+1}{-4x^3+3x+1} = \lim_{x\to0}\frac{2(x+1)^3-3(x+1)^2+1}{-4(x+1)^3+3(x+1)+1}$$
$$ = \lim_{x\to0}\frac{2x^3+3x^2}{-4x^3-12x^2-9x} = \lim_{x\to0}\frac{2x^2+3x}{-4x^2-12x-9} = \frac{\lim_{x\to0}2x^2+3x}{\lim_{x\to0}-4x^2-12x-9} = \frac{0}{9} =0$$

\subsubsection*{Exercice 54.2}
$$\lim_{x\to+\infty}\frac{2x+3}{3x^4+2}e^x = $$
Calculons 
$$\lim_{x\to+\infty}ln\left(\frac{2x+3}{3x^4+2}e^x\right) = \lim_{x\to+\infty}ln\left(\frac{2x+3}{3x^4+2}\right)+ln(e^x) = \lim_{x\to+\infty}ln\left(\frac{2x^{-3}+3x^{-4}}{3+2x{-4}}\right)+x $$
$$ = \ln(\lim_{x\to+\infty}2x^{-3}+3x^{-4})-\ln(\lim_{x\to+\infty}3+2x^{-4})+ \lim_{x\to+\infty}x = $$
$$ = \ln(\lim_{x\to+\infty}2x+3)-\ln(\lim_{x\to+\infty}x^{4})-\ln(3)+ \lim_{x\to+\infty}x = +\infty$$


\subsubsection*{Exercice 54.3}
$$\lim_{x\to+\infty}\frac{2x+3}{3x^4+2}e^{\ln(x)} = \lim_{x\to+\infty}\frac{2x+3}{3x^4+2}x = \lim_{x\to+\infty}\frac{2x^2+3x}{3x^4+2} = \lim_{x\to+\infty}\frac{2x^{-2}+3x^{-3}}{3+2x^{-4}} = \frac{\lim_{x\to+\infty}2x^{-2}+3x^{-3}}{\lim_{x\to+\infty}3+2x^{-4}} = \frac{0}{3} = 0$$


\subsection*{Exercice 55}
\subsubsection*{Exercice 55.1}
$$\lim_{x\to 0}\frac{\ln(\cos(x))}{\sin^2(x)} = \frac{\lim_{x\to 0}\ln(\cos(x))}{\lim_{x\to 0}sin^2(x)} = \frac{\ln(1)}{0+} = +\infty$$

\subsubsection*{Exercice 55.2}
$$\lim_{x\to+\infty}\frac{\ln(1+2x)}{\ln(1+x)} = \lim_{x\to+\infty}\frac{\ln((x^{-1}+2)x)}{\ln(1+x)} = \lim_{x\to+\infty}\frac{\ln((x^{-1}+2)+\ln(x)}{\ln(1+x}$$
$$\lim_{x\to+\infty}\frac{\ln((x^{-1}+2)}{\ln(1+x)} + \frac{\ln(x)}{\ln(1+x)} = \lim_{x\to+\infty}\frac{\ln(2)}{\ln(1+x)} + \frac{\ln(x)}{\ln(1+x)} = 1$$
car a l'infini $x \approx 1+x$.\\

\subsubsection*{Exercice 55.3}
$$\lim_{x\to+\infty}\frac{\sqrt(\ln(1+e^x))}{x^2}$$
On a l'infini $1+e^x \approx e^x$, donc $ln(1+e^x) \approx ln(e^x) = x$.\\
$$\lim_{x\to+\infty}\frac{\sqrt{\ln(1+e^x)}}{x^2} \approx \lim_{x\to+\infty}\frac{\sqrt{x}}{x^2} = 0$$

\subsubsection*{Exercice 55.4}
$$\lim_{x\to-\infty}\sin(x)\sin(\frac{1}{x^2}) = \lim_{x\to-\infty}\sin(x).\lim_{x\to-\infty}\sin(\frac{1}{x^2}) = [-1;1].0 = 0$$

QED

\end{document}

