\documentclass[]{book}

%These tell TeX which packages to use.
\usepackage{array,epsfig}
\usepackage{amsmath}
\usepackage{amsfonts}
\usepackage{amssymb}
\usepackage{amsxtra}
\usepackage{amsthm}
\usepackage{mathrsfs}
\usepackage{color}

%Here I define some theorem styles and shortcut commands for symbols I use often
\theoremstyle{definition}
\newtheorem{defn}{Definition}
\newtheorem{thm}{Theorem}
\newtheorem{cor}{Corollary}
\newtheorem*{rmk}{Remark}
\newtheorem{lem}{Lemma}
\newtheorem*{joke}{Joke}
\newtheorem{ex}{Example}
\newtheorem*{soln}{Solution}
\newtheorem{prop}{Proposition}

\newcommand{\lra}{\longrightarrow}
\newcommand{\ra}{\rightarrow}
\newcommand{\surj}{\twoheadrightarrow}
\newcommand{\graph}{\mathrm{graph}}
\newcommand{\bb}[1]{\mathbb{#1}}
\newcommand{\Z}{\bb{Z}}
\newcommand{\Q}{\bb{Q}}
\newcommand{\R}{\bb{R}}
\newcommand{\C}{\bb{C}}
\newcommand{\N}{\bb{N}}
\newcommand{\M}{\mathbf{M}}
\newcommand{\m}{\mathbf{m}}
\newcommand{\MM}{\mathscr{M}}
\newcommand{\HH}{\mathscr{H}}
\newcommand{\Om}{\Omega}
\newcommand{\Ho}{\in\HH(\Om)}
\newcommand{\bd}{\partial}
\newcommand{\del}{\partial}
\newcommand{\bardel}{\overline\partial}
\newcommand{\textdf}[1]{\textbf{\textsf{#1}}\index{#1}}
\newcommand{\img}{\mathrm{img}}
\newcommand{\ip}[2]{\left\langle{#1},{#2}\right\rangle}
\newcommand{\inter}[1]{\mathrm{int}{#1}}
\newcommand{\exter}[1]{\mathrm{ext}{#1}}
\newcommand{\cl}[1]{\mathrm{cl}{#1}}
\newcommand{\ds}{\displaystyle}
\newcommand{\vol}{\mathrm{vol}}
\newcommand{\cnt}{\mathrm{ct}}
\newcommand{\osc}{\mathrm{osc}}
\newcommand{\LL}{\mathbf{L}}
\newcommand{\UU}{\mathbf{U}}
\newcommand{\support}{\mathrm{support}}
\newcommand{\AND}{\;\wedge\;}
\newcommand{\OR}{\;\vee\;}
\newcommand{\Oset}{\varnothing}
\newcommand{\st}{\ni}
\newcommand{\wh}{\widehat}

%Pagination stuff.
\setlength{\topmargin}{-.3 in}
\setlength{\oddsidemargin}{0in}
\setlength{\evensidemargin}{0in}
\setlength{\textheight}{9.in}
\setlength{\textwidth}{6.5in}
\pagestyle{empty}



\begin{document}

\subsection*{Exercice 1}

La fonction $f(x)$ est paire ssi $\forall x, f(x) = f(-x)$, La fonction $f(x)$ est impaire ssi $\forall x, f(x) = -f(-x)$.\\

Soit une fonction $f(x)$,
$$f(x) = \frac{2f(x)}{2} = \frac{2f(x)+f(-x)-f(-x)}{2} = \frac{f(x)+f(-x)+f(x)-f(-x)}{2} = \frac{f(x)+f(-x)}{2} + \frac{f(x)-f(-x)}{2}$$

Soit la fonction $f_1(x) = \frac{f(x)+f(-x)}{2}$, la fonction $f_1(x)$ est paire car $f_1(-x) = \frac{f(-x)+f(--x)}{2} = \frac{f(-x)+f(x)}{2} = f_1(x)$.\\
Soit la fonction $f_2(x) = \frac{f(x)-f(-x)}{2}$, la fonction $f_2(x)$ est impaire car $f_2(-x) = \frac{f(-x)-f(--x)}{2} = \frac{f(-x)-f(x)}{2} = -\frac{f(x)-f(-x)}{2} = -f_2(x)$.\\

Pour toute fonction $f(x)$, on a trouv\'e une fonction $f_1(x)$ paire et une fonction $f_2(x)$ impaire tel que $f(x) = f_1(x) + f_2(x)$.


\subsection*{Buffon TD1 - Exercice 5}
Vrai pour $u_0$? $u_0 = \frac{0(0+1)}{2} = 0$, oui.\\
Admettons que $u_n = \frac{n(n+1)}{2}$, calculons $u_{n+1}$.
$$u_{n+1} = u_n + n + 1 = \frac{n(n+1)}{2} + n + 1 = \frac{n(n+1)}{2} + \frac{2(n+1)}{2}  = \frac{(n+1)(n+2)}{2} = \frac{(n+1)((n+1)+1)}{2}$$


Vrai pour $v_0$? $v_0 = \frac{0(0+1)(2.0+1)}{6} = 0$, oui.\\
Admettons que $v_n = \frac{n(n+1)(2n+1)}{6}$, calculons $v_{n+1}$.
$$v_{n+1} = v_n + n + 1 = \frac{n(n+1)(2n+1)}{6} + (n + 1)^2 = \frac{n(n+1)(2n+1)}{2} + \frac{6(n+1)^2}{6}  = \frac{(n+1)(n(2n+1) + 6(n+1))}{6}$$
$$v_{n+1} = \frac{(n+1)(2n^2 +n + 6n+ 6))}{6} = \frac{(n+1)(n+2)(2n+3)}{6} = \frac{(n+1)((n+1)+1)(2(n+1)+1)}{6}$$

Vrai pour $w_0$? $w_0 = \left(\frac{0(0+1)}{2}\right)^2 = 0$, oui.\\
Admettons que $w_n = \left(\frac{n(n+1)}{2}\right)^2$, calculons $w_{n+1}$.
$$w_{n+1} = w_n+ (n+1)^3 = \left(\frac{n(n+1)}{2}\right)^2 + (n+1)^3 = \left(\frac{n(n+1)}{2}\right)^2 + \frac{4(n+1)^3}{4} = \frac{n^2(n+1)^2 + 4(n+1)^3}{4}$$
$$w_{n+1} = \frac{(n+1)^2(n^2 + 4(n+1))}{4} = \frac{(n+1)^2(n+2)^2}{4} = \left(\frac{(n+1)((n+1)+1)}{2}\right)^2$$


\subsection*{Buffon TD1 - Exercice 6}
Preuve par r\'ecurrence, suppposons $\exists n_0 \in \N, \forall n \in \N, n \geq n_0 \implies n! \geq 2^{n}$, trouvons un entier $n$ pour lequel $n! \geq 2^n$ et v\'erifions la propri\'et'e pour $n+1$ avec l'hypoth\`ese de r\'ecurence $n! \geq 2^{n}$.
Prenons $n=4$, on a $4! = 24 \geq 16 = 2^4$. 

$$(n+1)! = n!(n+1) \geq 2^n(n+1) \text{ (hypoth\`ese de r\'ecurrence)}$$
$$ = n2^n + 2^n > 2^{n+1}, \text{ pour } n \geq 2$$
Pour $n_0 \geq 2$ la proposition est vraie. Donc il existe un $n_0$ (par exemple $n_0 =2$) pour lequel la proposition est vraie.

\subsection*{Buffon TD1 - Exercice 7}
$u_0 = 1, u_1 = u_0 = 1, u_2 = u_0+u_1 = 2, u_3 = u_0+u_1+u_2 = 4, u_4 = u_0+u_1+u_2+u_3 = 8$.\\

Vrai pour $u_0$?, $u_0 = 1 \leq 2^0=1$, Vrai.\\
Hypoth\`ese de r\'ecurrence: $u_n \leq 2^n$, calculons $u_{n+1}$.
$$u_{n+1} = u_0 + u_1 + ... + u_{n-1} + u_n = u_n + u_n = 2u_n \leq 2.2^n = 2^{n+1}$$
La proposition est vraie. 

\subsection*{Buffon TD1 - Exercice 8}
$$\forall x \in \R, \left(x+\frac{1}{x} \in \Z \implies \left( \forall n \in \N, x^n + \frac{1}{x^n} \in \Z \right) \right)$$

Pour $n=0$, on a $\forall x \in \R, \left(x+\frac{1}{x} \in \Z \implies \left( x^0 + \frac{1}{x^0} \in \Z \right) \right)$ qui est vrai.\\
Pour $n=1$, on a $\forall x \in \R, \left(x+\frac{1}{x} \in \Z \implies \left( x^1 + \frac{1}{x^1} \in \Z \right) \right)$ qui est vrai.\\

Supposons la propri\'et\'e vraie au rang $k<n$, calculons le rang $n$.
$$\left( x^{n-1} + \frac{1}{x^{n-1}} \right) \left( x + \frac{1}{x} \right) = x^n + x^{n-2} + \frac{1}{x^n} + \frac{1}{x^n-2} = \left( x^{n} + \frac{1}{x^{n}} \right) + \left( x^{n-2} + \frac{1}{x^{n-2}} \right)$$
Donc
$$ \left( x^{n} + \frac{1}{x^{n}} \right) = \left( x^{n-1} + \frac{1}{x^{n-1}} \right) - \left( x^{n-2} + \frac{1}{x^{n-2}} \right)$$

Par hypoth\`ese de r\'ecurence, on a $\left( x^{n-1} + \frac{1}{x^{n-1}} \right) \in \Z$ et $\left( x^{n-2} + \frac{1}{x^{n-2}} \right) \in \Z$. Donc $\left( x^{n} + \frac{1}{x^{n}} \right) \in \Z$ car la soustraction de deux nombres dans $\Z$.



\newpage
\subsection*{Buffon TD2 - Exercice 1.a}
$$(1-a)\sum_{k=1}^{n}ka^{k-1} = \sum_{k=1}^{n}ka^{k-1} - \sum_{k=1}^{n}aka^{k-1} = \sum_{k=1}^{n}ka^{k-1} - \sum_{k=1}^{n}ka^{k}$$

$$\sum_{k=1}^{n}{ka^{k-1} - ka^{k}} = \sum_{k=0}^{n-1}{a^k} - na^{n} = \sum_{k=0}^{n}{a^k} - (n+1)a^{n}$$
Donc
$$\sum_{k=1}^{n}ka^{k} = \sum_{k=1}^{n}ka^{k-1} - \sum_{k=0}^{n}{a^k} + (n+1)a^{n}$$

On a $\sum_{k=0}^{n}{a^k} = \frac{1-a^{n+1}}{1-a}$ et $\sum_{k=1}^{n}ka^{k-1} = \left( \sum_{k=0}^{n}{a^k} \right)' = \left( \frac{1-a^{n+1}}{1-a} \right)'$
Donc
$$\sum_{k=1}^{n}ka^{k} = \left( \frac{1-a^{n+1}}{1-a} \right)' - \frac{1-a^{n+1}}{1-a} + (n+1)a^{n} $$

\subsection*{Buffon TD2 - Exercice 1.b}
Trouver $a,b,c$, tel que $\forall x \in \R^{+*},\frac{1}{x(x+1)(x+2)} = \frac{a}{x}+ \frac{b}{x+1}+ \frac{c}{x+2}$.

$$\frac{a}{x}+ \frac{b}{x+1}+ \frac{c}{x+2} = \frac{a(x+1)(x+2)+bx(x+2)+cx(x+1)}{x(x+1)(x+2)} = \frac{a(x^2+3x+2)+b(x^2+2x)+c(x^2+x)}{x(x+1)(x+2)}$$
Donc 
$$
\left\{ 
\begin{array}{l}
a+b+c = 0 \\
3a+2b+c = 0 \\
2a = 1 \\
\end{array}
\right. 
$$
Et $a=\frac{1}{2}$, $b=-1$, $c=\frac{1}{2}$. Donc $\frac{1}{x(x+1)(x+2)} = \frac{1}{2x} - \frac{1}{x+1} + \frac{1}{2(x+2)}$.\\
$$\sum_{k=1}^{n}{\frac{1}{k(k+1)(k+2)}} = \sum_{k=1}^{n}{\frac{1}{2k} - \frac{1}{k+1} + \frac{1}{2(k+2)}} = \sum_{k=1}^{n}{\frac{1}{2k}} - \sum_{k=1}^{n}{\frac{1}{k+1}} + \sum_{k=1}^{n}{\frac{1}{2(k+2)}}$$
$$ = \frac{1}{2}\sum_{k=1}^{n}{\frac{1}{k}} - \sum_{k=1}^{n}{\frac{1}{k+1}} + \frac{1}{2}\sum_{k=1}^{n}{\frac{1}{k+2}}$$
$$ = \frac{1}{2}\sum_{k=1}^{n}{\frac{1}{k}} - \sum_{k=2}^{n+1}{\frac{1}{k}} + \frac{1}{2}\sum_{k=3}^{n+2}{\frac{1}{k}}$$
$$ = \frac{1}{2}(\sum_{k=3}^{n}{\frac{1}{k}}+1+\frac{1}{2}) - (\sum_{k=3}^{n}{\frac{1}{k}}+\frac{1}{2}+\frac{1}{n+1}) + \frac{1}{2}(\sum_{k=3}^{n}{\frac{1}{k}}+\frac{1}{n+1}+\frac{1}{n+2})$$
$$=\frac{1}{4}-\frac{1}{2(n+1)}+\frac{1}{2(n+2)}$$


\subsection*{Buffon TD2 - Exercice 3.1}
$$\sum_{k \in [1,2n], impair}{3^k} = 3^1+3^3+3^5+\ldots+3^{2n-1} = 3 + 3.9 + (3.9).9 + \ldots (((\ldots))).9$$
Soit la suite g\'eom\'etrique d\'efinit par $u_0 = 3$ et de raison $q=9$. La somme $\sum_{k \in[0,n]}{u_k} = u_0.\left(\frac{1-q^{n+1}}{1-q}\right)$.
Donc 
$$\sum_{k \in [1,2n], impair}{3^k} = 3.\left(\frac{1-9^{n-1}}{1-9}\right)$$

\subsection*{Buffon TD2 - Exercice 3.2}
$$\sum_{k=2}^{n}{\ln(1-\frac{1}{k^2}} = \sum_{k=2}^{n}{\ln(\frac{k^2 - 1}{k^2}} = \sum_{k=2}^{n}{\ln(\frac{(k - 1)(k+1)}{k^2}} = \sum_{k=2}^{n}{\ln(k-1) + \ln(k+1) - ln(k^2)} $$
$$ = \sum_{k=2}^{n}{\ln(k-1)} + \sum_{k=2}^{n}{\ln(k+1)} - 2\sum_{k=2}^{n}{ln(k)} = \sum_{k=1}^{n-1}{\ln(k)} + \sum_{k=3}^{n+1}{\ln(k)} - 2\sum_{k=2}^{n}{ln(k)}$$
$$ = \sum_{k=2}^{n-1}{\ln(k)} + \ln(1) + \sum_{k=2}^{n-1}{\ln(k)} -\ln(2) + \ln(n) + \ln(n+1) - 2\sum_{k=2}^{n-1}{\ln(k)} - 2\ln(n) = -\ln(2) + \ln(n) + \ln(n+1) - 2\ln(n)$$
$$ = -\ln(2) + \ln\left(\frac{n+1}{n}\right) $$


\subsection*{Buffon TD2 - Exercice 3.3}
$$\prod{k=1}{n}{\left(1-\frac{1}{2k}\right)} = (1-\frac{1}{2})(1-\frac{1}{4})(1-\frac{1}{6})\ldots(1-\frac{1}{2n}) = \frac{1}{2}.\frac{3}{4}.\frac{5}{6}.\ldots.\frac{2n-1}{2n} = \frac{\prod_{k=1}^{n}{2n-1}}{\prod_{k=1}^{n}{2n}} = \frac{\prod_{k=[1,2n],impair}{n}}{\prod_{k=[1,2n],pair}{n}}$$

Soit les s\'eries arithm\'etique $u_0=1$ et $u_{n+1} = u_{n} + 2$ et $v_0=0$ et $v_{n+1} = v_{n} + 2$.
$$\frac{\prod_{k=0}^{n}{u_n}}{\prod_{k=0}^{n}{v_n}}$$


\subsection*{Buffon TD2 - Exercice 3.4}
$$\prod_{k=1}^{n}{3^k} = 3 . 3^2 . 3^3 . \ldots . 3^n = 3^{1+2+3+\ldots+n} = 3^{\sum_{k=1}^{n}{k}} = 3^{\frac{n(n+1)}{2}}$$


\subsection*{Buffon TD2 - Exercice 3.6}
$$\sum_{k=1}^{n}{kk!} = \sum_{k=1}^{n}{((k+1)-1)k!} = \sum_{k=1}^{n}{(k+1)k!-k!} = \sum_{k=1}^{n}{(k+1)!}-\sum_{k=1}^{n}{k!} = (n-1)! -1$$


\newpage
\subsection*{Buffon TD3 - Exercice 1.1}
Soit $z=a+bi$,
$$z\bar{z} + 3(z-\bar{z}) = 4 -3i$$
$$(a+bi)(a-bi) + 3((a+bi)-(a-bi)) = 4 -3i$$
$$ a^2+b^2 + 3(2bi) = 4 -3i$$
Donc, on a $a^2+b^2 =4$ et $6b = -3$, ce qui fait $b=-\frac{1}{2}$ et $a=\frac{\sqrt{15}}{2}$.

\subsection*{Buffon TD3 - Exercice 1.3}
Soit $z=a+bi$,
$$|z| = z + \bar{z}$$
$$\sqrt{a^2+b^2} = (a+bi)+(a-bi) = 2a$$
$$b^2 = 3a^2$$

\subsection*{Buffon TD3 - Exercice 1.5}
Soit $z=a+bi$,
$$|(1+i)z-2i| = 2$$
$$|(1+i)(a+bi)-2i| = 2$$
$$|a+bi+ai-b-2i| = 2$$
$$\sqrt{(a-b)^2+(a+b-2)^2} = 2$$
$$a^2-2ab+b^2 +(a+b-2)^2 = 4$$
$$a^2-2ab+b^2 + a^2+b^2+2ab-4a-4b+4 = 4$$
$$2a^2+2b^2 -4a-4b = 0$$
$$a^2+b^2 -2a-2b = 0$$


\newpage
\subsection*{Rappel de cours}	
\begin{itemize}
\item la fonction $f \in F^{E}$ est injective si tout \'el\'ement de $F$ admet au plus un ant\'ec\'edent, $\forall (x_1, x_2) \in E^2, f(x_1) = f(x_2) \implies x_1 = x_2$ ou $\forall (x_1, x_2) \in E^2, x_1 \neq x_2 \implies f(x_1) \neq f(x_2)$.
\item la fonction $f \in F^{E}$ est surjective si tout \'el\'ement de $F$ admet au moins un ant\'ec\'edent, $\forall y \in F, \exists x \in E, y = f(x)$.
\item la fonction $f \in F^{E}$ est bijective si elle est injective et bijective.
\item soit $f \in F^{E}$ et $g \in G^{F}$, la compos\'ee des fonctions $f$ et $g$ not\'ee $g \circ f$ d\'efinie par $g \circ f : E \to G, x \to g(f(x))$.
\end{itemize}

\subsection*{Buffon TD5 - Exercice 4.1}
$P$: Si $g \circ f$ est injective alors $f$ aussi.\\
Comme $g \circ f$ est injective, on a $\forall (x_1, x_2) \in E^2, g(f(x_1)) = g(f(x_2)) \implies x_1 = x_2 \textrm{  : } [1]$.\\
Preuve par l'absurde.\\
Supposons que la fonction $f$ n'est pas injective. Donc $\exists (x_1, x_2) \in E^2, f(x_1)=f(x_2) \implies x_1 \neq x_2 \textrm{  : } [2]$.\\
En partant de $f(x_1)=f(x_2)$, on a $g(f(x_1))=g(f(x_2))$ car $g$ est une fonction. Donc de $[1]$, on a $x_1 = x_2$.
En partant de $f(x_1)=f(x_2)$ et de $[2]$, on a $x_1 \neq x_2$ ce qui contredit pr\'ec\'edemment.\\
Donc la proposition $P$ est vraie.

\subsection*{Buffon TD5 - Exercice 4.2}
$P$: Si $g \circ f$ est surjective alors $g$ aussi.\\
Comme $g \circ f$ est surjective, on a $\forall y \in G, \exists x \in E, y = g \circ f(x) = g(f(x)) \textrm{  : } [1]$.\\
Comme $f$ est une fonction donc $\forall x_e \in E, \exists !y_f \in F, y_f = f(x_e)$.
Donc, $\forall y \in G, y=g(f(x))$ de $[1]$, soit $b \in F, b = f(x)$, $b$ existe et est unique par $[2]$. \\
DOnc $ \forall y \in G, y=g(f(x))=g(b)$ ce qui est la d\'efinition de g est une fonction surjective.
Donc la proposition $P$ est vraie.

\subsection*{Buffon TD5 - Exercice 4.3}
P:si $g \circ f$ est injective et si $f$ est surjective alors $g$ est injective
Comme $g \circ f$ est injective, on a $\forall (x_1, x_2) \in E^2, g(f(x_1)) = g(f(x_2)) \implies x_1 = x_2 \textrm{  :[1]}$.\\
Comme $f$ est surjective, on a $\forall y \in F, \exists x \in E, y = f(x) \textrm{  : [2]}$.\\

Prenons $x_1, x_2$ tel que $g(x) = g(y) \textrm{  : [3]}$. Comme $f$ est surjective $[2]$, il existe $a,b$ tel que $x=f(a)$ et $y=f(b)$. Donc par $g(f(a) = g(f(b)$.\\
par [1], on a donc $a=b$. Par cons\'equent $f(a) = f(b)$ car $f$ est une fonction. Donc $x=y$ et $g(x) = g(y) \implies x = y$ qui es la d\'efinition de $g$ est injective.\\
Donc la proposition $P$ est vraie.

\subsection*{Buffon TD5 - Exercice 4.4}
P:  si $g \circ f$ est surjective et si $g$ est injective alors $f$ est surjective.\\
Comme $g \circ f$ est surjective, on a $\forall y \in G, \exists x \in E, y = g \circ f(x)=g(f(x)) \textrm{  : } [1]$.\\
Comme $g$ est injective, on a $\forall (x_1, x_2) \in F^2, g(x_1) = g(x_2) \implies x_1 = x_2 \textrm{  : }[2]$.\\

De $[1]$, $\forall z_1 \in G, \exists x_1 \in E, z_1 = g(f(x_1))$. donc $\exists y_1 \in F, y_1 = f(x_1)$.\\ 
De $[2]$, $\not \exists y_2 \in F, y_2 \neq y_1, g(y_1) = g(y_2)$. Donc $y_1$ est unique.\\
$\exists y_1 \in F, \forall x \in E, y_1 \neq f(x)$.


\newpage
\subsection*{Buffon TD7 - Exercice 1.1}
Etude de $\arcsin(\frac{1+x}{\sqrt{2(1+x^2)}})$.\\
1-d\'eriv\'ee de la fonction. On prend $g(x) = \arcsin(x)$ et $f(x)=\frac{1+x}{\sqrt{2(1+x^2)}}$. Donc
$$\left(arcsin(\frac{1+x}{\sqrt{2(1+x^2)}})\right)' = (g \circ f(x))' = f'(x).\frac{1}{\sqrt{1-f^2(x)}}$$

Donc calcul de $f'(x) = \left(\frac{1+x}{\sqrt{2(1+x^2)}}\right)' = \left(\frac{u}{v}\right)' = \frac{u'v-v'u}{v^2}$ avec $u(x) = 1+x$ et $v(x)=\sqrt{2(1+x^2)}$. Donc $u'(x) = 1$ et $v'(x) = \sqrt(2)w(x)^{1/2} = \sqrt(2).\frac{1}{2}.w^{\frac{1}{2}-1}w'(x)$ avec $w(x) = 1+x^2$ et $w'(x) = 2x$. Donc

$$f'(x) = \frac{1.\sqrt{1+x^2}-(1+x)\frac{\sqrt{2}x}{\sqrt{1+x^2}}}{2(1+x^2)}  = \frac{1-x}{2\sqrt{2}(1+x^2)^{\frac{3}{2}}}$$

Et
$$d(x) = \left(arcsin(\frac{1+x}{\sqrt{2(1+x^2)}})\right)' = \frac{1-x}{2\sqrt{2}(1+x^2)^{\frac{3}{2}}}.\frac{1}{\sqrt{1-\frac{(1+x)^2}{2(1+x^2)}}}$$

La d\'eriv\'ee de la fonction n'est pas d\'efinie en $x=1$.
Lorque $x<1$, $d(x) > 0$ et lorque $x>1$, $d(x) < 0$ Donc $f(1) = 1 $est un maximum.\\

Regardons le domaine de d\'efinition de $\arcsin(\frac{1+x}{\sqrt{2(1+x^2)}})$. $\arcsin(x)$ est d\'efinie pour $x \in ]-1,1[$. Calculons les $x$ tel que $-1 < \frac{1+x}{\sqrt{2(1+x^2)}} < 1$. On a $\lim_{x\to+\infty} f(x) = \frac{1}{\sqrt{2}}$ et $\lim_{x\to-\infty} f(x) = -\frac{1}{\sqrt{2}}$. Donc la fonction est d\'efinie sur $x \in ]-\infty, +\infty[$.\\

Donc $\lim_{x\to+\infty} \arcsin(f(x)) = \arcsin(\frac{1}{\sqrt{2}}) = \frac{\pi}{4}$, Donc $\lim_{x\to-\infty} \arcsin(f(x)) = \arcsin(-\frac{1}{\sqrt{2}}) = -\frac{\pi}{4}$ et $\arcsin(1) = \frac{\pi}{2}$ le maximum.


\newpage
\subsection*{Exercice Cauchy}
Solution de l'\'equation diff\'erentielle de la forme $(y'(t) = a(t)y(t) + b(t)$ est $y(t) = \lambda e^{A(t)}+y_1(t)$ avec $A(t)$ une primitive de la fonction $a(t)$ et $y_1(t)$ est une solution particuli\`ere de l'\'equation.\\

Pour $y'(t) - \frac{y(t)}{t^2} = e^{\frac{1}{t}}\sin(t)$, donc $a(t) = \frac{1}{t^2}$ et $b(t)= e^{\frac{1}{t}}\sin(t)$. On a $A(t) = -\frac{1}{t}$. Recherchons une  solution particuli\`ere de la forme $y_1(t) = \lambda(t)a^{A(t)}$ avec $\lambda'(t) = b(t)e^{A(t)}$.
$$\lambda'(t) = b(t)e^{A(t)} = e^{\frac{1}{t}}\sin(t)e^{-\frac{1}{t}} = sin(t)$$
donc $\lambda(t) =  -\cos(t)$ et $y_1(t) = -\cos(t)e^{-\frac{1}{t}}$\\
Par cons\'equent
$$y(t) = \lambda e^{A(t)}+y_1(t) = \lambda e^{-\frac{1}{n}} -\cos(t)e^{-\frac{1}{t}} = e^{-\frac{1}{n}}(\lambda-cos(t))$$

Calculons l'unique solution $y(2\pi) =  e^{-\frac{1}{2\pi}}(\lambda-\cos(2\pi)) = \lambda e^{-\frac{1}{2\pi}} = 0$.\\
Donc $\lambda = 0$ et l'unique solution est 
$$y(t) = -\cos(t)e^{-\frac{1}{t}}$$


QED

\end{document}

