\documentclass[]{book}

%These tell TeX which packages to use.
\usepackage{array,epsfig}
\usepackage{amsmath}
\usepackage{amsfonts}
\usepackage{amssymb}
\usepackage{amsxtra}
\usepackage{amsthm}
\usepackage{mathrsfs}
\usepackage{color}

%Here I define some theorem styles and shortcut commands for symbols I use often
\theoremstyle{definition}
\newtheorem{defn}{Definition}
\newtheorem{thm}{Theorem}
\newtheorem{cor}{Corollary}
\newtheorem*{rmk}{Remark}
\newtheorem{lem}{Lemma}
\newtheorem*{joke}{Joke}
\newtheorem{ex}{Example}
\newtheorem*{soln}{Solution}
\newtheorem{prop}{Proposition}

\newcommand{\lra}{\longrightarrow}
\newcommand{\ra}{\rightarrow}
\newcommand{\surj}{\twoheadrightarrow}
\newcommand{\graph}{\mathrm{graph}}
\newcommand{\bb}[1]{\mathbb{#1}}
\newcommand{\Z}{\bb{Z}}
\newcommand{\Q}{\bb{Q}}
\newcommand{\R}{\bb{R}}
\newcommand{\C}{\bb{C}}
\newcommand{\N}{\bb{N}}
\newcommand{\M}{\mathbf{M}}
\newcommand{\m}{\mathbf{m}}
\newcommand{\MM}{\mathscr{M}}
\newcommand{\HH}{\mathscr{H}}
\newcommand{\Om}{\Omega}
\newcommand{\Ho}{\in\HH(\Om)}
\newcommand{\bd}{\partial}
\newcommand{\del}{\partial}
\newcommand{\bardel}{\overline\partial}
\newcommand{\textdf}[1]{\textbf{\textsf{#1}}\index{#1}}
\newcommand{\img}{\mathrm{img}}
\newcommand{\ip}[2]{\left\langle{#1},{#2}\right\rangle}
\newcommand{\inter}[1]{\mathrm{int}{#1}}
\newcommand{\exter}[1]{\mathrm{ext}{#1}}
\newcommand{\cl}[1]{\mathrm{cl}{#1}}
\newcommand{\ds}{\displaystyle}
\newcommand{\vol}{\mathrm{vol}}
\newcommand{\cnt}{\mathrm{ct}}
\newcommand{\osc}{\mathrm{osc}}
\newcommand{\LL}{\mathbf{L}}
\newcommand{\UU}{\mathbf{U}}
\newcommand{\support}{\mathrm{support}}
\newcommand{\AND}{\;\wedge\;}
\newcommand{\OR}{\;\vee\;}
\newcommand{\Oset}{\varnothing}
\newcommand{\st}{\ni}
\newcommand{\wh}{\widehat}

%Pagination stuff.
\setlength{\topmargin}{-.3 in}
\setlength{\oddsidemargin}{0in}
\setlength{\evensidemargin}{0in}
\setlength{\textheight}{9.in}
\setlength{\textwidth}{6.5in}
\pagestyle{empty}



\begin{document}

\subsection*{Exercice 1}

La fonction $f(x)$ est paire ssi $\forall x, f(x) = f(-x)$, La fonction $f(x)$ est impaire ssi $\forall x, f(x) = -f(-x)$.\\

Soit une fonction $f(x)$,
$$f(x) = \frac{2f(x)}{2} = \frac{2f(x)+f(-x)-f(-x)}{2} = \frac{f(x)+f(-x)+f(x)-f(-x)}{2} = \frac{f(x)+f(-x)}{2} + \frac{f(x)-f(-x)}{2}$$

Soit la fonction $f_1(x) = \frac{f(x)+f(-x)}{2}$, la fonction $f_1(x)$ est paire car $f_1(-x) = \frac{f(-x)+f(--x)}{2} = \frac{f(-x)+f(x)}{2} = f_1(x)$.\\
Soit la fonction $f_2(x) = \frac{f(x)-f(-x)}{2}$, la fonction $f_2(x)$ est impaire car $f_2(-x) = \frac{f(-x)-f(--x)}{2} = \frac{f(-x)-f(x)}{2} = -\frac{f(x)-f(-x)}{2} = -f_2(x)$.\\

Pour toute fonction $f(x)$, on a trouv\'e une fonction $f_1(x)$ paire et une fonction $f_2(x)$ impaire tel que $f(x) = f_1(x) + f_2(x)$.


\subsection*{Buffon TD1 - Exercice 5}
Vrai pour $u_0$? $u_0 = \frac{0(0+1)}{2} = 0$, oui.\\
Admettons que $u_n = \frac{n(n+1)}{2}$, calculons $u_{n+1}$.
$$u_{n+1} = u_n + n + 1 = \frac{n(n+1)}{2} + n + 1 = \frac{n(n+1)}{2} + \frac{2(n+1)}{2}  = \frac{(n+1)(n+2)}{2} = \frac{(n+1)((n+1)+1)}{2}$$


Vrai pour $v_0$? $v_0 = \frac{0(0+1)(2.0+1)}{6} = 0$, oui.\\
Admettons que $v_n = \frac{n(n+1)(2n+1)}{6}$, calculons $v_{n+1}$.
$$v_{n+1} = v_n + n + 1 = \frac{n(n+1)(2n+1)}{6} + (n + 1)^2 = \frac{n(n+1)(2n+1)}{2} + \frac{6(n+1)^2}{6}  = \frac{(n+1)(n(2n+1) + 6(n+1))}{6}$$
$$v_{n+1} = \frac{(n+1)(2n^2 +n + 6n+ 6))}{6} = \frac{(n+1)(n+2)(2n+3)}{6} = \frac{(n+1)((n+1)+1)(2(n+1)+1)}{6}$$

Vrai pour $w_0$? $w_0 = \left(\frac{0(0+1)}{2}\right)^2 = 0$, oui.\\
Admettons que $w_n = \left(\frac{n(n+1)}{2}\right)^2$, calculons $w_{n+1}$.
$$w_{n+1} = w_n+ (n+1)^3 = \left(\frac{n(n+1)}{2}\right)^2 + (n+1)^3 = \left(\frac{n(n+1)}{2}\right)^2 + \frac{4(n+1)^3}{4} = \frac{n^2(n+1)^2 + 4(n+1)^3}{4}$$
$$w_{n+1} = \frac{(n+1)^2(n^2 + 4(n+1))}{4} = \frac{(n+1)^2(n+2)^2}{4} = \left(\frac{(n+1)((n+1)+1)}{2}\right)^2$$


\subsection*{Buffon TD1 - Exercice 6}
Preuve par r\'ecurrence, suppposons $\exists n_0 \in \N, \forall n \in \N, n \geq n_0 \implies n! \geq 2^{n}$, trouvons un entier $n$ pour lequel $n! \geq 2^n$ et v\'erifions la propri\'et'e pour $n+1$ avec l'hypoth\`ese de r\'ecurence $n! \geq 2^{n}$.
Prenons $n=4$, on a $4! = 24 \geq 16 = 2^4$. 

$$(n+1)! = n!(n+1) \geq 2^n(n+1) \text{ (hypoth\`ese de r\'ecurrence)}$$
$$ = n2^n + 2^n > 2^{n+1}, \text{ pour } n \geq 2$$
Pour $n_0 \geq 2$ la proposition est vraie. Donc il existe un $n_0$ (par exemple $n_0 =2$) pour lequel la proposition est vraie.

\subsection*{Buffon TD1 - Exercice 7}
$u_0 = 1, u_1 = u_0 = 1, u_2 = u_0+u_1 = 2, u_3 = u_0+u_1+u_2 = 4, u_4 = u_0+u_1+u_2+u_3 = 8$.\\

Vrai pour $u_0$?, $u_0 = 1 \leq 2^0=1$, Vrai.\\
Hypoth\`ese de r\'ecurrence: $u_n \leq 2^n$, calculons $u_{n+1}$.
$$u_{n+1} = u_0 + u_1 + ... + u_{n-1} + u_n = u_n + u_n = 2u_n \leq 2.2^n = 2^{n+1}$$
La proposition est vraie. 

\subsection*{Buffon TD1 - Exercice 8}
$$\forall x \in \R, \left(x+\frac{1}{x} \in \Z \implies \left( \forall n \in \N, x^n + \frac{1}{x^n} \in \Z \right) \right)$$

Pour $n=0$, on a $\forall x \in \R, \left(x+\frac{1}{x} \in \Z \implies \left( x^0 + \frac{1}{x^0} \in \Z \right) \right)$ qui est vrai.\\
Pour $n=1$, on a $\forall x \in \R, \left(x+\frac{1}{x} \in \Z \implies \left( x^1 + \frac{1}{x^1} \in \Z \right) \right)$ qui est vrai.\\

Supposons la propri\'et\'e vraie au rang $k<n$, calculons le rang $n$.
$$\left( x^{n-1} + \frac{1}{x^{n-1}} \right) \left( x + \frac{1}{x} \right) = x^n + x^{n-2} + \frac{1}{x^n} + \frac{1}{x^n-2} = \left( x^{n} + \frac{1}{x^{n}} \right) + \left( x^{n-2} + \frac{1}{x^{n-2}} \right)$$
Donc
$$ \left( x^{n} + \frac{1}{x^{n}} \right) = \left( x^{n-1} + \frac{1}{x^{n-1}} \right) - \left( x^{n-2} + \frac{1}{x^{n-2}} \right)$$

Par hypoth\`ese de r\'ecurence, on a $\left( x^{n-1} + \frac{1}{x^{n-1}} \right) \in \Z$ et $\left( x^{n-2} + \frac{1}{x^{n-2}} \right) \in \Z$. Donc $\left( x^{n} + \frac{1}{x^{n}} \right) \in \Z$ car la soustraction de deux nombres dans $\Z$.



\subsection*{Exercice Cauchy}
Solution de l'\'equation diff\'erentielle de la forme $(y'(t) = a(t)y(t) + b(t)$ est $y(t) = \lambda e^{A(t)}+y_1(t)$ avec $A(t)$ une primitive de la fonction $a(t)$ et $y_1(t)$ est une solution particuli\`ere de l'\'equation.\\

Pour $y'(t) - \frac{y(t)}{t^2} = e^{\frac{1}{t}}\sin(t)$, donc $a(t) = \frac{1}{t^2}$ et $b(t)= e^{\frac{1}{t}}\sin(t)$. On a $A(t) = -\frac{1}{t}$. Recherchons une  solution particuli\`ere de la forme $y_1(t) = \lambda(t)a^{A(t)}$ avec $\lambda'(t) = b(t)e^{A(t)}$.
$$\lambda'(t) = b(t)e^{A(t)} = e^{\frac{1}{t}}\sin(t)e^{-\frac{1}{t}} = sin(t)$$
donc $\lambda(t) =  -\cos(t)$ et $y_1(t) = -\cos(t)e^{-\frac{1}{t}}$\\
Par cons\'equent
$$y(t) = \lambda e^{A(t)}+y_1(t) = \lambda e^{-\frac{1}{n}} -\cos(t)e^{-\frac{1}{t}} = e^{-\frac{1}{n}}(\lambda-cos(t))$$

Calculons l'unique solution $y(2\pi) =  e^{-\frac{1}{2\pi}}(\lambda-\cos(2\pi)) = \lambda e^{-\frac{1}{2\pi}} = 0$.\\
Donc $\lambda = 0$ et l'unique solution est 
$$y(t) = -\cos(t)e^{-\frac{1}{t}}$$


QED

\end{document}

