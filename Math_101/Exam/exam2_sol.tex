\documentclass[]{book}

%These tell TeX which packages to use.
\usepackage{array,epsfig}
\usepackage{amsmath}
\usepackage{amsfonts}
\usepackage{amssymb}
\usepackage{amsxtra}
\usepackage{amsthm}
\usepackage{mathrsfs}
\usepackage{color}

%Here I define some theorem styles and shortcut commands for symbols I use often
\theoremstyle{definition}
\newtheorem{defn}{Definition}
\newtheorem{thm}{Theorem}
\newtheorem{cor}{Corollary}
\newtheorem*{rmk}{Remark}
\newtheorem{lem}{Lemma}
\newtheorem*{joke}{Joke}
\newtheorem{ex}{Example}
\newtheorem*{soln}{Solution}
\newtheorem{prop}{Proposition}

\newcommand{\lra}{\longrightarrow}
\newcommand{\ra}{\rightarrow}
\newcommand{\surj}{\twoheadrightarrow}
\newcommand{\graph}{\mathrm{graph}}
\newcommand{\bb}[1]{\mathbb{#1}}
\newcommand{\Z}{\bb{Z}}
\newcommand{\Q}{\bb{Q}}
\newcommand{\R}{\bb{R}}
\newcommand{\C}{\bb{C}}
\newcommand{\N}{\bb{N}}
\newcommand{\M}{\mathbf{M}}
\newcommand{\m}{\mathbf{m}}
\newcommand{\MM}{\mathscr{M}}
\newcommand{\HH}{\mathscr{H}}
\newcommand{\Om}{\Omega}
\newcommand{\Ho}{\in\HH(\Om)}
\newcommand{\bd}{\partial}
\newcommand{\del}{\partial}
\newcommand{\bardel}{\overline\partial}
\newcommand{\textdf}[1]{\textbf{\textsf{#1}}\index{#1}}
\newcommand{\img}{\mathrm{img}}
\newcommand{\ip}[2]{\left\langle{#1},{#2}\right\rangle}
\newcommand{\inter}[1]{\mathrm{int}{#1}}
\newcommand{\exter}[1]{\mathrm{ext}{#1}}
\newcommand{\cl}[1]{\mathrm{cl}{#1}}
\newcommand{\ds}{\displaystyle}
\newcommand{\vol}{\mathrm{vol}}
\newcommand{\cnt}{\mathrm{ct}}
\newcommand{\osc}{\mathrm{osc}}
\newcommand{\LL}{\mathbf{L}}
\newcommand{\UU}{\mathbf{U}}
\newcommand{\support}{\mathrm{support}}
\newcommand{\AND}{\;\wedge\;}
\newcommand{\OR}{\;\vee\;}
\newcommand{\Oset}{\varnothing}
\newcommand{\st}{\ni}
\newcommand{\wh}{\widehat}

%Pagination stuff.
\setlength{\topmargin}{-.3 in}
\setlength{\oddsidemargin}{0in}
\setlength{\evensidemargin}{0in}
\setlength{\textheight}{9.in}
\setlength{\textwidth}{6.5in}
\pagestyle{empty}



\begin{document}

\subsection*{Exercice 1}

Rappel de cours: 
\begin{itemize}
\item On appelle extraction toute application $\varphi : \N \to \N$ strictement croissante.
\item On appelle suite extraite (ou sous-suite) d'une suite $(x_n)_{n \in \N}$. toute suite de la forme
$(x_n)_{\varphi(n) \in \N}$ o\`u $\varphi$ est une extraction. Une suite extraite de $(x_n)_{n \in \N}$ est une suite obtenue à partir de celle-ci en n’en gardant que les éléments $\varphi(n)$, mais en nombre infini.
\item On appelle valeur d'adh\'erence d'une suite $(x_n)_{n \in \N}$ toute limite finie d'une suite extraite de
$(x_n)_{n \in \N}$.
\end{itemize}


Soit la fonction $f(n) = n*360$, la suite extraite $(cos_{f(n)})_{n \in \N}$ est une suite constante qui admet une valeur d'adh\'erence $1$.

Donc la proposition est fausse.

\subsection*{Exercice 2}
Soit les fonctions $f_1(n) = n*360$ et $f_2(n) = 90+n*360$, les suites $(cos_{f_1(n)})_{n \in \N}$ et $(cos_{f_2(n)})_{n \in \N}$ convergent respectivement vers les valeurs $1$ et $0$. La suite $(cos\, n)_{n \in \N}$ n'est pas convergente car elle admet 2 valeurs d'adh\'erence distinctes. 	

Donc la proposition est fausse.


\subsection*{Exercice 3}
Rappel de cours: \\
Deux suites r\'eelles $(a_n)_{n \in \N}$ et $(b_n)_{n \in \N}$ sont dites adjacentes si
\begin{enumerate}
\item l'une est croissante et l'autre d\'ecroissante,
\item $\lim_{n \to \infty}(b_n - a_n) = 0 $
\end{enumerate}

(a) $(S_n)_{n \in \N^{*}}$ est croissante ?\\
$$S_{n+1} - S_{n} = \sum_{k=1}^{n+1}\frac{1}{k^2} - \sum_{k=1}^{n}\frac{1}{k^2} = \frac{1}{(n+1)^2} > 0$$  
La suite $S_n$ est croissante.\\

(b) $(S_n+\frac{1}{n})_{n \in \N^{*}}$ est d\'ecroissante ?\\
$$(S_{n+1} +\frac{1}{n+1}) - (S_{n}+\frac{1}{n})) = \sum_{k=1}^{n+1}\frac{1}{k^2} - \sum_{k=1}^{n}\frac{1}{k^2} + \frac{1}{n+1} - \frac{1}{n} = \frac{1}{(n+1)^2} + \frac{1}{n+1} - \frac{1}{n}$$  
$$= \frac{n+n(n+1)-(n+1)^2}{n(n+1)^2} = \frac{n+n^2+n-n^2-2n-1}{n(n+1)^2} = \frac{-1}{n(n+1)^2} < 0$$
La suite $(S_n+\frac{1}{n})_{n \in \N^{*}}$ est d\'ecroissante.\\

(c) $\lim_{n \to \infty}((S_n + \frac{1}{n}) - S_n) = 0 $?
$$\lim_{n \to \infty}((S_n + \frac{1}{n}) - S_n) = \lim_{n \to \infty}\frac{1}{n} = 0 $$


Les suites $(S_n)_{n \in \N^{*}}$ et $(S_n+\frac{1}{n})_{n \in \N^{*}}$ sont adjacentes.

Donc la proposition est vraie.

\subsection*{Exercice 4}
$$\lim_{n \to +\infty} \frac{u_{n+1}}{u_{n}} = l \in ]-1,1[$$ 
alors $\exists N_0 \in \mathbb{N}$ tel que $\forall n > N_{0}, \frac{u_{n+1}}{u_{n}} \approx l$. \\
Soit $a = u_{N_0}$ alors $u_{N_0+1} \approx l.a, u_{N_0+2} \approx {l^2.a}, \ldots , u_{N_0+m} \approx {l^{m}.a}$. \\
Par consequent, la suite $u_n$ converge vers $0$ car $|l| < 1$.
\\ 
Donc la proposition est vraie.


\subsection*{Exercice 5}
la proposition est vraie. Voir d\'efinition du cours.

\subsection*{Exercice 6}
Soit la fonction 
$$f(x) = 
\left\{ 
\begin{array}{l l}
 x &  x \le 0\\
 x+1-sin\; x & x > 0\\
\end{array}
\right. 
$$
La fonction $f$ n'est pas continue en $0$ ($f(0^-) = 0$ et $f(0^+) = 1$) donc elle n'admet pas de limite en $0$.

Donc la proposition est fausse.

\subsection*{Exercice 7}
Rappel de cours:
$$\lim_{x \neq x_0, x \to x_0}(g \circ f)(x) = \lim_{x \neq x_0, x \to x_0}g(f(x)) = g(y_0)\; si\; \lim_{x \neq x_0, x \to x_0} f(x) = y_0 $$

Prenons $g(x) = sin\; x$ et $f(x) = \frac{\pi.x}{|2x|}$. 
$$ \lim_{x \neq 0, x \to 0} f(x) = \lim_{x \neq 0, x \to 0} \frac{\pi.x}{|2x|}$$
\begin{itemize}
\item Lorsque $x>0$, $\frac{\pi.x}{|2x|} = \frac{\pi}{2}$, donc $\lim_{x \neq 0, x \to 0} g(f(x)) = g(\frac{\pi}{2}) = 1$ 
\item lorsque $x<0$, $\frac{\pi.x}{|2x|} = \frac{-\pi}{2}$, donc $\lim_{x \neq 0, x \to 0} g(f(x)) = g(\frac{-\pi}{2}) = 1$.
\end{itemize}


Donc la proposition est vraie.


\subsection*{Exercice 8}
$$\lim_{x \to +\infty} \frac{ln(1+x^2)}{ln x} = 2 \;\;?$$
Application de la r\`egle de l'Hospital car $\lim_{x \to +\infty} ln(1+x2) = +\infty$ et $\lim_{x \to +\infty} ln\, x = +\infty$.\\

On calcule les deux d\'eriv\'es: $(ln(1+x^2))' = \frac{(1+x^2)'}{1+x^2} = \frac{2x}{1+x^2}$ et $(ln\, x)' = \frac{1}{x}$.
$$ \lim_{x \to +\infty} \frac{ln(1+x^2)}{ln x} = \lim_{x \to +\infty} \frac{\frac{2x}{1+x^2}}{\frac{1}{x}} = \lim_{x \to +\infty} \frac{2x^2}{1+x^2} = \lim_{x \to +\infty} \frac{2}{\frac{1}{x^2}+1} = 2$$


Donc la proposition est vraie.

\subsection*{Exercice 9}
$$ \lim_{x \neq 0, x \to 0} \frac{1}{x} \frac{1}{\sqrt{1+x^{-2}}}= 1 \;\;?$$

$$\lim_{x \neq 0,x \to 0} \frac{1}{x} = \infty$$
et 
$$\lim_{x \neq 0,x \to 0} \frac{1}{\sqrt{1+x^{-2}}} = 0$$

Limite ind\'etermin\'ee.

[1] $x>0$ alors $x = \sqrt{x^2}$.
$$ \lim_{x \neq 0, x \to 0^{+}} \frac{1}{x} \frac{1}{\sqrt{1+x^{-2}}} = \lim_{x \neq 0, x \to 0^{+}} \frac{1}{\sqrt{x^2}\sqrt{1+x^{-2}}}$$
$$ = \lim_{x \neq 0, x \to 0^{+}} \frac{1}{\sqrt{x^2+1}} = 1$$



[12] $x<0$ alors $x = -\sqrt{x^2}$.
$$ \lim_{x \neq 0, x \to 0^{-}} \frac{1}{x} \frac{1}{\sqrt{1+x^{-2}}} = \lim_{x \neq 0, x \to 0^{-}} \frac{1}{-\sqrt{x^2}\sqrt{1+x^{-2}}}$$
$$ = \lim_{x \neq 0, x \to 0^{-}} \frac{-1}{\sqrt{x^2+1}} = -1$$



Donc la proposition est fausse.


\subsection*{Exercice 10}
Rappel de cours:\\
Soit f une fonction d\'efinie au voisinage d'un point $x_0 \in \R$ (donc y compris en
$x_0$). On dit que $f$ est continue en $x_0$ si $x \ne x_0,\;\lim_{x \to x_0} f(x) = f(x_0)$.


$$(H \circ f) (x) = 
\left\{ 
\begin{array}{l l}
 1 + 1 -\frac{1}{2}cos^2\, x & 1 -\frac{1}{2}cos^2\, x \ge 0\\
 0 & 1 -\frac{1}{2}cos^2\, x < 0\\
\end{array}
\right. 
$$

Les fonctions $0$ et $2 -\frac{1}{2}cos^2\, x$ sont continues car elles sont une combinaison de fonctions continues. Il faut v\'erifier la continuit\'e de la fonction $(H \circ f) (x)$ en $1 -\frac{1}{2}cos^2\, x = 0$.
$$1 -\frac{1}{2}cos^2\, x < 0$$
$$\frac{1}{2}cos^2\, x > 1$$
$$cos^2\, x > 2$$
$$|cos\, x| > \sqrt{2}$$
Il n'existe pas de $x$ tel que $|cos\, x| > \sqrt{2}$. Donc $(H \circ f) (x) = 2 -\frac{1}{2}cos^2\, x$.


Donc la proposition est vraie.

\subsection*{Exercice 11}
Soit la fonction 
$$f(x) = 
\left\{ 
\begin{array}{l l}
 x+1 & x \ge 0\\
 x-1 & x < 0\\
\end{array}
\right. 
$$

La fonction $f(x)$ est croissante et n'est pas continue en 0.


Donc la proposition est fausse.

\subsection*{Exercice 12}
La fonction se prolonge par continuit\'e si $\exists l \in \R, \lim_{x \neq 0, x \to 0} g(x) = l$.
$$\lim_{x \neq 0, x \to 0} (e^{sin\; x}-1) = 0$$
$$\lim_{x \neq 0, x \to 0} cos\; \frac{1}{x} \in [-1;1]$$
$$\lim_{x \neq 0, x \to 0} ln(3+cos\; \frac{1}{x}) \in [ln\; 2; ln\; 4]$$
Donc
$$\lim_{x \neq 0, x \to 0} g(x) = 0$$

Le prolongement par continuit\'e de la fonction $g(x)$ est: 
$$p(x) = 
\left\{ 
\begin{array}{l l}
 g(x) & x \in \R^{*}\\
 0 & x = 0\\
\end{array}
\right. 
$$

Donc la proposition est vraie.


\subsection*{Exercice 13}
$f$ est une fonction $f$ continue alors 
$$\forall x_0 \in [2,3], \forall \epsilon >0, \exists \eta > 0\; tel\; que\; (\forall x \in [2,3] \cap ]x_0-\eta, x_0+\eta[, |f(x)-f(x_0)| < \epsilon) \;\;\; [1]$$ 

$f$ a pour limite $\infty$ en $x_0=\frac{5}{2}$ alors
$$\forall A \in \R, \exists \eta > 0,\; tel\; que\; (]x_0-\eta, x_0+\eta[ \setminus \{x_0\} \subset [2,3] \; et \; \forall x \in ]x_0-\eta, x_0+\eta[ \setminus \{x_0\},\; f(x) > A)$$
$$\forall A \in \R, \exists \eta > 0,\; tel\; que\; (\forall x \in [2,3] \cap ]x_0-\eta, x_0+\eta[ \setminus \{x_0\}, \; f(x) > A) \;\;\; [2]$$

Deux cas possibles, 
\begin{itemize}
\item la fonction $f$ est d\'efinie en $x_0$, alors $f(x_0) = \infty$ et la proposition [1] est fausse.
\item la fonction $f$ n'est pas d\'efinie en $x_0$, alors il faut trouver un prolongement par continuit\'e de la fonction $f$ en $x_0$. Il n'existe pas de valeur $l \in \R$ tel que $\lim_{x \neq x_0, x \to x_0} f(x) = l$ car $\lim_{x \neq x_0, x \to x_0} f(x) = \infty$. Donc,  la fonction $f$ n'est pas prolongeable en $x_0$.
\end{itemize}


Donc la proposition est fausse.


\subsection*{Exercice 14}
A faire
Preuve par l'absurde. Il faut admettre qu'une telle fonction existe pour montrer une contradiction. Donc elle n'existe pas.

Donc la proposition est fausse.

\subsection*{Exercice 15}
Admettons que la fonction n'est pas constante alors $\exists x_0, \exists \eta > 0\; t.q.\; f(x_0 - \eta) = Z_1\; et \; f(x_0 + \eta) = Z_2$ et $Z_1 \neq Z_2$.\\


$\forall x \in ]x_0 - \eta;x_0 + \eta[, \epsilon = 0.1, |f(x) - f(x_0)| \nless \epsilon$ ce qui contredit l'hypoth\`ese de la fonction continue.\\

Donc la proposition est vraie.

\subsection*{Exercice 16}
A faire
Quand $c_n \le \sqrt{2}$, on fait d\'ecroitre $bn$, quand $c_n > \sqrt{2}$, on fait croitre $a_n$, donc les deux suites convergent vers $\sqrt{2}$.

Donc la proposition est vraie.

QED

\end{document}

