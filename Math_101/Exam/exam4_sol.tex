\documentclass[]{book}

%These tell TeX which packages to use.
\usepackage{array,epsfig}
\usepackage{amsmath}
\usepackage{amsfonts}
\usepackage{amssymb}
\usepackage{amsxtra}
\usepackage{amsthm}
\usepackage{mathrsfs}
\usepackage{color}
\usepackage{pgfplots}

%Here I define some theorem styles and shortcut commands for symbols I use often
\theoremstyle{definition}
\newtheorem{defn}{Definition}
\newtheorem{thm}{Theorem}
\newtheorem{cor}{Corollary}
\newtheorem*{rmk}{Remark}
\newtheorem{lem}{Lemma}
\newtheorem*{joke}{Joke}
\newtheorem{ex}{Example}
\newtheorem*{soln}{Solution}
\newtheorem{prop}{Proposition}

\newcommand{\lra}{\longrightarrow}
\newcommand{\ra}{\rightarrow}
\newcommand{\surj}{\twoheadrightarrow}
\newcommand{\graph}{\mathrm{graph}}
\newcommand{\bb}[1]{\mathbb{#1}}
\newcommand{\Z}{\bb{Z}}
\newcommand{\Q}{\bb{Q}}
\newcommand{\R}{\bb{R}}
\newcommand{\C}{\bb{C}}
\newcommand{\N}{\bb{N}}
\newcommand{\M}{\mathbf{M}}
\newcommand{\m}{\mathbf{m}}
\newcommand{\MM}{\mathscr{M}}
\newcommand{\HH}{\mathscr{H}}
\newcommand{\Om}{\Omega}
\newcommand{\Ho}{\in\HH(\Om)}
\newcommand{\bd}{\partial}
\newcommand{\del}{\partial}
\newcommand{\bardel}{\overline\partial}
\newcommand{\textdf}[1]{\textbf{\textsf{#1}}\index{#1}}
\newcommand{\img}{\mathrm{img}}
\newcommand{\ip}[2]{\left\langle{#1},{#2}\right\rangle}
\newcommand{\inter}[1]{\mathrm{int}{#1}}
\newcommand{\exter}[1]{\mathrm{ext}{#1}}
\newcommand{\cl}[1]{\mathrm{cl}{#1}}
\newcommand{\ds}{\displaystyle}
\newcommand{\vol}{\mathrm{vol}}
\newcommand{\cnt}{\mathrm{ct}}
\newcommand{\osc}{\mathrm{osc}}
\newcommand{\LL}{\mathbf{L}}
\newcommand{\UU}{\mathbf{U}}
\newcommand{\support}{\mathrm{support}}
\newcommand{\AND}{\;\wedge\;}
\newcommand{\OR}{\;\vee\;}
\newcommand{\Oset}{\varnothing}
\newcommand{\st}{\ni}
\newcommand{\wh}{\widehat}

%Pagination stuff.
\setlength{\topmargin}{-.3 in}
\setlength{\oddsidemargin}{0in}
\setlength{\evensidemargin}{0in}
\setlength{\textheight}{9.in}
\setlength{\textwidth}{6.5in}
\pagestyle{empty}



\begin{document}

\subsection*{Rappel de cours}

\begin{itemize}
\item 
\end{itemize}

\subsection*{Exo 1}
Preuve par r\'ecurrence.\\
Proposition est vraie pour $u_0 = 0 = 2^0 -1$.\\
Supposons que $u_n = 2^n-1$ pour $n>0$, v\'erifions si $u_{n+1} = 2^{n+1} - 1$.
$$u_{n+1} = 2u_n + 1$$
$$u_{n+1} = 2(2^n - 1) + 1$$
$$u_{n+1} = 2*2^n - 1$$
$$u_{n+1} = 2^{n+1} - 1$$

La proposition est Vraie.

\subsection*{Exo 2}
Preuve par r\'ecurrence.\\
Proposition est vraie pour $u_0 = 3 = 3^{2*0}$.\\
Supposons que $u_n = 3^{2n}$ pour $n>0$, v\'erifions si $u_{n+1} = 3^{2(n+1)}$.
$$u_{n+1} = u_n^2$$
$$u_{n+1} = (3^{2n})^2$$
$$u_{n+1} = 3^{4n}$$

La proposition est Fausse.

\subsection*{Exo 3}
Prenons $f(x) = x^2 + 1$, et d\'eterminons le signe de $f(x) - x$ selon $x$.\\
$$f(x) - x = x^2 + 1 -x = x(x-1) + 1$$

$$f(x) - x,
\left\{ 
\begin{array}{l l}
 >0 & x \in ]-\infty,0[\\
 >0 & x = 0\\
 >0 & x \in ]0,1[ \\
 >0 & x = 1\\
 >0 & x \in ]1,+\infty \\ 
\end{array}
\right. 
$$

\begin{itemize} 
\item La fonction $f$ est continue sur $\R$ car c'est un assemblage de fonctions continues sur $\R$,
\item La fonction $f$ est stable sur $\R$ car $f(\R) \subset \R^+ \subset \R$.
\item La fonction $f$ est strictement croissante
\item La fonction $f$ admet un point fixe , donc la suite $u_n = u_n^2 + 1$ est strictement croissante donc tend vers $l \in \R \cup \{+\infty\}$
\end{itemize} 

En passant \`a la limite dans l'in\'egalit\'e $u_n > u_0$, on obtient $l > u_0$, et la suite $u_n$ n'est pas constante, on en d\'eduit que $l=+\infty$ donc, la suite $\lim_{n \to +\infty} u_n = \{+\infty\}$. \\


La proposition est Vraie.

\subsection*{Exo 4}
Prenons $f(x) = 1 + arctan(\frac{x}{2})$, et d\'eterminon le signe de $f(x) - x$ selon $x$.\\
$$g(x) = f(x) - x = 1 + arctan(\frac{x}{2}) - x$$

La fonction $g(x) = f(x) - x$ est strictement d\'ecroissante, positive $\forall x \in ]-\infty,x_{pf}[$, n\'egative $\forall ]x_{pf},+\infty[$, donc elle s'annule pour un point $x_{pf} in ]1-\frac{\pi}{2}, 1+\frac{\pi}{2}[$.

$$
\left\{ 
\begin{array}{l l}
 f(x)>x & x \in ]-\infty,x_{pf}[\\
 =0 & x_{pf} in ]1-\frac{\pi}{2}, 1+\frac{\pi}{2}[\\
 f(x)<x & x \in ]x_{pf},+\infty[ \\
\end{array}
\right. 
$$

\begin{itemize} 
\item La fonction $f$ est continue sur $\R$ car c'est un assemblage de fonctions continues sur $\R$,
\item La fonction $f$ est stable sur $\R$ car $f(\R) \subset ]1-\frac{\pi}{2},1+\frac{\pi}{2}[ \subset \R$.
\item La fonction $f$ est strictement croissante
\item La fonction $f$ admet un point fixe $x_{pf}$
\end{itemize} 

Cas $u_0=x_{pf}$, la suite est constante.\\
cas $u_0 \neq x_{pf}$. Comme la fonction $f$ est strictement croissante sur $\R$, on  $f'(x_{pf}) > 1$, donc le point $x_{pf}$ est r\'epulsif et la suite $u_n$ n'est pas convergente.\\

La proposition est Fausse.

\subsection*{Exo 5}

La proposition est Fausse.

\subsection*{Exo 6}

Prenons la valeur $x=\frac{3\pi}{4}$. On a :
$$sin(\frac{3\pi}{4}) = \frac{1}{\sqrt{2}}$$
$$arcsin(\frac{1}{\sqrt{2}}) = \frac{\pi}{4}$$

Donc $arcsin(sin(x)) \neq x$.\\

La proposition est Fausse.

\subsection*{Exo 7}
Rappel de cours:\\
\begin{itemize}
\item une fonction $f$ est bijective, si elle est injective et surjective
\item une fonction $f$ est injective, si $\forall (x_1, x_2) \in D x D, f(x_1) = f(x_2) \implies x_1 = x_2$ ou $\forall (x_1, x_2) \in D x D, x_1 \neq x_2 \implies f(x_1) \neq f(x_2)$
\item une fonction $f$ est surjective, si $\forall y \in A, \exists x \in D, f(x) = y$ 
\item une fonction $f$ est de classe $\mathcal{C}^1$ sur $D$, si la fonction $f$ est d\'erivable sur $D$ et sa d\'eriv\'ee $f'(x)$ est continue sur $D$.
\end{itemize}

La fonction $f$ est de classe $\mathcal{C}^1$ sur $]-1;1[$, donc $\forall x_0 \in ]-1;1[, f'(x_0) = \lim_{x \to x_0}\frac{f(x) - f(x_0)}{x-x_0}$.
On a 
$$\lim_{x \to x_0}\frac{f(x) - f(x_0)}{x-x_0} = \lim_{x \to x_0}\frac{f(x) - f(x_0)}{f^{-1}(f(x)) - f^{-1}(f(x_0))}$$

Comme $f$ est injective $x \neq x_0 \implies f(x) \neq f(x_0)$, donc $f(x)-f(x_0) \neq 0$.\\
$$\frac{1}{f'(x_0)} = \lim_{x \to x_0}\frac{f^{-1}(f(x)) - f^{-1}(f(x_0))}{f(x) - f(x_0)}$$
En prenant $f(x_0) = y_0$ et $f(x) = y$, on a 
$$\frac{1}{f'(x_0)} = \lim_{x \to x_0}\frac{f^{-1}(f(x)) - f^{-1}(f(x_0))}{f(x) - f(x_0)}$$
Comme la fonction $f$ est continue, lorsque $x \to x_0$, on a $f(x) \to f(x_0)$, donc $y \to y_0$.
$$\frac{1}{f'(x_0)} = \lim_{y \to y_0}\frac{f^{-1}(y) - f^{-1}(y_0))}{y - y_0} = (f^{-1})'(y_0)$$

La d\'eriv\'e de la function $f^{-1}$ existe et est continue car la d\'eriv\'ee de la fonction $f$ est continue (ie. la fonction $f$ ets de classe $\mathcal{C}^1$).

La proposition est Vraie.

\subsection*{Exo 8}

La proposition est Fausse.

\subsection*{Exo 9}
Rappel de cours:\\
\begin{itemize}
\item une fonction $f$ est impaire si $\forall x, f(-x) = -f(x)$
\item une fonction $f$ est bijective, si elle est injective et surjective
\item une fonction $f$ est injective, si $\forall (x_1, x_2) \in D x D, f(x_1) = f(x_2) \implies x_1 = x_2$ ou $\forall (x_1, x_2) \in D x D, x_1 \neq x_2 \implies f(x_1) \neq f(x_2)$
\item une fonction $f$ est surjective, si $\forall y \in A, \exists x \in D, f(x) = y$ 
\end{itemize}

La fonction $f(x)$ est impaire donc $f(-x) = -f(x)$. La fonction $f(x)$ est bijective donc la fonction $f^{-1}(x)$ existe.  on a $f(f^{-1}(x)) = x$ donc $-f(f^{-1}(x)) = -x$, comme $f$ est impaire $f(-f^{-1}(x)) = -x$.\\

Admettons que la fonction $f^{-1}$ ne soit pas impaire donc $f^{-1}(-x) = -f^{-1}(x)$. Comme la fonction $f$ est injective on a $-x = f(-f^{-1}(x)) \neq f(f^{-1}(-x)) = -x$. Contradiction, donc la fonction $f^{-1}$ est impaire.

La proposition est Vraie.

\subsection*{Exo 10}
Rappel de cours:
\begin{itemize}
\item Int\'egrale de Riemann. $\frac{b-a}{n}\sum_{k=1}^{n}f(a+k\frac{b-a}{n}) \to_{n \to \infty} \int_{a}^{b} f(x)dx$
\end{itemize}

On a $\frac{\pi}{n}\sum_{k=0}^{n-1} sin(\frac{\pi k}{2n})$.

En prenant $b=\pi$, $a=0$, $x=k/n$, on a $f(x) = sin(\frac{\pi}{2}x)$
$$\lim_{n \to \infty}\frac{\pi}{n}\sum_{k=0}^{n-1}sin(\frac{\pi k}{2n}) = \int_0^{\pi}sin(\frac{\pi}{2}x)dx$$

Int\'egrale par substitution: $u=\frac{\pi}{2}x$, donc $\frac{du}{dx} = \frac{\pi}{2}$ et $dx=\frac{2}{\pi}du$
$$\int sin(\frac{\pi}{2}x)dx = \frac{2}{\pi}\int sin(u)du = -\frac{2}{\pi}cos(u) = -\frac{2}{\pi}cos(\frac{\pi}{2}x)$$
$$\int_0^{\pi} sin(\frac{\pi}{2}x) = [-\frac{2}{\pi}cos(\frac{\pi}{2}x)]_0^{\pi} = -\frac{2}{\pi} + \frac{cos(\frac{\pi^2}{2})}{n} \neq 1$$


La proposition est Fausse.

\subsection*{Exo 11}
On a $$\sum_{k=1}^{n}\frac{k}{n^2+k^2} = \sum_{k=1}^{n}\frac{k}{n^2(1+k^2/n^2)} = \frac{1}{n^2}\sum_{k=1}^{n}\frac{k}{1+(k/n)^2} = \frac{1}{n}.\frac{1-0}{n}\sum_{k=1}^{n}\frac{k}{1+(k/n)^2}$$

En prenant: $a=0$, $b=1$ et $x=k/n$ on a $f(x)=\frac{nx}{1+x^2}$ donc
$$\lim_{n \to \infty}\sum_{k=1}^{n}\frac{k}{n^2+k^2} = lim_{n \to \infty}\frac{1}{n}.\int_0^1 \frac{nx}{1+x^2}$$
$$\int_0^1 \frac{x}{1+x^2} = [\frac{ln(x^2+1)}{2}]_0^1 = \frac{ln(2)}{2} - \frac{ln(1)}{2}$$

Donc
$$ \lim_{n \to \infty}\sum_{k=1}^{n}\frac{k}{n^2+k^2} \neq \frac{\pi}{4}$$

La proposition est Fausse.

\subsection*{Exo 12}
On a $$\sum_{k=1}^{n}\frac{n}{n^2+k^2} = \sum_{k=1}^{n}\frac{n}{n^2(1+k^2/n^2)} = \frac{1}{n}\sum_{k=1}^{n}\frac{1}{1+(k/n)^2} = \frac{1-0}{n}\sum_{k=1}^{n}\frac{1}{1+(k/n)^2}$$

En prenant: $a=0$, $b=1$ et $x=k/n$ on a $f(x)=\frac{1}{1+x^2}$ donc
$$\lim_{n \to \infty}\sum_{k=1}^{n}\frac{n}{n^2+k^2} = \int_0^1 \frac{1}{1+x^2}$$
$$\int_0^1 \frac{1}{1+x^2} = [arctan(x)]_0^1 = arctan(1) - arctan(0) = \frac{\pi}{4} - 0 = \frac{\pi}{4}$$

La proposition est Vraie.

\subsection*{Exo 13}
Rappel de cours:
\begin{itemize}
\item Int\'egrale par partie. $\int_{a}^{b} f(x)g'(x)dx = [f(x)g(x)]_{a}^{b} - \int_{a}^{b}f'(x)g(x)$
\end{itemize}

On prend
$$
\begin{array}{l l}
 f(x) = (x-1)^2 & g'(x) = e^x\\
 f'(x) = 2x-2   & g(x) = e^x\\
\end{array}
$$

\medskip
Donc $\int_0^1(x-1)^2e^x = [(x-1)^2e^x]_0^1 - \int_0^1 (2x-2)e^x = -1 - \int_0^1 (2x-2)e^x$

On prend
$$
\begin{array}{l l}
 f(x) = 2x-2 & g'(x) = e^x\\
 f'(x) = 2   & g(x) = e^x\\
\end{array}
$$

\medskip
Donc $\int_0^1 (2x-2)e^x = [(2x-2)e(x)]_0^1 - \int_0^1 2e^x = [(2x-2)e(x)]_0^1 - 2[e^x]_0^1 = 2 - 2e + 2 = 4 - 2e$.\\

Enfin $\int_0^1(x-1)^2e^x = -1 - (4 - 2e) = 2e - 5$


La proposition est Vraie.

\subsection*{Exo 14}

Soit $f(x)$ une fonction croissante et $F(x)$ une fonction tel que $F'(x) = f(x)$. On a
$$g(x) = \int_{x^3}^{x^3+1}f(t)dt = F(x^3+1) - F(x^3)$$
La valeur de la d\'eriv\'ee de la fonction $g(x)$ donne le sens de la fonction $g(x)$.
$$g'(x) = (F(x^3+1) - F(x^3))' = F'(x^3+1) - F'(x^3) = f(x^3+1) - f(x^3)$$
La fonction $f(x)$ est croissante donc $\forall x_1, x_2, x_1 \geq x_2 \implies f(x_1) \geq f(x_2)$ et $x^3+1 > x^3$. Donc $g'(x)$ est toujours positive ou nulle. Par cons\'equent la fonction $g(x)$ est croissante.

La proposition est Vraie.

\subsection*{Exo 15}

La proposition est Fausse.

\subsection*{Exo 16}

La proposition est Fausse.


\end{document}

