\documentclass[]{book}

%These tell TeX which packages to use.
\usepackage{array,epsfig}
\usepackage{amsmath}
\usepackage{amsfonts}
\usepackage{amssymb}
\usepackage{amsxtra}
\usepackage{amsthm}
\usepackage{mathrsfs}
\usepackage{color}
\usepackage{pgfplots}

%Here I define some theorem styles and shortcut commands for symbols I use often
\theoremstyle{definition}
\newtheorem{defn}{Definition}
\newtheorem{thm}{Theorem}
\newtheorem{cor}{Corollary}
\newtheorem*{rmk}{Remark}
\newtheorem{lem}{Lemma}
\newtheorem*{joke}{Joke}
\newtheorem{ex}{Example}
\newtheorem*{soln}{Solution}
\newtheorem{prop}{Proposition}

\newcommand{\lra}{\longrightarrow}
\newcommand{\ra}{\rightarrow}
\newcommand{\surj}{\twoheadrightarrow}
\newcommand{\graph}{\mathrm{graph}}
\newcommand{\bb}[1]{\mathbb{#1}}
\newcommand{\Z}{\bb{Z}}
\newcommand{\Q}{\bb{Q}}
\newcommand{\R}{\bb{R}}
\newcommand{\C}{\bb{C}}
\newcommand{\N}{\bb{N}}
\newcommand{\M}{\mathbf{M}}
\newcommand{\m}{\mathbf{m}}
\newcommand{\MM}{\mathscr{M}}
\newcommand{\HH}{\mathscr{H}}
\newcommand{\Om}{\Omega}
\newcommand{\Ho}{\in\HH(\Om)}
\newcommand{\bd}{\partial}
\newcommand{\del}{\partial}
\newcommand{\bardel}{\overline\partial}
\newcommand{\textdf}[1]{\textbf{\textsf{#1}}\index{#1}}
\newcommand{\img}{\mathrm{img}}
\newcommand{\ip}[2]{\left\langle{#1},{#2}\right\rangle}
\newcommand{\inter}[1]{\mathrm{int}{#1}}
\newcommand{\exter}[1]{\mathrm{ext}{#1}}
\newcommand{\cl}[1]{\mathrm{cl}{#1}}
\newcommand{\ds}{\displaystyle}
\newcommand{\vol}{\mathrm{vol}}
\newcommand{\cnt}{\mathrm{ct}}
\newcommand{\osc}{\mathrm{osc}}
\newcommand{\LL}{\mathbf{L}}
\newcommand{\UU}{\mathbf{U}}
\newcommand{\support}{\mathrm{support}}
\newcommand{\AND}{\;\wedge\;}
\newcommand{\OR}{\;\vee\;}
\newcommand{\Oset}{\varnothing}
\newcommand{\st}{\ni}
\newcommand{\wh}{\widehat}

%Pagination stuff.
\setlength{\topmargin}{-.3 in}
\setlength{\oddsidemargin}{0in}
\setlength{\evensidemargin}{0in}
\setlength{\textheight}{9.in}
\setlength{\textwidth}{6.5in}
\pagestyle{empty}



\begin{document}

\subsection*{Rappel de cours}

\begin{itemize}
\item 
\end{itemize}

\subsection*{Exo 1}
Preuve par r\'ecurrence.\\
Proposition vraie pour $u_0 = 0 = 2^0 -1$.\\
Supposons que $u_n = 2^n-1$ pour $n>0$, calculons $u_{n+1}$.
$$u_{n+1} = 2u_n + 1$$
$$u_{n+1} = 2(2^n - 1) + 1$$
$$u_{n+1} = 2*2^n - 1$$
$$u_{n+1} = 2^{n+1} - 1$$

La proposition est Vraie.

\subsection*{Exo 2}

La proposition est Fausse.

\subsection*{Exo 3}

La proposition est Fausse.

\subsection*{Exo 4}

La proposition est Fausse.

\subsection*{Exo 5}

La proposition est Fausse.

\subsection*{Exo 6}

La proposition est Fausse.

\subsection*{Exo 7}

La proposition est Fausse.

\subsection*{Exo 8}

La proposition est Fausse.

\subsection*{Exo 9}

La proposition est Fausse.

\subsection*{Exo 10}

La proposition est Fausse.

\subsection*{Exo 11}

La proposition est Fausse.

\subsection*{Exo 12}

La proposition est Fausse.

\subsection*{Exo 13}

La proposition est Fausse.

\subsection*{Exo 14}

La proposition est Fausse.

\subsection*{Exo 15}

La proposition est Fausse.

\subsection*{Exo 16}

La proposition est Fausse.


\end{document}

