\documentclass[]{book}

%These tell TeX which packages to use.
\usepackage{array,epsfig}
\usepackage{amsmath}
\usepackage{amsfonts}
\usepackage{amssymb}
\usepackage{amsxtra}
\usepackage{amsthm}
\usepackage{mathrsfs}
\usepackage{color}
\usepackage{pgfplots}

%Here I define some theorem styles and shortcut commands for symbols I use often
\theoremstyle{definition}
\newtheorem{defn}{Definition}
\newtheorem{thm}{Theorem}
\newtheorem{cor}{Corollary}
\newtheorem*{rmk}{Remark}
\newtheorem{lem}{Lemma}
\newtheorem*{joke}{Joke}
\newtheorem{ex}{Example}
\newtheorem*{soln}{Solution}
\newtheorem{prop}{Proposition}

\newcommand{\lra}{\longrightarrow}
\newcommand{\ra}{\rightarrow}
\newcommand{\surj}{\twoheadrightarrow}
\newcommand{\graph}{\mathrm{graph}}
\newcommand{\bb}[1]{\mathbb{#1}}
\newcommand{\Z}{\bb{Z}}
\newcommand{\Q}{\bb{Q}}
\newcommand{\R}{\bb{R}}
\newcommand{\C}{\bb{C}}
\newcommand{\N}{\bb{N}}
\newcommand{\M}{\mathbf{M}}
\newcommand{\m}{\mathbf{m}}
\newcommand{\MM}{\mathscr{M}}
\newcommand{\HH}{\mathscr{H}}
\newcommand{\Om}{\Omega}
\newcommand{\Ho}{\in\HH(\Om)}
\newcommand{\bd}{\partial}
\newcommand{\del}{\partial}
\newcommand{\bardel}{\overline\partial}
\newcommand{\textdf}[1]{\textbf{\textsf{#1}}\index{#1}}
\newcommand{\img}{\mathrm{img}}
\newcommand{\ip}[2]{\left\langle{#1},{#2}\right\rangle}
\newcommand{\inter}[1]{\mathrm{int}{#1}}
\newcommand{\exter}[1]{\mathrm{ext}{#1}}
\newcommand{\cl}[1]{\mathrm{cl}{#1}}
\newcommand{\ds}{\displaystyle}
\newcommand{\vol}{\mathrm{vol}}
\newcommand{\cnt}{\mathrm{ct}}
\newcommand{\osc}{\mathrm{osc}}
\newcommand{\LL}{\mathbf{L}}
\newcommand{\UU}{\mathbf{U}}
\newcommand{\support}{\mathrm{support}}
\newcommand{\AND}{\;\wedge\;}
\newcommand{\OR}{\;\vee\;}
\newcommand{\Oset}{\varnothing}
\newcommand{\st}{\ni}
\newcommand{\wh}{\widehat}

%Pagination stuff.
\setlength{\topmargin}{-.3 in}
\setlength{\oddsidemargin}{0in}
\setlength{\evensidemargin}{0in}
\setlength{\textheight}{9.in}
\setlength{\textwidth}{6.5in}
\pagestyle{empty}



\begin{document}

\subsection*{Rappel de cours}

\begin{itemize}
\item 
\end{itemize}

\subsection*{Exo 1}
Preuve par r\'ecurrence.\\
Proposition est vraie pour $u_0 = 0 = 2^0 -1$.\\
Supposons que $u_n = 2^n-1$ pour $n>0$, v\'erifions si $u_{n+1} = 2^{n+1} - 1$.
$$u_{n+1} = 2u_n + 1$$
$$u_{n+1} = 2(2^n - 1) + 1$$
$$u_{n+1} = 2*2^n - 1$$
$$u_{n+1} = 2^{n+1} - 1$$

La proposition est Vraie.

\subsection*{Exo 2}
Preuve par r\'ecurrence.\\
Proposition est vraie pour $u_0 = 3 = 3^{2*0}$.\\
Supposons que $u_n = 3^{2n}$ pour $n>0$, v\'erifions si $u_{n+1} = 3^{2(n+1)}$.
$$u_{n+1} = u_n^2$$
$$u_{n+1} = (3^{2n})^2$$
$$u_{n+1} = 3^{4n}$$

La proposition est Fausse.

\subsection*{Exo 3}
Prenons $f(x) = x^2 + 1$, et d\'eterminons le signe de $f(x) - x$ selon $x$.\\
$$f(x) - x = x^2 + 1 -x = x(x-1) + 1$$

$$f(x) - x,
\left\{ 
\begin{array}{l l}
 >0 & x \in ]-\infty,0[\\
 >0 & x = 0\\
 >0 & x \in ]0,1[ \\
 >0 & x = 1\\
 >0 & x \in ]1,+\infty \\ 
\end{array}
\right. 
$$

\begin{itemize} 
\item La fonction $f$ est continue sur $\R$ car c'est un assemblage de fonctions continues sur $\R$,
\item La fonction $f$ est stable sur $\R$ car $f(\R) \subset \R^+ \subset \R$.
\item La fonction $f$ est strictement croissante
\item La fonction $f$ admet un point fixe , donc la suite $u_n = u_n^2 + 1$ est strictement croissante donc tend vers $l \in \R \cup \{+\infty\}$
\end{itemize} 

En passant \`a la limite dans l'in\'egalit\'e $u_n > u_0$, on obtient $l > u_0$, et la suite $u_n$ n'est pas constante, on en d\'eduit que $l=+\infty$ donc, la suite $\lim_{n \to +\infty} u_n = \{+\infty\}$. \\


La proposition est Vraie.

\subsection*{Exo 4}
Prenons $f(x) = 1 + arctan(\frac{x}{2})$, et d\'eterminon le signe de $f(x) - x$ selon $x$.\\
$$g(x) = f(x) - x = 1 + arctan(\frac{x}{2}) - x$$

La fonction $g(x) = f(x) - x$ est strictement d\'ecroissante, positive $\forall x \in ]-\infty,x_{pf}[$, n\'egative $\forall ]x_{pf},+\infty[$, donc elle s'annule pour un point $x_{pf} in ]1-\frac{\pi}{2}, 1+\frac{\pi}{2}[$.

$$
\left\{ 
\begin{array}{l l}
 f(x)>x & x \in ]-\infty,x_{pf}[\\
 =0 & x_{pf} in ]1-\frac{\pi}{2}, 1+\frac{\pi}{2}[\\
 f(x)<x & x \in ]x_{pf},+\infty[ \\
\end{array}
\right. 
$$

\begin{itemize} 
\item La fonction $f$ est continue sur $\R$ car c'est un assemblage de fonctions continues sur $\R$,
\item La fonction $f$ est stable sur $\R$ car $f(\R) \subset ]1-\frac{\pi}{2},1+\frac{\pi}{2}[ \subset \R$.
\item La fonction $f$ est strictement croissante
\item La fonction $f$ admet un point fixe $x_{pf}$
\end{itemize} 

Cas $u_0=x_{pf}$, la suite est constante.\\
cas $u_0 \neq x_{pf}$. Comme la fonction $f$ est strictement croissante sur $\R$, on  $f'(x_{pf}) > 1$, donc le point $x_{pf}$ est r\'epulsif et la suite $u_n$ n'est pas convergente.\\

La proposition est Fausse.

\subsection*{Exo 5}

La proposition est Fausse.

\subsection*{Exo 6}

La proposition est Fausse.

\subsection*{Exo 7}

La proposition est Fausse.

\subsection*{Exo 8}

La proposition est Fausse.

\subsection*{Exo 9}

La proposition est Fausse.

\subsection*{Exo 10}

La proposition est Fausse.

\subsection*{Exo 11}

La proposition est Fausse.

\subsection*{Exo 12}

La proposition est Fausse.

\subsection*{Exo 13}

La proposition est Fausse.

\subsection*{Exo 14}

La proposition est Fausse.

\subsection*{Exo 15}

La proposition est Fausse.

\subsection*{Exo 16}

La proposition est Fausse.


\end{document}

