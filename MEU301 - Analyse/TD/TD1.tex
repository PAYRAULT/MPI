\documentclass[]{book}

%These tell TeX which packages to use.
\usepackage{array,epsfig}
\usepackage{amsmath}
\usepackage{amsfonts}
\usepackage{amssymb}
\usepackage{amsxtra}
\usepackage{amsthm}
\usepackage{mathrsfs}
\usepackage{color}
\usepackage[margin=2cm,top=2.5cm,headheight=16pt,headsep=0.1in,heightrounded]{geometry}
\usepackage{fancyhdr}
\pagestyle{fancy}
\usepackage{tikz}


%Here I define some theorem styles and shortcut commands for symbols I use often
\theoremstyle{definition}
\newtheorem{defn}{Definition}
\newtheorem{thm}{Theorem}
\newtheorem{cor}{Corollary}
\newtheorem*{rmk}{Remark}
\newtheorem{lem}{Lemma}
\newtheorem*{joke}{Joke}
\newtheorem{ex}{Example}
\newtheorem*{soln}{Solution}
\newtheorem{prop}{Proposition}

\newcommand{\lra}{\longrightarrow}
\newcommand{\ra}{\rightarrow}
\newcommand{\surj}{\twoheadrightarrow}
\newcommand{\graph}{\mathrm{graph}}
\newcommand{\bb}[1]{\mathbb{#1}}
\newcommand{\Z}{\bb{Z}}
\newcommand{\Q}{\bb{Q}}
\newcommand{\R}{\bb{R}}
\newcommand{\C}{\bb{C}}
\newcommand{\N}{\bb{N}}
\newcommand{\M}{\mathbf{M}}
\newcommand{\m}{\mathbf{m}}
\newcommand{\MM}{\mathscr{M}}
\newcommand{\HH}{\mathscr{H}}
\newcommand{\Om}{\Omega}
\newcommand{\Ho}{\in\HH(\Om)}
\newcommand{\bd}{\partial}
\newcommand{\del}{\partial}
\newcommand{\bardel}{\overline\partial}
\newcommand{\textdf}[1]{\textbf{\textsf{#1}}\index{#1}}
\newcommand{\img}{\mathrm{img}}
\newcommand{\ip}[2]{\left\langle{#1},{#2}\right\rangle}
\newcommand{\inter}[1]{\mathrm{int}{#1}}
\newcommand{\exter}[1]{\mathrm{ext}{#1}}
\newcommand{\cl}[1]{\mathrm{cl}{#1}}
\newcommand{\ds}{\displaystyle}
\newcommand{\vol}{\mathrm{vol}}
\newcommand{\cnt}{\mathrm{ct}}
\newcommand{\osc}{\mathrm{osc}}
\newcommand{\LL}{\mathbf{L}}
\newcommand{\UU}{\mathbf{U}}
\newcommand{\support}{\mathrm{support}}
\newcommand{\AND}{\;\wedge\;}
\newcommand{\OR}{\;\vee\;} 
\newcommand{\Oset}{\varnothing}
\newcommand{\st}{\ni}
\newcommand{\wh}{\widehat}
\newcommand{\vect}[1]{\overrightarrow{#1}}

%Pagination stuff.
\setlength{\topmargin}{-.3 in}
%\setlength{\oddsidemargin}{0in}
%\setlength{\evensidemargin}{0in}
\setlength{\textheight}{9.in}
\setlength{\textwidth}{6.5in}
\cfoot{page \thepage}
\lhead{MEU301 - Analyse}
\rhead{TD1}
\pagestyle{fancy}


\begin{document}

\subsection*{Rappel de cours}
\begin{defn}
Une suite de r\'eels est dite \emph{Suite de Cauchy} si
$$
\forall \epsilon > 0, \exists N \in \N, \forall m,n \in \N, (n \geq N \text{ et } m \geq N) \implies |u_n - u_m| \leq \epsilon 
$$
\end{defn}



\newpage
\subsection*{Exercice 1.1}
\subsection*{Exercice 1.1.1}
On a $m > 0$ et $n >0$ donc $\frac{m.n}{(n+m)^2} > 0$, donc 0 est un minorant. 
$$
\frac{m.n}{(n+m)^2} = \frac{1}{\frac{m}{n}+2+\frac{n}{m}}
$$
Il faut montrer que 
$$
\begin{array}{l l}
\frac{m}{n}+\frac{n}{m} \geq 2 \\
\frac{m^2+n^2}{m.n} \geq 2 \\
m^2+n^2-2m.n \geq 0 \\
(m-n)^2 \geq 0 \\
\end{array}
$$
Donc $1/4$ est un majorant.

On a $1/4$ est la borne sup\'erieure de A si il n'existe aucun majorant inf\'erieur \`a $1/4$. On a $1/4 \in A$ pour $m=n=1$. Donc il n'existe pas de plus petit majorant.

On a 0 est la borne inf\'erieure de A si il n'existe aucun minorant sup\'erieur \`a $0$. Quand $n=1$, on a 
$$\lim_{m \to \infty}\frac{m}{(m+1)^2} = \lim_{m \to \infty}\frac{1}{m} = 0$$
Donc il n'exise pas de minorant sup\'erieur \`a 0.

\subsection*{Exercice 1.1.2}
Montrons que 2 est un majorant et 0 un minorant.

On a $\frac{1}{n}+\frac{1}{m}>0$ car $n,m \in \N^{*}$. DOnc 0 est un minorant.

On a $\frac{1}{n}+\frac{1}{m} = \frac{m+n}{n.m}$, montrons que 
$$
\begin{array}{l l}
\frac{1}{m} +  \frac{1}{n} \leq 2 \\
\frac{m+n}{n.m} \leq 2 \\
m+n \leq 2m.n \\
m+n-2m.n \leq 0 \\
m(1-n)+n(1-m) \leq 0 \\
\end{array}
$$
Vrai car $(1-n) \leq 0$, $(1-m) \leq 0$ et $n,m \in \N^{*}$. Donc $2$ est un majorant.

On a 2 est la borne sup\'erieure de A si il n'existe aucun majorant inf\'erieur \`a $2$. On a $2 \in A$ pour $m=n=1$. Donc il n'existe pas de plus petit majorant.

On a 0 est la borne inf\'erieure de A si il n'existe aucun minorant sup\'erieur \`a $0$. On a 
$$\lim_{m \to \infty, n \to \infty,}\frac{1}{m} + \frac{1}{n} = \lim_{m \to \infty, n \to \infty,}\frac{1}{m} + \lim_{m \to \infty, n \to \infty,} \frac{1}{n} = 0$$
Donc il n'exise pas de minorant sup\'erieur \`a 0.


\subsection*{Exercice 1.1.3}
La fonction $f(x)=\frac{x+1}{x+2}$ est strictement croissante pour $x \leq -3$ ($f'(x) = \frac{(x+2)-(x+1)}{(x+2)^2} = \frac{1}{(x+2)^2} > 0$). Donc $f(-3) =  2$ est la borne sup\'erieure de $A$. On a 
$$\lim_{x \to -\infty} \frac{x+1}{x+2} = \lim_{x \to -\infty} \frac{1+1/x}{1+2/x} = 1$$
Donc 1 est la borne inf\'erieure.
Oui la borne sup\'erieure est atteinte pour $x=-3$ mais pas la borne inf\'erieure car c'est une limite.

Maintenant si on prend $x \leq 3$ c'est autre chose car $\sup(A) = \infty$ et $\inf(A) = -\infty$ quand $x \to -2$.

\subsection*{Exercice 1.1.4}
Comme l'ensemble $A$ est born\'e alors il existe $\sup(A)$ et $\inf(A)$. Divisons en 3 cas; $x < y$, $x=y$, et $x > y$.

Pour le cas $x=y$ on a $0 \in A$. Pas tr\`es int\'eressant car $|x-y| \geq 0$. Donc, 0 n'est pas un majorant. 

Pour le cas $x>y$ on a $|x-y| = x - y$. La plus grande valeur possible est quand $x=\sup(A)$ et $y = \inf(A)$ (ie. plus grand \'ecart possible) donc $|\sup(A)-inf(A)|$. \\
Pour le cas $x<y$ on a $|x-y| = y - x$. La plus grande valeur possible est quand $x=\inf(A)$ et $y = \sup(A)$ (ie. plus grand \'ecart possible) donc $|\sup(A)-inf(A)|$. \\


\subsection*{Exercice 1.1.5}
$\sup(|f(x)|) = 2$ car 
$$
\begin{array}{l l}
{]-\infty,-1[} & |f(x)| = 0 \\
{[-1,0[} & |f(x)| = 1 \\
{[0, 1]} & |f(x)| = 1 \\
{]1,2]} & |f(x)| = 2 \\
{]2,\infty[} & |f(x)| = 0 \\
\end{array}
$$

\subsection*{Exercice 1.2}
\subsection*{Exercice 1.2.1}
On voit bien que cela diverge, car $1/n$ diverge. Donc il faut trouver un contre-exemple pour $n$ et $m$.
Prenons pour commencer $m=n+1$ on a $|u_m-u_n| = \frac{1}{n+1}$, pour un $\epsilon$ donn\'e on peut uoujors trouver un $n$ tel que $1/n < \epsilon$. donc pas bon contre-exemple. Il faut \'eliminer les $n$ pour trouver une constante.

Prenons $m=2n$, $|u_m-u_n| = \frac{1}{n+1} + \frac{1}{n+2} + \ldots + \frac{1}{2n} > \frac{1}{2n} + \frac{1}{2n} \ldots + \frac{1}{2n} = \frac{n}{2n} = \frac{1}{2}$. L\`a, c'est mieux. On a, pour $m=2n$, $|u_m-u_n| > \frac{1}{2}$ donc la suite n'est pas de Cauchy (car si on prend $\epsilon = 1/3$, la propri\'et\'e n'est pas v\'erif\'ee pour $m=2n$).

La suite n'est pas de Cauchy et elle est croissante donc elle diverge. Par cons\'equent, $\lim_{n \to \infty} u_n = +\infty$.

\subsection*{Exercice 1.2.2}
$$
\begin{array}{l l}
u_2 & \frac{u_0+u_1}{2} \\
u_3 & \frac{u_1+u_2}{2} = \frac{u0+3u_1}{4}\\
u_4 & \frac{u_2+u_3}{2} = \frac{3u0+5u_1}{8}\\
u_5 & \frac{u_3+u_4}{2} = \frac{5u0+7u_1}{16}\\
\end{array}
$$
Calculons $|u_{n+1} - u_{n}|$
$$
|\frac{u_{n-1}+u_{n}}{2} - \frac{u_{n-2}+u_{n-1}}{2}| = |\frac{u_{n}-u_{n-2}}{2}| = |\frac{\frac{u_{n-2}+u_{n-1}}{2}-u_{n-2}}{2}| = |\frac{u_{n-1}-u_{n-2}}{2^2}|
$$ 

Si $n$ est pair alors 
$$|u_{n+1} - u_{n}| = |\frac{u_1 - u_0}{2^n}| = \frac{|u_1 - u_0|}{2^n}$$ 

si $n$ est impair alors
$$|u_{n+1} - u_{n}| = |\frac{u_2 - u_1}{2^(n-1)}| = |\frac{\frac{u_0+u_1}{2} - u_1}{2^(n-1)}| = |\frac{u_0 - u_1}{2^n}| = \frac{|u_0 - u_1|}{2^n}$$

\subsection*{Exercice 1.2.3}
Preuve par r\'ecurrence. Quand $p=n+2$ on a $u_p = \frac{u_n+u_{n+1}}{2}$ donc $u_p$ est la moyenne entre $u_n$ et $u_{n+1}$ donc c'est compris entre $u_n$ et $u_{n+1}$. Si $p>n+2$ alors $u_p$ compris entre $u_{n}$ et $u_{n+1}$ alors $u_{p+1}$ est compris entre $u_{n}$ et $u_{n+1}$. $u_{p+1} = \frac{u_{p-1} + u_{p}}{2}$, c'est la moyenne entre $u_{p-1}$ et $u_{p}$ mais $u_p$ $u_{p-1}$ compris entre $u_{n}$ et $u_{n+1}$ donc leur moyenne est aussi comptris entre $u_{n}$ et $u_{n+1}$.

\subsection*{Exercice 1.2.4}
Prenons $N$ tel que $\frac{|u_0-u_1|}{2^{N}} < \epsilon$, montrons que $\forall m,n > N, |u_n-u_m| \leq \epsilon$. On sait que $|u_{N+1}-u_N| \leq \epsilon$ et que $u_m$ et $u_n$ sont compris entre $u_{N}$ et $u_{N+1}$ donc $|u_n - u_m| \leq \epsilon$. Donc la suite est de Cauchy, donc elle converge.

\end{document}

