\documentclass[]{book}

%These tell TeX which packages to use.
\usepackage{array,epsfig}
\usepackage{amsmath}
\usepackage{amsfonts}
\usepackage{amssymb}
\usepackage{amsxtra}
\usepackage{amsthm}
\usepackage{mathrsfs}
\usepackage{color}
\usepackage[margin=2cm,top=2.5cm,headheight=16pt,headsep=0.1in,heightrounded]{geometry}
\usepackage{fancyhdr}
\pagestyle{fancy}
\usepackage{tikz}


%Here I define some theorem styles and shortcut commands for symbols I use often
\theoremstyle{definition}
\newtheorem{defn}{Definition}
\newtheorem{thm}{Theorem}
\newtheorem{cor}{Corollary}
\newtheorem*{rmk}{Remark}
\newtheorem{lem}{Lemma}
\newtheorem*{joke}{Joke}
\newtheorem{ex}{Example}
\newtheorem*{soln}{Solution}
\newtheorem{prop}{Proposition}

\newcommand{\lra}{\longrightarrow}
\newcommand{\ra}{\rightarrow}
\newcommand{\surj}{\twoheadrightarrow}
\newcommand{\graph}{\mathrm{graph}}
\newcommand{\bb}[1]{\mathbb{#1}}
\newcommand{\Z}{\bb{Z}}
\newcommand{\Q}{\bb{Q}}
\newcommand{\R}{\bb{R}}
\newcommand{\C}{\bb{C}}
\newcommand{\N}{\bb{N}}
\newcommand{\M}{\mathbf{M}}
\newcommand{\m}{\mathbf{m}}
\newcommand{\MM}{\mathscr{M}}
\newcommand{\HH}{\mathscr{H}}
\newcommand{\Om}{\Omega}
\newcommand{\Ho}{\in\HH(\Om)}
\newcommand{\bd}{\partial}
\newcommand{\del}{\partial}
\newcommand{\bardel}{\overline\partial}
\newcommand{\textdf}[1]{\textbf{\textsf{#1}}\index{#1}}
\newcommand{\img}{\mathrm{img}}
\newcommand{\ip}[2]{\left\langle{#1},{#2}\right\rangle}
\newcommand{\inter}[1]{\mathrm{int}{#1}}
\newcommand{\exter}[1]{\mathrm{ext}{#1}}
\newcommand{\cl}[1]{\mathrm{cl}{#1}}
\newcommand{\ds}{\displaystyle}
\newcommand{\vol}{\mathrm{vol}}
\newcommand{\cnt}{\mathrm{ct}}
\newcommand{\osc}{\mathrm{osc}}
\newcommand{\LL}{\mathbf{L}}
\newcommand{\UU}{\mathbf{U}}
\newcommand{\support}{\mathrm{support}}
\newcommand{\AND}{\;\wedge\;}
\newcommand{\OR}{\;\vee\;} 
\newcommand{\Oset}{\varnothing}
\newcommand{\st}{\ni}
\newcommand{\wh}{\widehat}
\newcommand{\vect}[1]{\overrightarrow{#1}}

%Pagination stuff.
\setlength{\topmargin}{-.3 in}
%\setlength{\oddsidemargin}{0in}
%\setlength{\evensidemargin}{0in}
\setlength{\textheight}{9.in}
\setlength{\textwidth}{6.5in}
\cfoot{page \thepage}
\lhead{MEU301 - Analyse}
\rhead{TD1}
\pagestyle{fancy}


\begin{document}

\subsection*{Rappel de cours}
\begin{defn}
Une suite de r\'eels est dite \emph{Suite de Cauchy} si
$$
\forall \epsilon > 0, \exists N \in \N, \forall m,n \in \N, (n \geq N \text{ et } m \geq N) \implies |u_n - u_m| \leq \epsilon 
$$
\end{defn}

\begin{defn}
Une fonction croissante $f(x)$ admet un point de discontinuit\'e en $x_0$ si 
$$\lim_{x \to x_{0^{+}}}f(x) - \lim_{x \to x_{0^{-}}}f(x) > 0$$
\end{defn}


\begin{defn}
Une fonction $f(x)$ d\'efinit sur une partie $X$ de $\R$ est dite Lipschitzienne de rappoprt $k$ si 
$$
\forall x_1, x_2 \in X, |f(x_2)-f(x_1)| < k.|x_2-x_1|
$$
\end{defn}

\begin{defn}
Une fonction $f(x)$ d\'efinie sur l'intervalle $I$ est continue en un point $a$ si 
$$
\forall \epsilon \in \R^{+}, \exists \eta  \in \R^{+}, \forall x \in I, |x-a| \leq \eta \implies  |f(x)-f(a)| \leq \epsilon
$$
\end{defn}

\begin{defn}
Une fonction $f(x)$ d\'efinie sur l'intervalle $I$ est uniform\'ement continue en un point $a$ si 
$$
\forall \epsilon \in \R^{+}, \exists \eta  \in \R^{+}, \forall x_1, x_2 \in I, |x_2-x_1| \leq \eta \implies  |f(x_2)-f(x_1)| \leq \epsilon
$$
\end{defn}


\begin{defn}
Convergence simple de la s\'erie de fonction $f_n(x)$ vers la fonction $f(x)$.
$$\forall \epsilon > 0, \forall x \geq 0, \exists n_0 \in \N \text{ tel que } \forall n \geq n_0, |f_n(x) - f(x)| \leq \epsilon$$
\end{defn}

\begin{defn}
Convergence uniforme de la s\'erie de fonction $f_n(x)$ vers la fonction $f(x)$.
$$\forall \epsilon > 0, \exists n_0 \in \N  \text{ tel que }  \forall x \geq 0, \forall n \geq n_0, |f_n(x) - f(x)| \leq \epsilon$$
\end{defn}




\newpage
\subsection*{Exercice 1.1}
\subsection*{Exercice 1.1.1}
On a $m > 0$ et $n >0$ donc $\frac{m.n}{(n+m)^2} > 0$, donc 0 est un minorant. 
$$
\frac{m.n}{(n+m)^2} = \frac{1}{\frac{m}{n}+2+\frac{n}{m}}
$$
Il faut montrer que 
$$
\begin{array}{l l}
\frac{m}{n}+\frac{n}{m} \geq 2 \\
\frac{m^2+n^2}{m.n} \geq 2 \\
m^2+n^2-2m.n \geq 0 \\
(m-n)^2 \geq 0 \\
\end{array}
$$
Donc $1/4$ est un majorant.

On a $1/4$ est la borne sup\'erieure de A si il n'existe aucun majorant inf\'erieur \`a $1/4$. On a $1/4 \in A$ pour $m=n=1$. Donc il n'existe pas de plus petit majorant.

On a 0 est la borne inf\'erieure de A si il n'existe aucun minorant sup\'erieur \`a $0$. Quand $n=1$, on a 
$$\lim_{m \to \infty}\frac{m}{(m+1)^2} = \lim_{m \to \infty}\frac{1}{m} = 0$$
Donc il n'exise pas de minorant sup\'erieur \`a 0.

\subsection*{Exercice 1.1.2}
Montrons que 2 est un majorant et 0 un minorant.

On a $\frac{1}{n}+\frac{1}{m}>0$ car $n,m \in \N^{*}$. DOnc 0 est un minorant.

On a $\frac{1}{n}+\frac{1}{m} = \frac{m+n}{n.m}$, montrons que 
$$
\begin{array}{l l}
\frac{1}{m} +  \frac{1}{n} \leq 2 \\
\frac{m+n}{n.m} \leq 2 \\
m+n \leq 2m.n \\
m+n-2m.n \leq 0 \\
m(1-n)+n(1-m) \leq 0 \\
\end{array}
$$
Vrai car $(1-n) \leq 0$, $(1-m) \leq 0$ et $n,m \in \N^{*}$. Donc $2$ est un majorant.

On a 2 est la borne sup\'erieure de A si il n'existe aucun majorant inf\'erieur \`a $2$. On a $2 \in A$ pour $m=n=1$. Donc il n'existe pas de plus petit majorant.

On a 0 est la borne inf\'erieure de A si il n'existe aucun minorant sup\'erieur \`a $0$. On a 
$$\lim_{m \to \infty, n \to \infty,}\frac{1}{m} + \frac{1}{n} = \lim_{m \to \infty, n \to \infty,}\frac{1}{m} + \lim_{m \to \infty, n \to \infty,} \frac{1}{n} = 0$$
Donc il n'exise pas de minorant sup\'erieur \`a 0.


\subsection*{Exercice 1.1.3}
La fonction $f(x)=\frac{x+1}{x+2}$ est strictement croissante pour $x \leq -3$ ($f'(x) = \frac{(x+2)-(x+1)}{(x+2)^2} = \frac{1}{(x+2)^2} > 0$). Donc $f(-3) =  2$ est la borne sup\'erieure de $A$. On a 
$$\lim_{x \to -\infty} \frac{x+1}{x+2} = \lim_{x \to -\infty} \frac{1+1/x}{1+2/x} = 1$$
Donc 1 est la borne inf\'erieure.
Oui la borne sup\'erieure est atteinte pour $x=-3$ mais pas la borne inf\'erieure car c'est une limite.

Maintenant si on prend $x \leq 3$ c'est autre chose car $\sup(A) = \infty$ et $\inf(A) = -\infty$ quand $x \to -2$.

\subsection*{Exercice 1.1.4}
Comme l'ensemble $A$ est born\'e alors il existe $\sup(A)$ et $\inf(A)$. Divisons en 3 cas; $x < y$, $x=y$, et $x > y$.

Pour le cas $x=y$ on a $0 \in A$. Pas tr\`es int\'eressant car $|x-y| \geq 0$. Donc, 0 n'est pas un majorant. 

Pour le cas $x>y$ on a $|x-y| = x - y$. La plus grande valeur possible est quand $x=\sup(A)$ et $y = \inf(A)$ (ie. plus grand \'ecart possible) donc $|\sup(A)-inf(A)|$. \\
Pour le cas $x<y$ on a $|x-y| = y - x$. La plus grande valeur possible est quand $x=\inf(A)$ et $y = \sup(A)$ (ie. plus grand \'ecart possible) donc $|\sup(A)-inf(A)|$. \\


\subsection*{Exercice 1.1.5}
$\sup(|f(x)|) = 2$ car 
$$
\begin{array}{l l}
{]-\infty,-1[} & |f(x)| = 0 \\
{[-1,0[} & |f(x)| = 1 \\
{[0, 1]} & |f(x)| = 1 \\
{]1,2]} & |f(x)| = 2 \\
{]2,\infty[} & |f(x)| = 0 \\
\end{array}
$$

\subsection*{Exercice 1.2}
\subsection*{Exercice 1.2.1}
On voit bien que cela diverge, car $1/n$ diverge. Donc il faut trouver un contre-exemple pour $n$ et $m$.
Prenons pour commencer $m=n+1$ on a $|u_m-u_n| = \frac{1}{n+1}$, pour un $\epsilon$ donn\'e on peut uoujors trouver un $n$ tel que $1/n < \epsilon$. donc pas bon contre-exemple. Il faut \'eliminer les $n$ pour trouver une constante.

Prenons $m=2n$, $|u_m-u_n| = \frac{1}{n+1} + \frac{1}{n+2} + \ldots + \frac{1}{2n} > \frac{1}{2n} + \frac{1}{2n} \ldots + \frac{1}{2n} = \frac{n}{2n} = \frac{1}{2}$. L\`a, c'est mieux. On a, pour $m=2n$, $|u_m-u_n| > \frac{1}{2}$ donc la suite n'est pas de Cauchy (car si on prend $\epsilon = 1/3$, la propri\'et\'e n'est pas v\'erif\'ee pour $m=2n$).

La suite n'est pas de Cauchy et elle est croissante donc elle diverge. Par cons\'equent, $\lim_{n \to \infty} u_n = +\infty$.

\subsection*{Exercice 1.2.2}
$$
\begin{array}{l l}
u_2 & \frac{u_0+u_1}{2} \\
u_3 & \frac{u_1+u_2}{2} = \frac{u0+3u_1}{4}\\
u_4 & \frac{u_2+u_3}{2} = \frac{3u0+5u_1}{8}\\
u_5 & \frac{u_3+u_4}{2} = \frac{5u0+7u_1}{16}\\
\end{array}
$$
Calculons $|u_{n+1} - u_{n}|$
$$
|\frac{u_{n-1}+u_{n}}{2} - \frac{u_{n-2}+u_{n-1}}{2}| = |\frac{u_{n}-u_{n-2}}{2}| = |\frac{\frac{u_{n-2}+u_{n-1}}{2}-u_{n-2}}{2}| = |\frac{u_{n-1}-u_{n-2}}{2^2}|
$$ 

Si $n$ est pair alors 
$$|u_{n+1} - u_{n}| = |\frac{u_1 - u_0}{2^n}| = \frac{|u_1 - u_0|}{2^n}$$ 

si $n$ est impair alors
$$|u_{n+1} - u_{n}| = |\frac{u_2 - u_1}{2^(n-1)}| = |\frac{\frac{u_0+u_1}{2} - u_1}{2^(n-1)}| = |\frac{u_0 - u_1}{2^n}| = \frac{|u_0 - u_1|}{2^n}$$

\subsection*{Exercice 1.2.3}
Preuve par r\'ecurrence. Quand $p=n+2$ on a $u_p = \frac{u_n+u_{n+1}}{2}$ donc $u_p$ est la moyenne entre $u_n$ et $u_{n+1}$ donc c'est compris entre $u_n$ et $u_{n+1}$. Si $p>n+2$ alors $u_p$ compris entre $u_{n}$ et $u_{n+1}$ alors $u_{p+1}$ est compris entre $u_{n}$ et $u_{n+1}$. $u_{p+1} = \frac{u_{p-1} + u_{p}}{2}$, c'est la moyenne entre $u_{p-1}$ et $u_{p}$ mais $u_p$ $u_{p-1}$ compris entre $u_{n}$ et $u_{n+1}$ donc leur moyenne est aussi comptris entre $u_{n}$ et $u_{n+1}$.

\subsection*{Exercice 1.2.4}
Prenons $N$ tel que $\frac{|u_0-u_1|}{2^{N}} < \epsilon$, montrons que $\forall m,n > N, |u_n-u_m| \leq \epsilon$. On sait que $|u_{N+1}-u_N| \leq \epsilon$ et que $u_m$ et $u_n$ sont compris entre $u_{N}$ et $u_{N+1}$ donc $|u_n - u_m| \leq \epsilon$. Donc la suite est de Cauchy, donc elle converge.

\subsection*{Exercice 1.3}
\subsection*{Exercice 1.3.1}
Faux pour une limite finie. La fonction $f(x) = \frac{1}{1-x}$ est croissante sur l'intervalle $[-1,1]$ mais n'admet pas de limite finie \`a gauche et \`a droite en $x=0$.

Mais Vrai pour une limite infinie car les valeurs $a$ et $b$ \'etant donn\'e (ie. pas \'egale \`a l'infini), il existe toujours une limite \`a gauche et \`a droite pour une valeur de l'intervalle $[a,b]$. En effet, si on prend une valeur $x$ dans $]a,b[$, on construit l'ensemble $\{f(y), y \in ]a,x[\}$. Cet ensemble est major\'ee par $f(x)$ car la fonction $f$ est croissante. Donc il existe une borne sup\'erieure $f(x_{-}) \leq f(x)$. Par la d\'efinition d'une borne sup\'erieure, on a $\forall \epsilon>0, \exists y < x, f(x_{-})-\epsilon < f(y) \leq f(x_{-})$. Comme la fonction $f$ est croissante, ceci montre que $\lim_{x \to x^{-}} f(x) = f(x_{-})$. Donc la limite \`a gauche existe. On fait de m\^eme pour la limite \`a droite avec $y \in ]x,b[$ et la borne inf\'erieure $f(x_{+})$.

Il faut qu'il apprenne \`a poser des questions.

\subsection*{Exercice 1.3.2}
Une fonction croissante $f(x)$ admet un point de discontinuit\'e en $x_0$ si 
$$\lim_{x \to x_{0^{+}}}f(x) - \lim_{x \to x_{0^{-}}}f(x) > 0$$
C'est \'a dire que la limite \`a gauche et \`a droite au point $x$ sont diff\'erentes.
La r\'eponse est vraie car tu peux faire d\'efinir une injection des points de discontinuit\'e vers les rationels. Coome l'ensemble des rationnels est d\'enombrable alors il existe un nombre fini de discontinuit\'e.

\subsection*{Exercice 1.3.3}
Supposons que la fonction $f$ est k-lipschitzienne donc 
$$
\forall x_1, x_2 \in X, |f(x_2)-f(x_1)| < k.|x_2-x_1|
$$

Il faut montrer sa continuit\'e.
$$
\forall \epsilon \in \R^{+}, \exists \eta  \in \R^{+}, \forall x \in I, |x-a| < \eta \implies  |f(x)-f(a)| < \epsilon
$$

On part de 
$$
\forall x_1 \in R, |f(x)-f(a)| < k.|x-a|
$$

Prenons $\eta = \frac{\epsilon}{k}$, il faut montrer 
$$
\forall x, |x-a| < \frac{\epsilon}{k} \implies  |f(x)-f(a)| < \epsilon
$$
Mais $f$ est k-lipschitzienne donc
$$
\forall x, |x-a| < \frac{\epsilon}{k} \implies  |f(x)-f(a)| < k.|x-a| < k. \frac{\epsilon}{k} = \epsilon
$$
Vrai.

Il faut montrer sa continuit\'e uniforme.
$$
\forall \epsilon \in \R^{+}, \exists \eta  \in \R^{+}, \forall x_1, x_2 \in I, |x_2-x_1| \leq \eta \implies  |f(x_2)-f(x_1)| \leq \epsilon
$$

Prenons $\eta = \frac{\epsilon}{k}$, il faut montrer: 
$$
\forall x_1, x_2 \in I, |x_2-x_1| \leq \frac{\epsilon}{k} \implies  |f(x_2)-f(x_1)| \leq \epsilon
$$
Mais $f$ est k-lipschitzienne donc
$$
\forall x_1, x_2 \in I, |x_2-x_1| \leq \frac{\epsilon}{k} \implies  |f(x_2)-f(x_1)| \leq k|x_2-x_1| < k\frac{\epsilon}{k} = \epsilon
$$
Vrai aussi.


\subsection*{Exercice 1.3.4}
Faux. La fonction $f(x) = \sqrt{x}$ est uniform\'ement continue mais pas k-lipschitzienne.

1- $f(x)$ est uniform\'ement continue? Il faut montrer que 
$$
\forall \epsilon \in \R^{+}, \exists \eta  \in \R^{+}, \forall x_1, x_2 \in I, |x_2-x_1| \leq \eta \implies  |f(x_2)-f(x_1)| \leq \epsilon
$$
Prenons $\eta \leq \epsilon^2$.
$$
\forall x_1, x_2 \in I, |x_2-x_1| \leq \epsilon^2 \implies  |\sqrt(x_2)-\sqrt(x_1)| \leq \epsilon
$$

Pour $x_1 \leq x_2$, on a $\sqrt(x_2)-\sqrt(x_1) \leq \sqrt{x_2 - x_1}$. Donc
$$
\forall x_1, x_2 \in I, |x_2-x_1| \leq \epsilon^2 \implies  |\sqrt(x_2)-\sqrt(x_1)| \leq \sqrt{x_2 - x_1} < \sqrt{\epsilon^2} = \epsilon
$$
Donc $\sqrt{x}$ est uniform\'ement continue.


2- $f(x)$ n'est pas k-lipschitzienne? Preuve par l'absurde. Supposons que $f(x)$ est k-lipschitzienne alors
$$
\forall x_1, x_2 \in \R^{+}, |\sqrt{x_2} - \sqrt{x_1}| \leq k.|x_2 - x_1|
$$

si $k = 0$, absurde car $\sqrt{x_1} \neq \sqrt{x_2}$. si $k>0$, prenons $x_1=0$ et $x_2=\frac{1}{4k^2}$. Donc $|\sqrt{\frac{1}{4k^2}} - \sqrt{0}| = \frac{1}{2k}$ et $k|\frac{1}{4k^2}-0| = \frac{1}{4k}$. d'o\'u, $\frac{1}{2k} < \frac{1}{4k}$ ce qui est absurde aussi.
donc $\sqrt{x}$ n'est pas k-lipschitzienne.



\subsection*{Exercice 1.4.1.1}
\begin{tikzpicture}[scale=3.0]
\draw[->] (-1.5,0) -- (3.5,0) node[above left]{\footnotesize $x$};
\draw[->] (0,-0.2) -- (0,1.5) node[below right]{\footnotesize $y$};

\draw[blue] plot[variable=\x,domain=-1:0,smooth] ({\x},{0)});
\draw[blue] plot[variable=\x,domain=0:1,smooth] ({\x},{\x)})  node[above] {$n=1$};
\draw[blue] plot[variable=\x,domain=1:2,smooth] ({\x},{2-\x)});
\draw[blue] plot[variable=\x,domain=2:3,smooth] ({\x},{0});

\draw[red] plot[variable=\x,domain=-1:0,smooth] ({\x},{0)});
\draw[red] plot[variable=\x,domain=0:0.5,smooth] ({\x},{2*\x)})  node[above] {$n=2$};
\draw[red] plot[variable=\x,domain=0.5:1,smooth] ({\x},{2-2*\x)});
\draw[red] plot[variable=\x,domain=1:3,smooth] ({\x},{0});

\draw[yellow] plot[variable=\x,domain=-1:0,smooth] ({\x},{0)});
\draw[yellow] plot[variable=\x,domain=0:0.2,smooth] ({\x},{5*\x)})  node[above] {$n=5$};
\draw[yellow] plot[variable=\x,domain=0.2:0.4,smooth] ({\x},{2-5*\x)});
\draw[yellow] plot[variable=\x,domain=0.4:3,smooth] ({\x},{0});

\draw[purple] plot[variable=\x,domain=-1:0,smooth] ({\x},{0)});
\draw[purple] plot[variable=\x,domain=0:0.1,smooth] ({\x},{10*\x)})  node[left] {$n=10$};
\draw[purple] plot[variable=\x,domain=0.1:0.2,smooth] ({\x},{2-10*\x)});
\draw[purple] plot[variable=\x,domain=0.2:3,smooth] ({\x},{0});

\end{tikzpicture}

Convergence simple. Pour un $x$ donn\'e, prenons $\epsilon > 0$ et $n_0 > 2/x$. Dans ce cas, $\forall n \geq n_0, f_n(x) = 0$ donc la s\'erie de fonction converge simplement vers la fonction $f(x) = 0$.

Convergence uniforme. Il n'existe pas de $n_0$ tel que pour tout $x$ la fonction $f_n(x)$ converge. En effet, pour un $n \geq n_0$ donn\'e, on a pour $0 \geq x \geq 1/n$ la fonction qui varie entre $0 < f_n(x) \leq 1$ et pour $1/n < x \leq 2/n$ la fonction qui varie entre $1 \geq f_n(x) \geq 0$. Donc pour un $\epsilon$ donn\'e, on ne peut pas d\'efinir un $n$ qui satisfasse la d\'efinition d'une suite convergente uniform\'ement.   

\subsection*{Exercice 1.4.1.2}

\begin{tikzpicture}[scale=3.0]
\draw[->] (-1.5,0) -- (3.5,0) node[above left]{\footnotesize $x$};
\draw[->] (0,-0.2) -- (0,1.5) node[below right]{\footnotesize $y$};

\draw[blue] plot[variable=\x,domain=0:1,smooth] ({\x},{\x)})  node[left] {$n=1$};
\draw[red] plot[variable=\x,domain=0:1,smooth] ({\x},{\x^2)})  node[right] {$n=2$};
\draw[yellow] plot[variable=\x,domain=0:1,smooth] ({\x},{\x^5)})  node[right] {$n=5$};
\draw[purple] plot[variable=\x,domain=0:1,smooth] ({\x},{\x^10)})  node[right] {$n=10$};

\end{tikzpicture}

Convergence simple. Pour un $x$ et un $\epsilon > 0$ donn\'es, prenons $n_0$ tel que $x^{n_0} > \epsilon$. Dans ce cas, $\forall n \geq n_0, f_n(x) = 0$ donc la s\'erie de fonction converge simplement vers la fonction $f(x) = 0$.


\subsection*{Exercice 1.4.1.3}

\begin{tikzpicture}[scale=3.0]
\draw[->] (-1.5,0) -- (3.5,0) node[above left]{\footnotesize $x$};
\draw[->] (0,-0.2) -- (0,1.5) node[below right]{\footnotesize $y$};

\draw[blue] plot[variable=\x,domain=-1:0,smooth] ({\x},{0)});
\draw[blue] plot[variable=\x,domain=0:1,smooth] ({\x},{1*\x)})  node[above] {$n=1$};
\draw[blue] plot[variable=\x,domain=1:3,smooth] ({\x},{1});

\draw[red] plot[variable=\x,domain=-1:0,smooth] ({\x},{0)});
\draw[red] plot[variable=\x,domain=0:0.5,smooth] ({\x},{2*\x)})  node[above] {$n=2$};
\draw[red] plot[variable=\x,domain=0.5:3,smooth] ({\x},{1});

\draw[yellow] plot[variable=\x,domain=-1:0,smooth] ({\x},{0)});
\draw[yellow] plot[variable=\x,domain=0:0.2,smooth] ({\x},{5*\x)})  node[above] {$n=5$};
\draw[yellow] plot[variable=\x,domain=0.2:3,smooth] ({\x},{1});

\draw[purple] plot[variable=\x,domain=-1:0,smooth] ({\x},{0)});
\draw[purple] plot[variable=\x,domain=0:0.1,smooth] ({\x},{10*\x)})  node[left] {$n=10$};
\draw[purple] plot[variable=\x,domain=0.1:3,smooth] ({\x},{1});

\end{tikzpicture}

Convergence simple. Pour un $x$ donn\'e, prenons  $\epsilon > 0$ et $n_0 > 1/x$. Dans ce cas, $\forall n \geq n_0, f_n(x) = 1$ donc la s\'erie de fonction converge simplement vers la fonction $f(x) = 1$.

Convergence uniforme. Il n'existe pas de $n_0$ tel que pour tout $x$ la fonction $f_n(x)$ converge. En effet, pour un $n \geq n_0$ donn\'e, on a pour $0 \geq x \geq 1/n$ la fonction qui varie entre $0 \geq f_n(x) \geq 1$. Donc pour un $\epsilon$ donn\'e, on ne peut pas d\'efinir un $n$ qui satisfasse la d\'efinition d'une suite convergente uniform\'ement.   

\subsection*{Exercice 1.4.2.1}
$$
f_n(x) = 1_{[-n, n]} = 
\left\{
\begin{array}{l l}
1 & x \in [-n,n] \\
0 & sinon
\end{array}
\right.
$$

Quand $n \to \infty$, on a $f_n(x) = 1$ car $x$ ne peut pas \^etre en dehors de l'ensemble $[-\infty, \infty]$


\subsection*{Exercice 1.4.2.4}
$$
k_n(x) = 1_{[-1/n, 1/n]} = 
\left\{
\begin{array}{l l}
1 & x \in [-1/n,1/n] \\
0 & sinon
\end{array}
\right.
$$

Quand $n \to \infty$, on a $k_n(x) = 0$ car l'ensemble $[-1/\infty, 1/\infty]$ se rapproche de $\emptyset$.

\subsection*{Exercice 1.4.3}
$f_n(x) = \frac{n \sin(x)}{1+n}$. On a $\forall x \in \R, -1 \leq \sin(x) \leq 1$ et $\lim_{n \to \infty} {\frac{n}{1+n}} = 1$. Donc quand $n \to \infty$, la fonction $f_n$ converge simplement vers la fonction $\sin(x)$. Pour un $\epsilon$ et un $x$ donn\'e, cherchons $n_0$ tel que $|f_{n_0}-sin(x)|<\epsilon$. 
$$
|f_{n_0}(0)-\sin(x)| = \left|\frac{n_0}{1+n_0}\sin(x)-\sin(x)\right| = \left|\frac{-\sin(x)}{1+n_0}\right| < \frac{1}{1+n_0}
$$

Prenons, il faut que $\frac{1}{1+n_0} < \epsilon$ donc $n_0 > \frac{1-\epsilon}{\epsilon}$

Comme $n_0$ ne d\'epend pas de $x$ la fonction converge uniform\'enent.

\subsection*{Exercice 1.4.4}
$f_n(x) = x(1-e^{-nx})$. Pour $x \in \R_{+}$, on a $\lim_{n \to \infty}{ e^{-nx}} = 0$, donc $\lim_{n \to \infty}{1 - e^{-nx}} = 1$. V\'erifions que $f_n(x)$ converge vers $f(x)=x$. $|f_{n_0}(x) - x| < \epsilon$
$$
|f_{n_0}(x) - x| = \left|x(1-e^{-n_0x} -x \right| = \left| xe^{-n_0x} \right| 
$$
Pour un $x$ et un $\epsilon$ donn\'e, prenons un $n_0$ tel que $\left| xe^{-n_0x} \right| < \epsilon$.

Pour la convergence uniforme, om ne peut pas trouver de $n_0$ qui ne d\'epende pas de $x$. Donc pas de convergence uniforme.



\end{document}

