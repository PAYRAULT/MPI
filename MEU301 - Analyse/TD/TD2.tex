\documentclass[]{book}

%These tell TeX which packages to use.
\usepackage{array,epsfig}
\usepackage{amsmath}
\usepackage{amsfonts}
\usepackage{amssymb}
\usepackage{amsxtra}
\usepackage{amsthm}
\usepackage{mathrsfs}
\usepackage{color}
\usepackage[margin=2cm,top=2.5cm,headheight=16pt,headsep=0.1in,heightrounded]{geometry}
\usepackage{fancyhdr}
\pagestyle{fancy}
\usepackage{tikz}


%Here I define some theorem styles and shortcut commands for symbols I use often
\theoremstyle{definition}
\newtheorem{defn}{Definition}
\newtheorem{thm}{Theorem}
\newtheorem{cor}{Corollary}
\newtheorem*{rmk}{Remark}
\newtheorem{lem}{Lemma}
\newtheorem*{joke}{Joke}
\newtheorem{ex}{Example}
\newtheorem*{soln}{Solution}
\newtheorem{prop}{Proposition}

\newcommand{\lra}{\longrightarrow}
\newcommand{\ra}{\rightarrow}
\newcommand{\surj}{\twoheadrightarrow}
\newcommand{\graph}{\mathrm{graph}}
\newcommand{\bb}[1]{\mathbb{#1}}
\newcommand{\Z}{\bb{Z}}
\newcommand{\Q}{\bb{Q}}
\newcommand{\R}{\bb{R}}
\newcommand{\C}{\bb{C}}
\newcommand{\N}{\bb{N}}
\newcommand{\M}{\mathbf{M}}
\newcommand{\m}{\mathbf{m}}
\newcommand{\MM}{\mathscr{M}}
\newcommand{\HH}{\mathscr{H}}
\newcommand{\Om}{\Omega}
\newcommand{\Ho}{\in\HH(\Om)}
\newcommand{\bd}{\partial}
\newcommand{\del}{\partial}
\newcommand{\bardel}{\overline\partial}
\newcommand{\textdf}[1]{\textbf{\textsf{#1}}\index{#1}}
\newcommand{\img}{\mathrm{img}}
\newcommand{\ip}[2]{\left\langle{#1},{#2}\right\rangle}
\newcommand{\inter}[1]{\mathrm{int}{#1}}
\newcommand{\exter}[1]{\mathrm{ext}{#1}}
\newcommand{\cl}[1]{\mathrm{cl}{#1}}
\newcommand{\ds}{\displaystyle}
\newcommand{\vol}{\mathrm{vol}}
\newcommand{\cnt}{\mathrm{ct}}
\newcommand{\osc}{\mathrm{osc}}
\newcommand{\LL}{\mathbf{L}}
\newcommand{\UU}{\mathbf{U}}
\newcommand{\support}{\mathrm{support}}
\newcommand{\AND}{\;\wedge\;}
\newcommand{\OR}{\;\vee\;} 
\newcommand{\Oset}{\varnothing}
\newcommand{\st}{\ni}
\newcommand{\wh}{\widehat}
\newcommand{\vect}[1]{\overrightarrow{#1}}

%Pagination stuff.
\setlength{\topmargin}{-.3 in}
%\setlength{\oddsidemargin}{0in}
%\setlength{\evensidemargin}{0in}
\setlength{\textheight}{9.in}
\setlength{\textwidth}{6.5in}
\cfoot{page \thepage}
\lhead{MEU301 - Analyse}
\rhead{TD1}
\pagestyle{fancy}


\begin{document}

\subsection*{Rappel de cours}

\newpage
\subsection*{Exercice 2.1}
\subsection*{Exercice 2.1.1}
$$\int_{0}^{\pi}{x\sin(x) dx}$$
Prenons $f = x$ et $g' = \sin(x)$ donc $f' = 1$ et $g = -\cos(x)$

$$\int{fg'} = fg - \int{f'g} = -x\cos(x) - \int{-\cos(x)} = \sin(x) - x\cos(x)$$
$$\int_{0}^{\pi}{x\sin(x) dx} = \left[ \sin(x) - x\cos(x) \right]_{0}^{\pi} = \pi$$

\subsection*{Exercice 2.1.2}
$$\int_{0}^{\ln(2)}{\frac{e^x}{\sqrt{e^x+1}}dx}$$
Prenons $u=e^x+1$, on a $\frac{du}{dx}=e^x$ donc $dx = e^{-x}du$.
$$\int{\frac{e^x}{\sqrt{e^x+1}}dx} = \int{\frac{e^x}{\sqrt{u}}e^{-x}du} = \int{\frac{1}{\sqrt{u}}du} = 2\sqrt{u} = 2\sqrt{e^x+1}$$

$$\int_{0}^{\ln(2)}{\frac{e^x}{\sqrt{e^x+1}}dx} = \left[ 2\sqrt{e^x+1} \right]_{0}^{\ln(2)} = 2(\sqrt{3} - \sqrt{2})$$


\subsection*{Exercice 2.1.3}
$$\int_{0}^{1}{\frac{4}{x^4-4}dx}$$

Substitution $u=x^4-4$ -> Dead end.

On a $x^4-4 = {x^2}^2 - 2^2 = (x^2-2)(x^2+2)$
$$\frac{1}{(x^2-2)(x^2+2)} = \frac{1}{4}(\frac{1}{x^2-2} - \frac{1}{x^2+2})$$
$$\int_{0}^{1}{\frac{4}{x^4-4}dx} = \int_{0}^{1}{\frac{1}{x^2-2} - \frac{1}{x^2+2}dx} = \int_{0}^{1}{\frac{1}{x^2-2}dx} - \int_{0}^{1}{\frac{1}{x^2+2}dx}$$

Donc
$$
\int_{0}^{1}{\frac{1}{x^2-2}dx} = \int_{0}^{1}{\frac{1}{x^2-\sqrt{2}^2}dx} = \int_{0}^{1}{\frac{1}{(x+\sqrt{2})(x-\sqrt{2})}dx} = \int_{0}^{1}{\frac{1}{2\sqrt{2}}\left(\frac{1}{x+\sqrt{2}} - \frac{1}{x-\sqrt{2}}\right)dx}
$$
Maintenant, prenons $u = x+\sqrt{2}$, $\frac{du}{dx} = 1$ donc $du=dx$ et 
$$
\int{\frac{1}{x+\sqrt{2}}dx} = \int{\frac{1}{u}du} =\ln(u) = \ln(x+\sqrt{2})
$$
de m\^eme avec $u = x-\sqrt{2}$
$$
\int{\frac{1}{x-\sqrt{2}}dx} = \int{\frac{1}{u}du} =\ln(u) = \ln(x-\sqrt{2})
$$
Donc
$$
\int_{0}^{1}{\frac{1}{x^2-2}dx} = \frac{1}{2\sqrt{2}}\left( \left[ \ln(x + \sqrt{2}) \right]_{0}^{1} - \left[ \ln(x - \sqrt{2} ) \right]_{0}^{1} \right)
$$

et pour la seconde partie
$$
\int{\frac{1}{x^2+2}dx}
$$
On connait l'int\'egrale de $\frac{1}{x^2+1}$, c'est presque pareil. Il faut juste \'eliminer le 2 avec un changement de variable judicieux (ou astucieux). Prenons $u = \frac{x}{\sqrt{2}}$, on a $\frac{du}{dx} = \frac{1}{\sqrt{2}}$ donc $dx=\sqrt{2}du$. Ce qui fait:
$$
\int{\frac{1}{x^2+2}dx} = \int{\frac{1}{(\sqrt{2}u)^2+2}\sqrt{2}du} = \int{\frac{\sqrt{2}}{2u^2+2}} = \frac{\sqrt{2}}{2}\int{\frac{1}{u^2+1}} = \frac{\sqrt{2}}{2}\arctan(u) = \frac{\sqrt{2}}{2}\arctan(\frac{x}{\sqrt{2}})
$$
Donc
$$
\int_{0}^{1}{\frac{1}{x^2-2}dx} = \left[ \frac{\sqrt{2}}{2}\arctan \left(\frac{x}{\sqrt{2}}\right) \right]_{0}^{1} = \frac{\sqrt{2}}{2}\arctan\left(\frac{1}{\sqrt{2}}\right)
$$

Pour finir
$$
\int_{0}^{1}{\frac{4}{x^4-4}dx} = bla bla - \frac{\sqrt{2}}{2}\arctan\left(\frac{1}{\sqrt{2}}\right)
$$


\subsection*{Exercice 2.1.4}
$$\int_{-1}^{1}{\ln(1+x^2)dx}$$
Prenons $f = \ln(1+x^2)$ et $g' = 1$ donc $f' = \frac{2x}{x^2+1}$ et $g = x$

$$\int{fg'} = fg - \int{f'g} = x\ln(x^2+1) - \int{\frac{2x^2}{x^2+1} dx}$$

Prenons $u=x^2+1$, on a $\frac{du}{dx} = 2x$ donc $dx = \frac{1}{2x}du$
$$\int{\frac{2x^2}{x^2+1} dx} = \int{\frac{2x^2}{u} \frac{1}{2x}du} = \int{\frac{x}{u}du}$$
Dead end.

On connait l'int\'egrale de $\frac{1}{x^2+1}$ 
$$\int{\frac{2x^2}{x^2+1} dx} = 2\int{\frac{x^2+1-1}{x^2+1} dx} = 2\int{\frac{x^2+1}{x^2+1} - \frac{1}{x^2+1} dx} = 2\int{1 - \frac{1}{x^2+1} dx} =  2\int{1 dx} - 2\int{\frac{1}{x^2+1} dx} = 2(x - \arctan(x))$$

donc
$$
\int_{-1}^{1}{\ln(1+x^2)dx} = \left[ x\ln(x^2+1) + 2(\arctan(x) - x) \right]_{-1}^{1} = \ln(2) +2\pi/4 -2 - (-ln(2) + 2 - 2\pi/4) = 2\ln(2) + \pi - 4
$$

\subsection*{Exercice 2.1.5}
$$\int_{0}^{1}{x^2e^x dx}$$
Prenons $f = x^2$ et $g' = e^x$ donc $f' = 2x$ et $g = e^x$

$$\int{fg'} = fg - \int{f'g} = x^2e^x - \int{2x e^x dx}$$

Prenons $f = 2x$ et $g' = e^x$ donc $f' = 2$ et $g = e^x$
$$\int{fg'} = fg - \int{f'g} = 2xe^x - \int{2 e^x dx} = 2xe^x - 2e^x$$

Donc
$$\int_{-1}^{1}{\ln(1+x^2)dx} = \left[ x^2e^x - 2xe^x + 2e^x \right] = e - 5e^{-1}$$


\subsection*{Exercice 2.2}
\subsection*{Exercice 2.2.1}
$f(x) = \int_{0}^{x^2} {\frac{dt}{1+t+t^2}}$, on a donc $f(x) = F(x^2) - F(0)$ avec $F(x)$ la fonction int\'egrale de $f(x)$. donc
$$
f'(x) = (F(x^2) - F(0))' = (F(x^2))' - (F(0))' = (x^2)'f(x^2) - 0f(0) = \frac{2x}{1+x^2+x^4}
$$


\subsection*{Exercice 2.2.2}
$f(x) = \int_{x}^{x^2} {\frac{dt}{1+t+t^2}}$, on a donc $f(x) = F(x^2) - F(0)$ avec $F(x)$ la fonction int\'egrale de $f(x)$. donc
$$
f'(x) = (F(x^2) - F(x))' = (F(x^2))' - (F(x))' = (x^2)'f(x^2) - f(0) = \frac{2x}{1+x^2+x^4} - 1
$$



\subsection*{Exercice 2.7}
\subsection*{Exercice 2.7.1}

\subsection*{Exercice 1.4.1.1}
\begin{tikzpicture}[scale=1.0]
\draw[->] (-1.5,0) -- (3.5,0) node[above left]{\footnotesize $x$};
\draw[->] (0,-0.2) -- (0,10.5) node[below right]{\footnotesize $y$};

\draw[blue] plot[variable=\x,domain=-1:0,smooth] ({\x},{0)});
\draw[blue] plot[variable=\x,domain=0:1,smooth] ({\x},{\x)})  node[above] {$n=1$};
\draw[blue] plot[variable=\x,domain=1:2,smooth] ({\x},{2-\x)});
\draw[blue] plot[variable=\x,domain=2:3,smooth] ({\x},{0});

\draw[red] plot[variable=\x,domain=-1:0,smooth] ({\x},{0)});
\draw[red] plot[variable=\x,domain=0:0.5,smooth] ({\x},{4*\x)})  node[above] {$n=2$};
\draw[red] plot[variable=\x,domain=0.5:1,smooth] ({\x},{4-4*\x)});
\draw[red] plot[variable=\x,domain=1:3,smooth] ({\x},{0});

\draw[yellow] plot[variable=\x,domain=-1:0,smooth] ({\x},{0)});
\draw[yellow] plot[variable=\x,domain=0:0.33333,smooth] ({\x},{9*\x)})  node[above] {$n=3$};
\draw[yellow] plot[variable=\x,domain=0.33333:0.666666,smooth] ({\x},{6-9*\x)});
\draw[yellow] plot[variable=\x,domain=0.66666:3,smooth] ({\x},{0});

\draw[purple] plot[variable=\x,domain=-1:0,smooth] ({\x},{0)});
\draw[purple] plot[variable=\x,domain=0:0.1,smooth] ({\x},{100*\x)})  node[above] {$n=10$};
\draw[purple] plot[variable=\x,domain=0.1:0.2,smooth] ({\x},{20-100*\x)});
\draw[purple] plot[variable=\x,domain=0.3:3,smooth] ({\x},{0});

\end{tikzpicture}


Convergence simple. Pour un x donn´e, prenons $\epsilon> $0 et $n_0 > 2/x$. Dans ce cas, $\forall n \geq n_0, f_n(x) = 0$ donc
la s´erie de fonction converge simplement vers la fonction $f(x) = 0$.

pour $n=1$, 
$$
\int_{0}^{1}{f_n(x)} = \int_{0}^{1}{x} = \left[ \frac{x^2}{2}\right]_{0}^{1} = \frac{1}{2}
$$

pour $n=2$, 
$$
\int_{0}^{1}{f_n(x)} = \int_{0}^{1/2}{4x} + \int_{1/2}^{1}{4-4x} = \left[ 2x^2 \right]_{0}^{1/2} + \left[ 4x - 2x^2 \right]_{1/2}^{1} = \frac{2}{4} + 4 - 2 - 2 + \frac{2}{4} = 1
$$

pour $n > 3$, 
$$
\int_{0}^{1}{f_n(x)} = \int_{0}^{1/n}{n^2x} + \int_{1/n}^{2/n}{2n-n^2x} + \int_{2/n}^{1}{0}= \left[ \frac{n^2x^2}{2} \right]_{0}^{1/n} + \left[ 2nx - \frac{n^2x^2}{2} \right]_{1/n}^{2/n} + 0 = \frac{1}{2} +4 - 2 - 2 + \frac{1}{2} = 1
$$

On a $\lim_{n \to \infty} f_n(x) = 1$.


\subsection*{Autres exercices 1}
$$x \to e^{-nx^2}$$
Convergence simple. Pour un x donn´e diff\'erent de 0, prenons $\epsilon> $0 et un $n_0$ tel que $\epsilon = e^{-n_0x}$ donc $n_0 = -\frac{\ln(\epsilon)}{x^2}$. Dans ce cas, $\forall n \geq n_0, |f_n(x)| < \epsilon$ donc la s´erie de fonction converge simplement. Lorsque $n \to \infty$ on a la fonction $f(x)$ qui tend vers 0. Pour $x=0$, la s\'erie de fonction converge vers 1.

Convergence uniforme. Pour un $n$ donn\'e et un $x \neq 0$, calculons $|f_n(x) - f(x)| = f_n(x)$. Pour un $n$ donn\'e on pourra toujours trouv\'e un $x$ tel que $e^{-nx^2} > \epsilon$. donc la s\'erie ne converge par uniform\'ement pour $x \neq 0$. Pour $x=0$ calculons $|f_n(x) - f(x)| = e^{-n0} -1 = 0$ donc la fonction converge uniform\'ement pour $x=0$.

\subsection*{Autres exercices 2}
$$x \to \frac{1}{1+nx^2}$$
Convergence simple. Pour un x donn´e diff\'erent de 0, prenons $\epsilon> $0 et et un $n_0$ tel que $\epsilon = \frac{1}{1+n_0x^2}$ donc $n_0 = \frac{1-\epsilon}{\epsilon x^2}$. Dans ce cas, $\forall n \geq n_0, |f_n(x)| < \epsilon$ donc la s´erie de fonction converge simplement. Lorsque $n \to \infty$ on a la fonction $f(x)$ qui tend vers 0. Pour $x=0$, la s\'erie de fonction converge vers 1.

Convergence uniforme. Pour un $n$ donn\'e et un $x \neq 0$, calculons $|f_n(x) - f(x)| = f_n(x)$. Pour un $n$ donn\'e on pourra toujours trouv\'e un $x$ tel que $\frac{1}{1+nx^2} > \epsilon$. donc la s\'erie ne converge par uniform\'ement pour $x \neq 0$. Pour $x=0$ calculons $|f_n(x) - f(x)| = e^{-n0} -1 = 0$ donc la fonction converge uniform\'ement pour $x=0$.

\subsection*{Autres exercices 3}
$$x \to \sin(x)^n \text{ sur } [0,\frac{\pi}{2}]$$
Convergence simple. Pour un x donn´e diff\'erent de $\frac{\pi}{2}$, prenons $\epsilon> 0$ et et un $n_0$ tel que $\epsilon = \sin(x)^n_0$ donc $n_0 = \frac{\ln(\epsilon)}{\ln(\sin(x)}$. Dans ce cas, $\forall n \geq n_0, |f_n(x)| < \epsilon$ donc la s´erie de fonction converge simplement. Lorsque $n \to \infty$ on a la fonction $f(x)$ qui tend vers 0. Pour $x=\frac{\pi}{2}$, la s\'erie de fonction converge vers 1.

Convergence uniforme. Pour un $n$ donn\'e et un $x \neq \frac{\pi}{2}$, calculons $|f_n(x) - f(x)| = f_n(x)$. Pour un $n$ donn\'e on pourra toujours trouv\'e un $x$ tel que $\frac{1}{1+nx^2} > \epsilon$. donc la s\'erie ne converge par uniform\'ement pour $x \neq \frac{\pi}{2}$. Pour $x=\frac{\pi}{2}$ calculons $|f_n(x) - f(x)| = sin(\frac{\pi}{2})^{n} -1 = 0$ donc la fonction converge uniform\'ement pour $x=\frac{\pi}{2}$.


\subsection*{Autres exercices 4}
$$x \to \frac{1}{n}\sin(nx) \text{ sur } [0,\frac{\pi}{2}]$$
Convergence simple. Pour un x donn´e diff\'erent de $\frac{\pi}{2}$, prenons $\epsilon> $0 et et un $n_0$ tel que $\epsilon > \frac{1}{n_0}\sin(n_0x)$ donc $n_0 = \frac{1}{\epsilon}$ car $0 \leq \sin(x) \leq 1$. Dans ce cas, $\forall n \geq n_0, |f_n(x)| < \epsilon$ donc la s´erie de fonction converge simplement. Lorsque $n \to \infty$ on a la fonction $f(x)$ qui tend vers 0. Pour $x=\frac{\pi}{2}$, la s\'erie de fonction converge vers 1.

Convergence uniforme. Pour un $n$ donn\'e et un $x \neq \frac{\pi}{2}$, calculons $|f_n(x) - f(x)| = f_n(x)$. Pour un $n$ donn\'e on pourra toujours trouv\'e un $x$ tel que $\frac{1}{1+nx^2} > \epsilon$. donc la s\'erie ne converge par uniform\'ement pour $x \neq \frac{\pi}{2}$. Pour $x=\frac{\pi}{2}$ calculons $|f_n(x) - f(x)| = sin(\frac{\pi}{2})^{n} -1 = 0$ donc la fonction converge uniform\'ement pour $x=\frac{\pi}{2}$.



\end{document}

