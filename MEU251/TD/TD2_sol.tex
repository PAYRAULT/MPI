\documentclass[]{book}

%These tell TeX which packages to use.
\usepackage{array,epsfig}
\usepackage{amsmath}
\usepackage{amsfonts}
\usepackage{amssymb}
\usepackage{amsxtra}
\usepackage{amsthm}
\usepackage{mathrsfs}
\usepackage{color}

%Here I define some theorem styles and shortcut commands for symbols I use often
\theoremstyle{definition}
\newtheorem{defn}{Definition}
\newtheorem{thm}{Theorem}
\newtheorem{cor}{Corollary}
\newtheorem*{rmk}{Remark}
\newtheorem{lem}{Lemma}
\newtheorem*{joke}{Joke}
\newtheorem{ex}{Example}
\newtheorem*{soln}{Solution}
\newtheorem{prop}{Proposition}

\newcommand{\lra}{\longrightarrow}
\newcommand{\ra}{\rightarrow}
\newcommand{\surj}{\twoheadrightarrow}
\newcommand{\graph}{\mathrm{graph}}
\newcommand{\bb}[1]{\mathbb{#1}}
\newcommand{\Z}{\bb{Z}}
\newcommand{\Q}{\bb{Q}}
\newcommand{\R}{\bb{R}}
\newcommand{\C}{\bb{C}}
\newcommand{\N}{\bb{N}}
\newcommand{\M}{\mathbf{M}}
\newcommand{\m}{\mathbf{m}}
\newcommand{\MM}{\mathscr{M}}
\newcommand{\HH}{\mathscr{H}}
\newcommand{\Om}{\Omega}
\newcommand{\Ho}{\in\HH(\Om)}
\newcommand{\bd}{\partial}
\newcommand{\del}{\partial}
\newcommand{\bardel}{\overline\partial}
\newcommand{\textdf}[1]{\textbf{\textsf{#1}}\index{#1}}
\newcommand{\img}{\mathrm{img}}
\newcommand{\ip}[2]{\left\langle{#1},{#2}\right\rangle}
\newcommand{\inter}[1]{\mathrm{int}{#1}}
\newcommand{\exter}[1]{\mathrm{ext}{#1}}
\newcommand{\cl}[1]{\mathrm{cl}{#1}}
\newcommand{\ds}{\displaystyle}
\newcommand{\vol}{\mathrm{vol}}
\newcommand{\cnt}{\mathrm{ct}}
\newcommand{\osc}{\mathrm{osc}}
\newcommand{\LL}{\mathbf{L}}
\newcommand{\UU}{\mathbf{U}}
\newcommand{\support}{\mathrm{support}}
\newcommand{\AND}{\;\wedge\;}
\newcommand{\OR}{\;\vee\;}
\newcommand{\Oset}{\varnothing}
\newcommand{\st}{\ni}
\newcommand{\wh}{\widehat}

%Pagination stuff.
\setlength{\topmargin}{-.3 in}
\setlength{\oddsidemargin}{0in}
\setlength{\evensidemargin}{0in}
\setlength{\textheight}{9.in}
\setlength{\textwidth}{6.5in}
\pagestyle{empty}



\begin{document}

\subsection*{Rappel de cours}
\begin{defn}
Soit $I$ un intervalle de $\R$,  $(U_n)$ une suite de fonctions d\'efinies sur $I$ et $f$ une fonction d\'efinie sur I.
On dit que $(u_n)$ converge \textbf{simplement} vers $f$ sur $I$ si pour tout $x \in I$, la suite $(U_n(x))$ converge vers $f(x)$.
\end{defn}

\begin{defn}
Soit $I$ un intervalle de $\R$,  $(U_n)$ une suite de fonctions d\'efinies sur $I$ et $f$ une fonction d\'efinie sur I.
On dit que $(u_n)$ converge \textbf{uniform\'ement} vers $f$ sur $I$ si
$$\forall \epsilon > 0, \exists n_0 \in \N, \forall x \in I, \forall n > n_0, |(U_n(x)) - f(x)| < \epsilon$$
\end{defn}

\begin{defn}
\end{defn}


\newpage
\subsection*{Exercice 2}
\subsubsection*{Exercice 2.1}
Il faut trouver une fonction $f(x)$ tel que $\forall x \in I$, la suite $(U_n(x))$ converge vers $f(x)$. On a $\forall n \in \N, (U_n(0)) = 0$, il faut v\'erifier si la suite de $(U_n)$ converge et trouver la fonction de convergence. Comme d'habitude, trouver une borne sup\'erieure qui converge. 
\begin{enumerate}
\item Trouver une suite de fonction $(V_n)$ et exprimer $(V_n)$ en fonction de $(U_n)$. Prendre la fonction $(V_n)$ tel que $V_n = g(x,n)U_n$ et $\frac{1}{g(x,n)}$ converge.
\item V\'erifier que $\lim_{n \to +\infty} V_n(x) = 0$ et que $V_n(x)$ est positive donc il existe un $c > 0$, tel que $\forall n>n_0, 0 \leq (V_n) \leq c$.
\item Remplacer $(V_n)$ dans l'expression, simplifier et regarder si la borne sup\'erieure converge.
\end{enumerate}

Prenons $V_n = U_n.\frac{n^2}{x^2}$, donc $V_n = \frac{n^3}{e^{x\sqrt{n}}}$\\
On a $\lim_{n \to +\infty} V_n(x) = 0$ et $\forall x \in [0,+\infty[, V_n(x) \geq 0$ et prenons $c=1$. Donc
$$0 \leq V_n(x) \leq 1$$
$$0 < U_n(x)\frac{n^2}{x^2} < 1$$
$$0 < U_n(x) < \frac{x^2}{n^2}$$
Comme la borne sup\'erieure converge alors $(U_n)$ converge aussi.


\subsubsection*{Exercice 2.2}
La d\'eriv\'ee de $\left(nx^2.e^{-x\sqrt{n}}\right)' = -x(n^{3/2}x-2n)e^{-x\sqrt{n}}$. La d\'eriv\'ee s'annule en 2 points $x=0$ et $x=\frac{2}{\sqrt{n}}$.
on a $f(0) = 0$ et 
$$f(\frac{2}{\sqrt{n}}) = n.\frac{4}{n}.e^{-\frac{2}{\sqrt{n}}.\sqrt{n}} = 4.e^{-2} = 0.5413$$. 

Donc $sup_{x\in[0,+\infty[}u_n(x) = 4.e^{-2}$, comme le sup est constant, la s\'erie de fonction $u_n$ ne converge pas.


QED

\end{document}

