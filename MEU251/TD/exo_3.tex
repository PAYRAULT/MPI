\documentclass[]{book}

%These tell TeX which packages to use.
\usepackage{array,epsfig}
\usepackage{amsmath}
\usepackage{amsfonts}
\usepackage{amssymb}
\usepackage{amsxtra}
\usepackage{amsthm}
\usepackage{mathrsfs}
\usepackage{color}

%Here I define some theorem styles and shortcut commands for symbols I use often
\theoremstyle{definition}
\newtheorem{defn}{Definition}
\newtheorem{thm}{Theorem}
\newtheorem{cor}{Corollary}
\newtheorem*{rmk}{Remark}
\newtheorem{lem}{Lemma}
\newtheorem*{joke}{Joke}
\newtheorem{ex}{Example}
\newtheorem*{soln}{Solution}
\newtheorem{prop}{Proposition}

\newcommand{\lra}{\longrightarrow}
\newcommand{\ra}{\rightarrow}
\newcommand{\surj}{\twoheadrightarrow}
\newcommand{\graph}{\mathrm{graph}}
\newcommand{\bb}[1]{\mathbb{#1}}
\newcommand{\Z}{\bb{Z}}
\newcommand{\Q}{\bb{Q}}
\newcommand{\R}{\bb{R}}
\newcommand{\C}{\bb{C}}
\newcommand{\N}{\bb{N}}
\newcommand{\M}{\mathbf{M}}
\newcommand{\m}{\mathbf{m}}
\newcommand{\MM}{\mathscr{M}}
\newcommand{\HH}{\mathscr{H}}
\newcommand{\Om}{\Omega}
\newcommand{\Ho}{\in\HH(\Om)}
\newcommand{\bd}{\partial}
\newcommand{\del}{\partial}
\newcommand{\bardel}{\overline\partial}
\newcommand{\textdf}[1]{\textbf{\textsf{#1}}\index{#1}}
\newcommand{\img}{\mathrm{img}}
\newcommand{\ip}[2]{\left\langle{#1},{#2}\right\rangle}
\newcommand{\inter}[1]{\mathrm{int}{#1}}
\newcommand{\exter}[1]{\mathrm{ext}{#1}}
\newcommand{\cl}[1]{\mathrm{cl}{#1}}
\newcommand{\ds}{\displaystyle}
\newcommand{\vol}{\mathrm{vol}}
\newcommand{\cnt}{\mathrm{ct}}
\newcommand{\osc}{\mathrm{osc}}
\newcommand{\LL}{\mathbf{L}}
\newcommand{\UU}{\mathbf{U}}
\newcommand{\support}{\mathrm{support}}
\newcommand{\AND}{\;\wedge\;}
\newcommand{\OR}{\;\vee\;}
\newcommand{\Oset}{\varnothing}
\newcommand{\st}{\ni}
\newcommand{\wh}{\widehat}

%Pagination stuff.
\setlength{\topmargin}{-.3 in}
\setlength{\oddsidemargin}{0in}
\setlength{\evensidemargin}{0in}
\setlength{\textheight}{9.in}
\setlength{\textwidth}{6.5in}
\pagestyle{empty}



\begin{document}


\subsection*{Exercice 3}
\subsubsection*{Exercice 3.1}
Posons $u_n(x) = \sin(\frac{1}{\sqrt{n}})x^n$ et calculons
$$
\lim_{n \to \infty} \left\lvert \frac{u_{n+1}}{u_n} \right\rvert =  \lim_{n \to \infty} \left\lvert \frac{\sin(\frac{1}{\sqrt{n+1}})x^{n+1}}{\sin(\frac{1}{\sqrt{n}})x^n} \right\rvert = \lim_{n \to \infty} \lvert x \rvert
$$
On en d\'eduit que la s\'erie est convergente pour $\lvert x \rvert < 1$, donc son rayon de convergence vaut 1.

\subsubsection*{Exercice 3.2}
Le rayon de convergence est \'egale \`a 1, donc la s\'erie enti\`ere convergence sur le domaine $]-1, 1[$. Il faut maintenant v\'erifier que la s\'erie enti\`ere converge pour $x = -1$. La s\'erie enti\`ere s\'ecrit donc $S(-1) = \sum_{n \ge 0} \sin(\frac{1}{\sqrt{n}})(-1)^n$.\\
C'est une s\'erie altern\'ee qui v\'erifie le crit\`ere CSSA car quand $n$ tend vers l'infini, la fonction $\\sin(\frac{1}{\sqrt{n}})$ est positive et d\'ecroissante. Donc la s\'erie enti\`ere $S(-1)$ converge. Par cons\'equent, la s\'erie enti\`ere $S(x)$ converge sur $[-1] \cup ]-1, 1[ = [-1, 1[$.

\subsubsection*{Exercice 3.3}
Une s\'erie enti\`ere $S(x)$ convergence normalement sur $[-1,1[$, ssi il existe une suite de r\'eels positifs 	$(a_n)_{n \in \N}$ telle que $\forall n \in N, \forall x \in [-1,1[$, on a $\lvert u_n(x) \rvert 	\le a_n$ et que la s\'erie num\'erique $\sum_{n \ge 0}a_n$ converge. Sur le domaine $[-1,1[$, on a $\sin(\frac{1}{\sqrt{n}})x^n < \sin(\frac{1}{\sqrt{n}})$. Il faut maintenant v\'erifier que la suite num\'erique $\sum_{n \ge 0}{\sin(\frac{1}{\sqrt{n}})}$ converge. \\
En prenant le d\'eveloppement limit\'e de $\sin(x)$ en 0 on a, $\sin(\frac{1}{\sqrt{n}}) = \frac{1}{\sqrt{n}} + O_2(x)$. Mais $\frac{1}{\sqrt{n}} > \frac{1}{n}$ et la s\'erie $\frac{1}{n}$ diverge. Donc la s\'erie enti\`ere $S(x)$ ne converge pas normalement sur $[-1,1[$.

\subsubsection*{Exercice 3.4}
La s\'erie $S(x)$  converge uniform\'ement sur $[-1,a]$ avec $a \in [0,1[$ vers une fonction $s(x)$ ssi $\forall \epsilon > 0, \exists n_{\epsilon} \in \N,  \forall n \in \N, (n > n_{\epsilon} \implies \forall x \in [-1,a], \lvert u_n(x) - s(x) \rvert \le \epsilon)$. \\

D'apr\`es le th\'eor\`eme du reste , la s\'erie $S(x)$  converge uniform\'ement sur $[-1,a]$ avec $a \in [0,1[$ si 
$$
\lim_{n \to \infty}\sup_{x \in [-1,a]} \left \lvert \sum_{k = n+1}^{\infty} {u_n(x)} \right \rvert = 0
$$	

On divise en 2 parties:
\begin{itemize}
\item convergence uniforme de $S(x)$ sur le domaine $[-1,0]$. Prenons $s(x) = 0$, $\forall \epsilon > 0, \exists n_{\epsilon} \in \N,  \forall n \in \N, (n > n_{\epsilon} \implies \forall x \in [-1,0], \lvert u_n(x) - 0 \rvert \le \epsilon)$. On a $\lvert u_n(x) \rvert \le |\sin(\frac{1}{\sqrt{n}})| $, donc cherchons $n_{\epsilon}$ tel que $\forall \epsilon > 0, \forall n \in \N, (n > n_{\epsilon} \implies \forall x \in [-1,0], \lvert \sin(\frac{1}{\sqrt{n}}) \rvert \le \epsilon)$, en prenant $n_{\epsilon} > \frac{1}{\arcsin^2(\epsilon)}$, la propri\'et\'e est v\'erifi\'ee. 
\item convergence uniforme de $S(x)$ sur le domaine $[0,a]$ pour $a \in [0,1[$. On a sur $[0,a]$, $\lvert u_x(x) \rvert \le x^n \le  a^n $, donc $\sup_{x \in [0,a]} \left \lvert \sum_{k = n+1}^{\infty} {u_n(x)} \right \rvert \le \frac{a^{n+1}}{1-a}$ et $\lim{n \to \infty}{\frac{a^{n+1}}{1-a}} = 0$ pour $a \in [0.1[$. La s\'erie converge uniform\'ement sur $[0,a]$ avec $a \in [0,1[$.
\end{itemize}

Donc la s\'eire $S(x)$ converge uniform\'ement.

QED

\end{document}

