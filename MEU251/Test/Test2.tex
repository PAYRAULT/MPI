\documentclass[]{book}

%These tell TeX which packages to use.
\usepackage{array,epsfig}
\usepackage{amsmath}
\usepackage{amsfonts}
\usepackage{amssymb}
\usepackage{amsxtra}
\usepackage{amsthm}
\usepackage{mathrsfs}
\usepackage{color}

%Here I define some theorem styles and shortcut commands for symbols I use often
\theoremstyle{definition}
\newtheorem{defn}{Definition}
\newtheorem{thm}{Theorem}
\newtheorem{cor}{Corollary}
\newtheorem*{rmk}{Remark}
\newtheorem{lem}{Lemma}
\newtheorem*{joke}{Joke}
\newtheorem{ex}{Example}
\newtheorem*{soln}{Solution}
\newtheorem{prop}{Proposition}

\newcommand{\lra}{\longrightarrow}
\newcommand{\ra}{\rightarrow}
\newcommand{\surj}{\twoheadrightarrow}
\newcommand{\graph}{\mathrm{graph}}
\newcommand{\bb}[1]{\mathbb{#1}}
\newcommand{\Z}{\bb{Z}}
\newcommand{\Q}{\bb{Q}}
\newcommand{\R}{\bb{R}}
\newcommand{\C}{\bb{C}}
\newcommand{\N}{\bb{N}}
\newcommand{\M}{\mathbf{M}}
\newcommand{\m}{\mathbf{m}}
\newcommand{\MM}{\mathscr{M}}
\newcommand{\HH}{\mathscr{H}}
\newcommand{\Om}{\Omega}
\newcommand{\Ho}{\in\HH(\Om)}
\newcommand{\bd}{\partial}
\newcommand{\del}{\partial}
\newcommand{\bardel}{\overline\partial}
\newcommand{\textdf}[1]{\textbf{\textsf{#1}}\index{#1}}
\newcommand{\img}{\mathrm{img}}
\newcommand{\ip}[2]{\left\langle{#1},{#2}\right\rangle}
\newcommand{\inter}[1]{\mathrm{int}{#1}}
\newcommand{\exter}[1]{\mathrm{ext}{#1}}
\newcommand{\cl}[1]{\mathrm{cl}{#1}}
\newcommand{\ds}{\displaystyle}
\newcommand{\vol}{\mathrm{vol}}
\newcommand{\cnt}{\mathrm{ct}}
\newcommand{\osc}{\mathrm{osc}}
\newcommand{\LL}{\mathbf{L}}
\newcommand{\UU}{\mathbf{U}}
\newcommand{\support}{\mathrm{support}}
\newcommand{\AND}{\;\wedge\;}
\newcommand{\OR}{\;\vee\;}
\newcommand{\Oset}{\varnothing}
\newcommand{\st}{\ni}
\newcommand{\wh}{\widehat}

%Pagination stuff.
\setlength{\topmargin}{-.3 in}
\setlength{\oddsidemargin}{0in}
\setlength{\evensidemargin}{0in}
\setlength{\textheight}{9.in}
\setlength{\textwidth}{6.5in}
\pagestyle{empty}



\begin{document}

\subsection*{Rappel de cours}
\begin{defn}
Soit $I$ un intervalle de $\R$,  $(U_n)$ une suite de fonctions d\'efinies sur $I$ et $f$ une fonction d\'efinie sur I.
On dit que $(u_n)$ converge \textbf{simplement} vers $f$ sur $I$ si pour tout $x \in I$, la suite $(U_n(x))$ converge vers $f(x)$.
\end{defn}

\begin{defn}
Soit $I$ un intervalle de $\R$,  $(U_n)$ une suite de fonctions d\'efinies sur $I$ et $f$ une fonction d\'efinie sur I.
On dit que $(u_n)$ converge \textbf{uniform\'ement} vers $f$ sur $I$ si
$$\forall \epsilon > 0, \exists n_0 \in \N, \forall x \in I, \forall n > n_0, |(U_n(x)) - f(x)| < \epsilon$$
\end{defn}

\begin{defn}
On dit que la s\'erie de fonctions $\sum_{n\geq0}U_n(x)$ converge \textbf{normalement} sur  $I$ si la s\'erie $\sum_{n\geq0}||U_n(x)||_{\infty}$ est convergente.
 \end{defn}

\begin{defn}
On dit que la s\'erie de fonctions $\sum_{n\geq0}U_n(x)$ converge \textbf{normalement} sur  $I$ si la s\'erie $\sum_{n\geq0}||U_n(x)||_{\infty}$ est convergente.
\end{defn}


\newpage
\subsection*{Exercice 1}
\subsection*{Exercice 2}
\subsection*{Exercice 3}
\subsection*{Exercice 4}
\subsubsection*{Exercice 4.1}
$$
\frac{a}{(x-2)^2} + \frac{b}{(x-2)} + \frac{c}{(x-1)} = \frac{a(x-1)}{(x-2)^2(x-1)} + \frac{b(x-2)(x-1)}{(x-2)^2(x-1)} + \frac{c(x-2)^2}{(x-2)^2(x-1)} =
\frac{a(x-1)+b(x-2)(x-1)+c(x-2)^2}{(x-2)^2(x-1)}
$$

$$
= \frac{ax-a+bx^2-bx-2bx+2b +cx^2-c4x+4c}{(x-2)^2(x-1)}
$$

Donc
$$ 
\left\{
\begin{array}{l l}
b + c = 0 & x^2\\
a -3b -4c = 0 & x\\
-a +2b +4c = 1 & x^0\\
\end{array}
\right.
$$

$$ 
\left\{
\begin{array}{l}
b = -c  \\
a = c   \\
c = 1 \\
\end{array}
\right.
$$

Donc $a=1, b=-1, c=1$.

$$
\frac{1}{(x-2)^2(x-1)} = \frac{1}{(x-2)^2} - \frac{1}{(x-2)} + \frac{1}{(x-1)}
$$

\subsubsection*{Exercice 4.2}
Utilisons le d\'eveloppement de Taylor sur $f_3(x) = \frac{1}{(x-1)}$ au voisinage de 0. On a 
$$\frac{f_3(0)}{0!} = -1$$
$$f'_3(x) = -\frac{1}{(x-1)^2}, \frac{f'_3(0)}{1!} = -1$$
$$f^{(2)}_3(x) = \frac{2}{(x-1)^3},\ \frac{f^{(2)}_3(0)}{2!} = -1$$
$$f^{(3)}_3(x) = \frac{6}{(x-1)^4},\ \frac{f^{(3)}_3(0)}{3!} = -1$$ 

Donc, au voisinage de 0
$$
\frac{1}{(x-1)} = \sum_{n \ge 0} -x^n
$$

Utilisons le d\'eveloppement de Taylor sur $f_2(x) = -\frac{1}{(x-2)}$ au voisinage de 0. On a 
$$\frac{f_2(0)}{0!} = \frac{1}{2}$$
$$f'_2(x) = \frac{1}{(x-2)^2}, \frac{f'_3(0)}{1!} = \frac{1}{4}$$
$$f^{(2)}_2(x) = -\frac{2}{(x-2)^3},\ \frac{f^{(2)}_3(0)}{2!} = \frac{1}{8}$$
$$f^{(3)}_2(x) = \frac{6}{(x-2)^4},\ \frac{f^{(3)}_3(0)}{3!} = \frac{1}{16}$$ 

Donc, au voisinage de 0
$$
-\frac{1}{(x-2)} = \sum_{n \ge 0} \frac{1}{2^{n+1}}x^n
$$

Utilisons le d\'eveloppement de Taylor sur $f_1(x) = \frac{1}{(x-2)^2}$ au voisinage de 0. On a 
$$\frac{f_2(0)}{0!} = \frac{1}{4}$$
$$f'_2(x) = -\frac{2}{(x-2)^3}, \frac{f'_3(0)}{1!} = \frac{2}{8} = \frac{1}{4}$$
$$f^{(2)}_2(x) = \frac{6}{(x-2)^4},\ \frac{f^{(2)}_3(0)}{2!} = \frac{3}{16}$$
$$f^{(3)}_2(x) = -\frac{24}{(x-2)^5},\ \frac{f^{(3)}_3(0)}{3!} = \frac{4}{32}$$ 

Donc, au voisinage de 0
$$
\frac{1}{(x-2)^2} = \sum_{n \ge 0} \frac{n+1}{2^{n+2}}x^n
$$

Donc 
$$
f(x) = \frac{1}{(x-2)^2} -\frac{1}{(x-2)} + \frac{1}{(x-1)} = \sum_{n \ge 0} \frac{n+1}{2^{n+2}}x^n + \frac{1}{2^{n+1}}x^n -x^n = \sum_{n \ge 0} \frac{n+3-2^{n+2}}{2^{n+2}}x^n
$$ 


Le rayon de convergence de $f_3(x)$ est 
$$R_3 = \lim_{n \to \infty} \frac{a_{n+1}}{a_n} = \lim_{n \to \infty} \frac{-1}{-1} = 1$$

Le rayon de convergence de $f_2(x)$ est 
$$R_2 = \lim_{n \to \infty} \frac{a_{n+1}}{a_n} = \lim_{n \to \infty} \frac{\frac{1}{2^{n+2}}}{\frac{1}{2^{n+1}}} = \frac{1}{2}$$

Le rayon de convergence de $f_1(x)$ est 
$$R_1 = \lim_{n \to \infty} \frac{a_{n+1}}{a_n} = \lim_{n \to \infty} \frac{\frac{n+2}{2^{n+3}}}{\frac{n+1}{2^{n+2}}} = \lim_{n \to \infty} \frac{n+2}{2(n+1)} = \lim_{n \to \infty} \frac{1+\frac{2}{n}}{2(1+\frac{1}{n})} = \frac{1}{2}$$

Le rayon de converge de $f(x)$ au voisinage de 0 est $min(R_1, R_2, R_3) = \frac{1}{2}$.

\subsection*{Exercice 5}


QED

\end{document}

