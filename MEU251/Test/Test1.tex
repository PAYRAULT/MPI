\documentclass[]{book}

%These tell TeX which packages to use.
\usepackage{array,epsfig}
\usepackage{amsmath}
\usepackage{amsfonts}
\usepackage{amssymb}
\usepackage{amsxtra}
\usepackage{amsthm}
\usepackage{mathrsfs}
\usepackage{color}

%Here I define some theorem styles and shortcut commands for symbols I use often
\theoremstyle{definition}
\newtheorem{defn}{Definition}
\newtheorem{thm}{Theorem}
\newtheorem{cor}{Corollary}
\newtheorem*{rmk}{Remark}
\newtheorem{lem}{Lemma}
\newtheorem*{joke}{Joke}
\newtheorem{ex}{Example}
\newtheorem*{soln}{Solution}
\newtheorem{prop}{Proposition}

\newcommand{\lra}{\longrightarrow}
\newcommand{\ra}{\rightarrow}
\newcommand{\surj}{\twoheadrightarrow}
\newcommand{\graph}{\mathrm{graph}}
\newcommand{\bb}[1]{\mathbb{#1}}
\newcommand{\Z}{\bb{Z}}
\newcommand{\Q}{\bb{Q}}
\newcommand{\R}{\bb{R}}
\newcommand{\C}{\bb{C}}
\newcommand{\N}{\bb{N}}
\newcommand{\M}{\mathbf{M}}
\newcommand{\m}{\mathbf{m}}
\newcommand{\MM}{\mathscr{M}}
\newcommand{\HH}{\mathscr{H}}
\newcommand{\Om}{\Omega}
\newcommand{\Ho}{\in\HH(\Om)}
\newcommand{\bd}{\partial}
\newcommand{\del}{\partial}
\newcommand{\bardel}{\overline\partial}
\newcommand{\textdf}[1]{\textbf{\textsf{#1}}\index{#1}}
\newcommand{\img}{\mathrm{img}}
\newcommand{\ip}[2]{\left\langle{#1},{#2}\right\rangle}
\newcommand{\inter}[1]{\mathrm{int}{#1}}
\newcommand{\exter}[1]{\mathrm{ext}{#1}}
\newcommand{\cl}[1]{\mathrm{cl}{#1}}
\newcommand{\ds}{\displaystyle}
\newcommand{\vol}{\mathrm{vol}}
\newcommand{\cnt}{\mathrm{ct}}
\newcommand{\osc}{\mathrm{osc}}
\newcommand{\LL}{\mathbf{L}}
\newcommand{\UU}{\mathbf{U}}
\newcommand{\support}{\mathrm{support}}
\newcommand{\AND}{\;\wedge\;}
\newcommand{\OR}{\;\vee\;}
\newcommand{\Oset}{\varnothing}
\newcommand{\st}{\ni}
\newcommand{\wh}{\widehat}

%Pagination stuff.
\setlength{\topmargin}{-.3 in}
\setlength{\oddsidemargin}{0in}
\setlength{\evensidemargin}{0in}
\setlength{\textheight}{9.in}
\setlength{\textwidth}{6.5in}
\pagestyle{empty}



\begin{document}

\subsection*{Rappel de cours}
\begin{defn}
Soit $I$ un intervalle de $\R$,  $(U_n)$ une suite de fonctions d\'efinies sur $I$ et $f$ une fonction d\'efinie sur I.
On dit que $(u_n)$ converge \textbf{simplement} vers $f$ sur $I$ si pour tout $x \in I$, la suite $(U_n(x))$ converge vers $f(x)$.
\end{defn}

\begin{defn}
Soit $I$ un intervalle de $\R$,  $(U_n)$ une suite de fonctions d\'efinies sur $I$ et $f$ une fonction d\'efinie sur I.
On dit que $(u_n)$ converge \textbf{uniform\'ement} vers $f$ sur $I$ si
$$\forall \epsilon > 0, \exists n_0 \in \N, \forall x \in I, \forall n > n_0, |(U_n(x)) - f(x)| < \epsilon$$
\end{defn}

\begin{defn}
On dit que la s\'erie de fonctions $\sum_{n\geq0}U_n(x)$ converge \textbf{normalement} sur  $I$ si la s\'erie $\sum_{n\geq0}||U_n(x)||_{\infty}$ est convergente.
 \end{defn}

\begin{defn}
On dit que la s\'erie de fonctions $\sum_{n\geq0}U_n(x)$ converge \textbf{normalement} sur  $I$ si la s\'erie $\sum_{n\geq0}||U_n(x)||_{\infty}$ est convergente.
\end{defn}


\newpage
\subsection*{Exercice 1}
Montrons que la fonction $f_n(x)=\frac{ne^{-x}+x^2}{n+x}$ converge simplement pour $x \in [0,1]$.
$$\lim_{n \to \infty}\frac{ne^{-x}+x^2}{n+x} = \lim_{n \to \infty}\frac{ne^{-x}+1}{n+1} = \lim_{n \to \infty}\frac{ne^{-x}}{n} = e^{-x}$$

La fonction $f_n(x)$ converge simplement vers $f(x) = e^{-x}$. Montrons maintenant que $f_n(x)$ converge uniform\'ement. Il faut que $\forall \epsilon > 0, \exists n_{\epsilon} \in \N, \forall n \in \N, (n \geq n_{\epsilon} \implies \sup_{x \in [0,1]} |f_n(x)-f(x)| < \epsilon)$.
Calculons 
$$|f_n(x)-f(x)| = \left|\frac{ne^{-x}+x^2}{n+x} - e^{-x}\right| = \left|\frac{ne^{-x}+x^2-(n+x)e^{-x}}{n+x}\right| = \left|\frac{x^2-xe^{-x}}{n+x}\right| < \frac{2}{n}$$
car $n+x \geq n$ et $|x^2-xe^{-x}| < 2$ comme $x^2 \in [0,1]$ et $xe^{-x} \in [0, \frac{1}{e}]$.\\ 
Prenons $\sup_{x \in [0,1]} |f_n(x)-f(x)| = \frac{2}{n}$, donc $n_{\epsilon}$ existe et doit e\^tre sup\'erieur \`a $\frac{2}{\epsilon}$.


\subsection*{Exercice 2}
\subsection*{Exercice 2.1}
Montrons que la fonction $f_n(x)=\ln\left(x+\frac{1}{n}\right)$ converge simplement pour $x \in ]0,+\infty]$.
$$\lim_{n \to \infty}\ln\left(x+\frac{1}{n}\right) = \ln\left(x+0\right) = \ln(x)$$

La fonction $f_n(x)$ converge simplement vers $f(x) = \ln(x)$. Montrons maintenant que $f_n(x)$ converge uniform\'ement sur $[a,+\infty]$. Il faut que $\forall \epsilon > 0, \exists n_{\epsilon} \in \N, \forall n \in \N, (n \geq n_{\epsilon} \implies \sup_{x \in [a,+\infty]} |f_n(x)-f(x)| < \epsilon)$.
Calculons 
$$|f_n(x)-f(x)| = \left|\ln\left(x+\frac{1}{n}\right) - \ln(x)\right| = \left|\ln\left(\frac{x+\frac{1}{n}}{x}\right)\right| = \left|\ln\left(1 + \frac{1}{nx}\right)\right| \leq \ln\left(1 + \frac{1}{na}\right)$$
car $\frac{1}{na} \geq \frac{1}{nx}$ pour $x \in [a,+\infty]$.\\ 
Prenons $\sup_{x \in [a,+\infty]} |f_n(x)-f(x)| = \ln\left(1 + \frac{1}{na}\right)$, donc $n_{\epsilon}$ existe et doit e\^tre tel que $\ln\left(1 + \frac{1}{n_{\epsilon}a}\right) < \epsilon$.

\subsection*{Exercice 2.2}
Non, car pour $x$ proche de 0, il n'existe pas de borne sup\'erieure pour $|f_n(x)-f(x)| = \left|\ln\left(1 + \frac{1}{nx}\right)\right|$.

\subsection*{Exercice 3}
\subsection*{Exercice 3.1}
Quand $x=0$, on a $f_n(x) = 0$ pour tout $n \geq 1$.\\
Pour $x>0$, on a 
$$\lim_{n \to \infty} xe^{-nx} = \lim_{n \to \infty} \frac{x}{e^{nx}} = 0$$
On en d\'eduit que la fonction $f_n(x)$ converge simplement vers la fonction 
$$f(x) = 
\left\{
\begin{array}{l l}
0 & x = 0\\
0 & x \in ]0, +\infty]\\
\end{array}
\right. 
= 0
$$

\subsection*{Exercice 3.2}
Pour que la fonction $f_n(x)$ converge uniform\'ement sur $[0,+\infty]$. Il faut que $\forall \epsilon > 0, \exists n_{\epsilon} \in \N, \forall n \in \N, (n \geq n_{\epsilon} \implies \sup_{x \in [0,+\infty]} |f_n(x)-0| < \epsilon)$. \\

Cela revient \`a trouver si la fonction $f_n(x)$ a un maximum pour $x \in [0,+\infty]$.\\
La fonction est continue et positive sur $[0,\infty]$, cherchons si il existe un maximum.\\

Calculons $f_{n}'(x) = e^{-x}(1-nx)$. on a $f_{n}'(x)$ qui s'annule lorsque $x = \frac{1}{n}$.\\

La fonction $f_{n}(x)$ croit entre $[0,\frac{1}{n}[$ jusqu'\`a la valeur $f_n(\frac{1}{n}) = \frac{1}{ne}$ et d\'ecroit ensuite pour $x \in ]\frac{1}{n}, \infty]$. Donc il existe une valeur sup\'erieure $\frac{1}{ne}$. \\

On a $\sup_{x \in [0,+\infty]}|f_n(x)| = \frac{1}{ne}$, donc il $n_{\epsilon}$  existe et doit \^etre sup\'erieure \`a $\frac{1}{e.\epsilon}$. Par cons\'equent, la fonction $f_n(x)$ converge uniform\'ement vers la fonction $0$. 



\subsection*{Exercice 4}
Pour $x \in [-r,r]$ avec $r > 0$ on a $0 \leq f_n(x) \leq \ln\left(1+\frac{r^2}{n^2}\right)$. On va montrer que 
$$\ln\left(1+\frac{r^2}{n^2}\right) \leq \frac{r^2}{n^2}$$
$$1+\frac{r^2}{n^2} \leq e^{\frac{r^2}{n^2}} = 1 + \frac{r^2}{n^2} + \frac{r^4}{2n^4} + O(\frac{r^2}{n^2})$$
$$0 \leq \frac{r^4}{2n^4} + O(\frac{r^2}{n^2})$$

Donc, pour $x \in [-r,r]$ avec $r > 0$ on a $0 \leq f_n(x) \leq \ln\left(1+\frac{r^2}{n^2}\right) \leq \frac{r^2}{n^2}$. Comme la s\'erie de fonctions $\frac{r^2}{n^2}$ converge normalement, on en d\'eduit par le crit\`ere d'\'equivalence que la s\'erie de fonctions $f_n(x) = \ln(1+\frac{x^2}{n^2})$ converge normalement pour tout $x \in [-r, r]$.\\

Les fonctions $f_n$ sont continues sur $x \in [-r,r]$ (assemblage de fonctions continues) et la s\'erie de fonctions $S = \sum_{n \geq 1}{f_n(x)}$ converge normalement pour $x \in [-r,r]$. On en d\'eduit que $S$ est continue pour $x \in [-r,r]$. 


\subsection*{Exercice 5}
\subsection*{Exercice 5.1}
On va montrer que 
$$\frac{x}{n(n+x)} < \frac{x}{n^2}$$
$$0 < \frac{x}{n^2} - \frac{x}{n(n+x)}$$
$$0 < \frac{x^2}{n^2(n+x)}$$

Qui est vrai, donc $f_n(x) < \frac{x}{n^2}$. On sait que la s\'erie de fonctions $\frac{x}{n^2}$ converge, donc par le crit\`ere d'\'equivalence on a la s\'erie de fonctions $f_n(x)$ qui converge.

\subsection*{Exercice 5.2}
\begin{itemize}
\item (H1) - fonction $f_n(x)$ d\'erivable. Oui. $f_n'(x) = \frac{1}{(n+x)^2}$.
\item (H2) - la s\'erie de fonctions $\sum {f_n}$ converge simplement sur une fonction S sur $[0,+\infty]$. Oui voir 5.1
\item (H3) - la s\'erie de fonctions $\sum {f_n'}$ converge uniform\'ement sur une fonction g sur $[0,+\infty]$.
\end{itemize}

Montrons (H3). on a $f_n'(x) = \frac{1}{(n+x)^2} < \frac{1}{n^2}$ et la s\'erie de fonctions $\frac{1}{n^2}$ converge uniform\'ement sur $\R^+$. Donc (H3) est v\'erifi\'ee.\\

Donc la s\'erie de fonctions $f_n$ est de classe $C^1$.

\subsection*{Exercice 6}
\subsection*{Exercice 6.1}
On si la $(fn)n_{\in \N}$ une suite de fonctions continues sur l'intervalle $[a,b]$ et que la s\'erie $f_n$ converge uniform\'ement sur $[a, b]$ vers sa somme $S$ qui est continue, alors
$$\int_{a}^{b} \sum_{n=0}^{+\infty}f_n(t) dt =  \sum_{n=0}^{+\infty} \int_{a}^{b} f_n(t) dt $$ 

Montrons que la s\'erie de fonctions $f_n(t)=\frac{t^{n-1}}{n}$ est continue pour $t \in ]0,\frac{1}{2}[$. 
\begin{itemize}
\item les fonction $f_n$ sont continu sur $]0,\frac{1}{2}[$
\item on a $\frac{t^{n-1}}{n} < t^{n-1}$ et $\sum_{n \geq 1}t^{n-1} = \frac{1}{1-x}$ lorsque $x \in [0,1[$, donc la s\'erie de fonction $f_n$ converge uniform\'ement sur $]0,\frac{1}{2}[$
\end{itemize}
Donc la s\'erie de fonction $f_n$ est continue.\\

Comme la la s\'erie de fonction est continue, 
$$\int_{x}^{1-x}{\frac{\ln(1-t)}{t}dt} = \int_{x}^{1-x}{\frac{-\sum_{n=1}^{+\infty}{\frac{t^n}{n}}}{t}dt} = \int_{x}^{1-x}{-\sum_{n=1}^{+\infty}{\frac{t^{n-1}}{n}}dt} = -\sum_{n=1}^{+\infty}{\int_{x}^{1-x}{\frac{t^{n-1}}{n}dt}} $$ 
$$ = -\sum_{n=1}^{+\infty}{\left[ \frac{t^n}{n^2} \right]_{x}^{1-x}} = -\sum_{n=1}^{+\infty}{\frac{x^n}{n^2}  - \frac{(1-x)^n}{n^2}} = -\sum_{n=1}^{+\infty}{\frac{1}{n^2} (x^n-(1-x)^n)} = \sum_{n=1}^{+\infty}{\frac{1}{n^2} ((1-x)^n-x^n)} = \sum_{n=1}^{+\infty}{g_n(x)}$$

\subsection*{Exercice 6.2}
$$\int_{0}^{1}{\frac{\ln(1-t)}{t}dt} = \lim_{x \to 0}\int_{x}^{1-x}{\frac{\ln(1-t)}{t}dt} = \lim_{x \to 0}\sum_{n=1}^{+\infty}{g_n(x)} = \lim_{x \to 0}\sum_{n=1}^{+\infty}{\frac{1}{n^2} ((1-x)^n-x^n)} = \sum_{n=1}^{+\infty}{\frac{1}{n^2}}$$

\subsection*{Exercice 7}
On a $\lim_{n \to +\infty}f_n(x) = \lim_{n \to +\infty}\ln(1+\frac{x}{n(1+x)}) = 0$. Gr\^ace aux crit\`ere des s\'eries altern\'ees, on voit que, pour tout $x \in \R^{+}$, la s\'eries de fonctions $\sum_{n \geq 1} {(-1)^nf_n(x)}$ converge simplement. En appliquant le crit\`ere de Cauchy uniforme, la s\'erie $\sum_{n\geq 1} f_n(x)$ converge uniform\'ement sur $\R^{+}$ ssi:
$$\forall \epsilon > 0, \exists n_{\epsilon} \in \N, (m > n \geq n_{\epsilon} \implies \forall x \in \R^{+}, |\sum_{k=n+1}^{m} f_k(x)|  \leq \epsilon)$$ 

On a 
$$|\sum_{k=n+1}^{m} f_k(x)| < \ln \left(1 + \frac{x}{n(1+x)} \right) < \ln \left(1 + \frac{1}{n}\frac{x}{1+x} \right) < \ln \left(1 + \frac{1}{n} \right)$$

Donc en prenant $n_{\epsilon}$, tel que $\ln \left(1 + \frac{1}{n_{\epsilon}} \right) < \epsilon$, le crit\`ere de Cauchy est v\'erifi\'e et la s\'erie de fonctions converge uniform\'ement sur $\R^{+}$.



\subsection*{Exercice 3 - autre}
\subsubsection*{Q3.1}
$$\lim_{n \to +\infty} \sqrt{x^2+\frac{1}{n}} = \sqrt{x^2} = |x|$$
La fonction $f_n(x)$ converge simplement vers la fonction $f(x) = |x|$.

\subsubsection*{Q3.2}
Il faut v\'erifier que
$$\forall \epsilon > 0, \exists n_0 \in \N, \forall x \in \R, \forall n > n_0, \left|\sqrt{x^2+\frac{1}{n}} - |x|\right| < \epsilon$$

Il faut trouver un majorant a $|\sqrt{x^2+\frac{1}{n}} - |x||$. On voit que $x^2 + \frac{1}{n} > x^2$ car $n \geq 1$. donc $x^2 + \frac{1}{n} - x^2 > 0$ et $|\sqrt{x^2+\frac{1}{n}} - |x|| = \sqrt{x^2+\frac{1}{n}} - |x|$.

Ensuite, le "truc".
$$\left(\sqrt{x^2+\frac{1}{n}}\right)^2 - (|x|)^2 = x^2+\frac{1}{n} - x^2 = \frac{1}{n}$$
donc
$$\left(\sqrt{x^2+\frac{1}{n}} - |x|\right)\left(\sqrt{x^2+\frac{1}{n}} + |x|\right) = \frac{1}{n}$$
$$\sqrt{x^2+\frac{1}{n}} - |x| = \frac{\frac{1}{n}}{\sqrt{x^2+\frac{1}{n}} + |x|}$$
$$\sqrt{x^2+\frac{1}{n}} - |x| < \frac{1}{n}$$

Donc il suffit de prendre $n_epsilon$ tel que $\frac{1}{n_{\epsilon}} < \epsilon$.

\subsubsection*{Q3.3}
$$f_n'(x) = \frac{x}{\sqrt{x+\frac{1}{n}}}$$
donc
$$\lim_{n \to +\infty} \frac{x}{\sqrt{x^2+\frac{1}{n}}} = \frac{x}{|x|}$$
Donc $f_n'(x)$ converge simplement vers la fonction 
$$g(x) = 
\left\{
\begin{array}{l l}
1 & x > 0\\
-1 & x < 0\\
\text{non d\'efini} & x = 0\\
\end{array}
\right.
$$

\subsubsection*{Q3.4}
Il faut v\'erifier que
$$\forall \epsilon > 0, \exists n_0 \in \N, \forall x \in \R, \forall n > n_0, \left|\frac{x}{\sqrt{x+\frac{1}{n}}} - g(x)\right| < \epsilon$$

Il n'existe pas de $\epsilon$ car $\frac{x}{\sqrt{x+\frac{1}{n}}}$ n'est pas born\'ee par 1.


\subsection*{Exercice 3 - Partiel 2020}
\subsubsection*{Q3.1}
Pour $x=1$ et $x=0$, $f_n(x)=0$.
Pour $x \in ]0,1[$ et $n \geq 3$, on a $f_n(x) > 0$ car $x^n >0$, $(1-x) >0$ et $\ln(n)^{\alpha} > 0$. 
Pour $x \in ]0,1[$,
$$\frac{f_{n+1}(x)}{f_n(x)} = \frac{\frac{x^{n+1}(1-x)}{(\ln(n+1))^{\alpha}}}{\frac{x^n(1-x)}{(\ln(n))^{\alpha}}} = \frac{(\ln(n))^{\alpha}(x^{n+1}(1-x))} {(\ln(n+1))^{\alpha}(x^n(1-x))} = \left(\frac{\ln(n))}{\ln(n+1))} \right)^{\alpha}x \in ]0,1[$$

Donc par le crit\`ere d'Alembert, la s\'erie de fonctions $f_n(x)$ converge simplement.


\subsubsection*{Q3.2}
Calcul de 
$$f_n'(x) = -\frac{x^{n-1}((n+1)x -n)}{(\ln(n+1))^{\alpha}}$$

$f_n'(x) = 0$ pour $x=0$ et $x = \frac{n}{n+1}$. Les 2 points sont dans $[0,1]$.
On a $f_n(0) = 0$ et $f_n(1) = 0$ et $f_n(x)$ est positive, donc $f_n(x)$ croit entre $[0,\frac{n}{n+1}[$ et d\'ecroit entre $]\frac{n}{n+1},1]$

Calcul du maximum
$$f_n\left(\frac{n}{n+1}\right) =  \frac{\left(\frac{n}{n+1}\right)^n\left(1-\left(\frac{n}{n+1}\right)\right)}{(\ln(n))^{\alpha}} = \left(\frac{n}{n+1}\right)^n\frac{1}{(n+1)}\frac{1}{(\ln(n))^{\alpha}}$$

Soit $c = \left(\frac{n}{n+1}\right)^n\frac{1}{(n+1)} $, on a $0 < c < 1$. 
donc $f_n\left(\frac{n}{n+1}\right) = \frac{c}{(\ln(n))^{\alpha}}$ cela converge ssi $\alpha > 1$ (sinon $\frac{c}{(\ln(n))^{\alpha}}$ croit lorsque $n$ croit). 

\subsubsection*{Q3.3}
Je ne sais pas faire.

QED

\end{document}

