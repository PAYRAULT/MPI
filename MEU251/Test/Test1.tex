\documentclass[]{book}

%These tell TeX which packages to use.
\usepackage{array,epsfig}
\usepackage{amsmath}
\usepackage{amsfonts}
\usepackage{amssymb}
\usepackage{amsxtra}
\usepackage{amsthm}
\usepackage{mathrsfs}
\usepackage{color}

%Here I define some theorem styles and shortcut commands for symbols I use often
\theoremstyle{definition}
\newtheorem{defn}{Definition}
\newtheorem{thm}{Theorem}
\newtheorem{cor}{Corollary}
\newtheorem*{rmk}{Remark}
\newtheorem{lem}{Lemma}
\newtheorem*{joke}{Joke}
\newtheorem{ex}{Example}
\newtheorem*{soln}{Solution}
\newtheorem{prop}{Proposition}

\newcommand{\lra}{\longrightarrow}
\newcommand{\ra}{\rightarrow}
\newcommand{\surj}{\twoheadrightarrow}
\newcommand{\graph}{\mathrm{graph}}
\newcommand{\bb}[1]{\mathbb{#1}}
\newcommand{\Z}{\bb{Z}}
\newcommand{\Q}{\bb{Q}}
\newcommand{\R}{\bb{R}}
\newcommand{\C}{\bb{C}}
\newcommand{\N}{\bb{N}}
\newcommand{\M}{\mathbf{M}}
\newcommand{\m}{\mathbf{m}}
\newcommand{\MM}{\mathscr{M}}
\newcommand{\HH}{\mathscr{H}}
\newcommand{\Om}{\Omega}
\newcommand{\Ho}{\in\HH(\Om)}
\newcommand{\bd}{\partial}
\newcommand{\del}{\partial}
\newcommand{\bardel}{\overline\partial}
\newcommand{\textdf}[1]{\textbf{\textsf{#1}}\index{#1}}
\newcommand{\img}{\mathrm{img}}
\newcommand{\ip}[2]{\left\langle{#1},{#2}\right\rangle}
\newcommand{\inter}[1]{\mathrm{int}{#1}}
\newcommand{\exter}[1]{\mathrm{ext}{#1}}
\newcommand{\cl}[1]{\mathrm{cl}{#1}}
\newcommand{\ds}{\displaystyle}
\newcommand{\vol}{\mathrm{vol}}
\newcommand{\cnt}{\mathrm{ct}}
\newcommand{\osc}{\mathrm{osc}}
\newcommand{\LL}{\mathbf{L}}
\newcommand{\UU}{\mathbf{U}}
\newcommand{\support}{\mathrm{support}}
\newcommand{\AND}{\;\wedge\;}
\newcommand{\OR}{\;\vee\;}
\newcommand{\Oset}{\varnothing}
\newcommand{\st}{\ni}
\newcommand{\wh}{\widehat}

%Pagination stuff.
\setlength{\topmargin}{-.3 in}
\setlength{\oddsidemargin}{0in}
\setlength{\evensidemargin}{0in}
\setlength{\textheight}{9.in}
\setlength{\textwidth}{6.5in}
\pagestyle{empty}



\begin{document}

\subsection*{Rappel de cours}
\begin{defn}
Soit $I$ un intervalle de $\R$,  $(U_n)$ une suite de fonctions d\'efinies sur $I$ et $f$ une fonction d\'efinie sur I.
On dit que $(u_n)$ converge \textbf{simplement} vers $f$ sur $I$ si pour tout $x \in I$, la suite $(U_n(x))$ converge vers $f(x)$.
\end{defn}

\begin{defn}
Soit $I$ un intervalle de $\R$,  $(U_n)$ une suite de fonctions d\'efinies sur $I$ et $f$ une fonction d\'efinie sur I.
On dit que $(u_n)$ converge \textbf{uniform\'ement} vers $f$ sur $I$ si
$$\forall \epsilon > 0, \exists n_0 \in \N, \forall x \in I, \forall n > n_0, |(U_n(x)) - f(x)| < \epsilon$$
\end{defn}

\begin{defn}
On dit que la s\'erie de fonctions $\sum_{n\geq0}U_n(x)$ converge \textbf{normalement} sur  $I$ si la s\'erie $\sum_{n\geq0}||U_n(x)||_{\infty}$ est convergente.
 \end{defn}


\newpage
\subsection*{Exercice 1}
Montrons que la fonction $f_n(x)=\frac{ne^{-x}+x^2}{n+x}$ converge simplement pour $x \in [0,1]$.
$$\lim_{n \to \infty}\frac{ne^{-x}+x^2}{n+x} = \lim_{n \to \infty}\frac{ne^{-x}+1}{n+1} = \lim_{n \to \infty}\frac{ne^{-x}}{n} = e^{-x}$$

La fonction $f_n(x)$ converge simplement vers $f(x) = e^{-x}$. Montrons maintenant que $f_n(x)$ converge uniform\'ement. Il faut que $\forall \epsilon > 0, \exists n_{\epsilon} \in \N, \forall n \in \N, (n \geq n_{\epsilon} \implies \sup_{x \in [0,1]} |f_n(x)-f(x)| < \epsilon)$.
Calculons 
$$|f_n(x)-f(x)| = \left|\frac{ne^{-x}+x^2}{n+x} - e^{-x}\right| = \left|\frac{ne^{-x}+x^2-(n+x)e^{-x}}{n+x}\right| = \left|\frac{x^2-xe^{-x}}{n+x}\right| < \frac{2}{n}$$
car $n+x \geq n$ et $|x^2-xe^{-x}| < 2$ comme $x^2 \in [0,1]$ et $xe^{-x} \in [0, \frac{1}{e}]$.\\ 
Prenons $\sup_{x \in [0,1]} |f_n(x)-f(x)| = \frac{2}{n}$, donc $n_{\epsilon}$ existe et doit e\^tre sup\'erieur \`a $\frac{2}{\epsilon}$.


\subsection*{Exercice 2}
\subsection*{Exercice 2.1}
Montrons que la fonction $f_n(x)=\ln\left(x+\frac{1}{n}\right)$ converge simplement pour $x \in ]0,+\infty]$.
$$\lim_{n \to \infty}\ln\left(x+\frac{1}{n}\right) = \ln\left(x+0\right) = \ln(x)$$

La fonction $f_n(x)$ converge simplement vers $f(x) = \ln(x)$. Montrons maintenant que $f_n(x)$ converge uniform\'ement sur $[a,+\infty]$. Il faut que $\forall \epsilon > 0, \exists n_{\epsilon} \in \N, \forall n \in \N, (n \geq n_{\epsilon} \implies \sup_{x \in [a,+\infty]} |f_n(x)-f(x)| < \epsilon)$.
Calculons 
$$|f_n(x)-f(x)| = \left|\ln\left(x+\frac{1}{n}\right) - \ln(x)\right| = \left|\ln\left(\frac{x+\frac{1}{n}}{x}\right)\right| = \left|\ln\left(1 + \frac{1}{nx}\right)\right| \leq \ln\left(1 + \frac{1}{na}\right)$$
car $\frac{1}{na} \geq \frac{1}{nx}$ pour $x \in [a,+\infty]$.\\ 
Prenons $\sup_{x \in [a,+\infty]} |f_n(x)-f(x)| = \ln\left(1 + \frac{1}{na}\right)$, donc $n_{\epsilon}$ existe et doit e\^tre tel que $\ln\left(1 + \frac{1}{n_{\epsilon}a}\right) < \epsilon$.

\subsection*{Exercice 2.2}
Non, car pour $x$ proche de 0, il n'existe pas de borne sup\'erieure pour $|f_n(x)-f(x)| = \left|\ln\left(1 + \frac{1}{nx}\right)\right|$.

\subsection*{Exercice 3}
\subsection*{Exercice 3.1}
Quand $x=0$, on a $f_n(x) = 0$ pour tout $n \geq 1$.\\
Pour $x>0$, on a 
$$\lim_{n \to \infty} xe^{-nx} = \lim_{n \to \infty} \frac{x}{e^{nx}} = 0$$
On en d\'eduit que la fonction $f_n(x)$ converge simplement vers la fonction 
$$f(x) = 
\left\{
\begin{array}{l l}
0 & x = 0\\
0 & x \in ]0, +\infty]\\
\end{array}
\right. 
= 0
$$

\subsection*{Exercice 3.2}
Pour que la fonction $f_n(x)$ converge uniform\'ement sur $[0,+\infty]$. Il faut que $\forall \epsilon > 0, \exists n_{\epsilon} \in \N, \forall n \in \N, (n \geq n_{\epsilon} \implies \sup_{x \in [0,+\infty]} |f_n(x)-0| < \epsilon)$. \\

Cela revient \`a trouver si la fonction $f_n(x)$ a un maximum pour $x \in [0,+\infty]$.\\
La fonction est continue et positive sur $[0,\infty]$, cherchons si il existe un maximum.\\

Calculons $f_{n}'(x) = e^{-x}(1-nx)$. on a $f_{n}'(x)$ qui s'annule lorsque $x = \frac{1}{n}$.\\

La fonction $f_{n}(x)$ croit entre $[0,\frac{1}{n}[$ jusqu'\`a la valeur $f_n(\frac{1}{n}) = \frac{1}{ne}$ et d\'ecroit ensuite pour $x \in ]\frac{1}{n}, \infty]$. Donc il existe une valeur sup\'erieure $\frac{1}{ne}$. \\

On a $\sup_{x \in [0,+\infty]}|f_n(x)| = \frac{1}{ne}$, donc il $n_{\epsilon}$  existe et doit \^etre sup\'erieure \`a $\frac{1}{e.\epsilon}$. Par cons\'equent, la fonction $f_n(x)$ converge uniform\'ement vers la fonction $0$. 



\subsection*{Exercice 4}
Pour $x \in [-r,r]$ avec $r > 0$ on a $0 \geq f_n(x) \geq ln\left(1+\frac{r^2}{n^2}\right)$. On va montrer que 
$$ln\left(1+\frac{r^2}{n^2}\right) \geq \frac{r^2}{n^2}$$
$$1+\frac{r^2}{n^2} \geq e^{\frac{r^2}{n^2}} = 1 + \frac{r^2}{n^2} + \frac{r^4}{2n^4} + O(\frac{r^2}{n^2})$$
$$0 \geq \frac{r^4}{2n^4} + O(\frac{r^2}{n^2})$$

Donc $x \in [-r,r]$ avec $r > 0$ on a $0 \geq f_n(x) \geq ln\left(1+\frac{r^2}{n^2}\right) \geq \frac{r^2}{n^2}$ et on sait que la s\'erie de fonction $\frac{r^2}{n^2}$ converge normalement. Donc s\'erie de fonction $f_n(x) = ln(1+\frac{x^2}{n^2}$ converge normalement pour tout $x \in [-r, r]$.\\

Les fonctions $f_n$ sont continues sur $x \in [-r,r]$ (assemblage de fonctions continues) et la s\'erie de fonctions $S = \sum_{n |geq 1}{f_n(x)}$ converge normalement pour $x \in [-r,r]$. On en d\'eduit que $S$ est continue pour $x \in [-r,r]$. 


\subsection*{Exercice 5}
 


QED

\end{document}

