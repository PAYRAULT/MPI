\documentclass[]{book}

%These tell TeX which packages to use.
\usepackage{array,epsfig}
\usepackage{amsmath}
\usepackage{amsfonts}
\usepackage{amssymb}
\usepackage{amsxtra}
\usepackage{amsthm}
\usepackage{mathrsfs}
\usepackage{color}

%Here I define some theorem styles and shortcut commands for symbols I use often
\theoremstyle{definition}
\newtheorem{defn}{Definition}
\newtheorem{thm}{Theorem}
\newtheorem{cor}{Corollary}
\newtheorem*{rmk}{Remark}
\newtheorem{lem}{Lemma}
\newtheorem*{joke}{Joke}
\newtheorem{ex}{Example}
\newtheorem*{soln}{Solution}
\newtheorem{prop}{Proposition}

\newcommand{\lra}{\longrightarrow}
\newcommand{\ra}{\rightarrow}
\newcommand{\surj}{\twoheadrightarrow}
\newcommand{\graph}{\mathrm{graph}}
\newcommand{\bb}[1]{\mathbb{#1}}
\newcommand{\Z}{\bb{Z}}
\newcommand{\Q}{\bb{Q}}
\newcommand{\R}{\bb{R}}
\newcommand{\C}{\bb{C}}
\newcommand{\N}{\bb{N}}
\newcommand{\M}{\mathbf{M}}
\newcommand{\m}{\mathbf{m}}
\newcommand{\MM}{\mathscr{M}}
\newcommand{\HH}{\mathscr{H}}
\newcommand{\Om}{\Omega}
\newcommand{\Ho}{\in\HH(\Om)}
\newcommand{\bd}{\partial}
\newcommand{\del}{\partial}
\newcommand{\bardel}{\overline\partial}
\newcommand{\textdf}[1]{\textbf{\textsf{#1}}\index{#1}}
\newcommand{\img}{\mathrm{img}}
\newcommand{\ip}[2]{\left\langle{#1},{#2}\right\rangle}
\newcommand{\inter}[1]{\mathrm{int}{#1}}
\newcommand{\exter}[1]{\mathrm{ext}{#1}}
\newcommand{\cl}[1]{\mathrm{cl}{#1}}
\newcommand{\ds}{\displaystyle}
\newcommand{\vol}{\mathrm{vol}}
\newcommand{\cnt}{\mathrm{ct}}
\newcommand{\osc}{\mathrm{osc}}
\newcommand{\LL}{\mathbf{L}}
\newcommand{\UU}{\mathbf{U}}
\newcommand{\support}{\mathrm{support}}
\newcommand{\AND}{\;\wedge\;}
\newcommand{\OR}{\;\vee\;}
\newcommand{\Oset}{\varnothing}
\newcommand{\st}{\ni}
\newcommand{\wh}{\widehat}

%Pagination stuff.
\setlength{\topmargin}{-.3 in}
\setlength{\oddsidemargin}{0in}
\setlength{\evensidemargin}{0in}
\setlength{\textheight}{9.in}
\setlength{\textwidth}{6.5in}
\pagestyle{empty}



\begin{document}

\subsection*{Rappel de cours}
\begin{defn}
Soit $I$ un intervalle de $\R$,  $(U_n)$ une suite de fonctions d\'efinies sur $I$ et $f$ une fonction d\'efinie sur I.
On dit que $(u_n)$ converge \textbf{simplement} vers $f$ sur $I$ si pour tout $x \in I$, la suite $(U_n(x))$ converge vers $f(x)$.
\end{defn}

\begin{defn}
Soit $I$ un intervalle de $\R$,  $(U_n)$ une suite de fonctions d\'efinies sur $I$ et $f$ une fonction d\'efinie sur I.
On dit que $(u_n)$ converge \textbf{uniform\'ement} vers $f$ sur $I$ si
$$\forall \epsilon > 0, \exists n_0 \in \N, \forall x \in I, \forall n > n_0, |(U_n(x)) - f(x)| < \epsilon$$
\end{defn}

\begin{defn}
On dit que la s\'erie de fonctions $\sum_{n\geq0}U_n(x)$ converge \textbf{normalement} sur  $I$ si la s\'erie $\sum_{n\geq0}||U_n(x)||_{\infty}$ est convergente.
 \end{defn}

\begin{defn}
On dit que la s\'erie de fonctions $\sum_{n\geq0}U_n(x)$ converge \textbf{normalement} sur  $I$ si la s\'erie $\sum_{n\geq0}||U_n(x)||_{\infty}$ est convergente.
\end{defn}


\newpage
\subsection*{Exercice 2}
\subsubsection*{Exercice 2.1.a}
Calculons $\left\lvert \frac{f_{n+1}(x)}{f_n(x)} \right\rvert$.
$$
L = \lim_{n \to \infty}\left\lvert \frac{(-1)^{n+1}2^{n+1}\frac{x^{n+2}}{n+2}}{(-1)^{n}2^{n}\frac{x^{n+1}}{n+1}} \right\rvert = \lim_{n \to \infty} \left\lvert -2x\frac{n+1}{n+2}  \right\rvert = \lvert -2x \rvert
$$

\subsubsection*{Exercice 2.1.b}
Calculons $\left\lvert \frac{f'_{n+1}(x)}{f'_n(x)} \right\rvert$. avec $f_{n}'(x) = (-1)^n2^nx^n$.
$$
L' = \lim_{n \to \infty}\left\lvert \frac{(-1)^{n+1}2^{n+1}x^{n+1}}{(-1)^{n}2^{n}x^{n}} \right\rvert = \lim_{n \to \infty} \left\lvert -2x \right\rvert = \lvert -2x \rvert
$$

\subsubsection*{Exercice 2.1.c}
D'apr\`es le crit\`ere d'Alembert, la s\'erie de terme g\'en\'erale $\lvert f_n \rvert$ converge normalement si $\left\lvert \frac{f_{n+1}}{f_n} \right\rvert < 1$. Donc, il faut trouver $x$ tel que  $\lvert -2x \rvert < 1$, cela fait $x \in ]-\frac{1}{2}, \frac{1}{2}[$ donc pour tous $[-a,a]$ avec $a \in [0,\frac{1}{2}[$


\subsubsection*{Exercice 2.2}
Les 2 s\'eries ne convergent pas car $x=\pm \frac{1}{2}$ car ces valeurs ne v\'erifient pas le crit\`ere d'Alembert. ???


\subsubsection*{Exercice 2.3}
Sur $x \in [0,\frac{1}{2}[$, la fonction $f_n(x) = 2^n\frac{x^{n+1}}{n+1}$ est d\'ecroissante et tend vers 0. La s\'erie num\'erique $\sum_{n \ge 0} f_n(x)$ est une s\'erie altern\'ee. En utilisant, le th\'eor\`eme du reste on a:
$$\lvert R_n(x) \rvert = \lvert \sum_{k \ge n+1}f_k(x) \rvert \le f_{n+1}(x)$$

Donc
$$
f_{n+1}(x) = 2^{n+1}\frac{x^{n+2}}{n+2} = \frac{(2x)^{n+2}}{2(n+2)}
$$
Pour $x \in [0,\frac{1}{2}[$, on a
$$\lim_{n \to \infty} \frac{(2x)^{n+2}}{2(n+2)} = 0$$
Donc la s\'erie converge uniform\'ement.

\subsubsection*{Exercice 2.4}
Pour $a \in ]0,\frac{1}{2}[$, Sur le domaine $[-a,0[$, on a
$$
|(-1)^n2^n\frac{x^{n+1}}{n+1}| = |(-1)^n\frac{(2x)^{n+1}}{2(n+1)}| < |(-1)^n\frac{(2a)^{n+1}}{2(n+1)}| = \frac{(2a)^{n+1}}{2(n+1)}
$$

et 
$$
\lim_{n \to \infty} \frac{(2a)^{n+1}}{2(n+1)} = 0
$$

Donc, la s\'erie $\sum f_n$ converge uniform\'ement sur $[-a,0[$, pour $a \in [0,\frac{1}{2}]$, comme elle convergence \'egalement sur le domaine $[0,\frac{1}{2}$, elle converge sur $[-a, \frac{1}{2}[$.

\subsubsection*{Exercice 2.5}

 
QED

\end{document}

