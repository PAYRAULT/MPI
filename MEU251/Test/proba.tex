\documentclass[]{book}

%These tell TeX which packages to use.
\usepackage{array,epsfig}
\usepackage{amsmath}
\usepackage{amsfonts}
\usepackage{amssymb}
\usepackage{amsxtra}
\usepackage{amsthm}
\usepackage{mathrsfs}
\usepackage{color}

%Here I define some theorem styles and shortcut commands for symbols I use often
\theoremstyle{definition}
\newtheorem{defn}{Definition}
\newtheorem{thm}{Theorem}
\newtheorem{cor}{Corollary}
\newtheorem*{rmk}{Remark}
\newtheorem{lem}{Lemma}
\newtheorem*{joke}{Joke}
\newtheorem{ex}{Example}
\newtheorem*{soln}{Solution}
\newtheorem{prop}{Proposition}

\newcommand{\lra}{\longrightarrow}
\newcommand{\ra}{\rightarrow}
\newcommand{\surj}{\twoheadrightarrow}
\newcommand{\graph}{\mathrm{graph}}
\newcommand{\bb}[1]{\mathbb{#1}}
\newcommand{\Z}{\bb{Z}}
\newcommand{\Q}{\bb{Q}}
\newcommand{\R}{\bb{R}}
\newcommand{\C}{\bb{C}}
\newcommand{\N}{\bb{N}}
\newcommand{\M}{\mathbf{M}}
\newcommand{\m}{\mathbf{m}}
\newcommand{\MM}{\mathscr{M}}
\newcommand{\HH}{\mathscr{H}}
\newcommand{\Om}{\Omega}
\newcommand{\Ho}{\in\HH(\Om)}
\newcommand{\bd}{\partial}
\newcommand{\del}{\partial}
\newcommand{\bardel}{\overline\partial}
\newcommand{\textdf}[1]{\textbf{\textsf{#1}}\index{#1}}
\newcommand{\img}{\mathrm{img}}
\newcommand{\ip}[2]{\left\langle{#1},{#2}\right\rangle}
\newcommand{\inter}[1]{\mathrm{int}{#1}}
\newcommand{\exter}[1]{\mathrm{ext}{#1}}
\newcommand{\cl}[1]{\mathrm{cl}{#1}}
\newcommand{\ds}{\displaystyle}
\newcommand{\vol}{\mathrm{vol}}
\newcommand{\cnt}{\mathrm{ct}}
\newcommand{\osc}{\mathrm{osc}}
\newcommand{\LL}{\mathbf{L}}
\newcommand{\UU}{\mathbf{U}}
\newcommand{\support}{\mathrm{support}}
\newcommand{\AND}{\;\wedge\;}
\newcommand{\OR}{\;\vee\;}
\newcommand{\Oset}{\varnothing}
\newcommand{\st}{\ni}
\newcommand{\wh}{\widehat}

%Pagination stuff.
\setlength{\topmargin}{-.3 in}
\setlength{\oddsidemargin}{0in}
\setlength{\evensidemargin}{0in}
\setlength{\textheight}{9.in}
\setlength{\textwidth}{6.5in}
\pagestyle{empty}



\begin{document}

\subsection*{Exercice 1}
D'abord toujours definir l'univers: Choisir 5 cartes parmi 32 cartes $card(\Omega)=\C_5^32$

\subsection*{Exercice 1.1}
Pour avoir \emph{une seule} paire, il faut 2 cartes de m\^eme hauteur, et 3 cartes qui n'ont pas cette hauteur. Pour une hauteur donn\'ee, il y a $\C_2^4$ possibilit\'es (2 couleurs parmi 4). Dans un jeu de 32 cartes, il y a 8 hauteurs diff\'erentes. Il faut 1 paire, donc $\C_1^8$. Maintenant, il faut s'occuper des 3 autres cartes. Comme il faut une \emph{seule} paire, les 3 autres cartes doivent \^etre diff\'erentes de la hauteur de la paire. Il reste donc 7 autres hauteurs donc $C_3^7$ et chacune des 3 cartes peut prendre n'importe qu'elle couleur donc $\C_1^4$. Donc
$$P(une_seule_paire) = \frac{\C_1^8.\C_2^4.\C_3^7.\C_1^4.\C_1^4.\C_1^4}{\C_5^{32}} = \frac{8.6.35.4.4.4}{201376} = 53.39\%$$

\subsection*{Exercice 1.2}
Pour avoir \emph{deux} paire, il faut 2 fois 2 cartes de m\^eme hauteur, et 1 cartes qui n'a pas ces hauteurs. Pour une hauteur donn\'ee, il y a $\C_2^4$ possibilit\'es (2 couleurs parmi 4). Dans un jeu de 32 cartes, il y a 8 hauteurs diff\'erentes. Il faut 1 paire, donc $\C_1^8$ et une seconde paire diff\'erente $\C_1^7$. Maintenant, il faut s'occuper de la derni\`ere carte. Comme il faut deux paire, la derni\`ere cartes doit \^etre diff\'erentes de la hauteur des 2 paires. Il reste donc 6 autres hauteurs donc $C_1^6$ et elle peut prendre n'importe qu'elle couleur donc $\C_1^4$. Donc
$$P(deux_paires) = \frac{\C_1^8.\C_2^4.\C_1^7.\C_2^4.\C_1^6.\C_1^4}{\C_5^{32}} = \frac{8.6.7.6.6.4}{201376} = $$

\subsection*{Exercice 1.3}
Pour avoir un brelan, il faut 3 cartes de m\^eme hauteur, et 2 cartes qui n'ont pas cette hauteur. Pour une hauteur donn\'ee, il y a $\C_3^4$ possibilit\'es (3 couleurs parmi 4). Dans un jeu de 32 cartes, il y a 8 hauteurs diff\'erentes. Il faut 1 brelan, donc $\C_1^8$. Maintenant, il faut s'occuper des 2 derni\`eres cartes. Comme il faut un brelan uniquement, les deux derni\`eres cartes doivent \^etre diff\'erentes de la hauteur du brelan. Il reste donc 7 autres hauteurs donc $C_2^7$ et elle peuvent prendre n'importe qu'elle couleur donc $\C_1^4$. Donc
$$P(un_brelan) = \frac{\C_1^8.\C_3^4.\C_2^7.\C_1^4.\C_1^4}{\C_5^{32}} = \frac{8.4.21.4.4}{201376} = 5.33\%$$


\subsection*{Exercice 1.4}
Pour avoir un carr\'e, il faut 4 cartes de m\^eme hauteur, et 1 cartes qui n'a pas cette hauteur. Pour une hauteur donn\'ee, il y a $\C_4^4$ possibilit\'es (4 couleurs parmi 4). Dans un jeu de 32 cartes, il y a 8 hauteurs diff\'erentes. Il faut 1 carr\'e, donc $\C_1^8$. Maintenant, il faut s'occuper de la derni\`ere carte. Comme il faut un carr\'e, la derni\`ere carte doit \^etre diff\'erente de la hauteur du carr\'e. Il reste donc 7 autres hauteurs donc $C_1^7$ et elle peuvent prendre n'importe qu'elle couleur donc $\C_1^4$. Donc
$$P(un_brelan) = \frac{\C_1^8.\C_4^4.\C_1^7.\C_1^4}{\C_5^{32}} = \frac{8.1.7.4}{224} = 0.11\%$$



QED

\end{document}

