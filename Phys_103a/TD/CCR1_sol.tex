\documentclass[]{book}

%These tell TeX which packages to use.
\usepackage{array,epsfig}
\usepackage{amsmath}
\usepackage{amsfonts}
\usepackage{amssymb}
\usepackage{amsxtra}
\usepackage{amsthm}
\usepackage{mathrsfs}
\usepackage{color}
\usepackage{tikz}


%Here I define some theorem styles and shortcut commands for symbols I use often
\theoremstyle{definition}
\newtheorem{defn}{Definition}
\newtheorem{thm}{Theorem}
\newtheorem{cor}{Corollary}
\newtheorem*{rmk}{Remark}
\newtheorem{lem}{Lemma}
\newtheorem*{joke}{Joke}
\newtheorem{ex}{Example}
\newtheorem*{soln}{Solution}
\newtheorem{prop}{Proposition}

\newcommand{\lra}{\longrightarrow}
\newcommand{\ra}{\rightarrow}
\newcommand{\surj}{\twoheadrightarrow}
\newcommand{\graph}{\mathrm{graph}}
\newcommand{\bb}[1]{\mathbb{#1}}
\newcommand{\Z}{\bb{Z}}
\newcommand{\Q}{\bb{Q}}
\newcommand{\R}{\bb{R}}
\newcommand{\C}{\bb{C}}
\newcommand{\N}{\bb{N}}
\newcommand{\M}{\mathbf{M}}
\newcommand{\m}{\mathbf{m}}
\newcommand{\MM}{\mathscr{M}}
\newcommand{\HH}{\mathscr{H}}
\newcommand{\Om}{\Omega}
\newcommand{\Ho}{\in\HH(\Om)}
\newcommand{\bd}{\partial}
\newcommand{\del}{\partial}
\newcommand{\bardel}{\overline\partial}
\newcommand{\textdf}[1]{\textbf{\textsf{#1}}\index{#1}}
\newcommand{\img}{\mathrm{img}}
\newcommand{\ip}[2]{\left\langle{#1},{#2}\right\rangle}
\newcommand{\inter}[1]{\mathrm{int}{#1}}
\newcommand{\exter}[1]{\mathrm{ext}{#1}}
\newcommand{\cl}[1]{\mathrm{cl}{#1}}
\newcommand{\ds}{\displaystyle}
\newcommand{\vol}{\mathrm{vol}}
\newcommand{\cnt}{\mathrm{ct}}
\newcommand{\osc}{\mathrm{osc}}
\newcommand{\LL}{\mathbf{L}}
\newcommand{\UU}{\mathbf{U}}
\newcommand{\support}{\mathrm{support}}
\newcommand{\AND}{\;\wedge\;}
\newcommand{\OR}{\;\vee\;}
\newcommand{\Oset}{\varnothing}
\newcommand{\st}{\ni}
\newcommand{\wh}{\widehat}

%Pagination stuff.
\setlength{\topmargin}{-.3 in}
\setlength{\oddsidemargin}{0in}
\setlength{\evensidemargin}{0in}
\setlength{\textheight}{9.in}
\setlength{\textwidth}{6.5in}
\pagestyle{empty}



\begin{document}


\subsection*{Rappel de cours}
\subsubsection*{Travail}

\begin{itemize}
\item La composante de la force d'un point $M$, $\vec{F}(M)$ sur l'axe $\mathcal{O}_x$ est donn\'ee par le produit scalaire $f(x) = \vec{F}(M).\vec{i}$. 
\item Le travail d'une force $\vec{F}$ sur un segment $\vec{AB}$ est donn\'e par :
$$W_{A \to B}(\vec{F}) = \vec{F}.\vec{AB} = \int_{A \to B} \vec{F}.\vec{i}dx = \int_{x_a}^{x_b} f(x)dx$$ 
\item On dira qu'une force est conservative si elle ne d\'epend que de la position et si son travail d'un point $A$ au point $B$ ne d\'epend pas du chemin suivi, ceci quels que soient les point $A$ et $B$.
$$\forall A,B,C,\, W_{A \to B} \vec{F} = W_{A \to C} \vec{F} + W_{C \to B} \vec{F}$$
\item Dans le cas o\`u le chemin est rectiligne, si une force est conservatrice alors l'\emph{\'energie potentielle} associ\'ee \`a la force $\vec{F}$ est not\'ee $E_p(x)$ est d\'efinie par :
$$W_{A \to B}(\vec{F}) = \int_{A \to B} \vec{F}.\vec{i}dx = E_p(x_b) - E_p(x_a)$$. 
\item Le travail du poids $\vec{P} = m\vec{g}$ sur le segment $\vec{AB}$ est $W_{A \to B}(\vec{P}) = -mg(z_b - z_a) = -mgh$.
\item Le travail de la force de rappel \'elastique d'un ressort de raideur $k$, $\vec{F} = -k.x\vec{i}$ est $W_{A \to B}(\vec{F}) = -\frac{1}{2}k(x_a^2 - x_b^2)$.
\end{itemize}


\subsubsection*{\'Energie}
\begin{itemize}
\item L'\'energie m\'ecanique d'un syst\`eme $E_m = E_c + E_p$ avec $E_c$ l'\'energie cin\'etique qui d\'epend de la masse et de la norme de la vitesse du syst\`eme physique \'etudi\'e et de l'\'energie potentielle $E_p$ qui correspond aux forces exerc\'ees sur le syst\`eme.
\item L'\'energie cin\'etique du syst\`eme $E_c = \frac{1}{2}mv^2$
\item L'\'energie potentielle qui correspond \`a l'ensemble des forces conservatives qui s'exercent sur le syst\'eme, $E_p(B) - E_p(A) = W_{A \to B}(\vec{F}_{conservatives})$
\item $E_m(B) - E_m(A) = W_{A \to B}(\vec{F}_{non\, conservatives})$
\end{itemize}

\subsubsection*{Puissance}
\begin{itemize}
\item La puissance $P$ repr\'esente l'\'energie transf\'er\'ee uniformement (ie. le travail) pendant une unit\'e de temps, $P =\frac{W}{\Delta t}$.
\item $1\,W = 1\,J.s^{-1} = 1\, N.m.s^{-1} = 1\, kg.m^{2}.s^{-3}$
\end{itemize}

\subsubsection*{Coordonn\'ees polaires}

$$
\begin{array}{l l}
\text{Polaire vers cart\'esienne} & \text{Cart\'esienne vers polaire} \\
\left\{
\begin{array}{l }
 x = \rho \cos(\theta) \\
 y = \rho \sin(\theta) \\
\end{array}
\right. 
&
\left\{
\begin{array}{l }
 \rho = \sqrt{x^2 + y^2} \\
 \theta = \arctan(\frac{y}{x}) \\
\end{array}
\right. 
\\
\end{array}
$$

\begin{itemize}
\item le vecteur dans le rep\`ere $(\vec{i},\vec{j})$ est $\vec{u}_{\rho} = \cos\theta \vec{i} + \sin\theta\vec{j}$ et $\vec{u}_{\theta} = -\sin\theta \vec{i} + \cos\theta\vec{j}$
\item Le vecteur position d'un point est $\vec{r}(t) = \rho(t)\vec{u}_{\rho}(t)$
\item Le vecteur vitesse est $\vec{v} = \frac{d\vec{r}}{dt} = \dot{\rho}(t)\vec{u}_{\rho} + \rho\dot{\theta}(t)\vec{u}_{\theta}$
\end{itemize}


\subsection*{Exo I}
\subsubsection*{I.1.a}
$$
\left\{
\begin{array}{l }
	x = 1 \cos(30) =  \frac{\sqrt{3}}{2}\, cm\\
	y = 1 \sin(30) =  \frac{1}{2}\, cm\\
\end{array}
\right. 
$$

\subsubsection*{I.1.b}
$$
\left\{
\begin{array}{l }
	x = 20 \cos(-30) =  20 \frac{\sqrt{3}}{2} = 10\sqrt{3}\, mm\\
	y = 20 \sin(-30) =  -20 \frac{1}{2} = -10\, mm\\
\end{array}
\right. 
$$

\subsubsection*{I.1.c}
$$
\left\{
\begin{array}{l }
	x = 8 \cos(120) =  -8 \frac{1}{2} = 4\, mm\\
	y = 8 \sin(120) =  8 \frac{\sqrt{3}}{2} = 4\sqrt{3}\, mm\\
\end{array}
\right. 
$$

\subsubsection*{I.1.d}
$$
\left\{
\begin{array}{l }
	x = 3 \cos(120) =  -3 \frac{1}{2} = -\frac{3}{2}\, cm\\
	y = 3 \sin(120) =  -3 \frac{\sqrt{3}}{2}\, cm\\
\end{array}
\right. 
$$

\subsubsection*{I.2.a}
$$
\left\{
\begin{array}{l }
 \rho = \sqrt{3^2 + 5^2} = \sqrt{34}\, cm\\
 \theta = \arctan(\frac{5}{3}) = 90 - \arctan(\frac{3}{5}) = 90 - 31 = 59^{\circ}\\
\end{array}
\right. 
$$

\subsubsection*{III}
\subsubsection*{III.1}

\begin{tikzpicture}[x=1cm,y=1cm]

  \draw[<->] (-4,0)--(4,0); % l'axe des abscisses
  \draw[<->] (0,-4)--(0,4); % l'axe des ordonnées

  \draw (0,0) -- (0:3cm) circle [radius=.1] node[right]{$t=0$};
  \draw (0,0) -- (25:3cm) circle [radius=.1] node[right]{$t=1$} ;
  \draw (0,0) -- (50:3cm) circle [radius=.1] node[right]{$t=2$};
  \draw (0,0) -- (75:3cm) circle [radius=.1] node[right]{$t=3$};
  \draw (0,0) -- (100:3cm) circle [radius=.1] node[right]{$t=4$};
    
\end{tikzpicture}

D\'eriv\'ees de $\rho$ et $\theta$
$$\dot{\rho}(t) = 0, \dot{\theta}(t) = \omega$$ 

La vitesse en coordonnn\'ees polaires
$$\vec{v} = \dot{\rho}(t)\vec{u}_{\rho} + \rho\dot{\theta}(t)\vec{u}_{\theta} = 0\vec{u}_{\rho} + \rho_{0}\omega\vec{u}_{\theta}$$

La vitesse en coordonnn\'ees cart\'esiennes
$$\vec{v} = 0\vec{u}_{\rho} + \rho_{0}\omega\vec{u}_{\theta} = 0(\cos\theta \vec{i} + \sin\theta\vec{j}) + \rho_{0}\omega(-\sin\theta \vec{i} + \cos\theta\vec{j}) = \rho_{0}\omega(-\sin\theta \vec{i} + \cos\theta\vec{j})$$



\end{document}

