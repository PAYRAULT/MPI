\documentclass[]{book}

%These tell TeX which packages to use.
\usepackage{array,epsfig}
\usepackage{amsmath}
\usepackage{amsfonts}
\usepackage{amssymb}
\usepackage{amsxtra}
\usepackage{amsthm}
\usepackage{mathrsfs}
\usepackage{color}

%Here I define some theorem styles and shortcut commands for symbols I use often
\theoremstyle{definition}
\newtheorem{defn}{Definition}
\newtheorem{thm}{Theorem}
\newtheorem{cor}{Corollary}
\newtheorem*{rmk}{Remark}
\newtheorem{lem}{Lemma}
\newtheorem*{joke}{Joke}
\newtheorem{ex}{Example}
\newtheorem*{soln}{Solution}
\newtheorem{prop}{Proposition}

\newcommand{\lra}{\longrightarrow}
\newcommand{\ra}{\rightarrow}
\newcommand{\surj}{\twoheadrightarrow}
\newcommand{\graph}{\mathrm{graph}}
\newcommand{\bb}[1]{\mathbb{#1}}
\newcommand{\Z}{\bb{Z}}
\newcommand{\Q}{\bb{Q}}
\newcommand{\R}{\bb{R}}
\newcommand{\C}{\bb{C}}
\newcommand{\N}{\bb{N}}
\newcommand{\M}{\mathbf{M}}
\newcommand{\m}{\mathbf{m}}
\newcommand{\MM}{\mathscr{M}}
\newcommand{\HH}{\mathscr{H}}
\newcommand{\Om}{\Omega}
\newcommand{\Ho}{\in\HH(\Om)}
\newcommand{\bd}{\partial}
\newcommand{\del}{\partial}
\newcommand{\bardel}{\overline\partial}
\newcommand{\textdf}[1]{\textbf{\textsf{#1}}\index{#1}}
\newcommand{\img}{\mathrm{img}}
\newcommand{\ip}[2]{\left\langle{#1},{#2}\right\rangle}
\newcommand{\inter}[1]{\mathrm{int}{#1}}
\newcommand{\exter}[1]{\mathrm{ext}{#1}}
\newcommand{\cl}[1]{\mathrm{cl}{#1}}
\newcommand{\ds}{\displaystyle}
\newcommand{\vol}{\mathrm{vol}}
\newcommand{\cnt}{\mathrm{ct}}
\newcommand{\osc}{\mathrm{osc}}
\newcommand{\LL}{\mathbf{L}}
\newcommand{\UU}{\mathbf{U}}
\newcommand{\support}{\mathrm{support}}
\newcommand{\AND}{\;\wedge\;}
\newcommand{\OR}{\;\vee\;}
\newcommand{\Oset}{\varnothing}
\newcommand{\st}{\ni}
\newcommand{\wh}{\widehat}

%Pagination stuff.
\setlength{\topmargin}{-.3 in}
\setlength{\oddsidemargin}{0in}
\setlength{\evensidemargin}{0in}
\setlength{\textheight}{9.in}
\setlength{\textwidth}{6.5in}
\pagestyle{empty}



\begin{document}

\subsection*{Exercice 1}


\subsection*{Exercice 2}
\subsubsection*{Exercice 2.1}
$$\langle u_1|u_1 \rangle  = \frac{1}{3}*\frac{1}{3} + \frac{2}{3}*\frac{2}{3} + \frac{2}{3}*\frac{2}{3} = \frac{1}{9}+\frac{4}{9}+\frac{4}{9} = 1$$

$$\langle u_1|u_2 \rangle  = \frac{1}{3}*\frac{2}{3} + \frac{2}{3}*\frac{-2}{3} + \frac{2}{3}*\frac{1}{3} = \frac{2}{9}+\frac{-4}{9}+\frac{2}{9} = 0$$

$$\langle u_1|u_3 \rangle  = \frac{1}{3}*\frac{2}{3} + \frac{2}{3}*\frac{1}{3} + \frac{2}{3}*\frac{-2}{3} = \frac{2}{9}+\frac{2}{9}+\frac{-4}{9} = 0$$

$$\langle u_2|u_2 \rangle  = \frac{2}{3}*\frac{2}{3} + \frac{-2}{3}*\frac{-2}{3} + \frac{1}{3}*\frac{1}{3} = \frac{4}{9}+\frac{4}{9}+\frac{1}{9} = 1$$

$$\langle u_2|u_3 \rangle  = \frac{2}{3}*\frac{2}{3} + \frac{-2}{3}*\frac{1}{3} + \frac{1}{3}*\frac{-2}{3} = \frac{4}{9}+\frac{-2}{9}+\frac{-2}{9} = 0$$

$$\langle u_3|u_3 \rangle  = \frac{2}{3}*\frac{2}{3} + \frac{1}{3}*\frac{1}{3} + \frac{-2}{3}*\frac{-2}{3} = \frac{4}{9}+\frac{1}{9}+\frac{4}{9} = 1$$

\subsubsection*{Exercice 2.2}
$C = (u_1, u_2 , u_3)$ est une base orthonorm\'ee de $\R^3$, si:
\begin{itemize}
\item tous les $u_i$ sont unitaires. Un vecteur $u$ est unitaire si $||u|| = \sqrt{\langle u|u \rangle } =1$. \`A la question 1, nous avons montr\'e que $\langle u_1|u_1 \rangle  = \langle u_2|u_2 \rangle  = \langle u_3|u_3 \rangle  = 1$. Donc tous les $u_i$ de $C$ sont unitaires.
\item $C$ est orthogonale dans $\R^3$, si $\forall i,j \in \{1,2,3\}, i < j \implies \langle u_i|u_j \rangle  = 0$. \`A la question 1, nous avons montr\'e que $\langle u_1|u_2 \rangle  = \langle u_1|u_3 \rangle  = \langle u_2|u_3 \rangle  = 0$. Donc $C$ est orhtogonal.
\end{itemize}

Par cons\'equent, $C$ est une base orthonorm\'ee de $\R^3$. 

\subsubsection*{Exercice 2.3}
On exprime le vecteur $x$ dans la base $\mathcal{C}$ comme 
$$x = \begin{pmatrix} \langle x|u_1 \rangle \\ \langle x|u_2 \rangle  \\ \langle x|u_3 \rangle  \end{pmatrix}_{\mathcal{C}}$$

$$x = \begin{pmatrix} 1.\frac{1}{3}+ 1.\frac{2}{3} + 0.\frac{2}{3} \\ 1.\frac{2}{3}+ 1.\frac{-2}{3} + 0.\frac{1}{3} \\ 1.\frac{2}{3}+ 1.\frac{1}{3} + 0.\frac{-2}{3} \end{pmatrix}_{\mathcal{C}}$$

$$x = \begin{pmatrix} 1 \\ 0 \\ 1 \end{pmatrix}_{\mathcal{C}}$$



\subsection*{Exercice 3}
La famille $(y_1,y_2)$ orthogonale associ\'ee aux vecteurs $x_1$ et $x_2$ de $\R^2$ est contruite pa r\'ecurrence
$$y_1 = \frac{x_1}{||x_1||}$$
Supposons\ que\ la \ famille\ $(y_1, ... y_{n-1})$ soit \ construite. 
$$y_n = \frac{x_n-\sum_{k=1}^{n-1}\langle x_n|y_k \rangle y_k}{||x_n-\sum_{k=1}^{n-1}\langle x_n|y_k \rangle y_k||}$$

Donc
$$y_1 = \frac{x_1}{||x_1||} = \frac{x_1}{\sqrt{\langle x_1 | x_1 \rangle}} = \frac{x_1}{\sqrt{1.1+(-1).(-1)}} = \frac{x_1}{\sqrt{2}} = \begin{pmatrix} \frac{1}{\sqrt{2}}\\ \frac{-1}{\sqrt{2}} \end{pmatrix}$$

Et

$$y_2 = \frac{x_2- \langle x_2|y_1 \rangle y_1}{||x_2- \langle x_2|y_1 \rangle y_1||}$$
$$x_2 - \langle x_2|y_1 \rangle y_1 = x_2 - (1.\frac{1}{\sqrt{2}} + 0.\frac{-1}{\sqrt{2}} )y_1 =  \begin{pmatrix} 1 \\ 0 \end{pmatrix} - \frac{1}{\sqrt{2}} \begin{pmatrix} \frac{1}{\sqrt{2}}\\ \frac{-1}{\sqrt{2}} \end{pmatrix} = \begin{pmatrix} \frac{1}{2} \\ \frac{1}{2} \end{pmatrix}$$
$$||x_2 - \langle x_2|y_1 \rangle y_1|| = ||\begin{pmatrix} \frac{1}{2} \\ \frac{1}{2} \end{pmatrix}|| = \sqrt{\frac{1}{2}*\frac{1}{2} + \frac{1}{2}*\frac{1}{2}} = \frac{1}{\sqrt{2}}$$
Donc 
$$y_2 = \frac{\begin{pmatrix} \frac{1}{2} \\ \frac{1}{2} \end{pmatrix}}{\frac{1}{\sqrt{2}}} = \begin{pmatrix} \frac{1}{\sqrt{2}} \\ \frac{1}{\sqrt{2}} \end{pmatrix}$$


QED


\end{document}

