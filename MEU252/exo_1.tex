\documentclass[]{book}

%These tell TeX which packages to use.
\usepackage{array,epsfig}
\usepackage{amsmath}
\usepackage{amsfonts}
\usepackage{amssymb}
\usepackage{amsxtra}
\usepackage{amsthm}
\usepackage{mathrsfs}
\usepackage{color}
\usepackage{tikz}
\usepackage{graphicx}

%Here I define some theorem styles and shortcut commands for symbols I use often
\theoremstyle{definition}
\newtheorem{defn}{Definition}
\newtheorem{thm}{Theorem}
\newtheorem{cor}{Corollary}
\newtheorem*{rmk}{Remark}
\newtheorem{lem}{Lemma}
\newtheorem*{joke}{Joke}
\newtheorem{ex}{Example}
\newtheorem*{soln}{Solution}
\newtheorem{prop}{Proposition}

\newcommand{\lra}{\longrightarrow}
\newcommand{\ra}{\rightarrow}
\newcommand{\surj}{\twoheadrightarrow}
\newcommand{\graph}{\mathrm{graph}}
\newcommand{\bb}[1]{\mathbb{#1}}
\newcommand{\Z}{\bb{Z}}
\newcommand{\Q}{\bb{Q}}
\newcommand{\R}{\bb{R}}
\newcommand{\C}{\bb{C}}
\newcommand{\N}{\bb{N}}
\newcommand{\M}{\mathbf{M}}
\newcommand{\m}{\mathbf{m}}
\newcommand{\MM}{\mathscr{M}}
\newcommand{\HH}{\mathscr{H}}
\newcommand{\Om}{\Omega}
\newcommand{\Ho}{\in\HH(\Om)}
\newcommand{\bd}{\partial}
\newcommand{\del}{\partial}
\newcommand{\bardel}{\overline\partial}
\newcommand{\textdf}[1]{\textbf{\textsf{#1}}\index{#1}}
\newcommand{\img}{\mathrm{img}}
\newcommand{\ip}[2]{\left\langle{#1},{#2}\right\rangle}
\newcommand{\inter}[1]{\mathrm{int}{#1}}
\newcommand{\exter}[1]{\mathrm{ext}{#1}}
\newcommand{\cl}[1]{\mathrm{cl}{#1}}
\newcommand{\ds}{\displaystyle}
\newcommand{\vol}{\mathrm{vol}}
\newcommand{\cnt}{\mathrm{ct}}
\newcommand{\osc}{\mathrm{osc}}
\newcommand{\LL}{\mathbf{L}}
\newcommand{\UU}{\mathbf{U}}
\newcommand{\support}{\mathrm{support}}
\newcommand{\AND}{\;\wedge\;}
\newcommand{\OR}{\;\vee\;}
\newcommand{\Oset}{\varnothing}
\newcommand{\st}{\ni}
\newcommand{\wh}{\widehat}

%Pagination stuff.
\setlength{\topmargin}{-.3 in}
\setlength{\oddsidemargin}{0in}
\setlength{\evensidemargin}{0in}
\setlength{\textheight}{9.in}
\setlength{\textwidth}{6.5in}
\pagestyle{empty}



\begin{document}

\subsection*{Exercice 1}
\subsubsection*{1.1}

\includegraphics{scheme.jpg}

Donc 
$$ \cos(\overrightarrow{v}, \overrightarrow{OP}) = \cos(\overrightarrow{v}, \overrightarrow{OP'})$$
$$ \sin(\overrightarrow{v}, \overrightarrow{OP}) = -\sin(\overrightarrow{v}, \overrightarrow{OP'})$$
Ou
$$ \overrightarrow{OP}.\overrightarrow{v} = \overrightarrow{OP'}.\overrightarrow{v} $$
$$ \overrightarrow{OP} \land \overrightarrow{v} = -\overrightarrow{OP'} \land \overrightarrow{v} $$

En equation:
$$
\left\{
\begin{array}{l l}
x_v.(x-x_0) + y_v.(y-y_0) + z_v.(z-z_0) & =  x_v.(x'-x_0) + y_v.(y'-y_0) + z_v.(z'-z_0) \\
y_v.(z-z_0) - z_v.(y-y_0) & =  -(y_v.(z'-z_0) - z_v.(y'-y_0)) \\
z_v.(x-x_0) - x_v.(z-z_0) & =  -(z_v.(x'-x_0) - x_v.(z'-z_0)) \\
x_v.(y-y_0) - y_v.(x-x_0) & =  -(y_v.(y'-y_0) - y_v.(x'-x_0)) \\
\end{array}
\right.
$$

La droite passe par le point $O = (0,0,0)$ et de vecteur directeur $D = Vect{(1,1,0)^T}$.

$$
\left\{
\begin{array}{l l}
x + y  & =  x' + y' \\
z & =  -z' \\
- z & =  z' \\
y - x & =  x' - y' \\
\end{array}
\right.
$$

Donc
$$
\begin{pmatrix} x \\ y \\ z \end{pmatrix} = \begin{pmatrix} 0 & 1 & 0 \\ 1 & 0 & 0 \\ 0 & 0 & -1 \end{pmatrix}.\begin{pmatrix} x' \\ y' \\ z' \end{pmatrix}
$$

\subsubsection*{1.2}
Matrice $Q$ orthogonale au plan $P$ d'\'equation $x-y+z=0$.
Soit $n = i-j+k$ un vecteur normal \`a P. La sym\'etrie orthogonale par rapport a $P$ est 
$$s(x) = x -2 \frac{(x|n)}{||n||^2}n$$

On prends $x=(x_1, x_2, x_3)$,  $||n||^2 = 3$, $n=(1,-1,1)$ et $(x|n) = x_1-x_2+x_3$. Donc
$$s(x) = (x_1, x_2, x_3) - \frac{2}{3}(x_1-x_2+x_3)(1,-1,1) $$
$$ = \frac{1}{3}(3x_1, 3x_2, 3x_3) - \frac{1}{3} (2x_1-2x_2+2x_3, -2x_1+2x_2-2x_3, 2x_1-2x_2+2x_3)$$
$$ = \frac{1}{3}(x_1+2x_2-2x_3, 2x_1+x_2+2x_3, -2x_1+2x_2+x_3)$$
Donc
$$ Q = \frac{1}{3}\begin{pmatrix} 1 & 2 & -2 \\ 2 & 1 & 2 \\ -2 & 2 & 1 \end{pmatrix} $$


\subsubsection*{1.3}
$P$ est une synm\'etrie orthogonale ssi $P.P^T = Id$.
$$
P.P^T = \begin{pmatrix} 0 & 1 & 0 \\ 1 & 0 & 0 \\ 0 & 0 & -1 \end{pmatrix}.\begin{pmatrix} 0 & 1 & 0 \\ 1 & 0 & 0 \\ 0 & 0 & -1 \end{pmatrix} = \begin{pmatrix} 1 & 0 & 0 \\ 0 & 1 & 0 \\ 0 & 0 & 1 \end{pmatrix}
$$

donc
$P$ et une sym\'etrie orthogonale.


$Q$ est une synm\'etrie orthogonale ssi $Qc = Id$.
$$Q.Q^T = \frac{1}{9}\begin{pmatrix} 1 & 2 & -2 \\ 2 & 1 & 2 \\ -2 & 2 & 1 \end{pmatrix}.\begin{pmatrix} 1 & 2 & -2 \\ 2 & 1 & 2 \\ -2 & 2 & 1 \end{pmatrix} = \frac{1}{9}\begin{pmatrix} 1+4+4 & 2+2-4 & -2+4-2 \\ 2+2-4 & 4+1+4 & -4+2+2 \\ -2+4-2 & -4+2+2 & 4+4+1 \end{pmatrix} $$
$$= \frac{1}{9}\begin{pmatrix} 9 & 0 & 0 \\ 0 & 9 & 0 \\ 0 & 0 & 9 \end{pmatrix} = Id$$
et 
$Q$ et une sym\'etrie orthogonale.

QED


\end{document}

