\documentclass[]{book}

%These tell TeX which packages to use.
\usepackage{array,epsfig}
\usepackage{amsmath}
\usepackage{amsfonts}
\usepackage{amssymb}
\usepackage{amsxtra}
\usepackage{amsthm}
\usepackage{mathrsfs}
\usepackage{color}

%Here I define some theorem styles and shortcut commands for symbols I use often
\theoremstyle{definition}
\newtheorem{defn}{Definition}
\newtheorem{thm}{Theorem}
\newtheorem{cor}{Corollary}
\newtheorem*{rmk}{Remark}
\newtheorem{lem}{Lemma}
\newtheorem*{joke}{Joke}
\newtheorem{ex}{Example}
\newtheorem*{soln}{Solution}
\newtheorem{prop}{Proposition}

\newcommand{\lra}{\longrightarrow}
\newcommand{\ra}{\rightarrow}
\newcommand{\surj}{\twoheadrightarrow}
\newcommand{\graph}{\mathrm{graph}}
\newcommand{\bb}[1]{\mathbb{#1}}
\newcommand{\Z}{\bb{Z}}
\newcommand{\Q}{\bb{Q}}
\newcommand{\R}{\bb{R}}
\newcommand{\C}{\bb{C}}
\newcommand{\N}{\bb{N}}
\newcommand{\M}{\mathbf{M}}
\newcommand{\m}{\mathbf{m}}
\newcommand{\MM}{\mathscr{M}}
\newcommand{\HH}{\mathscr{H}}
\newcommand{\Om}{\Omega}
\newcommand{\Ho}{\in\HH(\Om)}
\newcommand{\bd}{\partial}
\newcommand{\del}{\partial}
\newcommand{\bardel}{\overline\partial}
\newcommand{\textdf}[1]{\textbf{\textsf{#1}}\index{#1}}
\newcommand{\img}{\mathrm{img}}
\newcommand{\ip}[2]{\left\langle{#1},{#2}\right\rangle}
\newcommand{\inter}[1]{\mathrm{int}{#1}}
\newcommand{\exter}[1]{\mathrm{ext}{#1}}
\newcommand{\cl}[1]{\mathrm{cl}{#1}}
\newcommand{\ds}{\displaystyle}
\newcommand{\vol}{\mathrm{vol}}
\newcommand{\cnt}{\mathrm{ct}}
\newcommand{\osc}{\mathrm{osc}}
\newcommand{\LL}{\mathbf{L}}
\newcommand{\UU}{\mathbf{U}}
\newcommand{\support}{\mathrm{support}}
\newcommand{\AND}{\;\wedge\;}
\newcommand{\OR}{\;\vee\;}
\newcommand{\Oset}{\varnothing}
\newcommand{\st}{\ni}
\newcommand{\wh}{\widehat}

%Pagination stuff.
\setlength{\topmargin}{-.3 in}
\setlength{\oddsidemargin}{0in}
\setlength{\evensidemargin}{0in}
\setlength{\textheight}{9.in}
\setlength{\textwidth}{6.5in}
\pagestyle{empty}



\begin{document}

\subsection*{Exercice a}
Une matrice $P \in O(n)$ si $P^T = P^{-1}$. Donc la matrice $P^{-1} \in O(n)$ si on arrive \`a montrer que $(P^{-1})^T = (P^{-1})^{-1}$. On a $(P^{-1})^{-1} = Id$ (par d\'efinition) et comme $P \in O(n)$ on a $P^T = P^{-1}$. Donc $ (P^{-1})^T = (P^T)^T = Id$. Par cons\'equent $p \in O(n) \implies P^{-1} \in O(n)$. 

\subsection*{Exercice b}
Soit $\lambda$ la valeur propre de $u$, et $x$ un vecteur propre non nul, alors $u(x)=\lambda x$, et comme $u$ est une isom\'etrie, elle conserve la norme, $\Vert x \Vert = \Vert u(x) \Vert = \vert \lambda \vert \Vert x \Vert$, d'o\`u $\vert \lambda \vert =1$ et $\lambda = 1$ ou -1.

\subsection*{Exercice c}
\subsubsection*{Exercice c-2}
C'est une sym\'etrie axiale car elle est de la forme $\begin{pmatrix} \cos \theta & \sin \theta \\ -\sin \theta & \cos \theta \end{pmatrix}$. Comme $P\begin{pmatrix} 3 \\ 2 \end{pmatrix} = \begin{pmatrix} 3 \\ 2 \end{pmatrix}$ l'axe de la sym\'etrie est $\begin{pmatrix} 3 \\ 2 \end{pmatrix}$. 

\subsection*{Exercice d}
\subsubsection*{Exercice d-1}
Une rotation d'angle $-\frac{\pi}{3}$ de centre $O$ est 
$$
R = $\begin{pmatrix} \cos -\frac{\pi}{3} & -\sin -\frac{\pi}{3} \\ \sin -\frac{\pi}{3} & \cos -\frac{\pi}{3} \end{pmatrix}$
$$



\subsubsection*{Exercice d-2}
$$
\left\{
\begin{array}{l l}
x' = \\
y' = \\
\end{array}
\right.
$$

QED


\end{document}

