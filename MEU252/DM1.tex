\documentclass[]{book}

%These tell TeX which packages to use.
\usepackage{array,epsfig}
\usepackage{amsmath}
\usepackage{amsfonts}
\usepackage{amssymb}
\usepackage{amsxtra}
\usepackage{amsthm}
\usepackage{mathrsfs}
\usepackage{color}
\usepackage{tikz}
\usepackage{graphicx}

%Here I define some theorem styles and shortcut commands for symbols I use often
\theoremstyle{definition}
\newtheorem{defn}{Definition}
\newtheorem{thm}{Theorem}
\newtheorem{cor}{Corollary}
\newtheorem*{rmk}{Remark}
\newtheorem{lem}{Lemma}
\newtheorem*{joke}{Joke}
\newtheorem{ex}{Example}
\newtheorem*{soln}{Solution}
\newtheorem{prop}{Proposition}

\newcommand{\lra}{\longrightarrow}
\newcommand{\ra}{\rightarrow}
\newcommand{\surj}{\twoheadrightarrow}
\newcommand{\graph}{\mathrm{graph}}
\newcommand{\bb}[1]{\mathbb{#1}}
\newcommand{\Z}{\bb{Z}}
\newcommand{\Q}{\bb{Q}}
\newcommand{\R}{\bb{R}}
\newcommand{\C}{\bb{C}}
\newcommand{\N}{\bb{N}}
\newcommand{\M}{\mathbf{M}}
\newcommand{\m}{\mathbf{m}}
\newcommand{\MM}{\mathscr{M}}
\newcommand{\HH}{\mathscr{H}}
\newcommand{\Om}{\Omega}
\newcommand{\Ho}{\in\HH(\Om)}
\newcommand{\bd}{\partial}
\newcommand{\del}{\partial}
\newcommand{\bardel}{\overline\partial}
\newcommand{\textdf}[1]{\textbf{\textsf{#1}}\index{#1}}
\newcommand{\img}{\mathrm{img}}
\newcommand{\ip}[2]{\left\langle{#1},{#2}\right\rangle}
\newcommand{\inter}[1]{\mathrm{int}{#1}}
\newcommand{\exter}[1]{\mathrm{ext}{#1}}
\newcommand{\cl}[1]{\mathrm{cl}{#1}}
\newcommand{\ds}{\displaystyle}
\newcommand{\vol}{\mathrm{vol}}
\newcommand{\cnt}{\mathrm{ct}}
\newcommand{\osc}{\mathrm{osc}}
\newcommand{\LL}{\mathbf{L}}
\newcommand{\UU}{\mathbf{U}}
\newcommand{\support}{\mathrm{support}}
\newcommand{\AND}{\;\wedge\;}
\newcommand{\OR}{\;\vee\;}
\newcommand{\Oset}{\varnothing}
\newcommand{\st}{\ni}
\newcommand{\wh}{\widehat}

%Pagination stuff.
\setlength{\topmargin}{-.3 in}
\setlength{\oddsidemargin}{0in}
\setlength{\evensidemargin}{0in}
\setlength{\textheight}{9.in}
\setlength{\textwidth}{6.5in}
\pagestyle{empty}



\begin{document}

https://fr.wikipedia.org/wiki/Courbe\_du\_chien

\subsection*{Exercice 1}
On a $\alpha(t) = (x(t),y(t))$, $\beta(t) = (a,z(t))$, $\alpha(0) =(0,0)$ et $\alpha'(0) =(1,0)$. Donc on a $\beta(0) = (a,0)$ car $\alpha'(t)$ est orient\'e vers $\beta(t)$.

\subsubsection*{1.1 - Calculer $\alpha'(t)$}
Comme $\alpha'(t)$ est sur la droite allant de $(x(t),y(t))$ vers $(a,z(t))$, on a $\frac{dy(t)}{dx(t)} = \frac{z(t)-y(t)}{a-x(t)}$ et $\alpha'(t) = (1, \frac{z(t)-y(t)}{a-x(t)})$. 

\subsubsection*{1.2 - Calculer $\beta(t)$}
Le point $\beta(t)$ est \`a l'intersection de la droite verticale passant par $(a,0)$ et de la droite passant par le point $(x(t), y(t))$ et de coefficient directeur $\frac{dy(t)}{dx(t)}$. La droite s'\'ecrit $y = \frac{dy(t)}{dx(t)}.x + c$ avec $c = y(t) - \frac{dy(t)}{dx(t)}.x(t)$ \\
$\beta$ se deplacant sur la droite verticale $(a,0)$ et est au point (a,0) \`a $t_0$, on a $z(t) = y$
Donc $z(t) = y = \frac{dy(t)}{dx(t)}.a +  y(t) - \frac{dy(t)}{dx(t)}.x(t) = \frac{dy(t)}{dx(t)}(a - x(t)) + y(t)$.\\
On a $\beta(t) = (a,\alpha'(t)(a - x(t)) + y(t))$

\subsubsection*{1.3 - Calculer $\beta'(t)$}
$\beta$ se d\'eplacant sur un droite verticale, on a:
$$\beta'(t) = (0, (\frac{dy(t)}{dx(t)}(a - x(t)) + y(t))') = (0, -\frac{dy(t)}{dx(t)} + (a-x(t))\alpha''(t) + \frac{dy(t)}{dx(t)} = (0,(a-x(t))\alpha''(t))$$

\subsubsection*{1.4 - relation entre $\beta'(t)$ et $\alpha'(t)$}
La vitesse $\alpha'(t)$ est toujours proportionnelle \`a la vitesse $\lVert  \beta'(t) \rVert  = \frac{1}{k}.\lVert  \alpha'(t) \rVert $. Comme k=1 on a, $\lVert  \beta'(t) \rVert  = \lVert  \alpha'(t) \rVert $
La distance parcourue par $\beta$ est \'egalement proportionelle \`a la distance parcourue par $\alpha$.

$$
\sqrt{0^2 + ((a-x(t))\alpha''(t))^2} = (a-x(t))\alpha''(t) = (a-x(t))\frac{dy(t)}{d^2x(t)}
$$

$$
\sqrt{(\frac{z(t)-y(t)}{(a-x(t))})^2 + 1^2} = \sqrt{\left(\frac{\frac{dy(t)}{dx(t)}(a - x(t)) + y(t) -y(t)}{(a-x(t))}\right)^2 + 1^2} = \sqrt{\left(\frac{dy(t)}{dx(t)}\right)^2+1}
$$

Il reste a r\'esoudre l'\'equation diff\'erentielle:
$$
(a-x(t))\frac{dy(t)}{d^2x(t)} = \sqrt{\left(\frac{dy(t)}{dx(t)}\right)^2+1}
$$

Donc, renomage $y' = \frac{dy(t)}{dx(t)}$ et s\'eparation des variables:
$$
\frac{y'}{\sqrt{y'^2+1}} = \frac{dx}{a-x(t)}
$$
En int\'egrant on a:
$$
\ln(y'+{\sqrt  {1+y'^{2}}})=-\ln(a-x)+C
$$
\`A l'instant $t=0$, les conditions initiales sont $x=0$ et $y'=0$, ce qui donne pour la constante :
$$
C=\ln(a)
$$

Donc
$$
\ln(y'+{\sqrt  {1+y'^{2}}})=-\ln(a-x)+\ln(a) = \ln \left(\frac{a}{a-x}\right)
$$
ou
$$
y'+\sqrt  {1+y'^{2}}=\frac{a}{a-x}
$$

En faisant l'\'egalit\'e des oppos\'es on a:
$$
-\ln(y'+{\sqrt{1+y'^{2}}})= \ln\left(\frac{1}{y'+\sqrt{1+y'^{2}}}\right) = \ln\left(\frac{y'-\sqrt{1-y'^{2}}}{(y'+\sqrt{1+y'^{2}})(y'-\sqrt{1+y'^{2}})}\right) = \ln \left(\frac{y'-\sqrt{1- y'^{2}}}{y'^2-(1+y'^{2})}\right)
$$
$$
= \ln({-(y'-\sqrt  {1+y'^{2}}})) = - \ln \left(\frac{a}{a-x}\right) = \ln \left(\frac{a-x}{a}\right)
$$ 
Ou
$$
y'-\sqrt{1+y'^{2}}=-\frac{a}{a-x}
$$

En additionnant les deux on obtient
$$
2y' = \frac{a}{a-x} + \frac{a-x}{a}
$$

Enfin en int\'egrant:
$$
y={\frac  {x^{2}-2ax}{4a}}+{\frac  {a}{2}}\ln {{\frac{a}{a-x}}}.
$$

Cette \'equation est \'equivalente \`a celle \`a montrer.
$$
y(t)=\frac{a}{4}\left(\left(\frac{a-x(t)}{a}\right)^2- 1 - 2\ln\left(\frac{a-t}{a}\right)\right) = \frac{a}{4}\left(\frac{a^2-2ax(t)+x^2(t)}{a^2} - \frac{a^2}{a^2} - 2\ln\left(\frac{a-t}{a}\right)\right)
$$
$$
= \frac{-2ax(t)+x^2(t)}{4a}  - \frac{a}{2}\ln\left(\frac{a-t}{a}\right)
$$


\subsection*{Exercice 2}
Lorsque $k \neq 1$, il faut r\'esoudre l'\'equation (voir question pr\'ec\'edente).
$$
(a-x(t))\frac{dy(t)}{d^2x(t)} = \frac{1}{k}\sqrt{\left(\frac{dy(t)}{dx(t)}\right)^2+1}
$$

Donc, on recommance renomage $y' = \frac{dy(t)}{dx(t)}$ et s\'eparation des variables:
$$
\frac{y'}{\sqrt{y'^2+1}} = \frac{1}{k}.\frac{dx}{a-x(t)}
$$

En int\'egrant on a:
$$
\ln(y'+{\sqrt  {1+y'^{2}}})=-\frac{1}{k}\ln(a-x)+C
$$
\`A l'instant $t=0$, les conditions initiales sont $x=0$ et $y'=0$, ce qui donne pour la constante :
$$
C=\frac{1}{k}\ln(a)
$$

Donc
$$
\ln(y'+{\sqrt  {1+y'^{2}}})=-\frac{1}{k}\ln(a-x)+\frac{1}{k}\ln(a) = \ln \left(\sqrt[k]{\frac{a}{a-x}}\right)
$$
ou
$$
y'+\sqrt  {1+y'^{2}}=\sqrt[k]{\frac{a}{a-x}}
$$

De la m\^eme facon que pour la question 1, on obtient
$$
y'-\sqrt  {1+y'^{2}}=-\sqrt[k]{\frac{a-x}{a}}
$$

En additionnant les 2, on a:
$$
2y' = \sqrt[k]{\frac{a}{a-x}} - \sqrt[k]{\frac{a-x}{a}}
$$

Enfin en int\'egrant:
$$
y={\frac{k(a-x)}{2}}\left({\frac{1}{k+1}}\left({\frac{a-x}{a}}\right)^{{{\frac{1}{k}}}}+{\frac{1}{1-k}}\left({\frac{a-x}{a}}\right)^{{-{\frac  {1}{k}}}}\right)+C
$$

Pour Pour x = 0 ; y = 0 on a 
$$
C=\frac  {ka}{k^{2}-1}
$$

Donc
$$y={\frac{k(a-x)}{2}}\left({\frac{1}{k+1}}\left({\frac{a-x}{a}}\right)^{{{\frac{1}{k}}}}+{\frac{1}{1-k}}\left({\frac{a-x}{a}}\right)^{{-{\frac  {1}{k}}}}\right)+ \frac  {ka}{k^{2}-1}
$$

$$y={\frac{k(a-x)}{2(k+1)}}\left({\frac{a-x}{a}}\right)^{{{\frac{1}{k}}}}+{\frac{k(a-x)}{2(1-k)}}\left({\frac{a-x}{a}}\right)^{{-{\frac  {1}{k}}}} + \frac  {ka}{k^{2}-1}
$$

$$y={\frac{ka(a-x)}{2a(k+1)}}\left({\frac{a-x}{a}}\right)^{{{\frac{1}{k}}}}+{\frac{ka(a-x)}{2a(1-k)}}\left({\frac{a-x}{a}}\right)^{{-{\frac  {1}{k}}}} + \frac  {ka}{k^{2}-1}
$$

$$y={\frac{ka}{2(k+1)}}\left({\frac{a-x}{a}}\right)^{{{1+\frac{1}{k}}}}+{\frac{ka}{2(1-k)}}\left({\frac{a-x}{a}}\right)^{{1-{\frac  {1}{k}}}} + \frac  {ka}{k^{2}-1}
$$

Comme $a =1$, on obtient la m\^eme \'equation que celle demand\'ee:
$$
y(t) = \frac{k}{k^2-1} + \frac{k\,(1-t)^{1+1/k}}{2\,(1+k)}+\frac{k(1-t)^{1-1/k}}{2\,(1-k)}
$$



\subsection*{Exercice 3}
Le poursuivant rattrape le fugitif à l'instant $t$ si l'\'eqution $(x(t), y(t)) = (a, z(t))$ a une solution. Comme k=1, les deux ont touours la m\^eme vitesse. Par cons\'equent, la distance parcourue par le poursuivant est \'egale \`a la distance parcourue par le fugitif a tout instant. La distance minimale parcourue par le poursuivant \`a l'instant $t$ est \'egale \`a $\sqrt{(x(t)-x(t_0))^2 + (y(t)-y(t_0))^2}$ (ie ligne droite depuis $(x(t_0), y(t_0))$). La distance parcourue par le fugitif est \'egale \`a $\sqrt{(a-a)^2+(z(t)-z(t_0))^2}$. \`A l'instant $t_0$, le poursuivant est \`a $(0,0)$ et le fugitif \`a $(a,0)$ car le vecteur vitesse du poursuivant est $(1,0)$ et pointe toujours vers le fugitif. On a donc:
$$
\sqrt{x^2(t) + y^2(t)} = \sqrt{z^2(t)}
$$
On cherche l'instant $t$ tel que $(x(t), y(t)) = (a, z(t))$ donc
$$
\sqrt{a^2 + z^2(t)} = \sqrt{z^2(t)}
$$
Comme $a$ est diff\'erent de z\'ero, cette \'equation n'a pas de solution, donc le poursuivant ne peut pas rattraper le figutif lorsque $k=1$.


QED


\end{document}

