\documentclass[]{book}

%These tell TeX which packages to use.
\usepackage{array,epsfig}
\usepackage{amsmath}
\usepackage{amsfonts}
\usepackage{amssymb}
\usepackage{amsxtra}
\usepackage{amsthm}
\usepackage{mathrsfs}
\usepackage{color}
\usepackage{tikz}
\usepackage{graphicx}

%Here I define some theorem styles and shortcut commands for symbols I use often
\theoremstyle{definition}
\newtheorem{defn}{Definition}
\newtheorem{thm}{Theorem}
\newtheorem{cor}{Corollary}
\newtheorem*{rmk}{Remark}
\newtheorem{lem}{Lemma}
\newtheorem*{joke}{Joke}
\newtheorem{ex}{Example}
\newtheorem*{soln}{Solution}
\newtheorem{prop}{Proposition}

\newcommand{\lra}{\longrightarrow}
\newcommand{\ra}{\rightarrow}
\newcommand{\surj}{\twoheadrightarrow}
\newcommand{\graph}{\mathrm{graph}}
\newcommand{\bb}[1]{\mathbb{#1}}
\newcommand{\Z}{\bb{Z}}
\newcommand{\Q}{\bb{Q}}
\newcommand{\R}{\bb{R}}
\newcommand{\C}{\bb{C}}
\newcommand{\N}{\bb{N}}
\newcommand{\M}{\mathbf{M}}
\newcommand{\m}{\mathbf{m}}
\newcommand{\MM}{\mathscr{M}}
\newcommand{\HH}{\mathscr{H}}
\newcommand{\Om}{\Omega}
\newcommand{\Ho}{\in\HH(\Om)}
\newcommand{\bd}{\partial}
\newcommand{\del}{\partial}
\newcommand{\bardel}{\overline\partial}
\newcommand{\textdf}[1]{\textbf{\textsf{#1}}\index{#1}}
\newcommand{\img}{\mathrm{img}}
\newcommand{\ip}[2]{\left\langle{#1},{#2}\right\rangle}
\newcommand{\inter}[1]{\mathrm{int}{#1}}
\newcommand{\exter}[1]{\mathrm{ext}{#1}}
\newcommand{\cl}[1]{\mathrm{cl}{#1}}
\newcommand{\ds}{\displaystyle}
\newcommand{\vol}{\mathrm{vol}}
\newcommand{\cnt}{\mathrm{ct}}
\newcommand{\osc}{\mathrm{osc}}
\newcommand{\LL}{\mathbf{L}}
\newcommand{\UU}{\mathbf{U}}
\newcommand{\support}{\mathrm{support}}
\newcommand{\AND}{\;\wedge\;}
\newcommand{\OR}{\;\vee\;}
\newcommand{\Oset}{\varnothing}
\newcommand{\st}{\ni}
\newcommand{\wh}{\widehat}

%Pagination stuff.
\setlength{\topmargin}{-.3 in}
\setlength{\oddsidemargin}{0in}
\setlength{\evensidemargin}{0in}
\setlength{\textheight}{9.in}
\setlength{\textwidth}{6.5in}
\pagestyle{empty}



\begin{document}

https://fr.wikipedia.org/wiki/Courbe\_du\_chien

\subsection*{Exercice 1}
On a $\alpha(t) = (x(t),y(t))$, $\beta(t) = (a,z(t))$, $\alpha(0) =(0,0)$ et $\alpha'(0) =(1,0)$. Donc on a $\beta(0) = (a,0)$ car $\alpha'(t)$ est orient\'e vers $\beta(t)$.

\subsubsection*{1.1 - Calculer $\alpha'(t)$}
Comme $\alpha(t)$ est sur la droite allant de $(x(t),y(t))$ vers $(a,z(t))$, on a $\alpha'(t) = (a-x(t), z(t)-y(t))$. 

\subsubsection*{1.2 - Calculer $\beta(t)$}
Le point $\beta(t)$ est \`a l'intersection de la droite verticale passant par $(a,0)$ et de la droite passant par le point $(x(t), y(t))$ et de coefficient directeur $\alpha'(t)$. La droite s'\'ecrit $y = \alpha'(t).x + c$ avec $c = y(t) - \alpha'(t).x(t)$ \\
$\beta$ se deplacant sur la droite verticale $(a,0)$ et est au point (a,0) \`a $t_0$, on a $z(t) = y$
Donc $z(t) = y = \alpha'(t).a +  y(t) - \alpha'(t).x(t) = \alpha'(t)(a - x(t)) + y(t)$.\\
On a $\beta(t) = (a,\alpha'(t)(a - x(t)) + y(t))$

\subsubsection*{1.3 - Calculer $\beta'(t)$}
$\beta$ se d\'eplacant sur un droite verticale, on a:
$$\beta'(t) = (0, (\alpha'(t)(a - x(t)) + y(t)') = (0, -\alpha'(t) + (a-x(t))\alpha''(t) + y'(t))$$

\subsubsection*{1.4 - relation entre $\beta'(t)$ et $\alpha'(t)$}
La vitesse $\alpha'(t)$ est toujours proportionnelle \`a la vitesse $\lVert  \beta'(t) \rVert  = \frac{1}{k}.\lVert  \alpha'(t) \rVert $. Comme k=1 on a, $\lVert  \beta'(t) \rVert  = \lVert  \alpha'(t) \rVert $
La distance parcourue par $\beta$ est \'egalement proportionelle \`a la distance parcourue par $\alpha$.

$$
\sqrt{0^2 + -\alpha'(t) + (a-x(t))\alpha''(t) + y'(t))^2} = -\alpha'(t) + (a-x(t))\alpha''(t) + y'(t)
$$

$$
\sqrt{(z(t)-y(t))^2 + (a-x(t))^2} = \sqrt{(\alpha'(t)(a - x(t)) + y(t) -y(t))^2 + (a-x(t))^2} = (a-x(t))\sqrt{\alpha'^2(t)+1}
$$



QED


\end{document}

