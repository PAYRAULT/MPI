\documentclass[]{book}

%These tell TeX which packages to use.
\usepackage{array,epsfig}
\usepackage{amsmath}
\usepackage{amsfonts}
\usepackage{amssymb}
\usepackage{amsxtra}
\usepackage{amsthm}
\usepackage{mathrsfs}
\usepackage{color}
\usepackage[margin=2cm,top=2.5cm,headheight=16pt,headsep=0.1in,heightrounded]{geometry}
\usepackage{fancyhdr}
\pagestyle{fancy}
%\usepackage{tikz}
\usepackage{pgfplots}

%Here I define some theorem styles and shortcut commands for symbols I use often
\theoremstyle{definition}
\newtheorem{defn}{Definition}
\newtheorem{thm}{Theorem}
\newtheorem{cor}{Corollary}
\newtheorem*{rmk}{Remark}
\newtheorem{lem}{Lemma}
\newtheorem*{joke}{Joke}
\newtheorem{ex}{Example}
\newtheorem*{soln}{Solution}
\newtheorem{prop}{Proposition}

\newcommand{\lra}{\longrightarrow}
\newcommand{\ra}{\rightarrow}
\newcommand{\surj}{\twoheadrightarrow}
\newcommand{\graph}{\mathrm{graph}}
\newcommand{\bb}[1]{\mathbb{#1}}
\newcommand{\Z}{\bb{Z}}
\newcommand{\Q}{\bb{Q}}
\newcommand{\R}{\bb{R}}
\newcommand{\E}{\bb{E}}
\newcommand{\C}{\bb{C}}
\newcommand{\N}{\bb{N}}
\newcommand{\M}{\mathbf{M}}
\newcommand{\m}{\mathbf{m}}
\newcommand{\MM}{\mathscr{M}}
\newcommand{\HH}{\mathscr{H}}
\newcommand{\Om}{\Omega}
\newcommand{\Ho}{\in\HH(\Om)}
\newcommand{\bd}{\partial}
\newcommand{\del}{\partial}
\newcommand{\bardel}{\overline\partial}
\newcommand{\textdf}[1]{\textbf{\textsf{#1}}\index{#1}}
\newcommand{\img}{\mathrm{img}}
\newcommand{\ip}[2]{\left\langle{#1},{#2}\right\rangle}
\newcommand{\inter}[1]{\mathrm{int}{#1}}
\newcommand{\exter}[1]{\mathrm{ext}{#1}}
\newcommand{\cl}[1]{\mathrm{cl}{#1}}
\newcommand{\ds}{\displaystyle}
\newcommand{\vol}{\mathrm{vol}}
\newcommand{\cnt}{\mathrm{ct}}
\newcommand{\osc}{\mathrm{osc}}
\newcommand{\LL}{\mathbf{L}}
\newcommand{\UU}{\mathbf{U}}
\newcommand{\support}{\mathrm{support}}
\newcommand{\AND}{\;\wedge\;}
\newcommand{\OR}{\;\vee\;} 
\newcommand{\Oset}{\varnothing}
\newcommand{\st}{\ni}
\newcommand{\wh}{\widehat}
\newcommand{\vect}[1]{\overrightarrow{#1}}

%Pagination stuff.
%\setlength{\oddsidemargin}{0in}
%\setlength{\evensidemargin}{0in}
\setlength{\textheight}{9.in}
\setlength{\textwidth}{6.5in}
\cfoot{page \thepage}
\lhead{MEU302 - Alg\`ebre}
\rhead{TD2}
\pagestyle{fancy}


\begin{document}

\subsection*{Exercice 3}
\subsection*{Question 3.1}
Soit $Z_1, Z_2, Z_3...Z_n$, des variables al\'eatoire independantes suivant une loi normale standard $N(0,1)$.et posons $Y_n = \sum_{1}^{n}{Z_i^2}$. Calculons le moment de $Y$.
$$
M_{Y_n}(t) = M_{Z_1^2}(t).M_{Z_2^2}(t).M_{Z_3^2}(t) \ldots M_{Z_n^2}(t)
$$

Chaque $Z^2$ suit la loi chi-deux de degr\'es 1 (ie $\chi_1^2$). Cela doit \^etre un resultat du cours?? sinon demande moi.
Donc $M_{Z_1^2}(t) = (1-2t)^{-1/2}$ et 
$$
M_Y(t) = (1-2t)^{-n/2}
$$ 
Ceci est le moment de la fonction $\Gamma(\frac{n}{2},2)$  qui est \'egale \`a la loi chi-deux avec $n$ degr\'es de libert\'e. 

Comme $X_i$ est une variable al\'eatoire suit une loi normale d'esp\'erance 5 et de variance $\sigma^2$, posons $Zi=\frac{X_i-5}{\sigma}$ qui suit une loi normale standard d'esp\'erance 0 et de variance 1.
Donc comme $Y_n = \sum_{1}^{n}Z_i^2$ suit une loi chi-deux de $n$ degr\'es de libert\'e, 
$$
Q_n = \sum_{1}^{n}{\left(\frac{X_i-5}{\sigma}\right)^2}= \sum_{1}^{n}Z_i^2 = Y_n
$$
suit \'egalement une loi chi-deux de $n$ degr\'es de libert\'e,

\subsection*{Question 3.2}
$$
V_n^2 = \frac{1}{n}\sum_{1}^{n}{(X_i-5)^2}
$$
Calculons $E(V_n^2)$.

$$
E(V_n^2) = E(1/n\sum_{i=1}^{n}{(X_i-5)^2})=1/nE(\sum_{i=1}{n}{(X_i-5)^2})=1/n\sum_{i=1}^{n}E((X_i-5)^2)=1/n\sum_{i=1}^{n}{\sigma^2}=1/n.n\sigma^2=\sigma^2.
$$

Calculons son risque quadratique $E((V^2_n - \sigma)^2) =  Var(V_n^2)$ car $V_n$ est un estimateur sans biais.



\end{document}

