\documentclass[]{book}

%These tell TeX which packages to use.
\usepackage{array,epsfig}
\usepackage{amsmath}
\usepackage{amsfonts}
\usepackage{amssymb}
\usepackage{amsxtra}
\usepackage{amsthm}
\usepackage{mathrsfs}
\usepackage{color}
\usepackage[margin=2cm,top=2.5cm,headheight=16pt,headsep=0.1in,heightrounded]{geometry}
\usepackage{fancyhdr}
\pagestyle{fancy}
%\usepackage{tikz}
\usepackage{pgfplots}

%Here I define some theorem styles and shortcut commands for symbols I use often
\theoremstyle{definition}
\newtheorem{defn}{Definition}
\newtheorem{thm}{Theorem}
\newtheorem{cor}{Corollary}
\newtheorem*{rmk}{Remark}
\newtheorem{lem}{Lemma}
\newtheorem*{joke}{Joke}
\newtheorem{ex}{Example}
\newtheorem*{soln}{Solution}
\newtheorem{prop}{Proposition}

\newcommand{\lra}{\longrightarrow}
\newcommand{\ra}{\rightarrow}
\newcommand{\surj}{\twoheadrightarrow}
\newcommand{\graph}{\mathrm{graph}}
\newcommand{\bb}[1]{\mathbb{#1}}
\newcommand{\Z}{\bb{Z}}
\newcommand{\Q}{\bb{Q}}
\newcommand{\R}{\bb{R}}
\newcommand{\E}{\bb{E}}
\newcommand{\C}{\bb{C}}
\newcommand{\N}{\bb{N}}
\newcommand{\M}{\mathbf{M}}
\newcommand{\m}{\mathbf{m}}
\newcommand{\MM}{\mathscr{M}}
\newcommand{\HH}{\mathscr{H}}
\newcommand{\Om}{\Omega}
\newcommand{\Ho}{\in\HH(\Om)}
\newcommand{\bd}{\partial}
\newcommand{\del}{\partial}
\newcommand{\bardel}{\overline\partial}
\newcommand{\textdf}[1]{\textbf{\textsf{#1}}\index{#1}}
\newcommand{\img}{\mathrm{img}}
\newcommand{\ip}[2]{\left\langle{#1},{#2}\right\rangle}
\newcommand{\inter}[1]{\mathrm{int}{#1}}
\newcommand{\exter}[1]{\mathrm{ext}{#1}}
\newcommand{\cl}[1]{\mathrm{cl}{#1}}
\newcommand{\ds}{\displaystyle}
\newcommand{\vol}{\mathrm{vol}}
\newcommand{\cnt}{\mathrm{ct}}
\newcommand{\osc}{\mathrm{osc}}
\newcommand{\LL}{\mathbf{L}}
\newcommand{\UU}{\mathbf{U}}
\newcommand{\support}{\mathrm{support}}
\newcommand{\AND}{\;\wedge\;}
\newcommand{\OR}{\;\vee\;} 
\newcommand{\Oset}{\varnothing}
\newcommand{\st}{\ni}
\newcommand{\wh}{\widehat}
\newcommand{\vect}[1]{\overrightarrow{#1}}

%Pagination stuff.
%\setlength{\oddsidemargin}{0in}
%\setlength{\evensidemargin}{0in}
\setlength{\textheight}{9.in}
\setlength{\textwidth}{6.5in}
\cfoot{page \thepage}
\lhead{MEU302 - Alg\`ebre}
\rhead{TD2}
\pagestyle{fancy}


\begin{document}

\subsection*{Exercice 4}
\subsection*{Question 4.3.a}
Posons 
$$R = \frac{\frac{1224S^2}{\sigma^2}-1224}{\sqrt{2048}}$$

$R$ peut etre approxim\'ee par une loi normale r\'eduite si $E(R) = 0$ et $V(R) = 1$ quand le nombre d'\'echantillon est grand.
Calculons $E(R)$, quand il y a un grand nombre d'echantillon on a $S^2=\sigma^2$, donc 
$$
E(R) = E\left( \frac{\frac{1224S^2}{\sigma^2}-1224}{\sqrt{2048}}\right) = E(\frac{1224 -1224}{\sqrt{2048}}) = E(0) = 0
$$

Calculons $V(R)$. On sait que $\frac{(n-1)S^2}{\sigma^2} \approx \chi^2_{n-1}$ et que $V(\chi^2_{n-1}) = 2(n-1)$.
$$
V(R) = E\left( \frac{\frac{1224S^2}{\sigma^2}-1224}{\sqrt{2048}}\right) = \frac{1}{2048}V(\frac{1224S^2}{\sigma^2}-1224) = \frac{1}{2048}V(\frac{1224S^2}{\sigma^2}) = \frac{1}{2048}V(\chi^2_{1224}) = \frac{1}{2048}.2(1224) = 1
$$

Donc $R$ peut \^etre approxim\'e par une loi normale $N(0,1)$.

\subsection*{Question 4.3.b}
On cherche a v\'erifier l'hypoth\`ese "$H_0$: \'ecart-type des p\'erim\`etres cr\^aniens des garcons de la France enti\`ere = \'ecart-type des p\'erim\`etres cr\^aniens des garcons de la r\'egion parisienne" avec un risque de 5%.

Calcul de $R$ pour $S^2 = 2.17$ et $\sigma^2 = 1,35^2 = 1.82$. $R = 4.31$.
On sait que $R$ peut \^etre approxim\'e par une loi normale r\'eduite centr\'ee donc $N(0,1)$.

Si je prends un calculateur en ligne avec $DF = 1$, $CV=4.31$ j'obtiens $3\%$. Donc hypoth\`ese correcte???  

\end{document}

