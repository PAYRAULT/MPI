\documentclass[]{book}

%These tell TeX which packages to use.
\usepackage{array,epsfig}
\usepackage{amsmath}
\usepackage{amsfonts}
\usepackage{amssymb}
\usepackage{amsxtra}
\usepackage{amsthm}
\usepackage{mathrsfs}
\usepackage{color}
\usepackage[margin=2cm,top=2.5cm,headheight=16pt,headsep=0.1in,heightrounded]{geometry}
\usepackage{fancyhdr}
\pagestyle{fancy}
%\usepackage{tikz}
\usepackage{pgfplots}

%Here I define some theorem styles and shortcut commands for symbols I use often
\theoremstyle{definition}
\newtheorem{defn}{Definition}
\newtheorem{thm}{Theorem}
\newtheorem{cor}{Corollary}
\newtheorem*{rmk}{Remark}
\newtheorem{lem}{Lemma}
\newtheorem*{joke}{Joke}
\newtheorem{ex}{Example}
\newtheorem*{soln}{Solution}
\newtheorem{prop}{Proposition}

\newcommand{\lra}{\longrightarrow}
\newcommand{\ra}{\rightarrow}
\newcommand{\surj}{\twoheadrightarrow}
\newcommand{\graph}{\mathrm{graph}}
\newcommand{\bb}[1]{\mathbb{#1}}
\newcommand{\Z}{\bb{Z}}
\newcommand{\Q}{\bb{Q}}
\newcommand{\R}{\bb{R}}
\newcommand{\E}{\bb{E}}
\newcommand{\C}{\bb{C}}
\newcommand{\N}{\bb{N}}
\newcommand{\M}{\mathbf{M}}
\newcommand{\m}{\mathbf{m}}
\newcommand{\MM}{\mathscr{M}}
\newcommand{\HH}{\mathscr{H}}
\newcommand{\Om}{\Omega}
\newcommand{\Ho}{\in\HH(\Om)}
\newcommand{\bd}{\partial}
\newcommand{\del}{\partial}
\newcommand{\bardel}{\overline\partial}
\newcommand{\textdf}[1]{\textbf{\textsf{#1}}\index{#1}}
\newcommand{\img}{\mathrm{img}}
\newcommand{\ip}[2]{\left\langle{#1},{#2}\right\rangle}
\newcommand{\inter}[1]{\mathrm{int}{#1}}
\newcommand{\exter}[1]{\mathrm{ext}{#1}}
\newcommand{\cl}[1]{\mathrm{cl}{#1}}
\newcommand{\ds}{\displaystyle}
\newcommand{\vol}{\mathrm{vol}}
\newcommand{\cnt}{\mathrm{ct}}
\newcommand{\osc}{\mathrm{osc}}
\newcommand{\LL}{\mathbf{L}}
\newcommand{\UU}{\mathbf{U}}
\newcommand{\support}{\mathrm{support}}
\newcommand{\AND}{\;\wedge\;}
\newcommand{\OR}{\;\vee\;} 
\newcommand{\Oset}{\varnothing}
\newcommand{\st}{\ni}
\newcommand{\wh}{\widehat}
\newcommand{\vect}[1]{\overrightarrow{#1}}

%Pagination stuff.
%\setlength{\oddsidemargin}{0in}
%\setlength{\evensidemargin}{0in}
\setlength{\textheight}{9.in}
\setlength{\textwidth}{6.5in}
\cfoot{page \thepage}
\lhead{MEU302 - Alg\`ebre}
\rhead{TD2}
\pagestyle{fancy}


\begin{document}

\subsection*{Exercice 5}
\subsection*{Question 1}
On a

$$
E(X_1) = \int_{a}^{1}{t f(t) \,dt} = \int_{a}^{1}{t\frac{1}{1-a} \,dt} = \frac{1}{1-a}\frac{1-a^2}{2} = \frac{1+a}{2}
$$

On a
$$
E[X_1^2] = \int_{a}^{1}{t^2 f(t) \,dt} = \int_{a}^{1}{t^2\frac{1}{1-a} \,dt} = \frac{1}{1-a}\frac{1-a^3}{3} = \frac{1+a+a^2}{3}
$$

Donc
$$
V(X_1) = E[X_1^2]−(E[X_1])^2 = \frac{1+a+a^2}{3} -  \left( \frac{1+a}{2} \right)^2 = \frac{(1-a)^2}{12}
$$

\subsection*{Question 2}
On a
$$E[X_1] = \frac{1+a}{2} \implies a = 2E[X_1] - 1$$
Donc on prend comme EMM de $a$
$$
\tilde{a}_n = 2 \bar{a}_n - 1
$$
Mais $0 < a <1$, il faut donc que son estimateur soit aussi $0 < \tilde{a}_n <1$
Donc 
$$
0 < 2\bar{a}_n - 1 < 1 \implies 1 < 2\bar{a}_n < 2 \implies 1/2 < \bar{a}_n < 1
$$
Donc l'EMM est d\'efini uniquement si la moyenne de l'\'echantillon $\bar{a}_n$ est comprise entre 0.5 et 1. 

Consistance. En appliquant le Lemme de l'application Continue (LAC).
En prenant $h(x) = 2 x - 1$, pour $1/2 < x <1$. La fonction est continue. On a \'egalement, $\bar{a}_n \xrightarrow[n \to +\infty]{P} E[X_1]$ selon la loi des grands nombres. Donc $\tilde{a}_n = h(\bar{a}_n) \xrightarrow[n \to +\infty]{P} h(E[X_1]) = a$. Donc consistance.

En appliquant le Th\'eo\`eme Central Limite (TCL) avec $\mu = a$ et $\sigma^2 = \frac{(1-a)^2}{12}$ on a  
$$
\frac{\sqrt{n}(\bar{a}_n - a)}{\sqrt{\sigma^2}} \xrightarrow[n \to +\infty]{P} Z \sim \mathcal{N}(0,1)
$$

\subsection*{Question 3}
On calcule
$$
\mathcal{L}_a(x_i,\ldots,x_n) = \prod_{1}^{n}{\frac{1}{1-a}}1_{x_i \in [a,1]}(x_i) =
\frac{1}{(1-a)^n}\prod_{1}^{n}{1_{x_i \in [a,1]}(x_i)} = \frac{1}{(1-a)^n}{1_{min(x_i) \leq x_i \leq 1}(x_i)}
$$

Ce qui donne la fonction suivante:
$$
\mathcal{L}_a(x_i,\ldots,x_n) =     
\left\{
    \begin{array}{cc}
        1 &  a = 0 \\
        \frac{1}{(1-a)^n} & 0 < a \leq min(x_i)\\
        0 & min(x_i) < a < 1 \\
    \end{array}
\right.
$$ 

$\mathcal{L}_a(x_i,\ldots,x_n)$ est croissante sur $0 \leq a \leq min(x_i)$ et nulle quand $min(x_i) < a$ donc EMV est maximale lorsque $a=min(x_i)$. On a $\hat{a}_n = \frac{1}{(1-min(x_i))^n}$.  


Je ne comprends pas ce que represente $Z_{n}$.

\subsection*{Question 4}
Vitesse de convergence de $\tilde{a}_n$.
Il faut trouver le plus grand $d$ qui verifie:
$$
n^d(\tilde{a}_n-a) \xrightarrow[n \to +\infty]{P} 0
$$
$$
\forall \epsilon, \lim_{n \to \infty}{\mathbb{P}(n^d(\tilde{a}_n-a) \geq \epsilon)} = 0
$$
$$
\forall \epsilon, \lim_{n \to \infty}{\mathbb{P}((2\bar{a}_n) \geq \frac{\epsilon}{n^d} +a+1)} = 0
$$

Je ne comprends rien.

\subsection*{Exercice 6}
\subsection*{Question 1-a}

en prenant $k=1$ on a 
$$\mathbb{E}(U^{2k}) = \frac{(2k)!}{2^kk!} = \frac{2!}{2.1!} = \frac{1}{2}$$

en prenant $k=2$ on a 
$$\mathbb{E}(U^{2k}) = \frac{(2k)!}{2^kk!} = \frac{4!}{4.2!} = \frac{3}{2}$$

\subsection*{Question 1-b}

Je ne comprends meme pas la question. J'abandonne. Good luck.


\end{document}

