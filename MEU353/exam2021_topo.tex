\documentclass[]{book}

%These tell TeX which packages to use.
\usepackage{array,epsfig}
\usepackage{amsmath}
\usepackage{amsfonts}
\usepackage{amssymb}
\usepackage{amsxtra}
\usepackage{amsthm}
\usepackage{mathrsfs}
\usepackage{color}
\usepackage{tikz}
\usepackage{graphicx}

%Here I define some theorem styles and shortcut commands for symbols I use often
\theoremstyle{definition}
\newtheorem{defn}{Definition}
\newtheorem{thm}{Theorem}
\newtheorem{cor}{Corollary}
\newtheorem*{rmk}{Remark}
\newtheorem{lem}{Lemma}
\newtheorem*{joke}{Joke}
\newtheorem{ex}{Example}
\newtheorem*{soln}{Solution}
\newtheorem{prop}{Proposition}

\newcommand{\lra}{\longrightarrow}
\newcommand{\ra}{\rightarrow}
\newcommand{\surj}{\twoheadrightarrow}
\newcommand{\graph}{\mathrm{graph}}
\newcommand{\bb}[1]{\mathbb{#1}}
\newcommand{\Z}{\bb{Z}}
\newcommand{\Q}{\bb{Q}}
\newcommand{\R}{\bb{R}}
\newcommand{\C}{\bb{C}}
\newcommand{\N}{\bb{N}}
\newcommand{\M}{\mathbf{M}}
\newcommand{\m}{\mathbf{m}}
\newcommand{\MM}{\mathscr{M}}
\newcommand{\HH}{\mathscr{H}}
\newcommand{\Om}{\Omega}
\newcommand{\Ho}{\in\HH(\Om)}
\newcommand{\bd}{\partial}
\newcommand{\del}{\partial}
\newcommand{\bardel}{\overline\partial}
\newcommand{\textdf}[1]{\textbf{\textsf{#1}}\index{#1}}
\newcommand{\img}{\mathrm{img}}
\newcommand{\ip}[2]{\left\langle{#1},{#2}\right\rangle}
\newcommand{\inter}[1]{\mathrm{int}{#1}}
\newcommand{\exter}[1]{\mathrm{ext}{#1}}
\newcommand{\cl}[1]{\mathrm{cl}{#1}}
\newcommand{\ds}{\displaystyle}
\newcommand{\vol}{\mathrm{vol}}
\newcommand{\cnt}{\mathrm{ct}}
\newcommand{\osc}{\mathrm{osc}}
\newcommand{\LL}{\mathbf{L}}
\newcommand{\UU}{\mathbf{U}}
\newcommand{\support}{\mathrm{support}}
\newcommand{\AND}{\;\wedge\;}
\newcommand{\OR}{\;\vee\;}
\newcommand{\Oset}{\varnothing}
\newcommand{\st}{\ni}
\newcommand{\wh}{\widehat}

%Pagination stuff.
\setlength{\topmargin}{-.3 in}
\setlength{\oddsidemargin}{0in}
\setlength{\evensidemargin}{0in}
\setlength{\textheight}{9.in}
\setlength{\textwidth}{6.5in}
\pagestyle{empty}



\begin{document}


\subsection*{Exercice 1}
\subsection*{Exercice 1.1}

$$
\frac{\partial X}{\partial r} = \frac{2r\cos(\theta)}{\partial r} = 2\cos(\theta)
$$

$$
\frac{\partial X}{\partial \theta} = \frac{2r\cos(\theta)}{\partial \theta} = -2r\sin(\theta)
$$

$$
\frac{\partial Y}{\partial r} = \frac{3r\sin(\theta)}{\partial r} = 3\sin(\theta)
$$

$$
\frac{\partial Y}{\partial \theta} = \frac{3r\sin(\theta)}{\partial \theta} = 3r\cos(\theta)
$$

\subsection*{Exercice 1.2}
D\'erivable ??

La matrice Jacobienne de $F(r,\theta)$ est 
$$
\begin{vmatrix}
    \frac{\partial X}{\partial r} & \frac{\partial X}{\partial \theta} \\
    \frac{\partial Y}{\partial r} & \frac{\partial Y}{\partial \theta} \\
\end{vmatrix}
=
\begin{vmatrix}
    2\cos(\theta) & -2r\sin(\theta) \\
    3\sin(\theta) & 3r\cos(theta) \\
\end{vmatrix}
$$

Le d\'eterminant Jacobien est $(2\cos(\theta))(3r\cos(\theta)) - (-2r\sin(\theta))(3\sin(\theta)) = 6r\cos^2(\theta) + 6r\sin^2(\theta) = 6r$





\subsection*{Exercice 1.3}
D'apr\`es le th\'eor\`eme d'inversion locale, notons $x_0 = (1,0)$ si $DF(1,0)$ est inversible et $F$ est de classe $C^1$ alors $\exists r>0, \text{ tel que } B=B(x_0, r)$ et la restriction de $F$ \`a $B$ est un diff\'eomorphisme sur $B \to F(B)$. On sait que $F$ est de classe $C^1$, que $DF(1,0)$ est inversible car son d\'eterminant Jacobien est $6*1 = 6$ (diff\'erent de 0). $F(1,0) = (2*1*\cos(0), 3*1*\sin(0)) = (2,0)$. Donc $F$ est un diff\'eomorhisme sur $(1,0) \to (2,0)$.

\subsection*{Exercice 1.4}

Montrons que l'application $F(r,\theta)$ est bijective. SOit $(x,Y)$ tel que $F(r,\theta) = (x,y)$ on a alors 
$$
\left(\frac{x}{2}\right)^2+\left(\frac{y}{3}\right)^2 = (r^2\cos^2(\theta) + r^2\sin^2(\theta)) = r^2 
$$
donc on peut d\'efinir $r$ uniquement \'a partir de $(x,y)$. 


$$
\frac{2y}{3x} = \frac{6r\sin(\theta)}{6r\cos(\theta)} = \frac{\sin(\theta)}{\cos(\theta)} = \tan(\theta)
$$
donc on peut d\'efinir $\theta$ uniquement \`a partir de $(x,y)$. 

L'application r\'eciproque est 
$$
F^{-1}(x,y) = \left(\sqrt{\left(\frac{x}{2}\right)^2+\left(\frac{y}{3}\right)^2}, \arctan \left(\frac{2y}{3x}\right) \right)
$$

$$
    DF^{-1}(2,0) = \left( \frac{\partial F^{-1}}{\partial x}(2,0),  \frac{\partial F^{-1}}{\partial y}(2,0)\right) = \frac{1}{2}
$$

$$
\frac{\partial F^{-1}}{\partial x}(2,0) = \frac{x}{4\sqrt{ \frac{x^2}{4}+\frac{y^2}{9}}}(2,0) = 
$$

$$
\frac{\partial F^{-1}}{\partial y}(2,0) = \frac{6x}{4y^2+9x^2}(2,0) = \frac{1}{3}
$$

donc
$$
DF^{-1}(2,0) = \left(\frac{1}{2}, \frac{1}{3}\right)
$$


\subsection*{Exercice 1.5}
On a $X(1,0) + Y(1,0) = 2.1.\cos(0) + 3.1.\sin(0) = 2$. Donc l'\'equation $X(r,\theta)+Y(r,\theta) = 2$ admet au moins la solution $(1,0)$.

la suite???


\subsection*{Exercice 2}
\subsection*{Exercice 2.1}

$\frac{\partial^2 f}{\partial x^2}(x,y) +  \frac{\partial^2 f}{\partial y^2}(x,y) = 1$


Rien compris.

\subsection*{Exercice 3}
\subsection*{Exercice 3.1}
Aucune id\'ee

\subsection*{Exercice 3.2}
$$
\nabla f_1(x,y,z) = \left(\frac{\partial f_1(x,y,z)}{\partial x}, \frac{\partial f_1(x,y,z)}{\partial y}, \frac{\partial f_1(x,y,z)}{\partial z}\right) 
= \left(\frac{\partial e^{x+y+z}}{\partial x}, \frac{\partial e^{x+y+z}}{\partial y}, \frac{\partial e^{x+y+z}}{\partial z}\right)
= (e^{x+y+z},e^{x+y+z},e^{x+y+z}) 
$$


$$
\nabla f_2(x,y,z) = \left(\frac{\partial f_2(x,y,z)}{\partial x}, \frac{\partial f_2(x,y,z)}{\partial y}, \frac{\partial f_2(x,y,z)}{\partial z}\right)
= (3x^2, -3y^2, 1+3z^2)
$$

\subsection*{Exercice 3.3}

$$
\det \begin{vmatrix}
    e^{x+y+z} & e^{x+y+z} \\
    -3y^2 & 1 + 3z^2
\end{vmatrix}
=
e^{x+y+z}(1+3z^2 + 3y^2)
$$

C'est nul si $e^{x+y+z} = 0$ pas possible, ou $1+3z^2+3y^2 = 0$ impossible aussi. Donc le d\'eterminant est non toujours nul.


\subsection*{Exercice 3.4}
On peut prendre $F(x,y,z) = (f_1(x,y,z)-a_0) + (f_2(x,y,z)-b_0)$. car, $(x,y,z) \in \Gamma \implies f_1(x,y,z) - a_0 = 0 \land  f_2(x,y,z) - b_0 = 0$. 

\subsection*{Exercice 3.5}
???


\subsection*{Exercice 4}
\subsection*{Exercice 4.1}
On a $\frac{\partial^2 f(x)}{\partial x^2} \geq 1$, donc $\frac{\partial f(x)}{\partial x} \geq x + c_1$ et $f(x) \geq \frac{x^2}{2} + c_1x + c_0$. Ce qui fait
$$f\left(\frac{x+y}{2}\right) = \frac{(x+y)^2}{8} + \frac{c_1(x+y)}{2} + c_0 $$
et
$$\frac{1}{2}(f(x) + f(y)) - \eta\Vert x - y \Vert^2 = \frac{1}{2}\left(\frac{x^2}{2} + c_1x + c_0 + \frac{y^2}{2} + c_1y + c_0\right) - \eta(x-y)^2
$$
V\'erifions USC
$$
\frac{(x+y)^2}{8} + \frac{c_1(x+y)}{2} + c_0 \leq \frac{1}{2}\left(\frac{x^2}{2} + c_1x + c_0 + \frac{y^2}{2} + c_1y + c_0\right) - \eta(x-y)^2
$$
$$
\frac{(x+y)^2}{8} \leq \frac{1}{2}\left(\frac{x^2}{2} + \frac{y^2}{2} \right) - \eta(x-y)^2
$$
$$
x^2+2xy+y^2 \leq 2x^2 + 2y^2 - 8\eta(x^2-2xy+y^2)
$$
$$
0 \leq x^2(1-8\eta) + y^2(1-8\eta) - 2xy(1-8\eta)
$$
$$
0 \leq (1-8\eta)(x-y)^2
$$
Pour \^etre positif il faut que $0 \leq \eta \leq 1/8$.


\subsection*{Exercice 4.2}
On a $f(x) = \Vert x \Vert^2 = x_1^2 + x_2^2 + \ldots + x_n^2$ donc
$$
1/2(f(x)+f(y))- f((x+y)/2) 
= 1/2(x_1^2 + x_2^2 + \ldots + x_n^2) + 1/2(y_1^2 + y_2^2 + \ldots + y_n^2) - (\frac{x_1+y_1)^2}{4} + \frac{x_2+y_2)^2}{4} + \ldots + \frac{x_n+y_n)^2}{4})
$$
$$ 
= 1/2(x_1^2 + x_2^2 + \ldots + x_n^2) + 1/2(y_1^2 + y_2^2 + \ldots + y_n^2) - (x_1^2/4+2x_1y_1/4+y_1^2/4 + x_2^2/4+2x_2y_2/4+y_2^2/4 + \ldots + x_n^2/4+2x_1y_n/4+y_n^2/4)
$$
$$
= 1/4(x_1^2 + x_2^2 + \ldots + x_n^2 + y_1^2 + y_2^2 + \ldots + y_n^2) - x_1y_1/2 - x_2y_2/2 - \ldots - x_ny_n/2
$$
$$
= 1/4(x_1^2 + x_2^2 + \ldots + x_n^2 + y_1^2 + y_2^2 + \ldots + y_n^2 - 2x_1y_1 - 2x_2y_2 - \ldots - 2x_ny_n) = 1/4\Vert x - y \Vert^2
$$

On a $f((x+y)/2) = 1/2(f(x)+f(y)) - 1/4\Vert x - y \Vert^2$, donc $f(x) = \Vert x \Vert^2$ est USC avec $\eta = 1/4$.

\subsection*{Exercice 4.3}
Rappel de cours: une fonction $f$ est convexe lorsque $f((1-t)x + ty) \leq (1-t)f(x) + tf(y), \forall x, y \in E, 0 < t < 1$. ou $f$ est convexe si tous arcs de son graphe est en dessous de sa corde. 

Comme on a la fonction $F$ qui est USC, alors $f((x+y)/2) \leq 1/2(f(x)+f(y))$, sur la figure cela donne que la corde $f(x), f(y)$ est toujours au dessus de $f((x+y)/2)$. Donc $f$ est convexe. 

\begin{figure}[htbp]
    \centerline{\includegraphics[scale=.5]{convexe.jpg}}
    \caption{l'arc est toujours en dessous de la corde}
    \label{fig}
\end{figure}

Recopie internet.
On a $f$ qui est convexe, donc
$$
f((1-t)x+ty) \leq (1-t)f(x)+tf(y)
$$
$$
f(x-tx+ty) \leq f(x)-tf(x)+tf(y)
$$
comme $t \neq 0$
$$
\frac{f(x-tx-ty)-f(x)}{t} \leq f(y) - f(x)
$$
Quand $t \to 0$, on a 
$$
\langle \nabla f(x),y-x\rangle \leq f(y) - f(x)
$$
En prenant $x=0$ on a $\langle \nabla f(0),y\rangle \leq f(y) - f(0)$ ou $f(y) \geq f(0) - \langle \nabla f(0),y\rangle$.

\subsection*{Exercice 4.4}
Rien compris.


QED


\end{document}

