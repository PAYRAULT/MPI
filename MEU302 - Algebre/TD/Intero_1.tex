\documentclass[]{book}

%These tell TeX which packages to use.
\usepackage{array,epsfig}
\usepackage{amsmath}
\usepackage{amsfonts}
\usepackage{amssymb}
\usepackage{amsxtra}
\usepackage{amsthm}
\usepackage{mathrsfs}
\usepackage{color}
\usepackage[margin=2cm,top=2.5cm,headheight=16pt,headsep=0.1in,heightrounded]{geometry}
\usepackage{fancyhdr}
\pagestyle{fancy}
\usepackage{tikz}


%Here I define some theorem styles and shortcut commands for symbols I use often
\theoremstyle{definition}
\newtheorem{defn}{Definition}
\newtheorem{thm}{Theorem}
\newtheorem{cor}{Corollary}
\newtheorem*{rmk}{Remark}
\newtheorem{lem}{Lemma}
\newtheorem*{joke}{Joke}
\newtheorem{ex}{Example}
\newtheorem*{soln}{Solution}
\newtheorem{prop}{Proposition}

\newcommand{\lra}{\longrightarrow}
\newcommand{\ra}{\rightarrow}
\newcommand{\surj}{\twoheadrightarrow}
\newcommand{\graph}{\mathrm{graph}}
\newcommand{\bb}[1]{\mathbb{#1}}
\newcommand{\Z}{\bb{Z}}
\newcommand{\Q}{\bb{Q}}
\newcommand{\R}{\bb{R}}
\newcommand{\C}{\bb{C}}
\newcommand{\N}{\bb{N}}
\newcommand{\K}{\bb{K}}
\newcommand{\M}{\mathbf{M}}
\newcommand{\m}{\mathbf{m}}
\newcommand{\MM}{\mathscr{M}}
\newcommand{\HH}{\mathscr{H}}
\newcommand{\Om}{\Omega}
\newcommand{\Ho}{\in\HH(\Om)}
\newcommand{\bd}{\partial}
\newcommand{\del}{\partial}
\newcommand{\bardel}{\overline\partial}
\newcommand{\textdf}[1]{\textbf{\textsf{#1}}\index{#1}}
\newcommand{\img}{\mathrm{img}}
\newcommand{\ip}[2]{\left\langle{#1},{#2}\right\rangle}
\newcommand{\inter}[1]{\mathrm{int}{#1}}
\newcommand{\exter}[1]{\mathrm{ext}{#1}}
\newcommand{\cl}[1]{\mathrm{cl}{#1}}
\newcommand{\ds}{\displaystyle}
\newcommand{\vol}{\mathrm{vol}}
\newcommand{\cnt}{\mathrm{ct}}
\newcommand{\osc}{\mathrm{osc}}
\newcommand{\LL}{\mathbf{L}}
\newcommand{\UU}{\mathbf{U}}
\newcommand{\support}{\mathrm{support}}
\newcommand{\AND}{\;\wedge\;}
\newcommand{\OR}{\;\vee\;} 
\newcommand{\Oset}{\varnothing}
\newcommand{\st}{\ni}
\newcommand{\wh}{\widehat}
\newcommand{\vect}[1]{\overrightarrow{#1}}

%Pagination stuff.
\setlength{\topmargin}{-.3 in}
%\setlength{\oddsidemargin}{0in}
%\setlength{\evensidemargin}{0in}
\setlength{\textheight}{9.in}
\setlength{\textwidth}{6.5in}
\cfoot{page \thepage}
\lhead{MEU303 - Alg\`ebre}
\rhead{TD1}
\pagestyle{fancy}


\begin{document}

\subsection*{Rappel de cours}

\begin{defn}
Une famille de vecteurs d'un $\K$-espace vectoriel $E$ est not\'e $(e_i)_{i \in I}$. On note \'egalement $Vect((e_i)_{i \in I}) = \{\sum\lambda_i e_i, \text{ avec } (\lambda_i) \in \K\}$ 
\end{defn}

\begin{defn}
Base de $E$:
\begin{itemize}
\item $(e_i)_{i \in I}$ est libre ssi $\sum\lambda_i e_i = 0_E \implies \forall i, \lambda_i = 0_{\K}$
\item $(e_i)_{i \in I}$ est g\'en\'eratrice ssi $Vect((e_i)_{i \in I}) = E$
\end{itemize}
$(e_i)_{i \in I}$ est une base de $E$ si $(e_i)_{i \in I}$ est libre et g\'en\'eratrice.
\end{defn}

\begin{defn}
La dimension d'un $\K$-expace vectoriel $E$ est la cardinal d'une base quelconque de $E$. 
\end{defn}

\begin{defn}
Le rang d'une famille de vecteur, not\'e $rg((e_i)_{i \in I})$ est \'egal \`a la dimension de l'espace vectoriel engendr\'e par la famille de vecteurs $Vect((e_i)_{i \in I})$. 
\end{defn}

\begin{defn}
Soit $f$ une application lin\'eaire. Si l'image de $f$, $\text{Im} f$ est de dimension finie alors on dit que $f$ est de rang fini et son rang not\'e $rg(f)$ est \'egale \`a la dimension de son image $\text{Im} f$. Si $(e_1, e_2. \ldots, e_n)$ est une base d'une $\K$-espace vectoriel $E$ alors $\text{IM}f = Vect(f(e_1), f(e_2), \ldots, f(e_n))$ donc $dim \text{ Im}f \leq n$.
\end{defn}

\begin{defn}
Soit une matrice $A \in M_{n,p}(\K)$, on d\'efinit le rang de la matrice $A$, not\'e $rg(A)$ la dimension du sous-espace vectoriel de $\K^n$ engendr\'e par ses vecteurs colonnes. 
\end{defn}

\begin{defn}
Si $A = M_{\mathcal{B}, \mathcal{B'}}(f)$, la matrice d'un endomorphisme $f$ par rapport \`a deux bases $\mathcal{B}$ et $\mathcal{B'}$ alors $rg(A) = rg(f)$. 
\end{defn}

\begin{defn}
Si $A = M_{\mathcal{B}, \mathcal{B'}}(f)$, $rg(A) \leq \min(n,p)$. 
\end{defn}

\begin{defn}
Soit $E$ et $F$ deux $\K$-espaces vectoriel de dimension finie $n$ et $p$. Prenons 
\begin{itemize}
\item $f \in L(E,F)$, 
\item 2 bases de $E$, $\mathcal{B}$ et $\mathcal{B'}$,
\item $P = P_{\mathcal{B} \to \mathcal{B'}} \in GL_p(\K)$  
\item 2 bases de $F$, $\mathcal{C}$ et $\mathcal{C'}$
\item $Q = Q_{\mathcal{C} \to \mathcal{C'}} \in GL_n(\K)$  
\end{itemize}
On peut d\'efinir les matrices de changement de base $f$ $M=M_{\mathcal{B}, \mathcal{C}}(f)$ et $M'=M_{\mathcal{B'}, \mathcal{C'}}(f)$ et on a $M'=Q^{-1}MP$.
\end{defn}


\begin{defn}
Deux matrices $A$ et $B$ sont semblables ssi la matrice $A$ peut s'\'ecrire sous la forme $A = PBP^{-1}$. En effet, deux matrices sont semblables si elles repr\'esentent la m\^eme applocation lin\'eaire mais pas dans la m\^eme base. Soit $f$ une application lin\'eaire avec sa matrice caract\'eristique $X$, prenons $A= M_{\mathcal{B}}(f)$ (ie application lin'eaire $f$ dans la base $\mathcal{B}$) et prenons $B= M_{\mathcal{B'}}(f)$ (ie application lin'eaire $f$ dans la base $\mathcal{B'})$. On a $Y=AX$ pour la base $\mathcal{B}$ et $Y'=BX'$ pour la base $\mathcal{B'}$. Notons $P$, le changement de base de $\mathcal{B} \to \mathcal{B'}$. On a donc $X=PX'$ et $Y=PY'$, donc $PY' = AX = APX'$ ce qui fait $Y'=P^{-1}APX' = BX'$. Donc $B = P^{-1}AP$ ou $A = PBP^{-1}$.


 est d\'efini dans la base $\mathcal{B}$ et $B$ dans la base $\mathcal{B'}$. Prenons $P$ le changement de base $\mathcal{B} \to \mathcal{B'}$, donc $P^{-1}$ existe et est le changement de base r\'eciproque $\mathcal{B'} \to \mathcal{B}$. 
\end{defn}

\begin{defn}
La trace de la matrices $A$ est $Tr(a) = \sum_{i=1}^{n} A_{i,i}$
\end{defn}

\begin{defn}
Le polynome caract\'eristique d'une matrice carr\'e d'ordre $n$, $M$ not\'e $P_M(X) = \det(XI_n - M)$ avec $I_n$ la matrice identit\'e d'ordre $n$. 
\end{defn}







\newpage

\subsection{propri\'et\'e 3.2}
idem que pour les matrices semblables.

\subsection{exercice 7}
Si les matrices $A$ et $B$ sont \'equivalentes alors on a $A=QBP^{-1}$. 

Commencons par montrer que si $P$ est une matrice inversible alors $rg(PA) = rg(A)$. On a $rg(AB) \leq \min(rg(A), rg(B))$. Donc $rg(PA) \leq \min(rg(P), rg(A)) \leq rg(A)$ mais comme $P$ est inversible on a $A$ qui peut s'\'ecrire $A = P^{-1}PA$, donc $rg(A) = rg(P^{-1}PA) \leq \min(rg(P^{-1}), rg(PA)) \leq rg(PA)$ donc $rg(PA) \leq rg(A) \text{ et } rg(A) \leq rg(PA)$ donc $rg(PA) = rg(A)$. 

Comme les matrice $P$ et $Q$ sont inversibles car dans $GL(\K)$, on a $rg(B) = rg(QB) = rg(QBP^{-1} = rg(A)$.  

\subsection{exercice 8}
La relation "\^etre \'equivalente" est une relation d'\'equivalence ssi:
\begin{itemize}
\item Elle est r\'eflexive, en prenant les 2 matrices identit\'es $I_n$ et $I_p$ on a bien $ A = I_p A I_n$. 
\item Elle est sym\'etrique $A = QBP^{-1} \implies B = Q^{-1}AP$ vrai car $P$ et $Q$ sont inversibles. Il suffit de multiplier des 2 cot\'es
\item Elle est transitive $A = QBP^{-1}, B = RCT^{-1} \implies A = WCZ^{-1}$, on a $A=QRCT^{-1}P^{-1} = (QR)C(PT)^{-1}$ (attention $T^{-1}P^{-1} = (PT)^{-1}$ et non $(TP)^{-1}$. Vrai en prenant $W=QR$ et $Z=TP$.
\end{itemize}

\subsection*{Question 1.1}
Il faut montrer que si $A$ et $B$ sont semblables (ie $A = PBP^{-1}$) alors leur traces sont \'egales $Tr(A) = Tr(B)$. On sait que pour des endomorphismes on a $tr(AB) = Tr(BA)$, donc calculons la trace de la matrice $A$. 
$$
Tr(a) = Tr(PBP^{-1}) = Tr(PP^{-1}B) = Tr(IdB) = Tr(B) 
$$

\subsection*{Question 1.2}





\end{document}

