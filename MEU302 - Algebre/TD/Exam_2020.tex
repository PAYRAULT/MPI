\documentclass[]{book}

%These tell TeX which packages to use.
\usepackage{array,epsfig}
\usepackage{amsmath}
\usepackage{amsfonts}
\usepackage{amssymb}
\usepackage{amsxtra}
\usepackage{amsthm}
\usepackage{mathrsfs}
\usepackage{color}
\usepackage[margin=2cm,top=2.5cm,headheight=16pt,headsep=0.1in,heightrounded]{geometry}
\usepackage{fancyhdr}
\pagestyle{fancy}
\usepackage{tikz}


%Here I define some theorem styles and shortcut commands for symbols I use often
\theoremstyle{definition}
\newtheorem{defn}{Definition}
\newtheorem{thm}{Theorem}
\newtheorem{cor}{Corollary}
\newtheorem*{rmk}{Remark}
\newtheorem{lem}{Lemma}
\newtheorem*{joke}{Joke}
\newtheorem{ex}{Example}
\newtheorem*{soln}{Solution}
\newtheorem{prop}{Proposition}

\newcommand{\lra}{\longrightarrow}
\newcommand{\ra}{\rightarrow}
\newcommand{\surj}{\twoheadrightarrow}
\newcommand{\graph}{\mathrm{graph}}
\newcommand{\bb}[1]{\mathbb{#1}}
\newcommand{\Z}{\bb{Z}}
\newcommand{\Q}{\bb{Q}}
\newcommand{\R}{\bb{R}}
\newcommand{\E}{\bb{E}}
\newcommand{\C}{\bb{C}}
\newcommand{\N}{\bb{N}}
\newcommand{\M}{\mathbf{M}}
\newcommand{\m}{\mathbf{m}}
\newcommand{\MM}{\mathscr{M}}
\newcommand{\HH}{\mathscr{H}}
\newcommand{\Om}{\Omega}
\newcommand{\Ho}{\in\HH(\Om)}
\newcommand{\bd}{\partial}
\newcommand{\del}{\partial}
\newcommand{\bardel}{\overline\partial}
\newcommand{\textdf}[1]{\textbf{\textsf{#1}}\index{#1}}
\newcommand{\img}{\mathrm{img}}
\newcommand{\ip}[2]{\left\langle{#1},{#2}\right\rangle}
\newcommand{\inter}[1]{\mathrm{int}{#1}}
\newcommand{\exter}[1]{\mathrm{ext}{#1}}
\newcommand{\cl}[1]{\mathrm{cl}{#1}}
\newcommand{\ds}{\displaystyle}
\newcommand{\vol}{\mathrm{vol}}
\newcommand{\cnt}{\mathrm{ct}}
\newcommand{\osc}{\mathrm{osc}}
\newcommand{\LL}{\mathbf{L}}
\newcommand{\UU}{\mathbf{U}}
\newcommand{\support}{\mathrm{support}}
\newcommand{\AND}{\;\wedge\;}
\newcommand{\OR}{\;\vee\;} 
\newcommand{\Oset}{\varnothing}
\newcommand{\st}{\ni}
\newcommand{\wh}{\widehat}
\newcommand{\vect}[1]{\overrightarrow{#1}}

%Pagination stuff.
%\setlength{\oddsidemargin}{0in}
%\setlength{\evensidemargin}{0in}
\setlength{\textheight}{9.in}
\setlength{\textwidth}{6.5in}
\cfoot{page \thepage}
\lhead{MEU302 - Alg\`ebre}
\rhead{Interro 3}
\pagestyle{fancy}


\begin{document}

\subsection*{Rappel de cours}

\subsection*{propri\'et\'e 4.1}
Soit $b(x,y)$ une forme bilin\'eaire. D\'efinissons $f(x,y) = \frac{b(x,y)+b(y,x)}{2}$ on a 

$$\forall (x,y), f(x, y) = \frac{b(x,y)+b(y,x)}{2} = \frac{b(y,x)+b(x,y)}{2} = f(y,x)$$
donc $f(x,y)$ est sym\'etrique. D\'efinissons $g(x,y) = \frac{b(x,y) - b(y,x)}{2}$, on a 
$$g(x,y) =  \frac{b(x,y) - b(y,x)}{2} = -\frac{b(y,x) - b(x,y)}{2} = -g(x,y)$$
donc $g(x,y)$ est anti-sym\'etrique. On peut d\'ecomposer $b(x,y)$ en 
$$\frac{b(x,y) + b(y,x)}{2} + \frac{b(x,y) - b(y,x)}{2} = f(x,y) + g(x,y)$$



\newpage
\subsection*{Question 1}
Une forme quadratique $q$ $\R^n$ est dite \emph{d\'efinie n\'egative} si $\forall x \in E, q(x) \leq 0$. 


\subsection*{Exercice 1}
\subsection*{Exercice 1.1}
Le vecteur $\vect{ab} = (-2,2,1)$, le vecteur $\vect{ac} = (-10, 4, -1)$. Les 2 vecteurs ne sont pas colin\'eaire donc l'espace affine est d\'efinie par $F= a + \R \vect{ab} + \R \vect{ac}$. Sa repr\'esentation param\'etrique est

$$
\left\{
\begin{array}{l}
x = 5 -2\lambda_1 - 10\lambda_2\\
y = -1 +2\lambda_1 + 4\lambda_2\\
z = 1 +\lambda_1 - 1\lambda_2\\
\end{array}
\right.
$$

\subsection*{Exercice 1.2}
L'espace affine de G est d\'efinie par une droite $G = d + \lambda \vect{u}$. La dimension de lespace affine $F$ est 2, celle de $G$ est 1. L'intersection des espaces affines $F$ et $G$ est soit l'ensemble vide (dim = 0), soit un point (dim = 1). soit une droite (dim =2). 

\subsection*{Exercice 1.3}
L'\'equation param\'etrique de $G$ est 
$$
\left\{
\begin{array}{l}
x = 6 \lambda\\
y = 1\\
z = t + 3\lambda\\
\end{array}
\right.
$$

L'intersection de $F$ et $G$ est
$$
\left\{
\begin{array}{l}
 5 -2\lambda_1 - 10\lambda_2 = 6 \lambda\\
 -1 +2\lambda_1 + 4\lambda_2 = 1\\
 1 +\lambda_1 - 1\lambda_2 = t + 3 \lambda\\
\end{array}
\right.
$$

$$
\left\{
\begin{array}{l}
 6 \lambda + 2\lambda_1 + 10\lambda_2 = 5\\
 \lambda_1 + 2\lambda_2 = 1\\
 3 \lambda - \lambda_1 + \lambda_2 = 1 - t \\
\end{array}
\right.
$$

Avec $(1)-2(3)$ on a

$$
\left\{
\begin{array}{l}
 6 \lambda + 2\lambda_1 + 10\lambda_2 = 5\\
 \lambda_1 + 2\lambda_2 = 1\\
 4 \lambda_1 + 8 \lambda_2 = 3 + 2t \\
\end{array}
\right.
$$

Avec $4(2) - (3)$ on a 
$$
\left\{
\begin{array}{l}
 6 \lambda + 2\lambda_1 + 10\lambda_2 = 5\\
 \lambda_1 + 2\lambda_2 = 1\\
 0 = 1 - 2t \\
\end{array}
\right.
$$

Si $t \neq 1/2$, alors l'intersection est vide, sinon 

$$
\left\{
\begin{array}{l}
 6 \lambda  = 7 - 6\lambda_2\\
 \lambda_1  = 1 - 2\lambda_2\\
\end{array}
\right.
$$
Donc, l'intersection est la droite passant par $d$ et de vecteur $u$. 










\end{document}


