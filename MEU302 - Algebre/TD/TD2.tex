\documentclass[]{book}

%These tell TeX which packages to use.
\usepackage{array,epsfig}
\usepackage{amsmath}
\usepackage{amsfonts}
\usepackage{amssymb}
\usepackage{amsxtra}
\usepackage{amsthm}
\usepackage{mathrsfs}
\usepackage{color}
\usepackage[margin=2cm,top=2.5cm,headheight=16pt,headsep=0.1in,heightrounded]{geometry}
\usepackage{fancyhdr}
\pagestyle{fancy}
\usepackage{tikz}


%Here I define some theorem styles and shortcut commands for symbols I use often
\theoremstyle{definition}
\newtheorem{defn}{Definition}
\newtheorem{thm}{Theorem}
\newtheorem{cor}{Corollary}
\newtheorem*{rmk}{Remark}
\newtheorem{lem}{Lemma}
\newtheorem*{joke}{Joke}
\newtheorem{ex}{Example}
\newtheorem*{soln}{Solution}
\newtheorem{prop}{Proposition}

\newcommand{\lra}{\longrightarrow}
\newcommand{\ra}{\rightarrow}
\newcommand{\surj}{\twoheadrightarrow}
\newcommand{\graph}{\mathrm{graph}}
\newcommand{\bb}[1]{\mathbb{#1}}
\newcommand{\Z}{\bb{Z}}
\newcommand{\Q}{\bb{Q}}
\newcommand{\R}{\bb{R}}
\newcommand{\C}{\bb{C}}
\newcommand{\N}{\bb{N}}
\newcommand{\M}{\mathbf{M}}
\newcommand{\m}{\mathbf{m}}
\newcommand{\MM}{\mathscr{M}}
\newcommand{\HH}{\mathscr{H}}
\newcommand{\Om}{\Omega}
\newcommand{\Ho}{\in\HH(\Om)}
\newcommand{\bd}{\partial}
\newcommand{\del}{\partial}
\newcommand{\bardel}{\overline\partial}
\newcommand{\textdf}[1]{\textbf{\textsf{#1}}\index{#1}}
\newcommand{\img}{\mathrm{img}}
\newcommand{\ip}[2]{\left\langle{#1},{#2}\right\rangle}
\newcommand{\inter}[1]{\mathrm{int}{#1}}
\newcommand{\exter}[1]{\mathrm{ext}{#1}}
\newcommand{\cl}[1]{\mathrm{cl}{#1}}
\newcommand{\ds}{\displaystyle}
\newcommand{\vol}{\mathrm{vol}}
\newcommand{\cnt}{\mathrm{ct}}
\newcommand{\osc}{\mathrm{osc}}
\newcommand{\LL}{\mathbf{L}}
\newcommand{\UU}{\mathbf{U}}
\newcommand{\support}{\mathrm{support}}
\newcommand{\AND}{\;\wedge\;}
\newcommand{\OR}{\;\vee\;} 
\newcommand{\Oset}{\varnothing}
\newcommand{\st}{\ni}
\newcommand{\wh}{\widehat}
\newcommand{\vect}[1]{\overrightarrow{#1}}

%Pagination stuff.
\setlength{\topmargin}{-.3 in}
%\setlength{\oddsidemargin}{0in}
%\setlength{\evensidemargin}{0in}
\setlength{\textheight}{9.in}
\setlength{\textwidth}{6.5in}
\cfoot{page \thepage}
\lhead{MEU302 - Alg\`ebre}
\rhead{TD2}
\pagestyle{fancy}


\begin{document}

\subsection*{Rappel de cours}
\begin{defn}
Bla bla
\end{defn}



\newpage
\subsection*{Exercice 10}
\subsection*{Exercice 10.a}

$$\det
\begin{vmatrix}
1 & 2 & x \\
2 & 3 & y \\
3 & 4 & z \\
\end{vmatrix}
= 1. \begin{vmatrix} 3 & y \\ 4 & z \\ \end{vmatrix} -2. \begin{vmatrix} 2 & x \\ 4 & z \\ \end{vmatrix} + 3. \begin{vmatrix} 2 & x \\ 3 & y \\ \end{vmatrix} = 3z - 4y -4z + 8x +6y - 9x = -z + 2y - x
$$

\subsection*{Exercice 10.b}
Un vecteur $(x,y,z)$ appartient à $F$ si et seulement si il existe deux r\'eels $\lambda_1$ et $\lambda_2$ tel que 
$$
(x,y,z)=\lambda_1(1,2,3)+\lambda_2(2,3,4)
$$

$$
\left\{
\begin{array}{l}
x = \lambda_1 + 2 \lambda_2 \\
y = 2\lambda_1 + 3 \lambda_2 \\
z = 3\lambda_1 + 4 \lambda_2 \\
\end{array}
\right.
=
\left\{
\begin{array}{l}
x = \lambda_1 + 2 \lambda_2 \\
y = 2\lambda_1 + 3 \lambda_2 \\
z+x = 4\lambda_1 + 6 \lambda_2 \\
\end{array}
\right.
=
\left\{
\begin{array}{l}
x = \lambda_1 + 2 \lambda_2 \\
y = 2\lambda_1 + 3 \lambda_2 \\
-z-x+2y = -4\lambda_1 + -6 \lambda_2 + 4\lambda_1 + 6 \lambda_2 = 0 \\
\end{array}
\right.
$$

Le syst\`eme (H) a donc des solutions si et seulement si $-x+2y-z=0$. L'ensemble $F$ est donc l'ensemble des triplets $(x,y,z)$ tel que $-x+2y-z=0$. On dit que $F$ a pour \'equation : $-x+2y-z=0 = \det
\begin{vmatrix}
1 & 2 & x \\
2 & 3 & y \\
3 & 4 & z \\
\end{vmatrix}
$.

\subsection*{Exercice 10.c}
Si $H$ est le noyau d'une application lin\'eaire $f$ alors $H = \{v= (x, y, z) | f(v) = 0\}$.
Donc trouver $\lambda_1, \lambda_2, \lambda_3$ tel que 
$$
\left\{
\begin{array}{l}
\lambda_1 + 2 \lambda_2 + 3 \lambda_3 = 0\\
2\lambda_1 + 3 \lambda_2 + 4\lambda_3  =  0\\
\end{array}
\right.
=
\left\{
\begin{array}{l}
\lambda_1 + 2 \lambda_2 + 3 \lambda_3 = 0\\
 \lambda_2 + 2\lambda_3 = 0\\
\end{array}
\right.
=
\left\{
\begin{array}{l}
\lambda_1 - 4 \lambda_3 + 3 \lambda_3 = 0\\
\lambda_2 = -2\lambda_3 \\
\end{array}
\right.
==
\left\{
\begin{array}{l}
\lambda_1 = \lambda_3\\
\lambda_2 = -2\lambda_3 \\
\end{array}
\right.
$$

Le noyau est $(\lambda, -2 \lambda, \lambda)$ et l'application lin\'eaire $f(x,y,z) = \lambda x - 2 \lambda y + \lambda z$. 	


\subsection*{Exercice 12}
Si $E$ est un espace vectoriel de dimension $n$ alors on a une base $\{v_1, v_2, ...., v_n\}$ de $E$ et $\forall e \in E, e = \lambda_1 v_1 + \lambda_2 v_2 + \ldots + \lambda_n v_n$ et $E^{*} = L(E, \R)$ donc $\forall f \in E^{*}, f: E \to \R, \forall (e_1,e_2) \in E^2, \forall \lambda \in \R, f(\lambda e_1 + e_2) = \lambda f(e_1) + f(e_2)$. $E^{*}$ est un espace vectoriel ssi $(E^{*},+)$ est un groupe Ab\'elien et $\forall(f_1, f_2) \in {E^{*}}^2, \forall (\lambda_1, \lambda_2) \in \R^2$ 
\begin {enumerate}
\item $\lambda_1.(\lambda_2 . f_1) = (\lambda_1 \lambda_2) . f_1$
\item $1 . f_1 = f_1$
\item $(\lambda_1 + \lambda_2) f_1 = \lambda_1 . f_1 + \lambda_2 . f_1 $
\item $\lambda_1 ( f_1 + f_2) = \lambda_1 . f_1 + \lambda_1 . f_2$
\end{enumerate}

$(E^{*},+)$ est un groupe Ab\'elien ssi
\begin{enumerate}
\item cloture sur $E^{*}$, $\forall (f_1, f_2) \in {E^{*}}^2, f_1 + f_2 \in E^{*}$
\item \'el\'ement neutre, $\exists f_{id} \in E^{*}, \forall f \in E^{*}, f_{id} + f = f + f_{id} = f$
\item inverse, $\forall f_1 \in E^{*}, \exists f_2 \in E^{*}, f_1 + f_2 = f_2 + f_1 = f_{id}$
\item Commutativit\'e, $\forall f_1 \in E^{*}, \forall f_2 \in E^{*}, f_1 + f_2 = f_2 + f_1$
\end{enumerate}

Donc
\begin{enumerate}
\item $\forall (f_1, f_2) \in {E^{*}}^2, f_1(\lambda_1 e_1 + e_2) + f_2(\lambda_1 e_1 + e_2) =  \lambda_1 f_1(e_1) + f_1(e_2) + \lambda_1 f_2(e_1) + f_2(e_2) = \lambda(f_1(e_2) + f_2(e_1)) + (f_1(e_2) + f_2(e_2))$ Prenons $g(e) = f_1(e) + f_2(e)$ on a $g(\lambda e_1 + e_2) = \lambda g(e_1) + g(e_2)$ donc $g(e) \in E^{*}$, donc Vrai
\item Prenons $f_{id} : E \to \R, \forall e \in E, f(e) = 0$. On a $f_{id} \in E^{*}$ car $\forall (e_1,e_2) \in E^2, \forall \lambda \in \R, f_{id}(\lambda e_1 + e_2) = 0 = \lambda 0 + 0 = \lambda f(e_1) + f(e_2)$ et $\forall e \in E, f_{id}(e) + f(e) = 0 + f(e) = f(e)$ et $\forall e \in E, + f(e) + f_{id}(e)= f(e) + 0 = f(e)$ donc Vrai
\item Pour $f_1 \in E^{*}$, prenons $f_2 = -f_1$. $f_2 \in E^{*}$ car $f_2(\lambda e_1 + e_2) = -f_1(\lambda e_1 + e_2) = -(\lambda f_1(e_1) + f_1(e_2)) = -\lambda f_1(e_1) - f_1(e_2) = \lambda f_2(e_1) + f_2(e_2))$ et $f_1(e) + f_2(e) = f_1(e) - f_1(e) = 0 = f_{id}(e)$ et $f_2(e) + f_1(e) = -f_1(e) + f_1(e) = 0 = f_{id}(e)$ donc Vrai
\item $\forall f_1, f_2 \in {E^{*}}^2, f_1(\lambda_1 e_1 + e_2) + f_2(\lambda_2 e_3 + e_4) = \lambda_1 f_1(e_1) + f_1(e_2) + \lambda_2 f_2(e_3) + f_2(e_4) =  \lambda_2 f_2(e_3) + f(e_4) + \lambda_1 f_1(e_1) + f_1(e_2) = f_2(\lambda_2 e_3 + e_4) + f_1(\lambda_1 e_1 + e_2)$ donc Vrai
\end{enumerate}


Donc
\begin {enumerate}
\item $\lambda_1.(\lambda_2 . f_1(\lambda e_1 + e_2)) = \lambda_1.(\lambda_2 . (\lambda f_1(e_1) + f_1(e_2))) = \lambda_1.( \lambda_2 \lambda f_1(e_1) + \lambda_2 f_1(e_2)) = \lambda_1 \lambda_2 \lambda f_1(e_1) + \lambda_1 \lambda_2 f_1(e_2) = (\lambda_1 \lambda_2) . (\lambda f_1(e_1) + f_1(e_2)) = (\lambda_1 \lambda_2) . f_1(\lambda e_1 + e_2))$ donc Vrai
\item $1 . f_1 = f_1$
\item $(\lambda_1 + \lambda_2) f_1 = \lambda_1 . f_1 + \lambda_2 . f_1 $
\item $\lambda_1 ( f_1 + f_2) = \lambda_1 . f_1 + \lambda_1 . f_2$
\end{enumerate}





\end{document}

