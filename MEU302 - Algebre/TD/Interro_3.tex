\documentclass[]{book}

%These tell TeX which packages to use.
\usepackage{array,epsfig}
\usepackage{amsmath}
\usepackage{amsfonts}
\usepackage{amssymb}
\usepackage{amsxtra}
\usepackage{amsthm}
\usepackage{mathrsfs}
\usepackage{color}
\usepackage[margin=2cm,top=2.5cm,headheight=16pt,headsep=0.1in,heightrounded]{geometry}
\usepackage{fancyhdr}
\pagestyle{fancy}
\usepackage{tikz}


%Here I define some theorem styles and shortcut commands for symbols I use often
\theoremstyle{definition}
\newtheorem{defn}{Definition}
\newtheorem{thm}{Theorem}
\newtheorem{cor}{Corollary}
\newtheorem*{rmk}{Remark}
\newtheorem{lem}{Lemma}
\newtheorem*{joke}{Joke}
\newtheorem{ex}{Example}
\newtheorem*{soln}{Solution}
\newtheorem{prop}{Proposition}

\newcommand{\lra}{\longrightarrow}
\newcommand{\ra}{\rightarrow}
\newcommand{\surj}{\twoheadrightarrow}
\newcommand{\graph}{\mathrm{graph}}
\newcommand{\bb}[1]{\mathbb{#1}}
\newcommand{\Z}{\bb{Z}}
\newcommand{\Q}{\bb{Q}}
\newcommand{\R}{\bb{R}}
\newcommand{\E}{\bb{E}}
\newcommand{\C}{\bb{C}}
\newcommand{\N}{\bb{N}}
\newcommand{\M}{\mathbf{M}}
\newcommand{\m}{\mathbf{m}}
\newcommand{\MM}{\mathscr{M}}
\newcommand{\HH}{\mathscr{H}}
\newcommand{\Om}{\Omega}
\newcommand{\Ho}{\in\HH(\Om)}
\newcommand{\bd}{\partial}
\newcommand{\del}{\partial}
\newcommand{\bardel}{\overline\partial}
\newcommand{\textdf}[1]{\textbf{\textsf{#1}}\index{#1}}
\newcommand{\img}{\mathrm{img}}
\newcommand{\ip}[2]{\left\langle{#1},{#2}\right\rangle}
\newcommand{\inter}[1]{\mathrm{int}{#1}}
\newcommand{\exter}[1]{\mathrm{ext}{#1}}
\newcommand{\cl}[1]{\mathrm{cl}{#1}}
\newcommand{\ds}{\displaystyle}
\newcommand{\vol}{\mathrm{vol}}
\newcommand{\cnt}{\mathrm{ct}}
\newcommand{\osc}{\mathrm{osc}}
\newcommand{\LL}{\mathbf{L}}
\newcommand{\UU}{\mathbf{U}}
\newcommand{\support}{\mathrm{support}}
\newcommand{\AND}{\;\wedge\;}
\newcommand{\OR}{\;\vee\;} 
\newcommand{\Oset}{\varnothing}
\newcommand{\st}{\ni}
\newcommand{\wh}{\widehat}
\newcommand{\vect}[1]{\overrightarrow{#1}}

%Pagination stuff.
%\setlength{\oddsidemargin}{0in}
%\setlength{\evensidemargin}{0in}
\setlength{\textheight}{9.in}
\setlength{\textwidth}{6.5in}
\cfoot{page \thepage}
\lhead{MEU302 - Alg\`ebre}
\rhead{Interro 3}
\pagestyle{fancy}


\begin{document}

\subsection*{Rappel de cours}
\begin{defn}
    Soit $F$ un sous-espace vectoriel de $E$, l'ensemble de tous les vecteurs orthogonaux \`a F est appel\'e le compl\'ement orthogonal de F (ou orthogonal de F)
et est not\'e $F^{\perp}$:
$$
F^{\perp} = \{u \in E, u \perp w \forall w \in F\}
$$
\end{defn}


\newpage
\subsection*{Exercice 3}
\subsection*{Exercice 3.1}
$$
P_u(v) = <u,v>\frac{u}{\lVert u\rVert^2}
$$

\subsection*{Exercice 3.2}
Construction d'une base $Vect(u,v')$ orthogonale d'une base $Vect(u,v).$ On construit le vecteur $\vect{v'}$ dans $Vect(u, v)$ sous la forme $v+\lambda u$ de facon \`a avoir $<u,v'> = 0$ (proc\'ed\'e d’orthogonalisation
de Gram-Schmidt).

$$
<u,v+\lambda u> = <u,v> + <u,\lambda u> = <u,v> + \lambda<u,u> = <u,v> + \lambda\lVert u \rVert^2
$$
Donc $\lambda = -\frac{<u,v>}{\lVert u \rVert^2}$ et $v'= v -\frac{<u,v>}{\lVert u \rVert^2} u = v - P_u(v)$.

\subsection*{Exercice 3.2}


\subsection*{Exercice 4}
\subsection*{Exercice 4.1}
On utilise le proc\'ed\'e d’orthogonalisation de Gram-Schmidt, en construisant par r\'ecurence la base orthogonale de $F$. On note $F' = Vect(u'_1, u'_2, u'_3)$ la base orthogonale \`a $F$. On commence par prendre $u'_1 = u_1$. Donc $u'_1 = (-1, 2, 0, 2)$.


On a ensuite $u'_2 = u_2 - \frac{<u_2, u'_1>}{\lVert u'_1 \rVert^2}u'_1$  avec $<u_2, u'_1> = 0$ et $\lVert u'_1 \rVert^2 = 9$ donc $u'_2 = (0,2,1,-2) - \frac{0}{9} (-1, 2, 0, 2) = (0,2,1,-2)$.

On a ensuite $u'_3 = u_3 - \frac{<e_3, u'_1>}{\lVert u'_1 \rVert^2}u'_1 - \frac{<e_3, u'_2>}{\lVert u'_2 \rVert^2}u'_2$
Avec $<e_3, u'_1> = 1$, $\lVert u'_1 \rVert^2 = 9$, $<e_3, u'_2> = 3$ et $\lVert u'_2 \rVert^2 = 9$ donc $u'_3 = (1,1,1,0) - \frac{1}{9} (-1,2,0,2) - \frac{3}{9}(0,2,1,-2) = (\frac{10}{9}, \frac{1}{9}, \frac{6}{9}, \frac{4}{9})$.

\subsection*{Exercice 4.2}
On a $\dim(\R^4) = 4$ et $\dim(F) =3 $ donc $\dim(F^{\perp}) = 4 - 3 = 1$. On cherche donc un vecteur $w=(x,y,z,t)$ tel que $w \perp u'1$, $w \perp u'2$,$w \perp u'3$.

$$
\left\{
    \begin{array}{l}
        <w, u'_1> = 0 \\
        <w, u'_2> = 0 \\
        <w, u'_3> = 0 \\
    \end{array}
\right.
$$
$$
\left\{
    \begin{array}{l}
        -x + 2y + 2t = 0 \\
        2y - 1z -2t = 0 \\
        10x + 1y + 6z + 4t = 0 \\
    \end{array}
\right.
$$
$$
\left\{
    \begin{array}{l}
        -x + 2y + 2t = 0 \\
             2y - 1z -2t = 0 \\
             21y + 6z + 24t = 0 \\
    \end{array}
\right.
$$
$$
\left\{
    \begin{array}{l}
        -x + 2y + 2t = 0 \\
             2y - 1z -2t = 0 \\
                 33z + 90t = 0 \\
    \end{array}
\right.
$$


Donc $F^{\perp} = \{(30,8,-30,11)\} $.

\subsection*{Exercice 4.3}
On cherche une application lin\'eaire $\varphi(x)$ tel que $F = Ker(\varphi(x))$. On sait que $Ker(f) = \{u \in E, f(u) = 0 \}$. On sait aussi que lorsque 2 vecteurs $u$ et $v$ sont orthogonaux alors $<u, v> = 0$. Si on choisit $\varphi(x) $ l'application qui calcule le produit vectoriel par rapport \`a $w$, alors on a $Ker(\varphi(x))$ qui est l'ensemble des vecteurs orthogonaux \`a $w$. Mais cet ensemble est $F$ car $w = F^{\perp}$. Donc $\varphi(x) = <x, w>$.  


\subsection*{Exercice 2}
\subsection*{Exercice 2.1}
Une isom\'etrie est une application line\'aire qui conserve le produit scalaire. 
On a $(u + v)^2 = <u,u> + 2<u,v> + <v,v>$. donc $2<u,v> = \lVert u+v \rVert^2 - \lVert u \rVert^2 - \lVert v \rVert^2$ Donc $2<f(u),f(v)> = \lVert f(u+v) \rVert^2 - \lVert f(u) \rVert^2 - \lVert f(v) \rVert^2 = \lVert f(u)+f(v) \rVert^2 - \lVert f(u) \rVert^2 - \lVert f(v) \rVert^2$. Comme l'application est une isom\'etrie on a $\lVert u \rVert = \lVert f(u) \rVert$. Donc 
$$
<f(u), f(v)> = \lVert f(u)+f(v) \rVert^2 - \lVert f(u) \rVert^2 - \lVert f(v) \rVert^2 = \lVert u+v \rVert^2 - \lVert u \rVert^2 - \lVert v \rVert^2 = 2<u,v>
$$

Maintenant la preuve. Une application lin\'eaire converve le produit scalaire ssi $<f(u),f(v)> = <u,v>$. Une application est non inversible ssi $\forall u, f(f(u)) \neq u$ (ou $\forall x, f(f(x) \neq Id(x))$).
Calculons $<f(f(u)), f(f(v))>$ si l'application lin\'eaire n'est pas inversible on a $<u',v'> = <u,v>$ avec $u' \neq u$ et $v' \neq v$. Donc $\lVert u' \rVert . \lVert v' \rVert . \cos(u',v') = \lVert u \rVert . \lVert v \rVert . \cos(u,v)$. Vrai rien n'emp\`eche d'avoir une application qui v\'erifie cette equation.


\subsection*{Exercice 2.2}
Une isom\'etrie conserve les distances. La rotation autour d'un point concerve \'egalement mais n'est pas une r\'eflexion. Donc toutes les isom\'etrie ne sont pas des r\'eflexions.


\subsection*{Exercice 3}
\subsection*{Exercice 3.1}
Une r\'eflexion est une sym\'etrie orthogonale par rapport \`a une droite. Une r\'eflexion est une isom\'etrie donc elle conserve les distances. Une r\'eflexion vectorielle par rapport \'a une droite orthogonale au vecteur $k$ s'exprime comme 
$$
s(x) = x - 2\frac{<x|k>}{\lVert k \rVert^2} k
$$

$$
s(u) = u - 2\frac{<u|k>}{\lVert k \rVert^2} k = v
$$
donc
$$
u - v = 2\frac{<u|k>}{\lVert k \rVert^2} k
$$
Ceci donne la direction de $k$. L'axe de r\'eflexion est orthogonale \`a $k$.   

\subsection*{Exercice 4}
\subsection*{Exercice 4.1}
$$u'_1 = u_1 = (1,1,-1,-1)$$
$$u'_2 = u_2 - \frac{<u_2,u'_1>}{<u'_1,u'_1>}u'_1$$
Avec $u_2=(1,-1,0,0)$, $<u_2, u_1> = 0$ donc $u'_2 = u_2 = (1,-1,0,0)$
$$u'_3 = u_3 - \frac{<u_3,u'_1>}{<u'_1,u'_1>}u'_1 - \frac{<u_3,u'_2>}{<u'_2,u'_2>}u'_2$$
Avec $u_3=(4,0,-1,-3)$, $<u_3, u'_1> = 8$, $<u'_1, u'_1> = 4$, $<u_3, u'_2> = 4$, $<u'_2, u'_2> = 2$ donc 
$$u'3 = (4,0,-1,-3) - \frac{8}{4}(1,1,-1,-1) - \frac{4}{2} (1,-1,0,0) = (0, 0, 1, -1)$$

Une base orthogonale de $F = \{(1,1,-1,-1),  (1,-1,0,0), (0, 0, 1, -1)\}$
\subsection*{Exercice 4.2}
Projection orthogonale sur orthogonal de $F$ est d\'efinie par:
$$e'_4 = <e_4,u'_1>u'_1 + <e_4,u'_2>u'_2 + <e_4,u'_3>u'_3$$
Avec $<e_4,u'_1> = -1$, $<e_4,u'_2> = 0$, $<e_4,u'_3> = -1$, donc
$$
e'_4 = -1(1,1,-1,-1) + 0 (1,-1,0,0) -1 (0,0,1,-1) = (-1, -1, 0, 2)
$$

$$
\left\{
    \begin{array}{l}
        <w, u'_1> = 0 \\
        <w, u'_2> = 0 \\
        <w, u'_3> = 0 \\
    \end{array}
\right.
$$
$$
\left\{
    \begin{array}{l}
        x + y - z - t = 0 \\
        x - y = 0 \\
        z - t = 0 \\
    \end{array}
\right.
$$
$$
\left\{
    \begin{array}{l}
        x + y - z - t = 0 \\
        x  = y \\
        z  = t \\
    \end{array}
\right.
$$
$$
\left\{
    \begin{array}{l}
        2x - 2z  = 0 \\
        x  = y \\
        z  = t \\
    \end{array}
\right.
$$
$$
\left\{
    \begin{array}{l}
        x  = z \\
        x  = y \\
        z  = t \\
    \end{array}
\right.
$$

La vecteur $w = (1, 1, 1, 1)$ est dans $F^{\perp}$ comme $\dim(F^{\perp}) = \dim(\R^4)-dim(F) =  4 - 3$ on a $F^{\perp} = Vect(w)$.


\subsection*{Exercice 4.3}
Voir exercice 4.3 ci-dessus.

\end{document}

