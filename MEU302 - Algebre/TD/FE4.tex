\documentclass[]{book}

%These tell TeX which packages to use.
\usepackage{array,epsfig}
\usepackage{amsmath}
\usepackage{amsfonts}
\usepackage{amssymb}
\usepackage{amsxtra}
\usepackage{amsthm}
\usepackage{mathrsfs}
\usepackage{color}
\usepackage[margin=2cm,top=2.5cm,headheight=16pt,headsep=0.1in,heightrounded]{geometry}
\usepackage{fancyhdr}
\pagestyle{fancy}
\usepackage{tikz}


%Here I define some theorem styles and shortcut commands for symbols I use often
\theoremstyle{definition}
\newtheorem{defn}{Definition}
\newtheorem{thm}{Theorem}
\newtheorem{cor}{Corollary}
\newtheorem*{rmk}{Remark}
\newtheorem{lem}{Lemma}
\newtheorem*{joke}{Joke}
\newtheorem{ex}{Example}
\newtheorem*{soln}{Solution}
\newtheorem{prop}{Proposition}

\newcommand{\lra}{\longrightarrow}
\newcommand{\ra}{\rightarrow}
\newcommand{\surj}{\twoheadrightarrow}
\newcommand{\graph}{\mathrm{graph}}
\newcommand{\bb}[1]{\mathbb{#1}}
\newcommand{\Z}{\bb{Z}}
\newcommand{\Q}{\bb{Q}}
\newcommand{\R}{\bb{R}}
\newcommand{\C}{\bb{C}}
\newcommand{\N}{\bb{N}}
\newcommand{\M}{\mathbf{M}}
\newcommand{\m}{\mathbf{m}}
\newcommand{\MM}{\mathscr{M}}
\newcommand{\HH}{\mathscr{H}}
\newcommand{\Om}{\Omega}
\newcommand{\Ho}{\in\HH(\Om)}
\newcommand{\bd}{\partial}
\newcommand{\del}{\partial}
\newcommand{\bardel}{\overline\partial}
\newcommand{\textdf}[1]{\textbf{\textsf{#1}}\index{#1}}
\newcommand{\img}{\mathrm{img}}
\newcommand{\ip}[2]{\left\langle{#1},{#2}\right\rangle}
\newcommand{\inter}[1]{\mathrm{int}{#1}}
\newcommand{\exter}[1]{\mathrm{ext}{#1}}
\newcommand{\cl}[1]{\mathrm{cl}{#1}}
\newcommand{\ds}{\displaystyle}
\newcommand{\vol}{\mathrm{vol}}
\newcommand{\cnt}{\mathrm{ct}}
\newcommand{\osc}{\mathrm{osc}}
\newcommand{\LL}{\mathbf{L}}
\newcommand{\UU}{\mathbf{U}}
\newcommand{\support}{\mathrm{support}}
\newcommand{\AND}{\;\wedge\;}
\newcommand{\OR}{\;\vee\;} 
\newcommand{\Oset}{\varnothing}
\newcommand{\st}{\ni}
\newcommand{\wh}{\widehat}
\newcommand{\vect}[1]{\overrightarrow{#1}}

%Pagination stuff.
\setlength{\topmargin}{-.3 in}
%\setlength{\oddsidemargin}{0in}
%\setlength{\evensidemargin}{0in}
\setlength{\textheight}{9.in}
\setlength{\textwidth}{6.5in}
\cfoot{page \thepage}
\lhead{MEU303 - Alg\`ebre}
\rhead{TD1}
\pagestyle{fancy}


\begin{document}

\subsection*{Rappel de cours}
\begin{defn}
Bla bla
\end{defn}



\newpage
\subsection*{Exercice 1}
\subsection*{Exercice 1.1}

Les 2 premi\`eres colonnes de la matrice $M=\begin{vmatrix}
    2 & -1 & 3 \\
    1 & 4 & 6 \\
\end{vmatrix}$ sont lin\'eairement ind\'ependantes. Il s'en suit que l'application lin\'eaire de $\R^3$ vers $\R^2$ associ\'ee est surjective. C'est donc un sous-espace affine de $\R^3$.

Calcul de $Ker(M)$.
$$
\begin{vmatrix}
    2 & -1 & 3 \\
    1 & 4 & -6 \\
\end{vmatrix} . 
\begin{vmatrix}
    \lambda_1 \\
    \lambda_2 \\
    \lambda_3 \\
\end{vmatrix}
=
\begin{vmatrix}
    0 \\
    0 \\
\end{vmatrix}
$$
$$
\left\{ 
    \begin{array}{l}
        2\lambda_1 - \lambda_2 + 3 \lambda_3 = 0 \\
        \lambda_1 + 4 \lambda_2 - 6\lambda_3 = 0
    \end{array}
\right.
$$
$$
\left\{ 
    \begin{array}{l}
        2\lambda_1 - \lambda_2 + 3 \lambda_3 = 0 \\
         -9 \lambda_2 + 15\lambda_3 = 0
    \end{array}
\right.
$$
$$
\left\{ 
    \begin{array}{l}
        \lambda_1 = -\frac{2}{5} \lambda_2 \\
        \lambda_3 = \frac{3}{5} \lambda_2 
    \end{array}
\right.
$$

Donc $Ker(M) = \{(-2, 5, 3)\}$ donc $\dim Ker(M) = 1$

Il faut trouver une solution particuli\`ere
$$
\left\{ 
    \begin{array}{l}
        2\lambda_1 - \lambda_2 + 3 \lambda_3 = 1 \\
        \lambda_1 + 4 \lambda_2 - 6\lambda_3 = -2 \\
    \end{array}
\right.
$$
$$
\left\{ 
    \begin{array}{l}
        2\lambda_1 - \lambda_2 + 3 \lambda_3 = 1 \\
        -9 \lambda_2 +15 \lambda_3 = 5 \\
    \end{array}
\right.
$$
$$
\left\{ 
    \begin{array}{l}
        2\lambda_1 - \lambda_2 + 3 \lambda_3 = 1 \\
        3 \lambda_3 = 1 + \frac{9}{5} \lambda_2\\
    \end{array}
\right.
$$
$$
\left\{ 
    \begin{array}{l}
        3\lambda_3 = 1 + \frac{9}{5}\lambda_2 \\
        \lambda_1 = - \frac{2}{5}\lambda_2 \\
    \end{array}
\right.
$$

Une solution particuli\`ere est $\{0,0,\frac{1}{3}\}$. 

La nature est une droite affine.

L'\'equation param\'atrique est donc
$$
\begin{vmatrix}
    0 \\
    0 \\
    \frac{1}{3} \\
\end{vmatrix}
+
(\R\begin{vmatrix}
   -2 \\
   5 \\
   3 \\ 
\end{vmatrix})
$$

\subsection*{Exercice 1.2}
$f: \R^3 \to \R^2$ avec $f(x,y,z) = M.\begin{vmatrix}
    x \\ y \\ z \\
\end{vmatrix}
$.
et il y a une solution si $\exists (x,y,z) \in \R^3, f(x,y,z) = \begin{vmatrix}
    1 \\ 2 \\
\end{vmatrix}
$

\subsection*{Exercice 1.3}
Nommons $H_1 = \{ (x,y,z) \in \R^3 | 2x-y+3z = 1$ et  $H_2 = \{ (x,y,z) \in \R^3 | x+4y-6z = 2$. Chaque \'equation du sous-syst\`eme admet une solution (ie. $F$). Ceux sont donc des sous espaces affines. 
Il reste a montrer que ceux sont des plans. On peut dire que $F = H_1 \cup H_2$ car c'est la solution du syst\`eme.


\subsection*{Exercice 2}
\subsection*{Exercice 2.1}
Soit $M_1=\begin{vmatrix}
    -1 & \lambda & -3 \\
    1 & -3 & \lambda \\
\end{vmatrix}$. Calculons $Ker(M_1)$.

$$
\left\{ 
    \begin{array}{l}
        -x + \lambda y - 3 z = 0 \\
        x - 3 y + \lambda z = 0
    \end{array}
\right.
$$

$$
\left\{ 
    \begin{array}{l}
        -x + \lambda y - 3 z = 0 \\
        (\lambda - 3) y + (-3 +\lambda) z = 0
    \end{array}
\right.
$$

$$
\left\{ 
    \begin{array}{l}
        x  = -3z+3y \\
        0y + 0z= 0
    \end{array}
\right.
$$

Si $\lambda = 3$, on a $Ker(M_1) = \{(-3, 0, 1), (3, 1, 0)\}$, ce 'est pas une droite affine mais un plan affine car $\dim Ker(M_1) = 2$. Si $\lambda \neq 3$, on a $Ker(M_1) = \{(-\lambda -3, -1, 1)\}$. 

Soit $M_2=\begin{vmatrix}
    0 & 1 & 1 \\
    \lambda & 0 & -2 \\
\end{vmatrix}$. Calculons $Ker(M_2)$.

$$
\left\{ 
    \begin{array}{l}
        y + z = 0 \\
        \lambda x - 2 z = 0
    \end{array}
\right.
$$

$$
\left\{ 
    \begin{array}{l}
        y = -z \\
        \lambda x = 2 z
    \end{array}
\right.
$$

Si $\lambda = 0$, $Ker(M_2) =\{(1,0,0) \}$. Si $\lambda \neq 0$, $Ker(M_2) =\{(\frac{2}{\lambda},-1,1) \}$

\subsection*{Exercice 2.2}
Solution particul\`ere de $M_1$ quand $\lambda \neq 3$
$$
\left\{ 
    \begin{array}{l}
        -x + \lambda y - 3 z = \lambda - 1 \\
        x - 3 y + \lambda z = -2
    \end{array}
\right.
$$

$$
\left\{ 
    \begin{array}{l}
        -x + \lambda y - 3 z = \lambda - 1 \\
        (\lambda - 3) y + (\lambda - 3) z = (\lambda - 3)
    \end{array}
\right.
$$

$$
\left\{ 
    \begin{array}{l}
        x  = 1 - (\lambda + 3) z\\
         y = 1-z
    \end{array}
\right.
$$

Les points $A_1 = (1 - (\lambda + 3) z, 1-z, z)$. Donc, on a la droite affine $(1 - (\lambda + 3) z, 1-z, z) + \R(-\lambda -3, -1, 1)$

Solution particul\`ere de $M_2$ quand $\lambda = 0$
$$
\left\{ 
    \begin{array}{l}
        y+z  = 2\\
        -2z = 0
    \end{array}
\right.
$$

Les points $A_2 = (x,2,0)$. Donc, on a la droite affine $(x,2,0) + \R(1,0,0)$


Solution particul\`ere de $M_2$ quand $\lambda \neq 0$
$$
\left\{ 
    \begin{array}{l}
        y+z  = -\lambda + 2\\
        \lambda x -2z = 0
    \end{array}
\right.
$$

$$
\left\{ 
    \begin{array}{l}
        y = -\lambda + 2 - z\\
        x = \frac{2}{\lambda}z
    \end{array}
\right.
$$

Les points $A_2 = (\frac{2}{\lambda}z,-\lambda + 2 - z,z)$. Donc, on a la droite affine $(\frac{2}{\lambda}z,-\lambda + 2 - z,z) + \R(\frac{2}{\lambda},-1,1)$


\subsection*{Exercice 2.3}
Pour $M_1$ on a $\lambda \neq 3$. Premier cas $\lambda = 0$, on a donc 2 coefficients directeur de droites affines $(-3, -1, 1)$ pour $M_1$ et $(-1, 0, 0)$ pour $M_2$. Les 2 droites ne sont pas parall\`eles (car leurs coefficients directeurs ne peuvent pas \^etre \'egaux), donc non confondues aussi. Elles sont donc s\'ecantes.

Second cas $\lambda \neq 0$, on a donc 2 coefficients directeur de droites affines $(-\lambda -3, -1, 1)$ pour $M_1$ et $(\frac{2}{\lambda}, -1, 1)$ pour $M_2$. Pour que les droites soient parall\`eles il faut que $-\lambda -3 = \frac{2}{\lambda}$. Donc trouver les solutions de l'\'equation $|lambda^2 + 3\lambda + 2 = 0$, soit $\lambda = -2$ ou $\lambda = -1$. Pour que les droites soient confondues, il faut \'egalement que leurs points soient identiques, pour les valeurs de $lambda$. Quand $\lambda = 1$, on a $A_1 = (-3z,1-z,z)$ et $A_2 = (2z,1,z)$, il existe un point commun quand $z=0$. C'est le point $A=(0,1,0)$. Quand Quand $\lambda = 2$, on a $A_1 = (1-5z,-1,z)$ et $A_2 = (z,0,z)$. Il n'existe pas de point commun (\`a cause de y).

Pour r\'esumer:
\begin{itemize}
    \item $\lambda = 0$, droites s\'ecantes
    \item $\lambda = 1$, et point $(0,1,0)$, droites confondues
    \item $\lambda = 1$, et point non $(0,1,0)$, droites parall\`eles
    \item $\lambda = 2$, droites parall\`eles
    \item $\lambda \neq 3$, droites s\'ecantes
\end{itemize}

\subsection*{Exercice 5}
\subsection*{Exercice 5.1}
Le plan passe par l'origine donc Quand $x=y=z=0$, il faut que l'\'equation soit vraie. Donc le terme de gauche doit \^etre 0. Il faut donc
$$
\left\{ 
    \begin{array}{l}
        a + 2b + c = 0\\
        b + c = 0
    \end{array}
\right.
$$

$$
\left\{ 
    \begin{array}{l}
        a - c = 0\\
        b = - c
    \end{array}
\right.
$$

Donc $x - y + z = 0$ est une \'equation du plan passant par l'origine et de vecteurs $(1,2,1)$ et $(0,1,1)$.

\subsection*{Exercice 5.2}
Plan parall\`elle donc il doit v\'erifier l'\'equation $kx -ky + kz = b$. Il passe par le point $(0,0,1)$ donc
$k = b$. Ce qui fait que un plan d\'equation $kx - ky + kz = k$

Son \'equation param\'etrique
$$
\begin{vmatrix}
    0 \\
    0 \\
    1 \\
\end{vmatrix}
+(\R
    \begin{vmatrix}
        1 \\
        2 \\
        1 \\
    \end{vmatrix}
+    
\R
    \begin{vmatrix}
        0 \\
        1 \\
        1 \\
    \end{vmatrix}
)
$$

\subsection*{Exercice 5.3}
Droite passant par le point $(1,0,0)$ et dirig\'ee par le vecteur $(1,0,1)$. 
\'Equation param\'etrique 

$$
\begin{vmatrix}
    1 \\
    0 \\
    0 \\
\end{vmatrix}
+(\R
    \begin{vmatrix}
        1 \\
        0 \\
        1 \\
    \end{vmatrix}
)
$$

Donc son \'equation est 
$$
\left\{ 
    \begin{array}{l}
        x = 1 + t\\
        y = 0 + 0t\\
        z = 0 + t
    \end{array}
\right.
$$

Donc
$$
\left\{ 
    \begin{array}{l}
        x - z = 1 \\
        y = 0 \\
    \end{array}
\right.
$$

Intersection est donc
$$
\left\{ 
    \begin{array}{l}
        x - z = 1 \\
        y = 0 \\
        kx - ky + kz = k
    \end{array}
\right.
$$

$$
\left\{ 
    \begin{array}{l}
        x - z = 1 \\
        y = 0 \\
        x  + z = 1
    \end{array}
\right.
$$

Le point d'intersection est $(1,0,0)$.


\subsection*{Exercice 16}
La dimension d’un espace affine est celle de sa direction. Donc, un espace affine de dimension 1 est une droite.

$$
\left\{
\begin{array} {l l l}
    f: & \R \to & \R^n \\
     & x \to &\begin{pmatrix} a_1x + b_1 \\ \vdots \\ a_n x + b_n \end{pmatrix}
\end{array}
\right.
$$

\subsection*{Exercice 17}
Si une application affine $f$ commute avec toute translation $t$ on a $f \circ t = t \circ f$. Une tranlation de vecteur $v$ est caract\'eris\'ee par $t_v(a) = b$ avec $\vect{ab} = \vect{v}$ ou sous la forme d'une application affine $t_v(x) = x + \vect{v}$. Une application lin\'eaire s'\'ecrit sous la forme $f(x) = f(a) + \vect{f}(\vect{ax})$.
Donc
$$
f \circ t_v(x) = f(t_v(x)) = f(a) + \vect{f}(\vect{at(x)})
$$
$$
t_v \circ f(x) = t_v(f(x)) = f(a) + \vect{f}(\vect{ax})  + \vect{v}
$$
Donc par commutation
$$
f(a) + \vect{f}(\vect{at(x)}) = f(a) + \vect{f}(\vect{ax})  + \vect{v}
$$
$$
\vect{f}(\vect{at(x)}) = \vect{f}(\vect{ax})  + \vect{v}
$$
$$
\vect{f}(\vect{a(x+\vect{v})}) = \vect{f}(\vect{ax})  + \vect{v}
$$
$$
\vect{f}(\vect{ax}+\vect{v)}) = \vect{f}(\vect{ax})  + \vect{v}
$$
$\vect{f}$ est une application lin\'eaire donc
$$
\vect{f}(\vect{ax}) +\vect{f}(\vect{v}) = \vect{f}(\vect{ax})  + \vect{v}
$$
Donc $\vect{f}(x) = Id(x)$. Par cons\'equent $f(x) = f(a) + \vect{ax}$ qui est par d\'efinition une translation.

\subsection*{Exercice 18}
Si c'est une homot\'ethie de centre $O$ et de rapport $k$ alors on a $\vect{OA} = k\vect{OA'}$ et $\vect{OB}= k\vect{OB'}$. avec $\vect{OA} = (1-x_o, 1 - y_o)$, $\vect{OA'} = (-2-x_o, 2 - y_o)$, $\vect{OB} = (1-x_o, 3 - y_o)$ et $\vect{OB'} = (-2-x_o, 1 - y_o)$. Il faut r\'esoudre le syst\'emptyse


$$
\left\{
\begin{array} {l}
    1 - x_o = k(-2-x_o) \\
    1 - y_o = k(2-y_o) \\
    1 - x_o = k(-2 -x_o) \\
    3 - yo = k(1-y_o) \\
\end{array}
\right.
$$
(1) et (3) identiques
$$
\left\{
\begin{array} {l}
    1 - x_o = k(-2-x_o) \\
    1 - y_o = k(2-y_o) \\
    3 - y_o = k(1-y_o) \\
\end{array}
\right.
$$
$$
\left\{
\begin{array} {l}
    1 +2k + x_o(k-1) = 0 \\
    1 -2k + y_o(k-1) = 0 \\
    3 -k + y_o(k-1) = 0 \\
\end{array}
\right.
$$
(3)-(2) donne $k = -2$. Donc $x_o = -1$ et $y_o = 5/3$.

C'est une homot\'ethie de centre $O= (-1, 5/3)$ et de rapport $k=-2$.


\subsection*{Exercice 21}
\subsection*{Exercice 21.1}
On a $h: M_1 \to M_2, \vect{AM_2} = \lambda \vect{AM_1}$ et $h': M_1 \to M_2, \vect{BM_2} = \mu \vect{AM_1}$ et $\lambda\mu = 1$. Soit un point $O$ on a par $h$, $\vect{OM_2} = \vect{OA} + \vect{AM_2} = \vect{OA} + \lambda \vect{AM_1} = \vect{OA} + \lambda (\vect{AO} + \vect{OM_1}) = \lambda \vect{OM_1} + (1-\lambda)\vect{OA}$. de m\^eme on a par $h'$, $\vect{OM_2} = \lambda \vect{OM_1} + (1-\mu)\vect{OB}$.

Calculons $h \circ h'(M)$
$$
\vect{Oh \circ h'(M)} = \vect{Oh(h'(M))} = \lambda \vect{Oh'(M)} + (1-\lambda)\vect{OA} = \lambda(\mu\vect{OM} + (1-\mu)\vect{OB}) + (1-\lambda)\vect{OA}
$$
$$
= \lambda\mu\vect{OM} + \lambda(1-\mu)\vect{OB} + (1-\lambda)\vect{OA}
$$
comme on a $\lambda\mu = 1$ donc
$$
\vect{Oh \circ h'(M)} = \vect{OM} + (\lambda-1)\vect{OB} + (1-\lambda)\vect{OA} = \vect{OM} + (1-\lambda)\vect{BA}
$$

Donc
\begin{itemize}
    \item si $A=B$ on a $\vect{Oh \circ h'(M)} = \vect{OM}$, donc l'identit\'e.
    \item si $A \neq B$ on a $\vect{Oh \circ h'(M)} = \vect{OM} + (1-\lambda)\vect{BA}$, qui est une translation de vecteur $(1-\lambda)\vect{BA}$
\end{itemize}

Pour $h' \circ h(M)$, m\^eme r\'esultat raisonnement. 
Donc
\begin{itemize}
    \item si $A=B$ on a $\vect{Oh \circ h'(M)} = \vect{OM}$, donc l'identit\'e.
    \item si $A \neq B$ on a $\vect{Oh \circ h'(M)} = \vect{OM} + (1-\mu)\vect{AB}$, qui est une translation de vecteur $(1-\mu)\vect{AB}$
\end{itemize}

\subsection*{Exercice 21.2}
Avec $\lambda = 1/3$ et $\mu = 2$ on a 
$$
\vect{Oh \circ h'(M)} = \lambda\mu\vect{OM} + \lambda(1-\mu)\vect{OB} + (1-\lambda)\vect{OA} = 2/3\vect{OM} - 1/3\vect{OB}+ 2/3\vect{OA} = ??
$$

$$
\vect{Oh' \circ h(M)} = \mu\lambda\vect{OM} + \mu(1-\lambda)\vect{OA} + (1-\mu)\vect{OB} = 2/3\vect{OM} + 4/3\vect{OB}-  \vect{OA} = ??
$$

\end{document}

