\documentclass[]{book}

%These tell TeX which packages to use.
\usepackage{array,epsfig}
\usepackage{amsmath}
\usepackage{amsfonts}
\usepackage{amssymb}
\usepackage{amsxtra}
\usepackage{amsthm}
\usepackage{mathrsfs}
\usepackage{color}
\usepackage[margin=2cm,top=2.5cm,headheight=16pt,headsep=0.1in,heightrounded]{geometry}
\usepackage{fancyhdr}
\pagestyle{fancy}
\usepackage{tikz}


%Here I define some theorem styles and shortcut commands for symbols I use often
\theoremstyle{definition}
\newtheorem{defn}{Definition}
\newtheorem{thm}{Theorem}
\newtheorem{cor}{Corollary}
\newtheorem*{rmk}{Remark}
\newtheorem{lem}{Lemma}
\newtheorem*{joke}{Joke}
\newtheorem{ex}{Example}
\newtheorem*{soln}{Solution}
\newtheorem{prop}{Proposition}

\newcommand{\lra}{\longrightarrow}
\newcommand{\ra}{\rightarrow}
\newcommand{\surj}{\twoheadrightarrow}
\newcommand{\graph}{\mathrm{graph}}
\newcommand{\bb}[1]{\mathbb{#1}}
\newcommand{\Z}{\bb{Z}}
\newcommand{\Q}{\bb{Q}}
\newcommand{\R}{\bb{R}}
\newcommand{\C}{\bb{C}}
\newcommand{\N}{\bb{N}}
\newcommand{\M}{\mathbf{M}}
\newcommand{\m}{\mathbf{m}}
\newcommand{\MM}{\mathscr{M}}
\newcommand{\HH}{\mathscr{H}}
\newcommand{\Om}{\Omega}
\newcommand{\Ho}{\in\HH(\Om)}
\newcommand{\bd}{\partial}
\newcommand{\del}{\partial}
\newcommand{\bardel}{\overline\partial}
\newcommand{\textdf}[1]{\textbf{\textsf{#1}}\index{#1}}
\newcommand{\img}{\mathrm{img}}
\newcommand{\ip}[2]{\left\langle{#1},{#2}\right\rangle}
\newcommand{\inter}[1]{\mathrm{int}{#1}}
\newcommand{\exter}[1]{\mathrm{ext}{#1}}
\newcommand{\cl}[1]{\mathrm{cl}{#1}}
\newcommand{\ds}{\displaystyle}
\newcommand{\vol}{\mathrm{vol}}
\newcommand{\cnt}{\mathrm{ct}}
\newcommand{\osc}{\mathrm{osc}}
\newcommand{\LL}{\mathbf{L}}
\newcommand{\UU}{\mathbf{U}}
\newcommand{\support}{\mathrm{support}}
\newcommand{\AND}{\;\wedge\;}
\newcommand{\OR}{\;\vee\;} 
\newcommand{\Oset}{\varnothing}
\newcommand{\st}{\ni}
\newcommand{\wh}{\widehat}
\newcommand{\vect}[1]{\overrightarrow{#1}}

%Pagination stuff.
\setlength{\topmargin}{-.3 in}
%\setlength{\oddsidemargin}{0in}
%\setlength{\evensidemargin}{0in}
\setlength{\textheight}{9.in}
\setlength{\textwidth}{6.5in}
\cfoot{page \thepage}
\lhead{MEU303 - Alg\`ebre}
\rhead{Cours 1}
\pagestyle{fancy}


\begin{document}

\subsection*{Rappel de cours}
\begin{defn}
Bla bla
\end{defn}



\newpage
\subsection*{II.1 Exercice 1}
??
\subsection*{II.1 Proposition 1.1}
??

\subsection*{II.3 Proposition 1.1}
Soit $P$ une matrice de passage de $\mathcal{B}$ vers $\mathcal{B'}$ et $Q$ une matrice de passage de $\mathcal{B'}$ vers $\mathcal{B}$. On a $P.Q = I$, en effet appliquer le passage d'un vecteur $\vect{u}$ d'une base vers une autre et inversement retourne le m\^eme vecteur $\vect{u}$. Donc $Q=P^{-1}$, ce qui montre que la matrice $P$ est inversible.

\subsection*{II.3 Exercice 2.1}
$GL_n(\R)$ est un groupe multiplicatif (pour la multiplication des matrices) si la multiplication est associative, il existe un \'el\'ement neutre et un inverse pour chaque \'el\'ement de $GL_n(\R)$. 
\begin{itemize}
\item Une matrice de passage est une matrice et la multiplication de matrice est associative
\item La matrice identit\'e est dans l'ensemble $GL_n(\R)$. C'est la matrice de passage de la base $\mathcal{B}$ vers la base $\mathcal{B}$.
\item une matrice de passage est inversible (voir exercice pr\'ec'edent).
\end{itemize}

Donc $GL_n(\R)$ est un groupe multiplicatif (pour la multiplication des matrices).

\subsection*{II.3 Exercice 2.2}
$\mathcal{M}_{n,p}(\R)$ n'est pas un groupe multiplicatif car il existe des matrices non inversibles. \\

Un ensemble $S$ est stable par rapport \`a une op\'eration $*$ si $\forall a, b \in S, a * b \in S$. $\mathcal{M}_{n,p}(\R)$ est n'est pas stable pour la multiplication car la multiplication de la matrice $\mathcal{M}_{1,2}$ et $\mathcal{M}_{4,5}$ n'existe pas.

\subsection*{II.4 Exercice 4}
??
\subsection*{II.4 Exercice 5}
Il peut y avoir 0, 1 ou plusieurs applications lin\'eaire pour passer d'une famille de vecteurs \`a l'autre. (voir TD). 
\subsection*{II.4 Exercice 6}
??

\subsection*{III.1 Propri\'et\'e 1.3}
??



\end{document}

