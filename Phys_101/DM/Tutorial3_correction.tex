\documentclass[]{book}

%These tell TeX which packages to use.
\usepackage{array,epsfig}
\usepackage{amsmath}
\usepackage{amsfonts}
\usepackage{amssymb}
\usepackage{amsxtra}
\usepackage{amsthm}
\usepackage{mathrsfs}
\usepackage{color}
\usepackage{pgfplots}

%Here I define some theorem styles and shortcut commands for symbols I use often
\theoremstyle{definition}
\newtheorem{defn}{Definition}
\newtheorem{thm}{Theorem}
\newtheorem{cor}{Corollary}
\newtheorem*{rmk}{Remark}
\newtheorem{lem}{Lemma}
\newtheorem*{joke}{Joke}
\newtheorem{ex}{Example}
\newtheorem*{soln}{Solution}
\newtheorem{prop}{Proposition}

\newcommand{\lra}{\longrightarrow}
\newcommand{\ra}{\rightarrow}
\newcommand{\surj}{\twoheadrightarrow}
\newcommand{\graph}{\mathrm{graph}}
\newcommand{\bb}[1]{\mathbb{#1}}
\newcommand{\Z}{\bb{Z}}
\newcommand{\Q}{\bb{Q}}
\newcommand{\R}{\bb{R}}
\newcommand{\C}{\bb{C}}
\newcommand{\N}{\bb{N}}
\newcommand{\M}{\mathbf{M}}
\newcommand{\m}{\mathbf{m}}
\newcommand{\MM}{\mathscr{M}}
\newcommand{\HH}{\mathscr{H}}
\newcommand{\Om}{\Omega}
\newcommand{\Ho}{\in\HH(\Om)}
\newcommand{\bd}{\partial}
\newcommand{\del}{\partial}
\newcommand{\bardel}{\overline\partial}
\newcommand{\textdf}[1]{\textbf{\textsf{#1}}\index{#1}}
\newcommand{\img}{\mathrm{img}}
\newcommand{\ip}[2]{\left\langle{#1},{#2}\right\rangle}
\newcommand{\inter}[1]{\mathrm{int}{#1}}
\newcommand{\exter}[1]{\mathrm{ext}{#1}}
\newcommand{\cl}[1]{\mathrm{cl}{#1}}
\newcommand{\ds}{\displaystyle}
\newcommand{\vol}{\mathrm{vol}}
\newcommand{\cnt}{\mathrm{ct}}
\newcommand{\osc}{\mathrm{osc}}
\newcommand{\LL}{\mathbf{L}}
\newcommand{\UU}{\mathbf{U}}
\newcommand{\support}{\mathrm{support}}
\newcommand{\AND}{\;\wedge\;}
\newcommand{\OR}{\;\vee\;}
\newcommand{\Oset}{\varnothing}
\newcommand{\st}{\ni}
\newcommand{\wh}{\widehat}

%Pagination stuff.
\setlength{\topmargin}{-.3 in}
\setlength{\oddsidemargin}{0in}
\setlength{\evensidemargin}{0in}
\setlength{\textheight}{9.in}
\setlength{\textwidth}{6.5in}
\pagestyle{empty}



\begin{document}

\subsection*{Rappel de cours}
\begin{itemize}
\item Th\'eor\`eme de l'\'energie cin\'etique. $\sum_{\overrightarrow{F}} W_{A \to B}(\overrightarrow{F}) = E_c(t_b) - E_c(t_a)$ avec $E_c(t) = \frac{1}{2}\Vert v(t) \Vert^2$ avec $t_b>t_a$.
\item Le travail du poids $\overrightarrow{P} = m \overrightarrow{g}$ sur le segment $AB$ est $W_{A \to B}(\overrightarrow{F}) = -m.g(z_b-z_a)$ avec $z_{a(b)}$ l'altitude du point $a$ (resp. $b$). Ou $W_{A \to B}(\overrightarrow{F}) = -m.g.h$ avec $h$ la diff\'erence d'altitude entre les points $a$ et $b$.
\item L'\'energie potentielle due au poids est \'egale $m.g.z$ avec $z$ l'altitude de la masse $m$.
\end{itemize}


\subsection*{Exo 1}

\subsubsection*{Q 1.1.2}
D'apr\`es le th\'eor\`eme de l'\'energie cin\'etique. $\sum_{\overrightarrow{F}}W_{A \to B}(\overrightarrow{F}) = E_c(t_b) - E_c(t_a)$ avec $E_c(t) = \frac{1}{2}\Vert v(t) \Vert^2$. On a $t_a=0$, $v(t_a) = 0$ et $v(t_b) = 1m/s \,(i.e 3.6\, km/h)$ et comme on n\'eglige les frottements et que la route est horizontale (ie. \'energie potentielle est nulle), il n'y a que la force de pouss\'ee.\\
$$\sum_{\overrightarrow{F}}W_{A \to B}(\overrightarrow{F}) = \frac{1}{2}.1000.1^2 - \frac{1}{2}.1000.0^2 = 500J$$

\subsubsection*{Q 1.1.3}
On a $W_{A \to B}(\overrightarrow{F}) = \overrightarrow{F}.d$ avec $\Vert \overrightarrow{F} \Vert = 500$.
$$500 = 500.d$$
Donc il faut pousser la voiture sur $1m$.

\subsubsection*{Q 1.1.4}
On a $\overrightarrow{F} = m.\overrightarrow{a}$. Si la force est constante alors l'acc\'el\'eration est \'egalement constante car la masse ne change pas. Lorsque l'acc\'el\'eration est constante alors $v_f = v_i + a.t$ et la distance parcourue avec une acc\'el\'eration constante \`a partir d'une vitesse initiale $v_i$ est $l=v_it+\frac{1}{2}at^2$.
$$v_f^2 - v_i^2 = (v_i + at)^2 - v_i^2 = v_i^2 + 2 v_iat + a^2t^2 - v_i^2 = 2a(v_it + \frac{1}{2}at^2) = 2al$$
ou
L'acc\'el\'eration est constante donc $F = m.a$ et D'apr\`es le th\'eor\`eme de l'\'energie cin\'etique, on a
$$\sum_{\overrightarrow{F}}W_{A \to B}(\overrightarrow{F}) = F.l = E_c(t_f) - E_c(t_i) = \frac{1}{2}m.v_f^2 - \frac{1}{2}m.v_i^2$$
$$m.a.l = \frac{1}{2}m(v_f^2 - v_i^2)$$
$$(v_f^2 - v_i^2) = 2a.l$$


\subsubsection*{Q 1.1.5}
Si la vitesse est constante alors avec $v(t_b) = v(t_a)$, donc $\sum_{\overrightarrow{F}}W_{A \to B}(\overrightarrow{F}) = E_c(t_b) - E_c(t_a) = 0$. Le travail est nulle et comme il n'y a que la force de pouss'ee, alors cette force est nulle.\\

Ceci n'est pas en accord l'exp\'erience, car lorsque l'on pousse une voiture sur une route horizontale, il faut constamment la pousser pour qu'elle avance. 

\subsubsection*{Q 1.2.1}
Si la vitesse est constante alors avec $v(t_b) = v(t_a)$, donc $\sum_{\overrightarrow{F}}W_{A \to B}(\overrightarrow{F}) = W_{A \to B}(\overrightarrow{F}) + W_{A \to B}(\overrightarrow{F_f}) = E_c(t_b) - E_c(t_a) = 0$. Le travail est nulle et comme la force de frottement est n'egative (sens inverse du mouvement), alors il faut une force de pouss\'ee \'egale \`a la force de frottement.\\

\subsubsection*{Q 1.2.2}
Le travail fournie par la force de pouss\'ee est $W_{A \to B}(\overrightarrow{F}) = F.l$ avec $F = 50N$ et $l=100m$. Donc le travail fourni est \'egale \`a $5000J$.\\
L'\'energie fournie a servi \`a annuler la force de frottement des pneus sur la route.

\subsection*{Exo 2}
\subsubsection*{Q 2.2}
La variation de l'\'energie potentielle est donc $-m.g.(z_b - z_h)$ ou $-m.g.h$ avec $h = z_b - z_h = -9m$.

\subsubsection*{Q 2.3}
Le travail de la force de la pesanteur est $W_{A \to B}(\overrightarrow{P})= m.g.h = 90000$.

\subsubsection*{Q 2.3}
Selon le th\'eor\`eme de l\'energie m\'ecanique $\sum_{\overrightarrow{F}}W_{A \to B}(\overrightarrow{F}) = E_c(t_b) - E_c(t_a)$.
La seule force appliqu\'ee est la force de la pesanteur $W_{A \to B}(\overrightarrow{P})= 90000$, la vitesse initiale est de $v_i = 1m/s$, la masse du v\'ehicule est de $1000kg$. Donc la vitesse en bas de la pente est \'egale \`a $\sqrt{\frac{2.(90000+500)}{1000}} = 13.45m/s$ (ou $48.4km/h$).

\subsection*{Exo 3}
\subsubsection*{Q 3.1}
Non, car les d\'t\'erioration du v\'ehicule absorbent une partie de l'\'energie du v\'ehicule au moment du choc. Cette partie n'est pas transmise aux occupants du v\'ehicule. Donc, ils absorbent un choc moindre. C'est pour cela que les voitures sont fabriqu\'ees pour se d\'eformer en cas de choc tout en prot\'egeant l'habitacle o\'u les passagers sont pr\'esents.

\subsubsection*{Q 3.2.a}
On a $u_i = 0m/s$, la seconde voiture est en stationnement.
$$
\left\{ 
\begin{array}{l}
 M\overrightarrow{V_i} + m\overrightarrow{u_i} = M\overrightarrow{V_f} + m\overrightarrow{u_f} \\
 \frac{1}{2}M\Vert \overrightarrow{V_i} \Vert^2 + \frac{1}{2}m\Vert \overrightarrow{u_i} \Vert^2 = \frac{1}{2}M\Vert \overrightarrow{V_f} \Vert^2 + \frac{1}{2}m\Vert \overrightarrow{u_f} \Vert^2 + E_d\\
\end{array}
\right. 
$$

$$
\left\{ 
\begin{array}{l}
 M.V_i = M.V_f + m.u_f \\
 M.V_i^2  = M.V_f^2 + m.u_f^2 + 2E_d\\
\end{array}
\right. 
$$

$$
\left\{ 
\begin{array}{l}
 V_i - \frac{m}{M}u_f = V_f  \\
 M.V_i^2  = M.V_f^2 + m.u_f^2 + 2E_d\\
\end{array}
\right. 
$$

$$
\left\{ 
\begin{array}{l}
 V_i - \frac{m}{M}u_f = V_f  \\
 M.V_i^2  = M.(V_i - \frac{m}{M}u_f)^2 + m.u_f^2 + 2E_d\\
\end{array}
\right. 
$$

$$
\left\{ 
\begin{array}{l}
 V_i - \frac{m}{M}u_f = V_f  \\
 M.V_i^2  = M.V_i^2 - 2m.V_i.u_f + \frac{m^2}{M}u_f^2 + m.u_f^2 + 2E_d\\
\end{array}
\right. 
$$

$$
\left\{ 
\begin{array}{l}
 V_i - \frac{m}{M}u_f = V_f  \\
 0 = - 2m.V_i.u_f + \frac{m^2}{M}u_f^2 + m.u_f^2 + 2E_d\\
\end{array}
\right. 
$$

$$
\left\{ 
\begin{array}{l}
 V_i - \frac{m}{M}u_f = V_f  \\
 u_f^2(\frac{m^2}{M} + m) - 2m.V_i.u_f + 2E_d = 0\\
\end{array}
\right. 
$$

???????


\subsubsection*{Q 3.2.b}
La condition limite est lorsque les 2 voitures ont la m\^eme vitesse apr\'es le choc. Donc on a $V_f = u_f$.

$$
\left\{ 
\begin{array}{l}
 M\overrightarrow{V_i} + m\overrightarrow{u_i} = M\overrightarrow{V_f} + m\overrightarrow{u_f} \\
 \frac{1}{2}M\Vert \overrightarrow{V_i} \Vert^2 + \frac{1}{2}m\Vert \overrightarrow{u_i} \Vert^2 = \frac{1}{2}M\Vert \overrightarrow{V_f} \Vert^2 + \frac{1}{2}m\Vert \overrightarrow{u_f} \Vert^2 + E_d\\
\end{array}
\right. 
$$

$$
\left\{ 
\begin{array}{l}
 M.V_i =  M.V_f + m.u_f \\
 M.V_i^2  = M.V_f^2 + m.u_f^2 + 2E_d\\
\end{array}
\right. 
$$

$$
\left\{ 
\begin{array}{l}
 M.V_i =  V_f(M + m) \\
 M.V_i^2  = V_f^2(M + m) + 2E_d\\
\end{array}
\right. 
$$

$$
\left\{ 
\begin{array}{l}
 \frac{M}{M+m}V_i =  V_f \\
 M.V_i^2  = (\frac{M}{M+m}V_i)^2(M + m) + 2E_d\\
\end{array}
\right. 
$$

$$
\left\{ 
\begin{array}{l}
 \frac{M}{M+m}V_i =  V_f \\
 V_i^2(M -\frac{M^2}{M+m}) = 2E_d\\
\end{array}
\right. 
$$

$$
\left\{ 
\begin{array}{l}
 \frac{M}{M+m}V_i =  V_f \\
 \frac{M.m}{2(M+m)}V_i^2 = E_d\\
\end{array}
\right. 
$$

Donc ${M.m}{2(M+m)}V_i^2 > E_d$.

\subsubsection*{Q 3.2.c}
On a $M=1000kg$, $m=1700kg$, $V_i = 50km/h = 13.83m/s$, $u_i = 0$, et $E_d=20000J$.

$$
\left\{ 
\begin{array}{l}
 1000.13.83 = 1000V_f + 1700u_f \\
 1000.(13.83)^2  = 1000V_f^2 + 1700u_f^2 + 40000\\
\end{array}
\right. 
$$

$$
\left\{ 
\begin{array}{l}
 u_f = \frac{13830 - 1000V_f}{1700}\\
 151269  = 1000V_f^2 + 1700(\frac{13830- 1000V_f}{1700})^2\\
\end{array}
\right. 
$$

$$
\left\{ 
\begin{array}{l}
 u_f = \frac{13830 - 1000V_f}{1700}\\
 151269  = 1000V_f^2 + \frac{13830^2 - 2.13830.1000V_f + (1000V_f)^2}{1700}\\
\end{array}
\right. 
$$

$$
\left\{ 
\begin{array}{l}
 u_f = \frac{13830 - 1000V_f}{1700}\\
 0 = 2700000V_f^2 -65888400 - 27660000V_f \\
\end{array}
\right. 
$$

$$
\left\{ 
\begin{array}{l}
 u_f = 0.93m/s\\
 V_f = 12.24m/s \\
\end{array}
\right. 
$$
Ou
$$
\left\{ 
\begin{array}{l}
 u_f = 9.3m/s\\
 V_f = -2m/s \\
\end{array}
\right. 
$$

La premiere solution n'est pas possible car la seconde voiture ne peux pas passer au travers de la voiture en stationnement.

\subsubsection*{Q 3.2.d}
L'\'energie dissip'ee par la d\'eformation est $E_d = 20000J$ et $E_d = \overrightarrow{F}.d$. Donc la norme de la force est \'egale \`a $20000/.25 = 80.000N$.

\subsubsection*{Q 3.2.e}
 L'id\'ee du pare-choc en ressort n'est pas une bonne id\'ee car le ressort restitue enti\`erement l'\'energie re\c{c}ue. Donc le passager devra \'egalement absorber cette \'energie.\\
 Mais je ne comprends pas pourquoi les heurtoirs de train ont des ressorts. Ils servent \`a absorber le choc!!!\\
 
 
 
QED.

\end{document}

