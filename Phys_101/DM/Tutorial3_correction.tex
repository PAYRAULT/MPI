\documentclass[]{book}

%These tell TeX which packages to use.
\usepackage{array,epsfig}
\usepackage{amsmath}
\usepackage{amsfonts}
\usepackage{amssymb}
\usepackage{amsxtra}
\usepackage{amsthm}
\usepackage{mathrsfs}
\usepackage{color}
\usepackage{pgfplots}

%Here I define some theorem styles and shortcut commands for symbols I use often
\theoremstyle{definition}
\newtheorem{defn}{Definition}
\newtheorem{thm}{Theorem}
\newtheorem{cor}{Corollary}
\newtheorem*{rmk}{Remark}
\newtheorem{lem}{Lemma}
\newtheorem*{joke}{Joke}
\newtheorem{ex}{Example}
\newtheorem*{soln}{Solution}
\newtheorem{prop}{Proposition}

\newcommand{\lra}{\longrightarrow}
\newcommand{\ra}{\rightarrow}
\newcommand{\surj}{\twoheadrightarrow}
\newcommand{\graph}{\mathrm{graph}}
\newcommand{\bb}[1]{\mathbb{#1}}
\newcommand{\Z}{\bb{Z}}
\newcommand{\Q}{\bb{Q}}
\newcommand{\R}{\bb{R}}
\newcommand{\C}{\bb{C}}
\newcommand{\N}{\bb{N}}
\newcommand{\M}{\mathbf{M}}
\newcommand{\m}{\mathbf{m}}
\newcommand{\MM}{\mathscr{M}}
\newcommand{\HH}{\mathscr{H}}
\newcommand{\Om}{\Omega}
\newcommand{\Ho}{\in\HH(\Om)}
\newcommand{\bd}{\partial}
\newcommand{\del}{\partial}
\newcommand{\bardel}{\overline\partial}
\newcommand{\textdf}[1]{\textbf{\textsf{#1}}\index{#1}}
\newcommand{\img}{\mathrm{img}}
\newcommand{\ip}[2]{\left\langle{#1},{#2}\right\rangle}
\newcommand{\inter}[1]{\mathrm{int}{#1}}
\newcommand{\exter}[1]{\mathrm{ext}{#1}}
\newcommand{\cl}[1]{\mathrm{cl}{#1}}
\newcommand{\ds}{\displaystyle}
\newcommand{\vol}{\mathrm{vol}}
\newcommand{\cnt}{\mathrm{ct}}
\newcommand{\osc}{\mathrm{osc}}
\newcommand{\LL}{\mathbf{L}}
\newcommand{\UU}{\mathbf{U}}
\newcommand{\support}{\mathrm{support}}
\newcommand{\AND}{\;\wedge\;}
\newcommand{\OR}{\;\vee\;}
\newcommand{\Oset}{\varnothing}
\newcommand{\st}{\ni}
\newcommand{\wh}{\widehat}

%Pagination stuff.
\setlength{\topmargin}{-.3 in}
\setlength{\oddsidemargin}{0in}
\setlength{\evensidemargin}{0in}
\setlength{\textheight}{9.in}
\setlength{\textwidth}{6.5in}
\pagestyle{empty}



\begin{document}

\subsection*{Rappel de cours}
\begin{itemize}
\item Th\'eor\`eme de l'\'energie cin\'etique. $\sum_{\overrightarrow{F}} W_{A \to B}(\overrightarrow{F}) = E_c(t_b) - E_c(t_a)$ avec $E_c(t) = \frac{1}{2}\Vert v(t) \Vert^2$ avec $t_b>t_a$.
\end{itemize}


\subsection*{Exo 1}

\subsubsection*{Q 1.1.2}
D'apr\`es le th\'eor\`eme de l'\'energie cin\'etique. $\sum_{\overrightarrow{F}}W_{A \to B}(\overrightarrow{F}) = E_c(t_b) - E_c(t_a)$ avec $E_c(t) = \frac{1}{2}\Vert v(t) \Vert^2$. On a $t_a=0$, $v(t_a) = 0$ et $v(t_b) = 1m/s \,(i.e 3.6\, km/h)$ et comme on n\'eglige les frottements et que la route est horizontale (ie. \'energie potentielle est nulle), il n'y a que la force de pouss\'ee.\\
$$\sum_{\overrightarrow{F}}W_{A \to B}(\overrightarrow{F}) = \frac{1}{2}.1000.1^2 - \frac{1}{2}.1000.0^2 = 500J$$

\subsubsection*{Q 1.1.3}
On a $W_{A \to B}(\overrightarrow{F}) = \overrightarrow{F}.d$ avec $\Vert \overrightarrow{F} \Vert = 500$.
$$500 = 500.d$$
Donc il faut pousser la voiture sur $1m$.

\subsubsection*{Q 1.1.4}
On a $\overrightarrow{F} = m.\overrightarrow{a}$. Si la force est constante alors l'acc\'el\'eration est \'egalement constante car la masse ne change pas. Lorsque l'acc\'el\'eration est constante alors $v_f = v_i + a.t$ et la distance parcourue avec une acc\'el\'eration constante \`a partir d'une vitesse initiale $v_i$ est $l=v_it+\frac{1}{2}at^2$.
$$v_f^2 - v_i^2 = (v_i + at)^2 - v_i^2 = v_i^2 + 2 v_iat + a^2t^2 - v_i^2 = 2a(v_it + \frac{1}{2}at^2) = 2al$$
ou
L'acc\'el\'eration est constante donc $F = m.a$ et D'apr\`es le th\'eor\`eme de l'\'energie cin\'etique, on a
$$\sum_{\overrightarrow{F}}W_{A \to B}(\overrightarrow{F}) = F.l = E_c(t_f) - E_c(t_i) = \frac{1}{2}m.v_f^2 - \frac{1}{2}m.v_i^2$$
$$m.a.l = \frac{1}{2}m(v_f^2 - v_i^2)$$
$$(v_f^2 - v_i^2) = 2a.l$$


\subsubsection*{Q 1.1.5}
Si la vitesse est constante alors avec $v(t_b) = v(t_a)$, donc $\sum_{\overrightarrow{F}}W_{A \to B}(\overrightarrow{F}) = E_c(t_b) - E_c(t_a) = 0$. Le travail est nulle et comme il n'y a que la force de pouss'ee, alors cette force est nulle.\\

Ceci n'est pas en accord l'exp\'erience, car lorsque l'on pousse une voiture sur une route horizontale, il faut constamment la pousser pour qu'elle avance. 

\subsubsection*{Q 1.2.1}
Si la vitesse est constante alors avec $v(t_b) = v(t_a)$, donc $\sum_{\overrightarrow{F}}W_{A \to B}(\overrightarrow{F}) = W_{A \to B}(\overrightarrow{F}) + W_{A \to B}(\overrightarrow{F_f}) = E_c(t_b) - E_c(t_a) = 0$. Le travail est nulle et comme la force de frottement est n'egative (sens inverse du mouvement), alors il faut une force de pouss\'ee \'egale \`a la force de frottement.\\

\subsubsection*{Q 1.2.2}
Le travail fournie par la force de pouss\'ee est $W_{A \to B}(\overrightarrow{F}) = F.l$ avec $F = 50N$ et $l=100m$. Donc le travail fourni est \'egale \`a $5000J$.\\
L'\'energie fournie a servi \`a annuler la force de frottement des pneus sur la route.

\subsection*{Exo 2}
\subsubsection*{Q 2.1}


QED.

\end{document}

