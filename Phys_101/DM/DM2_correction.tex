\documentclass[]{book}

%These tell TeX which packages to use.
\usepackage{array,epsfig}
\usepackage{amsmath}
\usepackage{amsfonts}
\usepackage{amssymb}
\usepackage{amsxtra}
\usepackage{amsthm}
\usepackage{mathrsfs}
\usepackage{color}
\usepackage{pgfplots}

%Here I define some theorem styles and shortcut commands for symbols I use often
\theoremstyle{definition}
\newtheorem{defn}{Definition}
\newtheorem{thm}{Theorem}
\newtheorem{cor}{Corollary}
\newtheorem*{rmk}{Remark}
\newtheorem{lem}{Lemma}
\newtheorem*{joke}{Joke}
\newtheorem{ex}{Example}
\newtheorem*{soln}{Solution}
\newtheorem{prop}{Proposition}

\newcommand{\lra}{\longrightarrow}
\newcommand{\ra}{\rightarrow}
\newcommand{\surj}{\twoheadrightarrow}
\newcommand{\graph}{\mathrm{graph}}
\newcommand{\bb}[1]{\mathbb{#1}}
\newcommand{\Z}{\bb{Z}}
\newcommand{\Q}{\bb{Q}}
\newcommand{\R}{\bb{R}}
\newcommand{\C}{\bb{C}}
\newcommand{\N}{\bb{N}}
\newcommand{\M}{\mathbf{M}}
\newcommand{\m}{\mathbf{m}}
\newcommand{\MM}{\mathscr{M}}
\newcommand{\HH}{\mathscr{H}}
\newcommand{\Om}{\Omega}
\newcommand{\Ho}{\in\HH(\Om)}
\newcommand{\bd}{\partial}
\newcommand{\del}{\partial}
\newcommand{\bardel}{\overline\partial}
\newcommand{\textdf}[1]{\textbf{\textsf{#1}}\index{#1}}
\newcommand{\img}{\mathrm{img}}
\newcommand{\ip}[2]{\left\langle{#1},{#2}\right\rangle}
\newcommand{\inter}[1]{\mathrm{int}{#1}}
\newcommand{\exter}[1]{\mathrm{ext}{#1}}
\newcommand{\cl}[1]{\mathrm{cl}{#1}}
\newcommand{\ds}{\displaystyle}
\newcommand{\vol}{\mathrm{vol}}
\newcommand{\cnt}{\mathrm{ct}}
\newcommand{\osc}{\mathrm{osc}}
\newcommand{\LL}{\mathbf{L}}
\newcommand{\UU}{\mathbf{U}}
\newcommand{\support}{\mathrm{support}}
\newcommand{\AND}{\;\wedge\;}
\newcommand{\OR}{\;\vee\;}
\newcommand{\Oset}{\varnothing}
\newcommand{\st}{\ni}
\newcommand{\wh}{\widehat}

%Pagination stuff.
\setlength{\topmargin}{-.3 in}
\setlength{\oddsidemargin}{0in}
\setlength{\evensidemargin}{0in}
\setlength{\textheight}{9.in}
\setlength{\textwidth}{6.5in}
\pagestyle{empty}



\begin{document}

\subsection*{Rappel de cours}
\begin{itemize}
\item La composante de la force d'un point $M$, $\overrightarrow{F}(M)$ sur l'axe $\mathcal{O}_x$ est donn\'ee par le produit scalaire $f(x) = \overrightarrow{F}(M).\overrightarrow{i}$. 
\item Le travail d'une force $\overrightarrow{F}$ sur un segment $\overrightarrow{AB}$ est donn\'e par :
$$W_{A \to B}(\overrightarrow{F}) = \int_{A \to B} \overrightarrow{F}.\overrightarrow{i}dx = \int_{x_a}^{x_b} f(x)dx$$ 
\item On dira qu'une force est conservative si elle ne d\'epend que de la position et si son travail d'un point $A$ au point $B$ ne d\'epend pas du chemin suivi, ceci quels que soient les point $A$ et $B$.
$$\forall A,B,C,\, W_{A \to B} \overrightarrow{F} = W_{A \to C} \overrightarrow{F} + W_{C \to B} \overrightarrow{F}$$
\item Dans le cas o\`u le chemin est rectiligne, si une force est conservatrice alors l'\emph{\'energie potentielle} associ\'ee \`a la force $\overrightarrow{F}$ est not\'ee $E_p(x)$ est d\'efinie par :
$$W_{A \to B}(\overrightarrow{F}) = \int_{A \to B} \overrightarrow{F}.\overrightarrow{i}dx = E_p(x_a) - E_p(x_b)$$. 
\end{itemize}


\subsection*{Exo I}

\subsubsection*{Q 1.1 a et b}
Si la force $\overrightarrow{F_{el}}(x)$ est conservatrice alors on peut d\'efinir son \'energie potentielle associ\'ee $E_{el}(x)$. La $\overrightarrow{F_{el)}}(x)$ est conservatrice si son travail d'un point $A$ au point $B$ ne d\'epend pas du chemin suivi sur l'axe $\mathcal{O}_x$, soit $\forall A,B,C,\, W_{A \to B} \overrightarrow{F} = W_{A \to C} \overrightarrow{F} + W_{C \to B} \overrightarrow{F}$. Le travail de la force $\overrightarrow{F_{el}}(x)$ entre les points $A$ et $B$ sur l'axe $\mathcal{O}_x$ est donn\'e par $W_{A \to B} \overrightarrow{F} = \int_{x_a}^{x_b} f(x)dx$ avec $f(x) = -\frac{A}{x^2}$ car la force $\overrightarrow{F_{el}}(x)$ est parall\`ele \`a l'axe $\mathcal{O}_x$.\\

$$W_{A \to B} \overrightarrow{F_{el}} = \int_{x_a}^{x_b} -\frac{A}{x^2}dx = \left[ \frac{A}{x} \right]_{x_a}^{x_b} = \frac{A}{x_a} -\frac{A}{x_b}$$
Et,
$$W_{A \to C} \overrightarrow{F_{el}} + W_{C \to B} \overrightarrow{F_{el}} = (\frac{A}{x_a} -\frac{A}{x_c}) + (\frac{A}{x_c} -\frac{A}{x_b}) = \frac{A}{x_a} -\frac{A}{x_b} = W_{A \to B} \overrightarrow{F_{el}}$$
Donc la force $\overrightarrow{F_{el}}$ est conservatrice et $E_p(x) = \frac{A}{x}$.\\

Pour l'origine de l'\'energie potentielle associ\'ee \`a la force $\overrightarrow{F_{el}}$, je dirais \'a l'origine de l'axe $\mathcal{O}_x$  car comme les deux ions ne peuvent pas s'interpen\'etrer alors l'abscisse du ion $Na^{+}$ ne peut pas \^etre $0$.


\subsubsection*{Q 1.2}
On a: $$\frac{dE_{rep}(x)}{dx} = -\overrightarrow{F_{rep}}.\overrightarrow{i}$$
$$\frac{dE_{rep}(x)}{dx} = \frac{d(\frac{B}{x^8})}{dx} = -\frac{8B}{x^9}$$

Donc $\overrightarrow{F_{rep}}.\overrightarrow{i} = \frac{8B}{x^9}.\overrightarrow{i}$. La force $\overrightarrow{F_{rep}}$ est r\'epulsive car elle a le m\^eme sens que $\overrightarrow{i}$.


\subsubsection*{Q 1.3 a}
Rappel de cours:
\begin{itemize}
\item L'\'energie m\'ecanique d'un syst\`eme $E_m = E_c + E_p$ avec $E_c$ l'\'energie cin\'etique qui d\'epend de la masse et de la norme de la vitesse du syst\`eme physique \'etudi\'e et de l'\'energie potentielle $E_p$ qui correspond aux forces exerc\'ees sur le syst\`eme.
\end{itemize}

L'\'energie m\'ecanique est \'egale \`a $E_m = E_c + E_p$. Avec l'\'energie cin\'etique du syst\`eme $E_c = \frac{1}{2}mv_0^2$ et l'\'energie potentielle qui correspond aux 2 forces qui s'exercent sur le syst\'eme, $\overrightarrow{F_{el}}$, $\overrightarrow{F_{rep}}$, $E_p = \frac{A}{x} + \frac{B}{x^8}$. 

\subsubsection*{Q 1.3 b}
Les forces qui s'exercent surt le syst\'eme sont conservatrices donc l'\`energie m\'ecanique du syst\`eme est concerv\'ee.
$$E_m = \frac{1}{2}m\dot{x}^2 + \frac{A}{x} + \frac{B}{x^8}$$

Comme l'ion $Na$ est lanc\'e depuis l'infini vers l'ion $Cl$ alors \`a $t=0$, $E_m = \frac{1}{2}mv_0^2 = \frac{1}{2}3.84\times10^{-28}(-2\times10^6)^2 = 7.68\times10^{-16}J$.


\subsubsection*{Q 1.3 c}
On a 
$$E_m = \frac{1}{2}m\dot{x}^2 + \frac{A}{x} + \frac{B}{x^8}$$
$$\dot{x}^2 = \frac{2}{m}(E_m -\frac{A}{x} - \frac{B}{x^8})$$
$$\dot{x} = \sqrt{\frac{2}{m}(E_m -\frac{A}{x} - \frac{B}{x^8})}$$

Non, on ne peux pas d\'eterminer le sens du mouvement de l'ion $Na$. Pour cela il faut conna\^itre les conditions initiales du syst\`eme.\\

\subsubsection*{Q 1.3 d}
Lorsque la force \'electromagn\'etique de coh\'esion devient n\'egligeable devant la force de r\'epulsion alors 
$$\dot{x} = \sqrt{\frac{2}{m}(E_m - \frac{B}{x^8})}$$
Cette \'equation est n'est valide que si $\frac{2}{m}(E_m - \frac{B}{x^8}) > 0$ ou $E_m > \frac{B}{x^8})$, comme $E_m$ est constant et positif, et que la fonction $\frac{B}{x^8}$ est continue, lorsque $x(t=0) > 0$ alors on a $x(t) > 0$.\\

L'abscisse minimum $x_{min}$ est lorsque $E_m - \frac{B}{x^8} = 0$ ou $E_m = \frac{B}{x^8}$. Donc $x^8=\frac{B}{E_m}$ ou $x=\sqrt[8]{\frac{B}{E_m}} = 9.90\times10^{-11} \approx 10^{-10}$.\\

Donc \`a la distance minimum les atomes distants de leurs rayon atomique respectif. 

\subsubsection*{Q 1.3 e}
??

\subsubsection*{Q 1.4 a}
Lorsque le syst\`eme des deux ions est \`a l'\'equilibre on a :
$$\Vert \overrightarrow{F_{el}}\Vert = \Vert \overrightarrow{F_{rep}}\Vert$$.
$$\frac{A}{x^2} = \frac{8B}{x^9}$$
$$A = \frac{8B}{x^7}$$
$$x = \sqrt[7]{8\frac{B}{A}}$$
$$x = \sqrt[7]{8\frac{7.1\times10^{-96}}{2.3\times10^{-28}}} = 3.05\times10^{-10}m = 3.05 \textup{\AA}$$

\subsubsection*{Q 1.4 b}
Voir courbes


\subsubsection*{Q 1.4 c}
$$\frac{\left(7.1\cdot10^{-96}\right)}{\left(3.05\cdot10^{-10}\right)^{8}} - \frac{\left(2.3\cdot10^{-28}\right)}{\left(3.05\cdot10^{-10}\right)} = −6.59287163×10^{−19}$$


\subsection*{Exo II}

\subsubsection*{Q 1.5 a}
On a $F_{tot}(x) = F_{el}(x) + F_{rep}(x) = -\frac{A}{x^2} + \frac{8B}{x^9}$. Le d\'eveloppement limit\'e de $F_{tot}(x)$ autour du point $x_{eq}$:
$$F_{tot}(x) = F_{tot}(x_eq) + F'_{tot}(x_{eq})(x-x_{eq})$$
On a $F_{tot}(x_eq) = 0$ car $x_{eq}$ est la position d'\'equilibre.\\
On a $F'_{tot}(x) = \frac{2A}{x^3} - \frac{72B}{x^{10}}$. Donc
$$F_{tot}(x) = 0 + (\frac{2A}{x_{eq}^3} - \frac{72B}{x_{eq}^{10}})(x-x_{eq})$$
$$F_{tot}(x) = -(\frac{72B}{x_{eq}^{10}}-\frac{2A}{x_{eq}^3})(x-x_{eq})$$
Selon est de la forme
$$F_{tot}(x) = -k.(x-x_0)$$
Avec $k=\frac{72*7.1\times10^{-96}}{(3.05\times10^{-10})^{10}}-\frac{2*2.30\times10^{-28}}{(3.05\times10^{-10})^3} = 57.17 N.m$ et $x_0 = x_{eq}$.\\
En faisant le changement de variable $x-x_{eq} = \delta$ alors
$$F_{tot}(\delta) = -k.\delta$$

\subsubsection*{Q 1.5 b}
L'\'equation du mouvement de $Na^{+}$ est 
$$m\ddot{x}(\delta) = F_{tot}(\delta) = -k.\delta$$
$$\ddot{x}(\delta) + \frac{k}{m}\delta = 0$$

La r\'esolution de l'\'equation diff\'erentielle donne $\delta(t)=x_0cos(\sqrt{\frac{k}{m}}t)$. $\delta(t)$ est une fonction oscillante autour de la position $x_{eq}$.

\subsubsection*{Q 1.5 c}
La p\'eriode de l'oscillation est $T=\frac{2\pi}{\sqrt{\frac{k}{m}}} = 2\pi\sqrt{\frac{m}{k}}$ et la fr\'equence est $F=\frac{1}{T}$.\\

$$T = 2\pi \sqrt{\frac{3.84\times10^{-28}}{60}} = 1.59\times10^{-14}$$

\subsubsection*{Q 1.6 a}
L'ion $n$ suffit deux forces de sens contraire; la force provenant du ressort li\'e au ion $n+1$ et la force provenant de l'ion $n-1$. Ces forces sont propotionnels au coefficent du raideur $k$ du ressort et \`a l'extension de chaque ressort. L'extension du ressort entre $n-1$ et $n$ est $\delta_{n} - \delta_{n-1}$, celle avec le ressort $n+1$ est $\delta_{n+1} - \delta_{n}$. Donc la force totale est $k(\delta_{n+1} - \delta_{n}) - (k(\delta_{n} - \delta_{n-1}) = k(\delta_{n-1} -2\delta_{n} + \delta_{n-1})$.\\ 

Selon PFD on a $m\ddot{\delta_{n}} = k(\delta_{n-1} -2\delta_{n} + \delta_{n-1})$


\subsubsection*{Q 1.6 b}
On a $$\dot{\delta_{n}} = \alpha.\omega.i e^{i(wt-kx_n)}$$
et $$\ddot{\delta_{n}} = \alpha.\omega.i.\omega.i e^{i(wt-kx_n)} = -\alpha.\omega^2 e^{i(wt-kx_n)}$$
Donc
$$-m.\alpha.\omega^2 e^{i(wt-kx_n)} = k(\delta_{n-1} -2\delta_{n} + \delta_{n-1})$$
$$-m.\alpha.\omega^2 e^{i(wt-kx_n)} = k(\alpha e^{i(wt-kx_{n-1})} -2(\alpha e^{i(wt-kx_n)}) + \alpha e^{i(wt-kx_{n+1})})$$
$$-m.\alpha.\omega^2 e^{i(wt-kx_n)} = k.\alpha( e^{i(wt-kx_{n-1})} -2e^{i(wt-kx_n)} + e^{i(wt-kx_{n+1})})$$
$$m.\omega^2 e^{i(wt-kx_n)} = k( -e^{i(wt-kx_{n-1})} +2e^{i(wt-kx_n)} - e^{i(wt-kx_{n+1})})$$

\`A l'\'equilibre les ion $NA^{+}$ sont espac\'es par la distance $a$. Donc,
$$m.\omega^2 e^{i(wt-k.n.a)} = k( -e^{i(wt-k(n-1)a)} +2e^{i(wt-k.n.a)} - e^{i(wt-k(n+1)a)})$$
$$m.\omega^2 e^{i(wt-k.n.a)} = k( -e^{i(wt-k.n.a)}e^{ika} + 2e^{i(wt-k.n.a)} - e^{i(wt-k.n.a)}e^{-ika})$$
$$m.\omega^2 e^{i(wt-k.n.a)} = k.e^{i(wt-k.n.a)}( -e^{ika} + 2 - e^{-ika})$$
$$m.\omega^2 = k( -e^{ika} + 2 - e^{-ika})$$
$$m.\omega^2 = k( 2 - 2cos(ka))$$
$$m.\omega^2 = 2k( 1 - cos(ka))$$



QED.



\end{document}

