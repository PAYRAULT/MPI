\documentclass[]{book}

%These tell TeX which packages to use.
\usepackage{array,epsfig}
\usepackage{amsmath}
\usepackage{amsfonts}
\usepackage{amssymb}
\usepackage{amsxtra}
\usepackage{amsthm}
\usepackage{mathrsfs}
\usepackage{color}
\usepackage{pgfplots}

%Here I define some theorem styles and shortcut commands for symbols I use often
\theoremstyle{definition}
\newtheorem{defn}{Definition}
\newtheorem{thm}{Theorem}
\newtheorem{cor}{Corollary}
\newtheorem*{rmk}{Remark}
\newtheorem{lem}{Lemma}
\newtheorem*{joke}{Joke}
\newtheorem{ex}{Example}
\newtheorem*{soln}{Solution}
\newtheorem{prop}{Proposition}

\newcommand{\lra}{\longrightarrow}
\newcommand{\ra}{\rightarrow}
\newcommand{\surj}{\twoheadrightarrow}
\newcommand{\graph}{\mathrm{graph}}
\newcommand{\bb}[1]{\mathbb{#1}}
\newcommand{\Z}{\bb{Z}}
\newcommand{\Q}{\bb{Q}}
\newcommand{\R}{\bb{R}}
\newcommand{\C}{\bb{C}}
\newcommand{\N}{\bb{N}}
\newcommand{\M}{\mathbf{M}}
\newcommand{\m}{\mathbf{m}}
\newcommand{\MM}{\mathscr{M}}
\newcommand{\HH}{\mathscr{H}}
\newcommand{\Om}{\Omega}
\newcommand{\Ho}{\in\HH(\Om)}
\newcommand{\bd}{\partial}
\newcommand{\del}{\partial}
\newcommand{\bardel}{\overline\partial}
\newcommand{\textdf}[1]{\textbf{\textsf{#1}}\index{#1}}
\newcommand{\img}{\mathrm{img}}
\newcommand{\ip}[2]{\left\langle{#1},{#2}\right\rangle}
\newcommand{\inter}[1]{\mathrm{int}{#1}}
\newcommand{\exter}[1]{\mathrm{ext}{#1}}
\newcommand{\cl}[1]{\mathrm{cl}{#1}}
\newcommand{\ds}{\displaystyle}
\newcommand{\vol}{\mathrm{vol}}
\newcommand{\cnt}{\mathrm{ct}}
\newcommand{\osc}{\mathrm{osc}}
\newcommand{\LL}{\mathbf{L}}
\newcommand{\UU}{\mathbf{U}}
\newcommand{\support}{\mathrm{support}}
\newcommand{\AND}{\;\wedge\;}
\newcommand{\OR}{\;\vee\;}
\newcommand{\Oset}{\varnothing}
\newcommand{\st}{\ni}
\newcommand{\wh}{\widehat}

%Pagination stuff.
\setlength{\topmargin}{-.3 in}
\setlength{\oddsidemargin}{0in}
\setlength{\evensidemargin}{0in}
\setlength{\textheight}{9.in}
\setlength{\textwidth}{6.5in}
\pagestyle{empty}



\begin{document}

\subsection*{Rappel de cours}
\begin{itemize}
\item La composante de la force d'un point $M$, $\overrightarrow{F}(M)$ sur l'axe $\mathcal{O}_x$ est donn\'ee par le produit scalaire $f(x) = \overrightarrow{F}(M).\overrightarrow{i}$. 
\item Le travail d'une force $\overrightarrow{F}$ sur un segment $\overrightarrow{AB}$ est donn\'e par :
$$W_{A \to B}(\overrightarrow{F}) = \int_{A \to B} \overrightarrow{F}.\overrightarrow{i}dx = \int_{x_a}^{x_b} f(x)dx$$ 
\item On dira qu'une force est conservative si elle ne d\'epend que de la position et si son travail d'un point $A$ au point $B$ ne d\'epend pas du chemin suivi, ceci quels que soient les point $A$ et $B$.
$$\forall A,B,C,\, W_{A \to B} \overrightarrow{F} = W_{A \to C} \overrightarrow{F} + W_{C \to B} \overrightarrow{F}$$
\item Dans le cas o\`u le chemin est rectiligne, si une force est conservatrice alors l'\emph{\'energie potentielle} associ\'ee \`a la force $\overrightarrow{F}$ est not\'ee $E_p(x)$ est d\'efinie par :
$$W_{A \to B}(\overrightarrow{F}) = \int_{A \to B} \overrightarrow{F}.\overrightarrow{i}dx = E_p(x_a) - E_p(x_b)$$. 
\end{itemize}


\subsection*{Exo 1}

\subsubsection*{Q 1.1 a et b}
Si la force $\overrightarrow{F_{el}}(x)$ est conservatrice alors on peut d\'efinir son \'energie potentielle associ\'ee $E_{el}(x)$. La $\overrightarrow{F_{el)}}(x)$ est conservatrice si son travail d'un point $A$ au point $B$ ne d\'epend pas du chemin suivi sur l'axe $\mathcal{O}_x$, soit $\forall A,B,C,\, W_{A \to B} \overrightarrow{F} = W_{A \to C} \overrightarrow{F} + W_{C \to B} \overrightarrow{F}$. Le travail de la force $\overrightarrow{F_{el}}(x)$ entre les points $A$ et $B$ sur l'axe $\mathcal{O}_x$ est donn\'e par $W_{A \to B} \overrightarrow{F} = \int_{x_a}^{x_b} f(x)dx$ avec $f(x) = -\frac{A}{x^2}$ car la force $\overrightarrow{F_{el}}(x)$ est parall\`ele \`a l'axe $\mathcal{O}_x$.\\

$$W_{A \to B} \overrightarrow{F_{el}} = \int_{x_a}^{x_b} -\frac{A}{x^2}dx = \left[ \frac{A}{x} \right]_{x_a}^{x_b} = \frac{A}{x_a} -\frac{A}{x_b}$$
Et,
$$W_{A \to C} \overrightarrow{F_{el}} + W_{C \to B} \overrightarrow{F_{el}} = (\frac{A}{x_a} -\frac{A}{x_c}) + (\frac{A}{x_c} -\frac{A}{x_b}) = \frac{A}{x_a} -\frac{A}{x_b} = W_{A \to B} \overrightarrow{F_{el}}$$
Donc la force $\overrightarrow{F_{el}}$ est conservatrice et $E_p(x) = \frac{A}{x}$.\\

Pour l'origine de l'\'energie potentielle associ\'ee \`a la force $\overrightarrow{F_{el}}$, je dirais \'a l'origine de l'axe $\mathcal{O}_x$  car comme les deux ions ne peuvent pas s'interpen\'etrer alors l'abscisse du ion $Na^{+}$ ne peut pas \^etre $0$.


\subsubsection*{Q 1.2}
On a: $$\frac{dE_{rep}(x)}{dx} = -\overrightarrow{F_{rep}}.\overrightarrow{i}$$
$$\frac{dE_{rep}(x)}{dx} = \frac{d(\frac{B}{x^8})}{dx} = -\frac{8B}{x^9}$$

Donc $\overrightarrow{F_{rep}}.\overrightarrow{i} = \frac{8B}{x^9}.\overrightarrow{i}$. La force $\overrightarrow{F_{rep}}$ est r\'epulsive car elle a le m\^eme sens que $\overrightarrow{i}$.


QED.

\end{document}

