\documentclass[]{book}

%These tell TeX which packages to use.
\usepackage{array,epsfig}
\usepackage{amsmath}
\usepackage{amsfonts}
\usepackage{amssymb}
\usepackage{amsxtra}
\usepackage{amsthm}
\usepackage{mathrsfs}
\usepackage{color}

%Here I define some theorem styles and shortcut commands for symbols I use often
\theoremstyle{definition}
\newtheorem{defn}{Definition}
\newtheorem{thm}{Theorem}
\newtheorem{cor}{Corollary}
\newtheorem*{rmk}{Remark}
\newtheorem{lem}{Lemma}
\newtheorem*{joke}{Joke}
\newtheorem{ex}{Example}
\newtheorem*{soln}{Solution}
\newtheorem{prop}{Proposition}

\newcommand{\lra}{\longrightarrow}
\newcommand{\ra}{\rightarrow}
\newcommand{\surj}{\twoheadrightarrow}
\newcommand{\graph}{\mathrm{graph}}
\newcommand{\bb}[1]{\mathbb{#1}}
\newcommand{\Z}{\bb{Z}}
\newcommand{\Q}{\bb{Q}}
\newcommand{\R}{\bb{R}}
\newcommand{\C}{\bb{C}}
\newcommand{\N}{\bb{N}}
\newcommand{\M}{\mathbf{M}}
\newcommand{\m}{\mathbf{m}}
\newcommand{\MM}{\mathscr{M}}
\newcommand{\HH}{\mathscr{H}}
\newcommand{\Om}{\Omega}
\newcommand{\Ho}{\in\HH(\Om)}
\newcommand{\bd}{\partial}
\newcommand{\del}{\partial}
\newcommand{\bardel}{\overline\partial}
\newcommand{\textdf}[1]{\textbf{\textsf{#1}}\index{#1}}
\newcommand{\img}{\mathrm{img}}
\newcommand{\ip}[2]{\left\langle{#1},{#2}\right\rangle}
\newcommand{\inter}[1]{\mathrm{int}{#1}}
\newcommand{\exter}[1]{\mathrm{ext}{#1}}
\newcommand{\cl}[1]{\mathrm{cl}{#1}}
\newcommand{\ds}{\displaystyle}
\newcommand{\vol}{\mathrm{vol}}
\newcommand{\cnt}{\mathrm{ct}}
\newcommand{\osc}{\mathrm{osc}}
\newcommand{\LL}{\mathbf{L}}
\newcommand{\UU}{\mathbf{U}}
\newcommand{\support}{\mathrm{support}}
\newcommand{\AND}{\;\wedge\;}
\newcommand{\OR}{\;\vee\;}
\newcommand{\Oset}{\varnothing}
\newcommand{\st}{\ni}
\newcommand{\wh}{\widehat}

%Pagination stuff.
\setlength{\topmargin}{-.3 in}
\setlength{\oddsidemargin}{0in}
\setlength{\evensidemargin}{0in}
\setlength{\textheight}{9.in}
\setlength{\textwidth}{6.5in}
\pagestyle{empty}



\begin{document}


\subsection*{Exo 3.3.2}
\subsubsection*{Q1}
La balance mesure la compression du ressort d\^u au poids (m.g) de l'objet pos\'e sur le plateau. Comme la force de rappel du ressort est proportionelle \`a la variation de la longueur du ressort ($k.\Delta l$), il est possible de facilement mesurer le poids de l'objet.


\subsubsection*{Q2}
\emph{Identification du syst\`eme} : Le plateau de la balance.\\


\emph{Bilan des forces} : Le plateau est soumis \`a deux forces : 
\begin{itemize}
\item La force gravitationnelle; $F_g = m.\overrightarrow{g}$
\item La force de rappel du ressort; $F_r = k.\Delta l \overrightarrow{i}$ 
\end{itemize}


\emph{PFD} :  
Quand le syst\`eme est \`a l'\'equilibre, on a $F_g = F_r$. Donc, $m.g = k.\Delta l$. La masse de l'objet peut \^etre d\'etermin\'ee par l'\'equation $m = \frac{k.\Delta l}{g}$ avec $k$ le coefficient de raideur du ressort.


\subsubsection*{Q2}

\begin{itemize}
\item Dans la phase 1, la gravit\'e dans l'ascenseur est \'egale \`a $g + 2\, m/s$ car l'ascenceur acc\'el\`ere dans le sens contraire \`a la gravit\'e.
\item Dans la phase 2, la gravit\'e dans l'ascenseur est \'egale \`a $g + 0\, m/s$ car l'ascenceur n'acc\'el\`ere pas.
\item Dans la phase 3, la gravit\'e dans l'ascenseur est \'egale \`a $g - 2\, m/s$ car l'ascenceur acc\'el\`ere dans le sens identique \`a la gravit\'e.
\end{itemize}

La masse de la personne et le coefficient de raideur du ressort sont identiques durant chaque phase de l'ascenseur. Seul son poids est diff\'erent. On a $\frac{m}{k} = \frac{\Delta l}{g(p)}$. 

\begin{itemize}
\item Dans la phase 1, $\Delta l_1 = \Delta l.\frac{g+2}{g}$. Ceci fait, $\Delta l_1 > \Delta l$. Donc la balance va afficher un poid sup\'erieur \`a 70kg.
\item Dans la phase 2, $\Delta l_2 = \Delta l.\frac{g}{g}$. Ceci fait, $\Delta l_2 = \Delta l$. Donc la balance va afficher 70kg.
\item Dans la phase 3, $\Delta l_3 = \Delta l.\frac{g-2}{g}$. Ceci fait, $\Delta l_3 < \Delta l$. Donc la balance va afficher un poid inf\'erieur \`a 70kg.
\end{itemize}

\subsubsection*{Q3}
Lorsque le cable casse, l'acc\'el\'eration de la cabine est \'egale \`a la gravit\'e. Donc la gravit\'e dans l'ascenseur est nulle. Le poids affich\'e sur la balance est \'egale \`a $0$ car $k.\Delta l = m.0$.

\end{document}

