\documentclass[]{book}

%These tell TeX which packages to use.
\usepackage{array,epsfig}
\usepackage{amsmath}
\usepackage{amsfonts}
\usepackage{amssymb}
\usepackage{amsxtra}
\usepackage{amsthm}
\usepackage{mathrsfs}
\usepackage{color}

%Here I define some theorem styles and shortcut commands for symbols I use often
\theoremstyle{definition}
\newtheorem{defn}{Definition}
\newtheorem{thm}{Theorem}
\newtheorem{cor}{Corollary}
\newtheorem*{rmk}{Remark}
\newtheorem{lem}{Lemma}
\newtheorem*{joke}{Joke}
\newtheorem{ex}{Example}
\newtheorem*{soln}{Solution}
\newtheorem{prop}{Proposition}

\newcommand{\lra}{\longrightarrow}
\newcommand{\ra}{\rightarrow}
\newcommand{\surj}{\twoheadrightarrow}
\newcommand{\graph}{\mathrm{graph}}
\newcommand{\bb}[1]{\mathbb{#1}}
\newcommand{\Z}{\bb{Z}}
\newcommand{\Q}{\bb{Q}}
\newcommand{\R}{\bb{R}}
\newcommand{\C}{\bb{C}}
\newcommand{\N}{\bb{N}}
\newcommand{\M}{\mathbf{M}}
\newcommand{\m}{\mathbf{m}}
\newcommand{\MM}{\mathscr{M}}
\newcommand{\HH}{\mathscr{H}}
\newcommand{\Om}{\Omega}
\newcommand{\Ho}{\in\HH(\Om)}
\newcommand{\bd}{\partial}
\newcommand{\del}{\partial}
\newcommand{\bardel}{\overline\partial}
\newcommand{\textdf}[1]{\textbf{\textsf{#1}}\index{#1}}
\newcommand{\img}{\mathrm{img}}
\newcommand{\ip}[2]{\left\langle{#1},{#2}\right\rangle}
\newcommand{\inter}[1]{\mathrm{int}{#1}}
\newcommand{\exter}[1]{\mathrm{ext}{#1}}
\newcommand{\cl}[1]{\mathrm{cl}{#1}}
\newcommand{\ds}{\displaystyle}
\newcommand{\vol}{\mathrm{vol}}
\newcommand{\cnt}{\mathrm{ct}}
\newcommand{\osc}{\mathrm{osc}}
\newcommand{\LL}{\mathbf{L}}
\newcommand{\UU}{\mathbf{U}}
\newcommand{\support}{\mathrm{support}}
\newcommand{\AND}{\;\wedge\;}
\newcommand{\OR}{\;\vee\;}
\newcommand{\Oset}{\varnothing}
\newcommand{\st}{\ni}
\newcommand{\wh}{\widehat}

%Pagination stuff.
\setlength{\topmargin}{-.3 in}
\setlength{\oddsidemargin}{0in}
\setlength{\evensidemargin}{0in}
\setlength{\textheight}{9.in}
\setlength{\textwidth}{6.5in}
\pagestyle{empty}



\begin{document}


\subsection*{Exo 2.2.1}
\subsubsection*{Q1}

$x(t) = -2asin^2(\omega t))$, et $\left(f^n\right)' = nf'f^{n-1}$, avec $n=2$
et $f=sin(\omega t)$. De plus, $\left(sin(f)\right)' = f'cos(f)$ avec
$f=\omega t$ donc
$v_x(t) = x'(t) = -2a*2 \omega cos(\omega t) sin(\omega t)$.


$y(t)=2asin(\omega t)cos(\omega t)$ et $\left(fg\right)' = f'g+g'f$,
avec $f=sin(\omega t)$ et $g=cos((\omega t)$. Donc  $f'=\omega
cos(\omega t)$ et $g'=-\omega sin(\omega t)$, alors $2a(\omega cos(\omega t) cos(\omega t)- \omega sin(\omega t) sin(\omega t))$.

$$
v(t) = \left\{
\begin{array}{l} 
  v_x(t) = x'(t) = -4a\omega cos(\omega t) sin(\omega t) \\
  v_y(t) = y'(t) =  2a\omega (cos^2(\omega t) - sin^2(\omega t)) \\
  v_z(t) = z'(t) =  0 \\
\end{array}
\right\}
$$

$a_x(t) = v'_x(t)$ et $v_x(t)= -4a \omega cos(\omega t) sin(\omega t)$ et
$\left(fg\right)' = f'g+g'f$, avec 
avec $f=cos(\omega t)$ et $g=sin((\omega t)$. Donc  $f'=-\omega sin(\omega t)$ et $g'=\omega cos(\omega t)$, alors
$-4a \omega ((-\omega sin(\omega t)) sin(\omega t) + \omega
cos(\omega t) cos(\omega t))$. On a aussi $sin^2(x) + cos^2(x) = 1$.

$a_y(t)= v'_y(t)$ et $v_y(t)= 2a\omega (cos^2(\omega t) - sin^2(\omega t))$ avec
$\left(f+g\right)' = f'+g'$, on connait la d\'eriv\'ee de
$sin^2(\omega t)$. Il faut calculer la d\'eriv\'ee de
$cos^2(\omega t)$. La d\'eriv\'ee est $2 \omega (-sin(\omega t))
cos(\omega t)$. 

$$a(t) = \left\{
\begin{array}{l} 
  a_x(t) = v'_x(t) = 4a \omega^2\\
  a_y(t) = v'_y(t) = -8a \omega^2 sin(\omega t) cos(\omega t)\\
  a_z(t) = v'_z(t) = 0 \\
\end{array}
\right\}
$$

\subsubsection*{Q2}

$$\Vert v(t) \Vert = \sqrt{v_x^2(t) + v_y^2(t) + v_z^2(t)} $$
$$= \sqrt{(-4a\omega cos(\omega t) sin(\omega t))^ 2 + (2a\omega
  (cos^2(\omega t) - sin^2(\omega t))^2 + 0^2)}$$
$$= \sqrt{16a^2\omega^2 cos^2(\omega t) sin^2(\omega t) +
  4a^2\omega^2(cos^4(\omega t)-2cos^2(\omega t)sin^2(\omega t) + sin^4(\omega t))}$$
$$= \sqrt{8a^2\omega^2 cos^2(\omega t) sin^2(\omega t) +
  4a^2\omega^2cos^4(\omega t)+ 4a^2\omega sin^4(\omega t))}$$
$$= 2a\omega \sqrt{2cos^2(\omega t) sin^2(\omega t) +
  cos^4(\omega t)+ sin^4(\omega t))}$$
$$= 2a\omega \sqrt{(cos^2(\omega t) +sin^2(\omega t))^2}$$
$$= 2a\omega \sqrt{1^2}$$
$$= 2a\omega$$

\subsubsection*{Q3}
Le vecteur vitesse est \'egale \`a $= 2a\omega$ qui est constant. Donc
le mouvement est uniforme.

\subsubsection*{Q4}
$$sin(2x) = 2sin(x)cos(x)$$
$$cos(2x) = cos^2(x)-sin^2(x) = 1 - 2 sin^2(x) =  1 - 2 cos^2(x)$$
$$R=\Vert AM \Vert = \sqrt{(x-a)^2 + (y-b)^2}$$
$$R^2=(x-a)^2 + (y-b)^2$$
ou en p\^olaire
$$ x = a + Rcos(\alpha), y = b + Rsin(\alpha), $$


\subsubsection*{Q5}
Supposons que la voiture d\'ecrit un cercle de centre $(x_c, y_c)$ et
de rayon $R$. Donc

$$
\begin{array}{l} 
  x_c + Rcos(\alpha) = -2asin^2( \omega t)\\
  y_c + Rsin(\alpha)= -2asin( \omega t)cos( \omega t)\\
\end{array}
$$

$$
\begin{array}{l} 
  x_c + Rcos(\alpha) = -2a(\frac{1-cos(2 \omega t)}{2})\\
  y_c + Rsin(\alpha)= asin( 2 \omega t)\\
\end{array}
$$

$$
\begin{array}{l} 
  x_c + Rcos(\alpha) = -2a + acos(2 \omega t)\\
  y_c + Rsin(\alpha)= 0+ asin( 2 \omega t)\\
\end{array}
$$

$$x_c = -2a, y_c = 0, R = a, \alpha = 2 \omega t$$

La voiture d\'ecrit un cercle de centre $(-2a,0)$ et de rayon $a$.


\subsubsection*{Q6}
La p\'eriode $T$ est $\frac{\omega}{\pi}$.

\subsubsection*{Q7}
$$
\begin{array}{c | c | c}
  t & x_c & y_c  \\
  \hline
  0 & -2a + a cos(2 \omega * 0) = -a = -1 & 0 + a
  sin(2 \omega * 0) = 0\\
  \frac{\pi}{4 \omega}  & -2a + a cos(2 \omega  \frac{\pi}{4 \omega})
  = -2a = -2 & 0 + a
  sin(2 \omega \frac{\pi}{4 \omega}) = a = 1\\
  \frac{\pi}{2 \omega}  & -2a + a cos(2 \omega  \frac{\pi}{2 \omega})
  = -3a = -3 & 0 + a
  sin(2 \omega \frac{\pi}{2 \omega}) = 0  \\
\end{array}
$$


\end{document}

