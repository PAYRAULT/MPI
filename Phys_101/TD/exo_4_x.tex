\documentclass[]{book}

%These tell TeX which packages to use.
\usepackage{array,epsfig}
\usepackage{amsmath}
\usepackage{amsfonts}
\usepackage{amssymb}
\usepackage{amsxtra}
\usepackage{amsthm}
\usepackage{mathrsfs}
\usepackage{color}

%Here I define some theorem styles and shortcut commands for symbols I use often
\theoremstyle{definition}
\newtheorem{defn}{Definition}
\newtheorem{thm}{Theorem}
\newtheorem{cor}{Corollary}
\newtheorem*{rmk}{Remark}
\newtheorem{lem}{Lemma}
\newtheorem*{joke}{Joke}
\newtheorem{ex}{Example}
\newtheorem*{soln}{Solution}
\newtheorem{prop}{Proposition}

\newcommand{\lra}{\longrightarrow}
\newcommand{\ra}{\rightarrow}
\newcommand{\surj}{\twoheadrightarrow}
\newcommand{\graph}{\mathrm{graph}}
\newcommand{\bb}[1]{\mathbb{#1}}
\newcommand{\Z}{\bb{Z}}
\newcommand{\Q}{\bb{Q}}
\newcommand{\R}{\bb{R}}
\newcommand{\C}{\bb{C}}
\newcommand{\N}{\bb{N}}
\newcommand{\M}{\mathbf{M}}
\newcommand{\m}{\mathbf{m}}
\newcommand{\MM}{\mathscr{M}}
\newcommand{\HH}{\mathscr{H}}
\newcommand{\Om}{\Omega}
\newcommand{\Ho}{\in\HH(\Om)}
\newcommand{\bd}{\partial}
\newcommand{\del}{\partial}
\newcommand{\bardel}{\overline\partial}
\newcommand{\textdf}[1]{\textbf{\textsf{#1}}\index{#1}}
\newcommand{\img}{\mathrm{img}}
\newcommand{\ip}[2]{\left\langle{#1},{#2}\right\rangle}
\newcommand{\inter}[1]{\mathrm{int}{#1}}
\newcommand{\exter}[1]{\mathrm{ext}{#1}}
\newcommand{\cl}[1]{\mathrm{cl}{#1}}
\newcommand{\ds}{\displaystyle}
\newcommand{\vol}{\mathrm{vol}}
\newcommand{\cnt}{\mathrm{ct}}
\newcommand{\osc}{\mathrm{osc}}
\newcommand{\LL}{\mathbf{L}}
\newcommand{\UU}{\mathbf{U}}
\newcommand{\support}{\mathrm{support}}
\newcommand{\AND}{\;\wedge\;}
\newcommand{\OR}{\;\vee\;}
\newcommand{\Oset}{\varnothing}
\newcommand{\st}{\ni}
\newcommand{\wh}{\widehat}

%Pagination stuff.
\setlength{\topmargin}{-.3 in}
\setlength{\oddsidemargin}{0in}
\setlength{\evensidemargin}{0in}
\setlength{\textheight}{9.in}
\setlength{\textwidth}{6.5in}
\pagestyle{empty}



\begin{document}


\subsection*{Rappel de cours}
\subsubsection*{Travail}

\begin{itemize}
\item La composante de la force d'un point $M$, $\overrightarrow{F}(M)$ sur l'axe $\mathcal{O}_x$ est donn\'ee par le produit scalaire $f(x) = \overrightarrow{F}(M).\overrightarrow{i}$. 
\item Le travail d'une force $\overrightarrow{F}$ sur un segment $\overrightarrow{AB}$ est donn\'e par :
$$W_{A \to B}(\overrightarrow{F}) = \overrightarrow{F}.\overrightarrow{AB} = \int_{A \to B} \overrightarrow{F}.\overrightarrow{i}dx = \int_{x_a}^{x_b} f(x)dx$$ 
\item On dira qu'une force est conservative si elle ne d\'epend que de la position et si son travail d'un point $A$ au point $B$ ne d\'epend pas du chemin suivi, ceci quels que soient les point $A$ et $B$.
$$\forall A,B,C,\, W_{A \to B} \overrightarrow{F} = W_{A \to C} \overrightarrow{F} + W_{C \to B} \overrightarrow{F}$$
\item Dans le cas o\`u le chemin est rectiligne, si une force est conservatrice alors l'\emph{\'energie potentielle} associ\'ee \`a la force $\overrightarrow{F}$ est not\'ee $E_p(x)$ est d\'efinie par :
$$W_{A \to B}(\overrightarrow{F}) = \int_{A \to B} \overrightarrow{F}.\overrightarrow{i}dx = E_p(x_b) - E_p(x_a)$$. 
\item Le travail du poids $\overrightarrow{P} = m\overrightarrow{g}$ sur le segment $\overrightarrow{AB}$ est $W_{A \to B}(\overrightarrow{P}) = -mg(z_b - z_a) = -mgh$.
\item Le travail de la force de rappel \'elastique d'un ressort de raideur $k$, $\overrightarrow{F} = -k.x\overrightarrow{i}$ est $W_{A \to B}(\overrightarrow{F}) = -\frac{1}{2}k(x_a^2 - x_b^2)$.
\end{itemize}


\subsubsection*{\'Energie}
\begin{itemize}
\item L'\'energie m\'ecanique d'un syst\`eme $E_m = E_c + E_p$ avec $E_c$ l'\'energie cin\'etique qui d\'epend de la masse et de la norme de la vitesse du syst\`eme physique \'etudi\'e et de l'\'energie potentielle $E_p$ qui correspond aux forces exerc\'ees sur le syst\`eme.
\item L'\'energie cin\'etique du syst\`eme $E_c = \frac{1}{2}mv^2$
\item L'\'energie potentielle qui correspond \`a l'ensemble des forces conservatives qui s'exercent sur le syst\'eme, $E_p(B) - E_p(A) = W_{A \to B}(\overrightarrow{F}_{conservatives})$
\item $E_m(B) - E_m(A) = W_{A \to B}(\overrightarrow{F}_{non\, conservatives})$
\end{itemize}

\subsubsection*{Puissance}
\begin{itemize}
\item La puissance $P$ repr\'esente l'\'energie transf\'er\'ee uniformement (ie. le travail) pendant une unit\'e de temps, $P =\frac{W}{\Delta t}$.
\item $1\,W = 1\,J.s^{-1} = 1\, N.m.s^{-1} = 1\, kg.m^{2}.s^{-3}$
\end{itemize}


\subsection*{Exo 4.1.1}
L'\'electron est soumis a la force gravitationnelle. Donc $E_m = E_c + E_p$ avec $E_c = \frac{1}{2}mv^2$. On n\'eglige l'\'energie de la force gravitationnelle devant celle de l'\'energie cim\'etique. On a $E_m = \frac{1}{2}mv^2 = 18\,keV$, donc $v = \sqrt{\frac{2*18\,keV}{m}}$ et $m=9.10\,10^{-31}kg$, $18\,keV = 2.88\,10^{-12}$. 


\subsection*{Exo 4.1.2}
Il faut monter une masse de $m=500+5*70 = 850\,kg$ \`a une vitesse de $v=25/60 = 0.41\,m/s$. La puissance n\'ecessaire est $P=m.g.v = 850*9.81*0.41 = 3418\,Watt$.\\
 
En l'abscence de frottements la puissance n\'ecessaire pour lever la cabine d'ascenseur est \'egale \`a la puissance fournie par le moteur. Le puissance du poids est r\'esistante, celle du moteur est motrice.\\
v
On a $P =\frac{W}{\Delta_t}$, donc $W = P*\Delta_t = 3418*60 = 205kJ$.

\subsection*{Exo 4.1.3}
\subsubsection{Q1}
$TWh$ repr\'esente des $10^{12}Wh$.

\subsubsection{Q2}
On a $1W = 1J/s$. On produit $429.10^{12}Wh$ pour une ann\'ee, donc on a produit $\frac{429.10^{12}}{24*365.25} = 48.10^{9} W$. Ce qui fait $48.10^{9}*(24*365.25*3600) = 1.510^{18}J$.

\subsubsection{Q3}
La puissance \'electrique moyenne d'un r\'eacteur est de $\frac{48.10^{9}}{58} = 0.82 MW$.  

\subsection*{Exo 4.2}
\subsubsection{Q1}
Le travail accompli par la force est $\overrightarrow{F}.\overrightarrow{AB}$

\subsubsection{Q2}
On a $E_m = E_c + E_p$. L'\'energie du syst\`eme $E_m$ est conserv\'ee . Donc $E_{c0} + E_p = E_{cf} + E_{p_f}$ avec $E_{p_f} = 0$ car aucune force ne s'exerce sur la masse. Donc l'accroissement de l'\'energie cin\'etique est $E_{cf} - E_{c0} = E_p$.

\subsubsection{Q3}
La puissance moyenne d\'evelopp\'e est $P=\frac{\Delta_W}{\Delta_t} = \frac{E_p}{T} = \frac{\overrightarrow{F}.\overrightarrow{AB}}{T}$. 

\subsection*{Exo 4.2.3 - Traineau}
\subsubsection{Q1.a}
3 forces s'exercent sur la masse $m$; la force de la pesanteur (verticale) $\overrightarrow{P}$, la force de r\'eaction $\overrightarrow{R}$ (perpendiculaire au plan inclin\'e) et la force de frottement statique $\overrightarrow{f}$ (parall\`ele au plan inclin\'e). Comme la masse ne bouge pas, la somme des 3 forces est nulle.\\
On d\'efinit un rep\`ere orthonorm\'e centr\'e sur la masse $m$ et parall\`ele au plan inclin\'e. Dans ce rep\`ere, les 3 forces ont les coordonn\'ees:

$$
\left\{
\begin{array}{l l l}
 P_x = -m.g.sin(\beta) & R_x = 0 & f_x = \|\overrightarrow{f}\| \\
 P_y = -m.g.cos(\beta) & R_y = \|\overrightarrow{R}\| & f_y = 0\\
\end{array}
\right. 
$$

Donc, on a $ \|\overrightarrow{R}\| = m.g.cos(\beta)$ et $\|\overrightarrow{f}\| = m.g.sin(\beta)$.

\subsubsection{Q1.b}
Le coefficient de frottement statique $k_s$ peut \^etre d\'etermin\'e pour l'angle minimum du plan inclin\'e $\beta_{min}$ faisant boug'e la masse $m$.\\
$$k_s = \frac{\overrightarrow{f_{max}}}{\overrightarrow{R}}$$
$$k_s = \frac{-m.g.sin(\beta_{min})}{-m.g.cos(\beta_{min})}$$
$$k_s = \tan(\beta_{min})$$

\subsubsection{Q1.c}
Ne marche pas pour $\beta = 60^\circ$. Mais $\tan(6^\circ) = 0.1$.

\subsubsection{Q2.a}
Le travail $W_{C \to D} = W_{C \to D}(\overrightarrow{f\prime}) + W_{C \to D}(\overrightarrow{R}) + W_{C \to D}(\overrightarrow{P})$. On a 

$$
\left\{
\begin{array}{l l}
W_{C \to D}(\overrightarrow{R}) = 0 & La\, force\, \overrightarrow{R}\, est\, toujours\, perpendiculaire\, au\, deplacement.\\
W_{C \to D}(\overrightarrow{f^\prime}) = \overrightarrow{f^\prime}.l = -k_d.m.g.\cos(\beta).l & \\
W_{C \to D}(\overrightarrow{P}) = -m.g.h = m.g.l.\sin(\beta) & \\
\end{array}
\right. 
$$

Donc, $W_{C \to D} = m.g.l.(sin(\beta) - k_d.cos(\beta))$.


\subsubsection{Q2.b}
Le travail $W_{D \to E} = W_{D \to E}(\overrightarrow{f^\prime}) + W_{D \to E}(\overrightarrow{R}) + W_{D \to E}(\overrightarrow{P})$. On a 

$$
\left\{
\begin{array}{l l}
W_{D \to E}(\overrightarrow{R}) = 0 & La\, force\, \overrightarrow{R}\, est\, toujours\, perpendiculaire\, au\, deplacement.\\
W_{D \to E}(\overrightarrow{f^\prime}) = \overrightarrow{f^\prime}.l = -k_d.m.g.l^\prime & \\
W_{D \to E}(\overrightarrow{P}) = -m.g.h = 0 & \\
\end{array}
\right. 
$$

Donc, $W_{C \to D} = -k_d.m.g.l^\prime$.

\subsubsection{Q2.c}
Les forces \'etant toutes conservatives on a $E_m = W_{C \to D} + W_{D \to E} = 0$
$$ m.g.l.(sin(\beta) - k_d.cos(\beta)) -k_d.m.g.l^\prime = 0$$
$$ l.(sin(\beta) - k_d.cos(\beta)) -k_d.l^\prime = 0$$
$$ (sin(\beta) - k_d.cos(\beta)) -k_d\frac{l^\prime}{l} = 0$$
$$ sin(\beta) - k_d.cos(\beta) -k_d.r = 0$$
$$ sin(\beta) - k_d(r+cos(\beta)) = 0$$
$$ k_d = \frac{sin(\beta)}{r+cos(\beta)}$$

\subsubsection{Q2.d}
On a 
$$l^\prime = \frac{l.(\sin(\beta) - k_d.\cos(\beta))}{k_d} $$

Donc $l^\prime = 5\frac{\sin(6^\circ) - 0.05\cos(6^\circ)}{0.05} = 5.48m$.

\subsection*{Exo 4.3.1}
\subsubsection{Q1}
On a $F_g = G\frac{M_t.m}{(R_t+h)^2}$ avec $G= 6.674\,10^{-11} m^3.kg^{-1}.s^{-2}$, $R_t$ le rayon de la terre et $M_t$ la masse de la terre.

\subsubsection{Q2}
Le d\'eveloppement limit\'e de $f(x)$ au point $x_0$ est $f(x)=f(x_{0})+f'(x_{0}) \cdot (x-x_{0})+o(x-x_{0})$. On a $f'(x) = n.(1+x)^{n-1}$. Donc,
$$f(x) = (1+x_0)^n + n(1+x_0)^{n-1}.(x-x_0) + o(x-x_{0})$$

En prenant $x_0 = 0$, cela fait $f(x) = 1 + n.x^{n-1} + o(x)$.

\subsubsection{Q3}
On a $F_g = mg = G\frac{M_t.m}{(R_t+h)^2}$. Donc \`a petite altitude, 
$$g=G\frac{M_t}{R_t^2(1+\frac{h}{R_t})^2} = G\frac{M_t}{R_t(1+2\frac{h}{R_t})} = G\frac{M_t}{R_t^2}$$
$$g = 9.82$$
avec $R_t = 6370\,km$ et $M_t = 5.972\,10^{24}\,kg$. 

\subsubsection{Q4}
On a $\int \frac{K}{x^2} = K \int \frac{1}{x^2} = \frac{-K}{x} + C$

\subsubsection{Q5}
On a 
$$W_{0 \to h} = \int_{0}^{h} f(x)dx = \int_{0}^{h} G\frac{M_t.m}{(R_t+x)^2} dx$$

Changement de variable $X=R_t+x$ donc $dX = dx$.

$$W_{0 \to h} = \int_{R_t}^{R_t+h} G\frac{M_t.m}{X^2} dX = [-G\frac{M_t.m}{X} + C]_{R_t}^{R_t+h}$$
$$W_{0 \to h} = -G\frac{M_t.m}{R_t+h} + G\frac{M_t.m}{R_t} = -G.M_t.m(\frac{1}{R_t} - \frac{1}{R_t+h})$$
$$W_{0 \to h} =  -G.M_t.m(\frac{R_t+h -R_t}{R_t(R_t+h)}) = -G.M_t.m(\frac{h}{R_t(R_t+h)})$$
Lorsque $h$ est petit devant $R_t$ on a:
$$W_{0 \to h} =  -G.M_t.m(\frac{h}{R_t^2}) = -m.g.h$$


\subsection*{Exo 4.3.2}
\subsubsection{Q1}
$$\int ax^2+bx+c = \frac{ax^3}{3} + \frac{bx^2}{2} + cx + d$$

\subsubsection{Q2}
On a $\overrightarrow{F} = -k.x.\overrightarrow{i}$, donc
$$E_p(x) = W_{0 \to x}F(x) = \int_0^x F(x) = \int_0^x -k.x = [\frac{-k}{2}x^2]_0^x = -\frac{1}{2}k.x^2$$

\subsubsection{Q3}
L'\'energie potentielle gravitationnelle est $E_p(x) = mgx$.\\
La forme de l'energie potentielle totale est $E_p = mgx - -\frac{1}{2}k.x^2$.

\subsection*{Exo 4.4.1}
\subsubsection{Q1}
$$\int 2\, dx = 2x +C $$
$$\int Ax\, dx = \frac{1}{2}A x^2 + C$$
$$\int -\frac{K}{x^2}\, dx = \frac{K}{x} + C$$

\subsubsection{Q2}
Une force est conservative si le calcul de son travail ne d\'epend pas du chemin parcouru. Donc, il suffit de montrer que
$$W_{A \to C}\overrightarrow{F} = W_{A \to B}\overrightarrow{F} + W_{B \to C}\overrightarrow{F}$$

\subsubsection{Q3.a}
$$W_{x_1 \to x_2}\overrightarrow{F} = \int_{x_1 \to x_2} -\frac{GmM}{x^2} dx = [\frac{GmM}{x}]_{x_1}^{x_2}$$

$$W_{A \to C}\overrightarrow{F} = \frac{GmM}{C} - \frac{GmM}{A}$$
$$W_{A \to B}\overrightarrow{F} + W_{B \to C}\overrightarrow{F} = \frac{GmM}{B} - \frac{GmM}{A} + \frac{GmM}{C} - \frac{GmM}{B} = \frac{GmM}{C} - \frac{GmM}{A} = W_{A \to C}\overrightarrow{F}$$

Conservative.

\subsubsection{Q3.c}
$$W_{x_1 \to x_2}\overrightarrow{F} = \int_{x_1 \to x_2} -kx\, dx = [\frac{1}{2}kx^2]_{x_1}^{x_2}$$

$$W_{A \to C}\overrightarrow{F} = \frac{1}{2}kC^2 - \frac{1}{2}kA^2$$
$$W_{A \to B}\overrightarrow{F} + W_{B \to C}\overrightarrow{F} = \frac{1}{2}kB^2 - \frac{1}{2}kA^2 + \frac{1}{2}kC^2 - \frac{1}{2}kB^2 = \frac{1}{2}kC^2 - \frac{1}{2}kA^2 = W_{A \to C}\overrightarrow{F}$$

Conservative.

\subsubsection{Q3.d}
$$W_{x_1 \to x_2}\overrightarrow{F} = \int_{x_1 \to x_2} -\gamma \dot{x}\, dx$$
Cette int\'egrale d'epend de la vitesse au point $x$. La vitesse est ind\'ependente de la position de $x$ donc la force est non conservative.

\subsubsection{Q3.e}
$$W_{x_1 \to x_2}\overrightarrow{F} = \int_{x_1 \to x_2} -\Lambda x\, dx = [\frac{1}{2}\Lambda x^2]_{x_1}^{x_2}$$

$$W_{A \to C}\overrightarrow{F} = \frac{1}{2}\Lambda C^2 - \frac{1}{2}\Lambda A^2$$
$$W_{A \to B}\overrightarrow{F} + W_{B \to C}\overrightarrow{F} = \frac{1}{2}\Lambda B^2 - \frac{1}{2}\Lambda A^2 + \frac{1}{2}\Lambda C^2 - \frac{1}{2}\Lambda B^2 = \frac{1}{2}\Lambda C^2 - \frac{1}{2}\Lambda A^2 = W_{A \to C}\overrightarrow{F}$$

Conservative.

\subsection*{Exo 4.4.2}
\subsubsection{Q1}
L'\'energie potentielle d'une masse $m$ est \`a 1000m d'altitude est:
$E_p = W_{0 \to 1000} mg\, dx = [mgx]_0^{1000} = 1.g.1000 - 1.g.0 = 9810$

\subsubsection{Q2}
L'\'energie cin\'etique au moment de l'impact est $E_c = \frac{1}{2}mv^2$. Comme il n'y a pas de frottement et que les forces sont conservatives alors l'\'energie est conserv\'ee. Donc 
$$E_p = E_c$$
$$\frac{1}{2}mv^2 = 9810$$
$$v = \sqrt{2*9810} = 140 m/s$$

\subsubsection{Q3}
Durant la chute l'\'energie potentielle est de :
$$E_p(x) = W_{x \to 1000} mg\, dx = [mgx]_x^{1000} = 1.g.1000 - 1gx = 9810 - gx$$

L'\'energie cin\'etique est:
$$E_c(x) = \frac{1}{2}mv(x)^2 = \frac{1}{2}v(x)^2$$

L'\'energie est conserv'ee donc
$$E_c(x) = E_p(x)$$
$$v(x) = \sqrt{2(9810 - gx)}$$


\subsection*{Exo 4.4.3}
\subsubsection{Q1}
La force du poids est une force conservative donc elle ne d\'epend pas du chemin parcourue mais uniquement du point de d\'epart et d'arriv\'ee. 

\subsubsection{Q2}
La puissance repr\'esente l'\'energie transf\'er\'ee uniformement (ie. le travail) pendant une unit\'e de temps, $P =\frac{W}{\Delta t}$. La vitesse est identique mais la distance parcourue est diff\'erente  pour monter une force verticalement ou sur un plan inclin\'e (elle est plus courte verticalement). Donc le temps est 'egalement plus court. $P_{vert} = \frac{W}{\Delta_{t1}}$ et $P_{incline} = \frac{W}{\Delta_{t2}}$ avec $t1 < t2$. Il faut donc une plus grande puissance pour monter une charge verticalement.

\subsubsection{Q3}
Le travail est \'egale \`a $W_{A \to B}(\overrightarrow{P}) = -mg(z_b - z_a) = -mgh$. Il est n\'egatif.\\
Si il existait des frottements alors le travail du poids serait identique mais le travail global n\'ecessaire pour monter la charge serait plus grand. 


\subsection*{Exo 4.4.4}
\subsubsection{Q1}
Le travail est \'egale \`a $W_{A \to B}(\overrightarrow{P}) = -mg(z_b - z_a) = -mgh$.

\subsubsection{Q2}
La puissance repr\'esente l'\'energie transf\'er\'ee uniformement (ie. le travail) pendant une unit\'e de temps. Comme l'acc\'el\'eration est constante, la vitesse croit uniform\'ement. Pour une m\^eme unit\'e de temps, la distance parcourue est plus grande (ie. la vitesse augmente), donc le travail n\'ecessaire plus grand \'egalement. La puissance n'est dons pas constante mais augmente au fil du temps.

\subsubsection{Q3}
La variation d'\'energie cin\'etique est $\frac{1}{2}m(v_i + a.t)^2 - \frac{1}{2}mv_i^2 = \frac{1}{2}ma^2t^2 + mv_iat$\\
La variation d'\'energie potentielle est $-mg(z_b - z_a) = -mg((z_a + \frac{1}{2}at^2 - z_a) = -\frac{1}{2}mgat^2$\\

\subsubsection{Q4}
Lorsque on lache la masse \`a $t_l$, on a $a_l=0$, $v_l = a.t_l$ (vitesse atteinte depuis le d\'ebut de l'acc\'el\'eration) et $x_l = \frac{1}{2}a{t_l^2}$ ((altitude atteinte depuis le d\'ebut de l'acc\'el\'eration). Il n'y a pas de frottement, donc $E_{m_{t_l}} = \frac{1}{2}ma^2t_l^2 - \frac{1}{2}mga{t_l^2} = \frac{1}{2}mat_l^2(a-g)$. \\
Et $E_m(t) = E_c(t) + E_p(t) = \frac{1}{2}mv^2(t) - mg(x_l + x(t)- x_l) = \frac{1}{2}mv^2(t) - mgx(t)$. \\
La masse monte jusqu'\`a ce que sa vitesse soit nulle (ie v(t) = 0) et l'\'energie m'ecanique est pr\'eserv\'ee.\\
$$E_{m_{t_l}} = \frac{1}{2}mat_l^2(a-g) = E_m(t_f) = -mgx(t_f) $$
$$x(t_f) = \frac{at_l^2(g-a)}{2g}$$


\subsection*{Exo 4.4.5}
\`A $t_0$, lorsque la masse est lach\'ee, son son \'energie potentielle $E_p = -mgh$ et son \'energie cin\'etique est $E_c = 0$, donc $E_{m_0} = -mgh$.
\`A $t_1$, lLorsque la masse arrive sur le ressort son \'energie potentielle $E_p = -mgl_0$ et son \'energie cin\'etique est $E_c = \frac{1}{2}mv_1^2$. $E_{m_1} = \frac{1}{2}mv_1^2-mgl_0$. \\
\`A $t$, lorsque la masse est sur le ressort, il y a 2 forces qui s'exerce sur la masse; le poids et la force de rappel du ressort. $F_p(t)$ = mg$ et $F_r(t) = k(l_0-x(t))$. Donc $E_p(t) = -mgx(t) + kl_0x -\frac{1}{2}kx^2(t)$.


\subsection*{Exo 4.4.7}
\subsubsection{Q1}
Mouvement rectiligne et ??.

\subsubsection{Q2}
Le positron est soumis \`a une seule force $\overrightarrow{F}$, donc $\overrightarrow{F} = m\overrightarrow{a}$. L'\'equation diff\'erentielle est $m\ddot{x} - \frac{ke^2}{x^2} = 0$. On ne sait pas r\'esoudre cette \'equation diff\'erentielle.

\subsubsection{Q3}
$$W_{A \to B}\overrightarrow{F} = \int_{A \to B} \frac{ke^2}{x^2} = [-\frac{ke^2}{x}]_{x_a}^{x_b} = -\frac{ke^2}{x_b} + \frac{ke^2}{x_a}$$

Quand $B$ tend vers l'infini, le travail devient constant et \'egal \`a $\frac{ke^2}{x_a}$.

\subsubsection{Q4}
L'\'energie potentielle $E_p(x) = W_{\infty \to x}\overrightarrow{F} = -\frac{ke^2}{x} + \frac{ke^2}{\infty} = -\frac{ke^2}{x}$.\\

\subsubsection{Q4}
On a $E_m = E_p + E_c$, les forces \'etant conservatives, l'\'energie m\'ecanique est conserv\'ee. Quand $x \to \infty$, on a $E_m = E_c = \frac{1}{2}mv_0^2$ car $\lim_{x \to \infty}U(x) = 0$.\\
Au point $x$, on a $E_m(x) = E_c(x) + E_p(x) = \frac{1}{2}mv^2(x) - \frac{ke^2}{x} = \frac{1}{2}mv_0^2$. 
 

\end{document}

