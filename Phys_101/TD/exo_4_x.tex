\documentclass[]{book}

%These tell TeX which packages to use.
\usepackage{array,epsfig}
\usepackage{amsmath}
\usepackage{amsfonts}
\usepackage{amssymb}
\usepackage{amsxtra}
\usepackage{amsthm}
\usepackage{mathrsfs}
\usepackage{color}

%Here I define some theorem styles and shortcut commands for symbols I use often
\theoremstyle{definition}
\newtheorem{defn}{Definition}
\newtheorem{thm}{Theorem}
\newtheorem{cor}{Corollary}
\newtheorem*{rmk}{Remark}
\newtheorem{lem}{Lemma}
\newtheorem*{joke}{Joke}
\newtheorem{ex}{Example}
\newtheorem*{soln}{Solution}
\newtheorem{prop}{Proposition}

\newcommand{\lra}{\longrightarrow}
\newcommand{\ra}{\rightarrow}
\newcommand{\surj}{\twoheadrightarrow}
\newcommand{\graph}{\mathrm{graph}}
\newcommand{\bb}[1]{\mathbb{#1}}
\newcommand{\Z}{\bb{Z}}
\newcommand{\Q}{\bb{Q}}
\newcommand{\R}{\bb{R}}
\newcommand{\C}{\bb{C}}
\newcommand{\N}{\bb{N}}
\newcommand{\M}{\mathbf{M}}
\newcommand{\m}{\mathbf{m}}
\newcommand{\MM}{\mathscr{M}}
\newcommand{\HH}{\mathscr{H}}
\newcommand{\Om}{\Omega}
\newcommand{\Ho}{\in\HH(\Om)}
\newcommand{\bd}{\partial}
\newcommand{\del}{\partial}
\newcommand{\bardel}{\overline\partial}
\newcommand{\textdf}[1]{\textbf{\textsf{#1}}\index{#1}}
\newcommand{\img}{\mathrm{img}}
\newcommand{\ip}[2]{\left\langle{#1},{#2}\right\rangle}
\newcommand{\inter}[1]{\mathrm{int}{#1}}
\newcommand{\exter}[1]{\mathrm{ext}{#1}}
\newcommand{\cl}[1]{\mathrm{cl}{#1}}
\newcommand{\ds}{\displaystyle}
\newcommand{\vol}{\mathrm{vol}}
\newcommand{\cnt}{\mathrm{ct}}
\newcommand{\osc}{\mathrm{osc}}
\newcommand{\LL}{\mathbf{L}}
\newcommand{\UU}{\mathbf{U}}
\newcommand{\support}{\mathrm{support}}
\newcommand{\AND}{\;\wedge\;}
\newcommand{\OR}{\;\vee\;}
\newcommand{\Oset}{\varnothing}
\newcommand{\st}{\ni}
\newcommand{\wh}{\widehat}

%Pagination stuff.
\setlength{\topmargin}{-.3 in}
\setlength{\oddsidemargin}{0in}
\setlength{\evensidemargin}{0in}
\setlength{\textheight}{9.in}
\setlength{\textwidth}{6.5in}
\pagestyle{empty}



\begin{document}


\subsection*{Rappel de cours}
\subsubsection*{Travail}

\begin{itemize}
\item La composante de la force d'un point $M$, $\overrightarrow{F}(M)$ sur l'axe $\mathcal{O}_x$ est donn\'ee par le produit scalaire $f(x) = \overrightarrow{F}(M).\overrightarrow{i}$. 
\item Le travail d'une force $\overrightarrow{F}$ sur un segment $\overrightarrow{AB}$ est donn\'e par :
$$W_{A \to B}(\overrightarrow{F}) = \overrightarrow{F}.\overrightarrow{AB} = \int_{A \to B} \overrightarrow{F}.\overrightarrow{i}dx = \int_{x_a}^{x_b} f(x)dx$$ 
\item On dira qu'une force est conservative si elle ne d\'epend que de la position et si son travail d'un point $A$ au point $B$ ne d\'epend pas du chemin suivi, ceci quels que soient les point $A$ et $B$.
$$\forall A,B,C,\, W_{A \to B} \overrightarrow{F} = W_{A \to C} \overrightarrow{F} + W_{C \to B} \overrightarrow{F}$$
\item Dans le cas o\`u le chemin est rectiligne, si une force est conservatrice alors l'\emph{\'energie potentielle} associ\'ee \`a la force $\overrightarrow{F}$ est not\'ee $E_p(x)$ est d\'efinie par :
$$W_{A \to B}(\overrightarrow{F}) = \int_{A \to B} \overrightarrow{F}.\overrightarrow{i}dx = E_p(x_b) - E_p(x_a)$$. 
\item Le travail du poids $\overrightarrow{P} = m\overrightarrow{g}$ sur le segment $\overrightarrow{AB}$ est $W_{A \to B}(\overrightarrow{P}) = -mg(z_b - z_a) = -mgh$.
\item Le travail de la force de rappel \'elastique d'un ressort de raideur $k$, $\overrightarrow{F} = -k.x\overrightarrow{i}$ est $W_{A \to B}(\overrightarrow{F}) = -\frac{1}{2}k(x_a^2 - x_b^2)$.
\end{itemize}


\subsubsection*{\'Energie}
\begin{itemize}
\item L'\'energie m\'ecanique d'un syst\`eme $E_m = E_c + E_p$ avec $E_c$ l'\'energie cin\'etique qui d\'epend de la masse et de la norme de la vitesse du syst\`eme physique \'etudi\'e et de l'\'energie potentielle $E_p$ qui correspond aux forces exerc\'ees sur le syst\`eme.
\item L'\'energie cin\'etique du syst\`eme $E_c = \frac{1}{2}mv^2$
\item L'\'energie potentielle qui correspond \`a l'ensemble des forces conservatives qui s'exercent sur le syst\'eme, $E_p(B) - E_p(A) = W_{A \to B}(\overrightarrow{F}_{conservatives})$
\item $E_m(B) - E_m(A) = W_{A \to B}(\overrightarrow{F}_{non\, conservatives})$
\end{itemize}

\subsubsection*{Puissance}
\begin{itemize}
\item La puissance $P$ repr\'esente l'\'energie transf\'er\'ee uniformement (ie. le travail) pendant une unit\'e de temps, $P =\frac{W}{\Delta t}$.
\item $1\,W = 1\,J.s^{-1} = 1\, N.m.s^{-1} = 1\, kg.m^{2}.s^{-3}$
\end{itemize}


\subsection*{Exo 4.1.1}
L'\'electron est soumis a la force gravitationnelle. Donc $E_m = E_c + E_p$ avec $E_c = \frac{1}{2}mv^2$. On n\'eglige l'\'energie de la force gravitationnelle devant celle de l'\'energie cim\'etique. On a $E_m = \frac{1}{2}mv^2 = 18\,keV$, donc $v = \sqrt{\frac{2*18\,keV}{m}}$ et $m=9.10\,10^{-31}kg$, $18\,keV = 2.88\,10^{-12}$. 


\subsection*{Exo 4.1.2}
Il faut monter une masse de $m=500+5*70 = 850\,kg$ \`a une vitesse de $v=25/60 = 0.41\,m/s$. La puissance n\'ecessaire est $P=m.g.v = 850*9.81*0.41 = 3418\,Watt$.\\
 
En l'abscence de frottements la puissance n\'ecessaire pour lever la cabine d'ascenseur est \'egale \`a la puissance fournie par le moteur. Le puissance du poids est r\'esistante, celle du moteur est motrice.\\
v
On a $P =\frac{W}{\Delta_t}$, donc $W = P*\Delta_t = 3418*60 = 205kJ$.

\subsection*{Exo 4.1.3}
\subsubsection{Q1}
$TWh$ repr\'esente des $10^{12}Wh$.

\subsubsection{Q2}
On a $1W = 1J/s$. On produit $429.10^{12}Wh$ pour une ann\'ee, donc on a produit $\frac{429.10^{12}}{24*365.25} = 48.10^{9} W$. Ce qui fait $48.10^{9}*(24*365.25*3600) = 1.510^{18}J$.

\subsubsection{Q3}
La puissance \'electrique moyenne d'un r\'eacteur est de $\frac{48.10^{9}}{58} = 0.82 MW$.  

\subsection*{Exo 4.2}
\subsubsection{Q1}
Le travail accompli par la force est $\overrightarrow{F}.\overrightarrow{AB}$

\subsubsection{Q2}
On a $E_m = E_c + E_p$. L'\'energie du syst\`eme $E_m$ est conserv\'ee . Donc $E_{c0} + E_p = E_{cf} + E_{p_f}$ avec $E_{p_f} = 0$ car aucune force ne s'exerce sur la masse. Donc l'accroissement de l'\'energie cin\'etique est $E_{cf} - E_{c0} = E_p$.

\subsubsection{Q3}
La puissance moyenne d\'evelopp\'e est $P=\frac{\Delta_W}{\Delta_t} = \frac{E_p}{T} = \frac{\overrightarrow{F}.\overrightarrow{AB}}{T}$. 

\subsection*{Exo 4.2.3}
\subsubsection{Q1.a}
3 forces s'exercent sur la masse $m$; la force de la pesanteur (verticale) $\overrightarrow{P}$, la force de r\'eaction $\overrightarrow{R}$ (perpendiculaire au plan inclin\'e) et la force de frottement statique $\overrightarrow{f}$ (parall\`ele au plan inclin\'e). Comme la masse ne bouge pas, la somme des 3 forces est nulle.\\
On d\'efinit un rep\`ere orthonorm\'e centr\'e sur la masse $m$ et parall\`ele au plan inclin\'e. Dans ce rep\`ere, les 3 forces ont les coordonn\'ees:

$$
\left\{
\begin{array}{l l l}
 P_x = -m.g.sin(\beta) & R_x = 0 & f_x = \|\overrightarrow{f}\| \\
 P_y = -m.g.cos(\beta) & R_y = \|\overrightarrow{R}\| & f_y = 0\\
\end{array}
\right. 
$$

Donc, on a $ \|\overrightarrow{R}\| = m.g.cos(\beta)$ et $\|\overrightarrow{f}\| = m.g.cos(\beta)$.

\subsubsection{Q1.b}
Le coefficient de frottement statique $k_s$ peut \^etre d\'etermin\'e pour l'angle minimum du plan inclin\'e $\beta_{min}$ faisant boug'e la masse $m$.\\
$$k_s = \frac{\overrightarrow{f_{max}}}{\overrightarrow{R}}$$
$$k_s = \frac{-m.g.sin(\beta_{min})}{-m.g.cos(\beta_{min})}$$
$$k_s = \tan(\beta_{min})$$

\subsubsection{Q1.c}
Ne marche pas pour $\beta = 60^\circ$. Mais $\tan(6^\circ) = 0.1$.

\subsubsection{Q2.a}
Le travail $W_{C \to D} = W_{C \to D}(\overrightarrow{f\prime}) + W_{C \to D}(\overrightarrow{R}) + W_{C \to D}(\overrightarrow{P})$. On a 

$$
\left\{
\begin{array}{l l}
W_{C \to D}(\overrightarrow{R}) = 0 & La\, force\, \overrightarrow{R}\, est\, toujours\, perpendiculaire\, au\, deplacement.\\
W_{C \to D}(\overrightarrow{f^\prime}) = \overrightarrow{f^\prime}.l = -k_d.m.g.\cos(\beta).l & \\
W_{C \to D}(\overrightarrow{P}) = -m.g.h = m.g.l.\sin(\beta) & \\
\end{array}
\right. 
$$

Donc, $W_{C \to D} = m.g.l.(sin(\beta) - k_d.cos(\beta))$.


\subsubsection{Q2.b}
Le travail $W_{D \to E} = W_{D \to E}(\overrightarrow{f^\prime}) + W_{D \to E}(\overrightarrow{R}) + W_{D \to E}(\overrightarrow{P})$. On a 

$$
\left\{
\begin{array}{l l}
W_{D \to E}(\overrightarrow{R}) = 0 & La\, force\, \overrightarrow{R}\, est\, toujours\, perpendiculaire\, au\, deplacement.\\
W_{D \to E}(\overrightarrow{f^\prime}) = \overrightarrow{f^\prime}.l = -k_d.m.g.l^\prime & \\
W_{D \to E}(\overrightarrow{P}) = -m.g.h = 0 & \\
\end{array}
\right. 
$$

Donc, $W_{C \to D} = -k_d.m.g.l^\prime$.

\subsubsection{Q2.c}
Les forces \'etant toutes conservatives on a $E_m = W_{C \to D} + W_{D \to E} = 0$
$$ m.g.l.(sin(\beta) - k_d.cos(\beta)) -k_d.m.g.l^\prime = 0$$
$$ l.(sin(\beta) - k_d.cos(\beta)) -k_d.l^\prime = 0$$
$$ (sin(\beta) - k_d.cos(\beta)) -k_d\frac{l^\prime}{l} = 0$$
$$ sin(\beta) - k_d.cos(\beta) -k_d.r = 0$$
$$ sin(\beta) - k_d(r+cos(\beta)) = 0$$
$$ k_d = \frac{sin(\beta)}{r+cos(\beta)}$$

\subsubsection{Q2.d}
On a 
$$l^\prime = \frac{l.(\sin(\beta) - k_d.\cos(\beta))}{k_d} $$

Donc $l^\prime = 5\frac{\sin(6^\circ) - 0.05\cos(6^\circ)}{0.05} = 5.48m$.

\end{document}

