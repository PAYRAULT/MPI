\documentclass[]{book}

%These tell TeX which packages to use.
\usepackage{array,epsfig}
\usepackage{amsmath}
\usepackage{amsfonts}
\usepackage{amssymb}
\usepackage{amsxtra}
\usepackage{amsthm}
\usepackage{mathrsfs}
\usepackage{color}

%Here I define some theorem styles and shortcut commands for symbols I use often
\theoremstyle{definition}
\newtheorem{defn}{Definition}
\newtheorem{thm}{Theorem}
\newtheorem{cor}{Corollary}
\newtheorem*{rmk}{Remark}
\newtheorem{lem}{Lemma}
\newtheorem*{joke}{Joke}
\newtheorem{ex}{Example}
\newtheorem*{soln}{Solution}
\newtheorem{prop}{Proposition}

\newcommand{\lra}{\longrightarrow}
\newcommand{\ra}{\rightarrow}
\newcommand{\surj}{\twoheadrightarrow}
\newcommand{\graph}{\mathrm{graph}}
\newcommand{\bb}[1]{\mathbb{#1}}
\newcommand{\Z}{\bb{Z}}
\newcommand{\Q}{\bb{Q}}
\newcommand{\R}{\bb{R}}
\newcommand{\C}{\bb{C}}
\newcommand{\N}{\bb{N}}
\newcommand{\M}{\mathbf{M}}
\newcommand{\m}{\mathbf{m}}
\newcommand{\MM}{\mathscr{M}}
\newcommand{\HH}{\mathscr{H}}
\newcommand{\Om}{\Omega}
\newcommand{\Ho}{\in\HH(\Om)}
\newcommand{\bd}{\partial}
\newcommand{\del}{\partial}
\newcommand{\bardel}{\overline\partial}
\newcommand{\textdf}[1]{\textbf{\textsf{#1}}\index{#1}}
\newcommand{\img}{\mathrm{img}}
\newcommand{\ip}[2]{\left\langle{#1},{#2}\right\rangle}
\newcommand{\inter}[1]{\mathrm{int}{#1}}
\newcommand{\exter}[1]{\mathrm{ext}{#1}}
\newcommand{\cl}[1]{\mathrm{cl}{#1}}
\newcommand{\ds}{\displaystyle}
\newcommand{\vol}{\mathrm{vol}}
\newcommand{\cnt}{\mathrm{ct}}
\newcommand{\osc}{\mathrm{osc}}
\newcommand{\LL}{\mathbf{L}}
\newcommand{\UU}{\mathbf{U}}
\newcommand{\support}{\mathrm{support}}
\newcommand{\AND}{\;\wedge\;}
\newcommand{\OR}{\;\vee\;}
\newcommand{\Oset}{\varnothing}
\newcommand{\st}{\ni}
\newcommand{\wh}{\widehat}

%Pagination stuff.
\setlength{\topmargin}{-.3 in}
\setlength{\oddsidemargin}{0in}
\setlength{\evensidemargin}{0in}
\setlength{\textheight}{9.in}
\setlength{\textwidth}{6.5in}
\pagestyle{empty}



\begin{document}


\subsection*{Exo 3.2.1}
	
\subsubsection*{Q1}
\emph{Identification du syst\`eme} : Le syst\`eme \'etudi\'e est un projectile ponctuel M de masse $m$ lanc\'e avec une vitesse initiale $v_0$ avec un angle de $\alpha_0$, sans frottement. 


\emph{Bilan des forces} : Comme il n'y a pas de frottement, le projectile est uniquement soumis \`a la pesanteur.
 

\emph{Composantes dans un rep\`ere cart\'esien} : \`a $t=0$, le projectile se trouve \`a l'origine du rep\`ere. La force de la pesanteur est verticale, donc la du projectile est :
$$
\left\{
\begin{array}{l}
 v_x(t) = v_0*cos\, \alpha_0 \\
 v_y(t) = 0 \\
 v_z(t) = v_0*sin\, \alpha_0 - g*t \\
\end{array}
\right. 
$$

L'\'equation horaire est
$$
\left\{
\begin{array}{l}
 x(t) = v_0*cos\, \alpha_0\, *t + C_z\\
 y(t) = 0*t + C_y \\
 z(t) = v_0*sin\, \alpha_0\, *t - \frac{1}{2}g*t^2 + C_x\\
\end{array}
\right. 
$$

\`A $t=0$, le projectile se trouve aux coordonn\'ees $(0,0,0)$. Donc, $C_x=C_y=C_z=0$.

$$
\left\{
\begin{array}{l}
 x(t) = v_0*cos\, \alpha_0\, *t\\
 y(t) = 0\\
 z(t) = v_0*sin\, \alpha_0\, *t - \frac{g*t^2}{2} \\
\end{array}
\right. 
$$

\'Equation de la trajectoire: $z = \frac{(v_0*sin\, \alpha_0)}{(v_0*cos\, \alpha_0)}*x - \frac{g*x^2}{2v_0^2cos^2\alpha_0} = tan\, \alpha_0*x - \frac{g*x^2}{2v_0^2cos^2\alpha_0}$. \\

Fin de la trajectoire quand $z=0$. 
$$tan\, \alpha_0*x - \frac{g*x^2}{2v_0^2cos^2\alpha_0} = 0$$
$$\frac{sin\, \alpha_0}{cos\, \alpha_0} - \frac{g*x}{2v_0^2cos^2\alpha_0} = 0$$
$$sin\, \alpha_0 = \frac{g*x}{2*v_0^2*cos\, \alpha_0}$$
$$g*x= 2*sin\, \alpha_0*v_0^2*cos\, \alpha_0$$
$$x= \frac{2*sin\, \alpha_0*v_0^2*cos\, \alpha_0}{g}$$
$$x= \frac{sin\, 2\alpha_0*v_0^2}{g}$$

La distance maximale est lorsque $sin\, 2\alpha_0 = 1$ soit $\alpha_0 = 45^{o}$.

\end{document}

