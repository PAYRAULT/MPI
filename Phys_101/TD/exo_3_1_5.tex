\documentclass[]{book}

%These tell TeX which packages to use.
\usepackage{array,epsfig}
\usepackage{amsmath}
\usepackage{amsfonts}
\usepackage{amssymb}
\usepackage{amsxtra}
\usepackage{amsthm}
\usepackage{mathrsfs}
\usepackage{color}

%Here I define some theorem styles and shortcut commands for symbols I use often
\theoremstyle{definition}
\newtheorem{defn}{Definition}
\newtheorem{thm}{Theorem}
\newtheorem{cor}{Corollary}
\newtheorem*{rmk}{Remark}
\newtheorem{lem}{Lemma}
\newtheorem*{joke}{Joke}
\newtheorem{ex}{Example}
\newtheorem*{soln}{Solution}
\newtheorem{prop}{Proposition}

\newcommand{\lra}{\longrightarrow}
\newcommand{\ra}{\rightarrow}
\newcommand{\surj}{\twoheadrightarrow}
\newcommand{\graph}{\mathrm{graph}}
\newcommand{\bb}[1]{\mathbb{#1}}
\newcommand{\Z}{\bb{Z}}
\newcommand{\Q}{\bb{Q}}
\newcommand{\R}{\bb{R}}
\newcommand{\C}{\bb{C}}
\newcommand{\N}{\bb{N}}
\newcommand{\M}{\mathbf{M}}
\newcommand{\m}{\mathbf{m}}
\newcommand{\MM}{\mathscr{M}}
\newcommand{\HH}{\mathscr{H}}
\newcommand{\Om}{\Omega}
\newcommand{\Ho}{\in\HH(\Om)}
\newcommand{\bd}{\partial}
\newcommand{\del}{\partial}
\newcommand{\bardel}{\overline\partial}
\newcommand{\textdf}[1]{\textbf{\textsf{#1}}\index{#1}}
\newcommand{\img}{\mathrm{img}}
\newcommand{\ip}[2]{\left\langle{#1},{#2}\right\rangle}
\newcommand{\inter}[1]{\mathrm{int}{#1}}
\newcommand{\exter}[1]{\mathrm{ext}{#1}}
\newcommand{\cl}[1]{\mathrm{cl}{#1}}
\newcommand{\ds}{\displaystyle}
\newcommand{\vol}{\mathrm{vol}}
\newcommand{\cnt}{\mathrm{ct}}
\newcommand{\osc}{\mathrm{osc}}
\newcommand{\LL}{\mathbf{L}}
\newcommand{\UU}{\mathbf{U}}
\newcommand{\support}{\mathrm{support}}
\newcommand{\AND}{\;\wedge\;}
\newcommand{\OR}{\;\vee\;}
\newcommand{\Oset}{\varnothing}
\newcommand{\st}{\ni}
\newcommand{\wh}{\widehat}

%Pagination stuff.
\setlength{\topmargin}{-.3 in}
\setlength{\oddsidemargin}{0in}
\setlength{\evensidemargin}{0in}
\setlength{\textheight}{9.in}
\setlength{\textwidth}{6.5in}
\pagestyle{empty}



\begin{document}


\subsection*{Exo 3.1.5}
	
\subsubsection*{Q1}
\emph{Identification du syst\`eme} : Le syst\`eme \'etudi\'e est l'anneau de masse
$m$ qui est accroch\'e aux 2 ressorts.


\emph{Bilan des forces} : Au repos, l'anneau est soumis \`a deux forces (la force de gravit\'e est annul\'ee par l'absence de frottement): 
\begin{itemize}
\item la force de rappel du ressort du premier ressort qui est proportionnelle \`a l'allongement
du ressort $F_{r_1} = -k_1.(l^{e}_{1} - l^{o}_{1})\overrightarrow{i}$, avec $l^{e}_{1}$ la longueur du ressort \`a l'\'equilibre.
\item la force de rappel du ressort du second ressort qui est proportionnelle \`a l'allongement
du ressort $F_{r_2} = k_2.(l^{e}_{2} - l^{o}_{2})\overrightarrow{i}$, avec $l^{e}_{2}$ la longueur du ressort \`a l'\'equilibre.
\end{itemize}


\emph{Composantes dans un rep\`ere cart\'esien} : Le mouvement est rectiligne.
Il a lieu le long de l'axe $O_x$. L'origine est pris \`a la position d'\'equilibre. Il existe une relation entre la position d'\'equilibre et la longueur totale; $l_{1} + l_{2} = l^{e}_{1} + l^{e}_{2}$


\emph{PFD} : La r\'esulante des forces est \'egale \`a z\'ero car le syst\`eme est \`a l'\'equilibre. Donc 
$-k_1.(l^{e}_{1}- l^{o}_{1})\overrightarrow{i} + k_2.(l^{e}_{2}- l^{o}_{2})\overrightarrow{i}=\overrightarrow{0}$. 


\emph{Solution}
(a) \\
On a le syst\`eme suivant \`a r\'esoudre:
$$
\left\{
\begin{array}{l}
 k_1.(l^{e}_{1}- l^{o}_{1})\overrightarrow{i} = k_2.(l^{e}_{2}- l^{o}_{2})\overrightarrow{i} \\
 l = l^{e}_{1} + l^{e}_{2}\\
\end{array}
\right. 
$$

$$
\left\{
\begin{array}{l}
 k_1.l^{e}_{1} - k_1.l^{o}_{1} = k_2.l^{e}_{2} - k_2.l^{o}_{2} \\
 l^{e}_{1}  = l - l^{e}_{2}\\
\end{array}
\right. 
$$

$$
\left\{
\begin{array}{l}
 k_1.(l - l^{e}_{2}) - k_1.l^{o}_{1} = k_2.l^{e}_{2} - k_2.l^{o}_{2} \\
 l^{e}_{1}  = l - l^{e}_{2}\\
\end{array}
\right. 
$$

$$
\left\{
\begin{array}{l}
 k_1.l - k_1.l^{e}_{2} - k_1.l^{o}_{1} = k_2.l^{e}_{2} - k_2.l^{o}_{2} \\
 l^{e}_{1}  = l - l^{e}_{2}\\
\end{array}
\right. 
$$

$$
\left\{
\begin{array}{l}
 k_2.l^{e}_{2} + k_1.l^{e}_{2} = k_1.l  - k_1.l^{o}_{1} + k_2.l^{o}_{2}\\
 l^{e}_{1}  = l - l^{e}_{2}\\
\end{array}
\right. 
$$

$$
\left\{
\begin{array}{l}
 l^{e}_{2} (k_1+k_2) = k_1.l  - k_1.l^{o}_{1} + k_2.l^{o}_{2}\\
 l^{e}_{1}  = l - l^{e}_{2}\\
\end{array}
\right. 
$$

$$
\left\{
\begin{array}{l}
 l^{e}_{2} =  \frac{k_1.l - k_1.l^{o}_{1} + k_2.l^{o}_{2}}{k_2 + k_1} \\
 l^{e}_{1} = l - l^{e}_{2}\\
\end{array}
\right. 
$$

$$
\left\{
\begin{array}{l}
 l^{e}_{1} = \frac{k_2.l - k_2.l^{o}_{2} + k_1.l^{o}_{1}}{k_2 + k_1} \\
 l^{e}_{2} = \frac{k_1.l - k_1.l^{o}_{1} + k_2.l^{o}_{2}}{k_2 + k_1} \\
\end{array}
\right. 
$$


\subsubsection*{Q2}
\emph{Identification du syst\`eme} : Le syst\`eme \'etudi\'e est l'anneau de masse
$m$ qui est accroch\'e aux 2 ressorts.


\emph{Bilan des forces} : L'anneau est soumis \`a deux forces : 
\begin{itemize}
\item la force de rappel du ressort du premier ressort qui est proportionnelle \`a l'allongement
du ressort $F_{r_1}(t) = -k_1.x(t)\overrightarrow{i}$.
\item la force de rappel du ressort du second ressort qui est proportionnelle \`a l'allongement
du ressort $F_{r_2}(t) = -k_2.x(t)\overrightarrow{i}$.
\end{itemize}


\emph{Composantes dans un rep\`ere cart\'esien} : Le mouvement est rectiligne.
Il a lieu le long de l'axe $O_x$. L'origine est pris \`a la position d'\'equilibre. $x(t) = (\overrightarrow{OM}(t)) - \overrightarrow{OM_{e}})$.


\emph{PFD} : La r\'esulante des forces est \'egale \`a $m.a(t) = m.\frac{d^2x(t)}{d^2t}$. 


\emph{Solution}
(a) \\
$$m.\frac{d^2x(t)}{d^2t} = -k_1.x(t)-k_2.x(t)$$
$$\frac{d^2x(t)}{d^2t} = -\frac{k_1+k_2}{m}.x(t)$$



(b)La solution de cette \'equation diff\'erentielle est $x(t) = C_1.cos(at) + C_2.sin(at)$ avec $a = \sqrt{\frac{k_1+k_2}{m}}$.


(c) Les conditions initiales sont \`a $t=0, x(0) = (M_0-M_e), v(0) = 0$. Donc,
$$x(0) = C_1.cos(a.0) + C_2.sin(a.0) = M_0-M_e$$
$$x(0) = C_1 = M_0-M_e$$
et
$$v(0) = x'(0) = -C_1.sin(a.0) +C_2.cos(a.0) = 0$$
$$v(0) = C_2 = 0$$

Donc la solution de l'\'equation est :
$$x(t) = (M_0-M_e).sin(\sqrt{\frac{k_1+k_2}{m}}.t)$$

L'amplitude est $2|M_0-M_e|$ car le sinus est compris entre $-1$ et $1$. \\
La p\'eriode est $\frac{2\pi}{\sqrt{\frac{k_1+k_2}{m}}}$.

\end{document}

