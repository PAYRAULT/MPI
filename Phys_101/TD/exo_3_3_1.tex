\documentclass[]{book}

%These tell TeX which packages to use.
\usepackage{array,epsfig}
\usepackage{amsmath}
\usepackage{amsfonts}
\usepackage{amssymb}
\usepackage{amsxtra}
\usepackage{amsthm}
\usepackage{mathrsfs}
\usepackage{color}

%Here I define some theorem styles and shortcut commands for symbols I use often
\theoremstyle{definition}
\newtheorem{defn}{Definition}
\newtheorem{thm}{Theorem}
\newtheorem{cor}{Corollary}
\newtheorem*{rmk}{Remark}
\newtheorem{lem}{Lemma}
\newtheorem*{joke}{Joke}
\newtheorem{ex}{Example}
\newtheorem*{soln}{Solution}
\newtheorem{prop}{Proposition}

\newcommand{\lra}{\longrightarrow}
\newcommand{\ra}{\rightarrow}
\newcommand{\surj}{\twoheadrightarrow}
\newcommand{\graph}{\mathrm{graph}}
\newcommand{\bb}[1]{\mathbb{#1}}
\newcommand{\Z}{\bb{Z}}
\newcommand{\Q}{\bb{Q}}
\newcommand{\R}{\bb{R}}
\newcommand{\C}{\bb{C}}
\newcommand{\N}{\bb{N}}
\newcommand{\M}{\mathbf{M}}
\newcommand{\m}{\mathbf{m}}
\newcommand{\MM}{\mathscr{M}}
\newcommand{\HH}{\mathscr{H}}
\newcommand{\Om}{\Omega}
\newcommand{\Ho}{\in\HH(\Om)}
\newcommand{\bd}{\partial}
\newcommand{\del}{\partial}
\newcommand{\bardel}{\overline\partial}
\newcommand{\textdf}[1]{\textbf{\textsf{#1}}\index{#1}}
\newcommand{\img}{\mathrm{img}}
\newcommand{\ip}[2]{\left\langle{#1},{#2}\right\rangle}
\newcommand{\inter}[1]{\mathrm{int}{#1}}
\newcommand{\exter}[1]{\mathrm{ext}{#1}}
\newcommand{\cl}[1]{\mathrm{cl}{#1}}
\newcommand{\ds}{\displaystyle}
\newcommand{\vol}{\mathrm{vol}}
\newcommand{\cnt}{\mathrm{ct}}
\newcommand{\osc}{\mathrm{osc}}
\newcommand{\LL}{\mathbf{L}}
\newcommand{\UU}{\mathbf{U}}
\newcommand{\support}{\mathrm{support}}
\newcommand{\AND}{\;\wedge\;}
\newcommand{\OR}{\;\vee\;}
\newcommand{\Oset}{\varnothing}
\newcommand{\st}{\ni}
\newcommand{\wh}{\widehat}

%Pagination stuff.
\setlength{\topmargin}{-.3 in}
\setlength{\oddsidemargin}{0in}
\setlength{\evensidemargin}{0in}
\setlength{\textheight}{9.in}
\setlength{\textwidth}{6.5in}
\pagestyle{empty}



\begin{document}


\subsection*{Exo 3.3.1}
\subsubsection*{Q1}
Dans la main.\\

Je suis dans le r\'ef\'erentiel du tapis roulant et le ballon \'egalement. Le ballon est soumis uniquement \`a la force gravitationelle (les frottements de l'air sont n\'eglig\'es). Donc $m.\overrightarrow{a} = m.\overrightarrow{g}$.

$$a(t)
\left\{
\begin{array}{l}
 a_{x}(t) = 0 \\
 a_{y}(t) = -g \\
\end{array}
\right. 
$$

$$v(t)
\left\{
\begin{array}{l}
 v_{x}(t) =  C_{vx} \\
 v_{y}(t) =  C_{vy} -gt \\
\end{array}
\right. 
$$
\`A l'initialisation, $v_{x}(0) = 0\, m/s$, $v_{y}(0) = 1\, m/s$. Donc $C_{vx} = 0$ et $C_{vy} = 1$.\\

$$
\left\{
\begin{array}{l}
 x(t) =  C_{x} \\
 y(t) =  C_{y} + t -\frac{g}{2}t^2 \\
\end{array}
\right. 
$$
\`A l'initialisation, $x(0) = 0$, $y(0) = 0$. Donc $C_{x} = 0$ et $C_{y} = 0$.\\

Donc,
$$
\left\{
\begin{array}{l}
 x(t) =  0 \\
 y(t) =  t -\frac{g}{2}t^2 \\
\end{array}
\right. 
$$


\subsubsection*{Q2}
Dans le r\'ef\'erentiel de la personne immobile, le r\'ef\'erentiel du tapis bouge \`a $2m/s$ sur l'axe $O_x$. Donc, 
$$v_{r}(t)
\left\{
\begin{array}{l}
 x(t) =  2 \\
 y(t) =  0 \\
\end{array}
\right. 
$$

L'\'equation horaire du r\'ef\'erentiel du tapis est
$$
\left\{
\begin{array}{l}
 x_r(t) =  2t + C_1 \\
 y_r(t) =  0 + C_2\\
\end{array}
\right. 
$$

\`A l'initialisation mon r\'ef\'erentiel est \`a $(0,0)$ dans le r\'ef\'erentiel de la personne immobile. Donc, $C_1 = 0$ et $C_2 = 0$.
$$
\left\{
\begin{array}{l}
 x_r(t) =  2t \\
 y_r(t) =  0 \\
\end{array}
\right. 
$$

Vis-\`a-vis du la personne immobile, l'\'equation horaire du ballon ets la somme des 2 r\'ef\'erentiels:
$$
\left\{
\begin{array}{l}
 x_b(t) =  2t \\
 y_b(t) =  t -\frac{g}{2}t^2 \\
 \\
\end{array}
\right. 
$$

La trajectoire du ballon est 
$$Y(t) = \frac{x(t)}{2} - \frac{g}{2}(\frac{x(t)}{2})^2$$
$$Y(t) = \frac{x(t)}{2} - \frac{g.x^2(t)}{8}$$

La trajectoire est une parabole.


\end{document}

