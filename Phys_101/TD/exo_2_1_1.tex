\documentclass[]{book}

%These tell TeX which packages to use.
\usepackage{array,epsfig}
\usepackage{amsmath}
\usepackage{amsfonts}
\usepackage{amssymb}
\usepackage{amsxtra}
\usepackage{amsthm}
\usepackage{mathrsfs}
\usepackage{color}

%Here I define some theorem styles and shortcut commands for symbols I use often
\theoremstyle{definition}
\newtheorem{defn}{Definition}
\newtheorem{thm}{Theorem}
\newtheorem{cor}{Corollary}
\newtheorem*{rmk}{Remark}
\newtheorem{lem}{Lemma}
\newtheorem*{joke}{Joke}
\newtheorem{ex}{Example}
\newtheorem*{soln}{Solution}
\newtheorem{prop}{Proposition}

\newcommand{\lra}{\longrightarrow}
\newcommand{\ra}{\rightarrow}
\newcommand{\surj}{\twoheadrightarrow}
\newcommand{\graph}{\mathrm{graph}}
\newcommand{\bb}[1]{\mathbb{#1}}
\newcommand{\Z}{\bb{Z}}
\newcommand{\Q}{\bb{Q}}
\newcommand{\R}{\bb{R}}
\newcommand{\C}{\bb{C}}
\newcommand{\N}{\bb{N}}
\newcommand{\M}{\mathbf{M}}
\newcommand{\m}{\mathbf{m}}
\newcommand{\MM}{\mathscr{M}}
\newcommand{\HH}{\mathscr{H}}
\newcommand{\Om}{\Omega}
\newcommand{\Ho}{\in\HH(\Om)}
\newcommand{\bd}{\partial}
\newcommand{\del}{\partial}
\newcommand{\bardel}{\overline\partial}
\newcommand{\textdf}[1]{\textbf{\textsf{#1}}\index{#1}}
\newcommand{\img}{\mathrm{img}}
\newcommand{\ip}[2]{\left\langle{#1},{#2}\right\rangle}
\newcommand{\inter}[1]{\mathrm{int}{#1}}
\newcommand{\exter}[1]{\mathrm{ext}{#1}}
\newcommand{\cl}[1]{\mathrm{cl}{#1}}
\newcommand{\ds}{\displaystyle}
\newcommand{\vol}{\mathrm{vol}}
\newcommand{\cnt}{\mathrm{ct}}
\newcommand{\osc}{\mathrm{osc}}
\newcommand{\LL}{\mathbf{L}}
\newcommand{\UU}{\mathbf{U}}
\newcommand{\support}{\mathrm{support}}
\newcommand{\AND}{\;\wedge\;}
\newcommand{\OR}{\;\vee\;}
\newcommand{\Oset}{\varnothing}
\newcommand{\st}{\ni}
\newcommand{\wh}{\widehat}

%Pagination stuff.
\setlength{\topmargin}{-.3 in}
\setlength{\oddsidemargin}{0in}
\setlength{\evensidemargin}{0in}
\setlength{\textheight}{9.in}
\setlength{\textwidth}{6.5in}
\pagestyle{empty}



\begin{document}


\subsection*{Exo 2.1.1}
\subsubsection*{Q1}
$x(t)=\mu*t^2+\nu$, $x(t)$ est une distance exprim\'ee en metre $m$. Donc
la dimension de $\nu$ est \'egalement en $m$ et $\mu$ est en $m/s^2$.

\subsubsection*{Q2}
(a) \\
La d\'eriv\'ee de $f(x)$xs en $x_0$ est
$$f'(x_0) = \lim_{h \to 0} \frac{f(x_0+h)-f(x_0)}{h}$$


(b)
$$f'(2) = \lim_{h \to 0} \frac{f(2+h)-f(2)}{h}$$
$$f'(2) = \lim_{h \to 0} \frac{5(2+h)^2+3-(5*2^2+3)}{h}$$
$$f'(2) = \lim_{h \to 0} \frac{5h^2+20h+20+3-23}{h}$$
$$f'(2) = \lim_{h \to 0} \frac{5h^2+20h}{h}$$
$$f'(2) = \lim_{h \to 0} 5h+20 = 20$$

(c)
$$f'(x) = \lim_{h \to 0} \frac{f(x+h)-f(x)}{h}$$
$$f'(x) = \lim_{h \to 0} \frac{5(x+h)^2+3-(5x^2+3)}{h}$$
$$f'(x) = \lim_{h \to 0} \frac{5h^2+10xh+5x^2+3-5x^2 -3}{h}$$
$$f'(x) = \lim_{h \to 0} \frac{5h^2+10xh}{h}$$
$$f'(x) = \lim_{h \to 0} 5h+10x = 10x$$

\subsubsection*{Q3}
(a)
\begin{center}
\begin{tabular}{c|c|c} 
                         $t_0$         & $t_1$  & $v _{moy}$ \\
      \hline
     $2.0000$ & $3.0000$ & $25$ \\
     $2.0000$ & $2.1000$ & $20.5$ \\
     $2.0000$ & $2.0100$ & $20.05$ \\
     $2.0000$ & $2.0010$ & $20.005$ \\
     $2.0000$ & $2.0001$ & $20.0005$ \\
\end{tabular}
\end{center}

(b) La vitesse instantan\'ee est 
$$v(t)=lim_{\Delta_t \to 0} \frac{x(t+\Delta_t)-x(t)}{\Delta_t}$$ 
L'expression de la vitesse instantan\'ee est identique \`a la
d\'efinition de la d\'eriv\'ee d'une fonction (voir Question 2). Donc 
$$v(t_0)=x'(t_0)=10t_0$$
$$v(2)=20$$

(c)
Plus la valeur de $\Delta_t$ diminue, plus la valeur de la vitesse
moyenne tend vers la vitesse instantan\'ee.

\subsubsection*{Q3}
(a)
\'Equation d'une droite $y = ax + b$, la valeur de $a$ est $20$ et la
droite passe par le point $(2,23)$. Donc
$$23 = 20 * 2 - b$$
$$b = -17$$
L'\'equation de la droite est $y = 20x - 17$.

(b)
La tangente et la droite sont confondues. La vitesse instantan\'ee au
point $t_0$ est $20m/s$. Donc on peut conclure que le coefficient
directeur de la tangente correspond \`a la vitesse instantan\'ee \`a
ce point.


\end{document}

