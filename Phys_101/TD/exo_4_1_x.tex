\documentclass[]{book}

%These tell TeX which packages to use.
\usepackage{array,epsfig}
\usepackage{amsmath}
\usepackage{amsfonts}
\usepackage{amssymb}
\usepackage{amsxtra}
\usepackage{amsthm}
\usepackage{mathrsfs}
\usepackage{color}

%Here I define some theorem styles and shortcut commands for symbols I use often
\theoremstyle{definition}
\newtheorem{defn}{Definition}
\newtheorem{thm}{Theorem}
\newtheorem{cor}{Corollary}
\newtheorem*{rmk}{Remark}
\newtheorem{lem}{Lemma}
\newtheorem*{joke}{Joke}
\newtheorem{ex}{Example}
\newtheorem*{soln}{Solution}
\newtheorem{prop}{Proposition}

\newcommand{\lra}{\longrightarrow}
\newcommand{\ra}{\rightarrow}
\newcommand{\surj}{\twoheadrightarrow}
\newcommand{\graph}{\mathrm{graph}}
\newcommand{\bb}[1]{\mathbb{#1}}
\newcommand{\Z}{\bb{Z}}
\newcommand{\Q}{\bb{Q}}
\newcommand{\R}{\bb{R}}
\newcommand{\C}{\bb{C}}
\newcommand{\N}{\bb{N}}
\newcommand{\M}{\mathbf{M}}
\newcommand{\m}{\mathbf{m}}
\newcommand{\MM}{\mathscr{M}}
\newcommand{\HH}{\mathscr{H}}
\newcommand{\Om}{\Omega}
\newcommand{\Ho}{\in\HH(\Om)}
\newcommand{\bd}{\partial}
\newcommand{\del}{\partial}
\newcommand{\bardel}{\overline\partial}
\newcommand{\textdf}[1]{\textbf{\textsf{#1}}\index{#1}}
\newcommand{\img}{\mathrm{img}}
\newcommand{\ip}[2]{\left\langle{#1},{#2}\right\rangle}
\newcommand{\inter}[1]{\mathrm{int}{#1}}
\newcommand{\exter}[1]{\mathrm{ext}{#1}}
\newcommand{\cl}[1]{\mathrm{cl}{#1}}
\newcommand{\ds}{\displaystyle}
\newcommand{\vol}{\mathrm{vol}}
\newcommand{\cnt}{\mathrm{ct}}
\newcommand{\osc}{\mathrm{osc}}
\newcommand{\LL}{\mathbf{L}}
\newcommand{\UU}{\mathbf{U}}
\newcommand{\support}{\mathrm{support}}
\newcommand{\AND}{\;\wedge\;}
\newcommand{\OR}{\;\vee\;}
\newcommand{\Oset}{\varnothing}
\newcommand{\st}{\ni}
\newcommand{\wh}{\widehat}

%Pagination stuff.
\setlength{\topmargin}{-.3 in}
\setlength{\oddsidemargin}{0in}
\setlength{\evensidemargin}{0in}
\setlength{\textheight}{9.in}
\setlength{\textwidth}{6.5in}
\pagestyle{empty}



\begin{document}


\subsection*{Rappel de cours}
\subsubsection*{Travail}

\begin{itemize}
\item La composante de la force d'un point $M$, $\overrightarrow{F}(M)$ sur l'axe $\mathcal{O}_x$ est donn\'ee par le produit scalaire $f(x) = \overrightarrow{F}(M).\overrightarrow{i}$. 
\item Le travail d'une force $\overrightarrow{F}$ sur un segment $\overrightarrow{AB}$ est donn\'e par :
$$W_{A \to B}(\overrightarrow{F}) = \int_{A \to B} \overrightarrow{F}.\overrightarrow{i}dx = \int_{x_a}^{x_b} f(x)dx$$ 
\item On dira qu'une force est conservative si elle ne d\'epend que de la position et si son travail d'un point $A$ au point $B$ ne d\'epend pas du chemin suivi, ceci quels que soient les point $A$ et $B$.
$$\forall A,B,C,\, W_{A \to B} \overrightarrow{F} = W_{A \to C} \overrightarrow{F} + W_{C \to B} \overrightarrow{F}$$
\item Dans le cas o\`u le chemin est rectiligne, si une force est conservatrice alors l'\emph{\'energie potentielle} associ\'ee \`a la force $\overrightarrow{F}$ est not\'ee $E_p(x)$ est d\'efinie par :
$$W_{A \to B}(\overrightarrow{F}) = \int_{A \to B} \overrightarrow{F}.\overrightarrow{i}dx = E_p(x_a) - E_p(x_b)$$. 
\end{itemize}

\subsubsection*{\'Energie}
\begin{itemize}
\item L'\'energie m\'ecanique d'un syst\`eme $E_m = E_c + E_p$ avec $E_c$ l'\'energie cin\'etique qui d\'epend de la masse et de la norme de la vitesse du syst\`eme physique \'etudi\'e et de l'\'energie potentielle $E_p$ qui correspond aux forces exerc\'ees sur le syst\`eme.
\item L\'energie m\'ecanique est \'egale \`a $E_m = E_c + E_p$. Avec l'\'energie cin\'etique du syst\`eme $E_c = \frac{1}{2}mv_0^2$
\item L'\'energie potentielle qui correspond \`a l'ensemble des forces qui s'exercent sur le syst\'eme
\end{itemize}

\subsubsection*{Puissance}
\begin{itemize}
\item La puissance $P$ repr\'esente l'\'energie transf\'er\'ee uniformement (ie. le travail) pendant une unit\'e de temps, $P =\frac{W}{\Delta t}$.
\item $1\,W = 1\,J.s^{-1} = 1\, N.m.s^{-1} = 1\, kg.m^{2}.s^{-3}$
\end{itemize}



\subsection*{Exo 4.1.1}
L'\'electron est soumis a la force gravitationnelle. Donc $E_m = E_c + E_p$ avec $E_c = \frac{1}{2}mv^2$. On n\'eglige l'\'energie de la force gravitationnelle devant celle de l'\'energie cim\'etique. On a $E_m = \frac{1}{2}mv^2 = 18\,keV$, donc $v = \sqrt{\frac{2*18\,keV}{m}}$ et $m=9.10\,10^{-31}kg$, $18\,keV = 2.88\,10^{-12}$. 


\subsection*{Exo 4.1.2}
Il faut monter une masse de $m=500+5*70 = 850\,kg$ \`a une vitesse de $v=25/60 = 0.41\,m/s$. La puissance n\'ecessaire est $P=m.g.v = 850*9.81*0.41 = 3418\,Watt$.\\
 
En l'abscence de frottements la puissance n\'ecessaire pour lever la cabine d'ascenseur est \'egale \`a la puissance fournie par le moteur. Le puissance du poids est r\'esistante, celle du moteur est motrice.\\

On a $P =\frac{W}{\Delta_t}$, donc $W = P*\Delta_t = 3418*60 = 205kJ$.

\subsection*{Exo 4.1.3}
\subsubsection{Q1}
$TWh$ repr\'esente des $10^{12}Wh$.

\subsubsection{Q2}
On a $1W = 1J/s$. On produit $429.10^{12}Wh$ pour une ann\'ee, donc on a produit $\frac{429.10^{12}}{24*365.25} = 48.10^{9} W$. Ce qui fait $48.10^{9}*(24*365.25*3600) = 1.510^{18}J$.

\subsubsection{Q3}
La puissance \'electrique moyenne d'un r\'eacteur est de $\frac{48.10^{9}}{58} = 0.82 MW$.  



\end{document}

