\documentclass[]{book}

%These tell TeX which packages to use.
\usepackage{array,epsfig}
\usepackage{amsmath}
\usepackage{amsfonts}
\usepackage{amssymb}
\usepackage{amsxtra}
\usepackage{amsthm}
\usepackage{mathrsfs}
\usepackage{color}
\usepackage{pgfplots}
\usepackage{fancyhdr}

%Here I define some theorem styles and shortcut commands for symbols I use often
\theoremstyle{definition}
\newtheorem{defn}{Definition}
\newtheorem{thm}{Theorem}
\newtheorem{cor}{Corollary}
\newtheorem*{rmk}{Remark}
\newtheorem{lem}{Lemma}
\newtheorem*{joke}{Joke}
\newtheorem{ex}{Example}
\newtheorem*{soln}{Solution}
\newtheorem{prop}{Proposition}

\newcommand{\lra}{\longrightarrow}
\newcommand{\ra}{\rightarrow}
\newcommand{\surj}{\twoheadrightarrow}
\newcommand{\graph}{\mathrm{graph}}
\newcommand{\bb}[1]{\mathbb{#1}}
\newcommand{\Z}{\bb{Z}}
\newcommand{\Q}{\bb{Q}}
\newcommand{\R}{\bb{R}}
\newcommand{\C}{\bb{C}}
\newcommand{\N}{\bb{N}}
\newcommand{\M}{\mathbf{M}}
\newcommand{\m}{\mathbf{m}}
\newcommand{\MM}{\mathscr{M}}
\newcommand{\HH}{\mathscr{H}}
\newcommand{\Om}{\Omega}
\newcommand{\Ho}{\in\HH(\Om)}
\newcommand{\bd}{\partial}
\newcommand{\del}{\partial}
\newcommand{\bardel}{\overline\partial}
\newcommand{\textdf}[1]{\textbf{\textsf{#1}}\index{#1}}
\newcommand{\img}{\mathrm{img}}
\newcommand{\ip}[2]{\left\langle{#1},{#2}\right\rangle}
\newcommand{\inter}[1]{\mathrm{int}{#1}}
\newcommand{\exter}[1]{\mathrm{ext}{#1}}
\newcommand{\cl}[1]{\mathrm{cl}{#1}}
\newcommand{\ds}{\displaystyle}
\newcommand{\vol}{\mathrm{vol}}
\newcommand{\cnt}{\mathrm{ct}}
\newcommand{\osc}{\mathrm{osc}}
\newcommand{\LL}{\mathbf{L}}
\newcommand{\UU}{\mathbf{U}}
\newcommand{\support}{\mathrm{support}}
\newcommand{\AND}{\;\wedge\;}
\newcommand{\OR}{\;\vee\;}
\newcommand{\Oset}{\varnothing}
\newcommand{\st}{\ni}
\newcommand{\wh}{\widehat}

%Pagination stuff.
\setlength{\topmargin}{-.3 in}
\setlength{\oddsidemargin}{0in}
\setlength{\evensidemargin}{0in}
\setlength{\textheight}{9.in}
\setlength{\textwidth}{6.5in}
\pagestyle{fancy}
\fancyhf{}
\rhead{Math\_132}
\lhead{Exam}
\rfoot{Page \thepage}


\begin{document}

\subsection*{Rappel de cours}

\begin{itemize}
\item 
\end{itemize}


\subsection*{Exercice 1}
\subsubsection*{Exercice 1.1.a}
La d\'efinition de "$f$ est d\'erivable en $x_0$" (note $f'(x_0)$) si la limite existe et est finie.
$$
f'(x_0) = \lim_{x \to x_0} \frac{f(x)-f(x_0)}{x-x_0}
$$

Pour $x_0 =0$, on a 
$$
f'(0) = \lim_{x \to 0} \frac{f(x)-f(0)}{x}
$$

\subsubsection*{Exercice 1.1.b}
$$
l = \lim_{x \to 0} \frac {f(2x) -f(x)}{x} = \lim_{x \to 0} 2\frac {f(2x) -f(0)}{2x} - \frac {f(x) -f(0)}{x} = 2 \lim_{\frac{X}{2} \to 0} \frac {f(X) -f(0)}{X} - \lim_{x \to 0} \frac {f(x) -f(0)}{x}
$$
$$
= 2 \lim_{X \to 0} \frac {f(X) -f(0)}{X} - \lim_{x \to 0} \frac {f(x) -f(0)}{x} = 2f'(0) - f'(0) = f'(0)
$$

\subsubsection*{Exercice 1.2.a}
1 - Montrons que $f$ d\'erivable en 0 $\implies$ $g$ d\'erivable en 0.\\
$g$ d\'erivable en 0 si il existe $a = \lim_{x \to 0} \frac{g(x) - g(0)}{x}$.

$$\lim_{x \to 0} \frac{g(x) - g(0)}{x} = \lim_{x \to 0} \frac{f(x) - lx - f(0)}{x} = \lim_{x \to 0} \frac{f(x)- f(0)}{x} -l$$

$f$ est d\'erivable en 0 donc $\lim_{x \to 0} \frac{f(x)- f(0)}{x}$ existe, l existe \'egalement donc a existe.\\

2 - Montrons que $g$ d\'erivable en 0 $\implies$ $f$ d\'erivable en 0.\\
$g$ d\'erivable en 0 donc il existe $a = \lim_{x \to 0} \frac{g(x) - g(0)}{x}$
$$a = \lim_{x \to 0} \frac{g(x) - g(0)}{x} = \lim_{x \to 0} \frac{f(x) - lx - f(0)}{x} = \lim_{x \to 0} \frac{f(x)- f(0)}{x} -l$$

Par hypoth\`ese $a$ et $l$ existe, par cons\'equent $\lim_{x \to 0} \frac{f(x)- f(0)}{x}$ existe ($= a+l$). Donc $f$ est d\'erivable en 0.

\subsubsection*{Exercice 1.2.b}
$$\forall x \in \R, \lim_{n \to +\infty}g\left(\frac{x}{2^n}\right) = \lim_{X \to 0}g(X)$$

Car $2^n$ est toujours tr\`es grand devant $x$ quand $n$ tend vers $+\infty$.
  
$$\lim_{X \to 0}g(X) = \lim_{X \to 0} f(X) - lX = \lim_{X \to 0} f(X) - l\lim_{X \to 0}X = \lim_{X \to 0} f(X) = f(0)$$

Car comme la fonction $f$ est d\'erivable en 0, elle est continue en 0.
On a $g(0) = f(0) - l0 = f(0)$ donc $\forall x \in \R, \lim_{n \to +\infty}g\left(\frac{x}{2^n}\right) = g(0)$.

\subsubsection*{Exercice 1.3.a}
La fonction $h$ est continue en 0, si $\lim_{x \to 0} h(x) = 0$.
$$\lim_{x \to 0}h(x) = \lim_{x \to 0}\frac{f(2x)-2lx - f(x) + lx}{x} = \lim_{x \to 0}\frac{f(2x) - f(x)}{x} - l = l - l = 0$$
Donc $h$ est continue en 0.\\

$\lim_{x \to 0}h(x) = 0$ est \'equivalent \`a $\forall \epsilon > 0, \exists \sigma > 0, |x|<\sigma \implies |h(x)| < \epsilon$ et $\forall x \in [a,b], |h(x)| < Sup|h(x)|$. \\

\subsubsection*{Exercice 1.3.b}

On a $\forall k, \forall x \in [-\alpha_{\epsilon}, \alpha_{\epsilon}], X = 
\frac{x}{2^k} \in [-\frac{\alpha_{\epsilon}}{2^k}, \frac{\alpha_{\epsilon}}{2^k}] \in [-\alpha_{\epsilon}, \alpha_{\epsilon}]$. Donc
$$|h(X)| < \epsilon$$
$$\sum_{k=1}^{n}{|h(\frac{x}{2^k})}| < \sum_{k=1}^{n}{\epsilon}$$
$$\sum_{k=1}^{n}{\frac{1}{2^k}|h(\frac{x}{2^k})|} < \sum_{k=1}^{n}{\frac{1}{2^k} \epsilon}$$
$$\sum_{k=1}^{n}{\frac{1}{2^k}|h(\frac{x}{2^k})|} < \epsilon \sum_{k=1}^{n}{\frac{1}{2^k}}$$
Par in\'egalit\'e triangulaire
$$|\sum_{k=1}^{n}{\frac{1}{2^k}h(\frac{x}{2^k})}| < \epsilon \sum_{k=1}^{n}{\frac{1}{2^k}}$$

\subsubsection*{Exercice 1.3.c}
Soit la suite $v_{n} = \frac{1}{2^n}$. Suite g\'eom\'etique de raison $\frac{1}{2}$.

$$\sum_{k=1}^{n} v_n = \frac{\frac{1}{2}(1-\frac{1}{2^n})}{1-\frac{1}{2}} = 1-\frac{1}{2^n}$$
$$\lim_{n \to \infty} \sum_{k=1}^{n} v_n = \lim_{n \to \infty} 1-\frac{1}{2^n} = 1$$

La suite $u_n$ est strictement croissante, $u_1=1/2$ et sa limite est 1, donc la suite $u_n$ est major\'ee par 1.


\subsubsection*{Exercice 1.4.a}
Soit la relation $g(x) - g(\frac{x}{2^n}) = x\sum_{k=1}^{n} {\frac{1}{2^k}h\left(\frac{x}{2^k}\right)}$, montrons $g(x) - g(\frac{x}{2^{n+1}}) = x\sum_{k=1}^{n+1} {\frac{1}{2^k}h\left(\frac{x}{2^k}\right)}$.\\
Pour $x=0$, on a $g(0)- g(\frac{0}{2^n}) = 0\sum_{k=1}^{n} {\frac{1}{2^k}h\left(\frac{x}{2^k}\right)}$.\\
Vrai\\

Pour $x \neq 0$,
$$g(x) - g(\frac{x}{2^{n+1}}) = x\sum_{k=1}^{n+1} {\frac{1}{2^k}h\left(\frac{x}{2^k}\right)} = x\left(\sum_{k=1}^{n} {\frac{1}{2^k}h\left(\frac{x}{2^k}\right)} + \frac{1}{2^{n+1}}h\left(\frac{x}{2^{n+1}}\right)\right)$$

$$g(x) - g(\frac{x}{2^{n+1}}) = g(x) - g(\frac{x}{2^{n}}) + \frac{x}{2^{n+1}}h\left(\frac{x}{2^{n+1}}\right)$$

$$ g(\frac{x}{2^{n}}) - g(\frac{x}{2^{n+1}}) = \frac{x}{2^{n+1}}h\left(\frac{x}{2^{n+1}}\right)$$

$$ \frac{g(\frac{2x}{2^{n+1}}) - g(\frac{x}{2^{n+1}})}{\frac{x}{2^{n+1}}} = h\left(\frac{x}{2^{n+1}}\right)$$

Vrai.

\subsubsection*{Exercice 1.4.b}
$$g(x) - g(\frac{x}{2^n}) = x\sum_{k=1}^{n} {\frac{1}{2^k}h\left(\frac{x}{2^k}\right)}$$
Pour $x \neq 0$, 
$$\frac{g(x) - g(\frac{x}{2^n})}{x} = \sum_{k=1}^{n} {\frac{1}{2^k}h\left(\frac{x}{2^k}\right)}$$

$$\left| \frac{g(x) - g(\frac{x}{2^n})}{x} \right| = \left| \sum_{k=1}^{n} {\frac{1}{2^k}h\left(\frac{x}{2^k}\right)} \right| $$

De 1.3.b
$$\left| \frac{g(x) - g(\frac{x}{2^n})}{x} \right| = \left| \sum_{k=1}^{n} {\frac{1}{2^k}h\left(\frac{x}{2^k}\right)} \right|  < \epsilon \sum_{k=1}^{n}{\frac{1}{2^k}}$$

Quand $n \to \infty$, par 1.3.c
$$\left| \frac{g(x) - g(\frac{x}{2^n})}{x} \right| < \epsilon $$
$$\left| \frac{g(x) - g(0)}{x} \right| < \epsilon $$

\subsubsection*{Exercice 1.4.c}
La fonction $h$ est d\'erivable en 0.\\

\subsection*{Exercice 2}
\subsubsection*{Exercice 2.1.a}
Si la  fonction $f$ est d\'erivable sur $\R$ alors $\lim_{h \to 0} \frac{f(x+h) - f(x)}{h}$ existe pour tout $x \in \R$. \\
$$\lim_{h \to 0} \frac{f(x+h) - f(x)}{h} = l_1$$
Et
$$\lim_{h \to 0} \frac{f(x) - f(x-h)}{h} = l_2$$
Donc
$$\lim_{h \to 0} \frac{f(x+h) - f(x)}{h} + \lim_{h \to 0} \frac{f(x) - f(x-h)}{h} = 2l$$
$$\lim_{h \to 0} \frac{f(x+h) - f(x-h)}{h} = l_1 + l_2$$

$$\lim_{h \to 0} \frac{f(x+h) - f(x-h)}{2h} = \frac{l_1 + l_2}{2}$$

$$f'(x) = \tilde{f}(x)$$
???


\subsubsection*{Exercice 2.1.b}
$$
f(x) = \left\{ 
\begin{array}{l l}
0 & x \neq 0 \\
1 & x = 0 \\
\end{array}
\right.
$$

On a 
$$\tilde{f}(0)= \lim_{h \to 0} \frac{f(0+h)-f(0-h)}{2h} = \lim_{h \to 0} \frac{0 - 0}{2h} = 0$$

\subsubsection*{Exercice 2.1.c}
$$f(x) = |x|$$

$f(x)$ est continu en 0, non d\'erivable en 0 et pseudo-d\'erivable en 0.
$$\tilde{f}(0)= \lim_{h \to 0} \frac{f(0+h)-f(0-h)}{2h} = \lim_{h \to 0} \frac{|h|-|-h|}{2h} = \lim_{h \to 0} 0 = 0$$


\subsubsection*{Exercice 2.2.a}

On suppose que la fonction $f$ n'est pas croissante $\exists a,b \in \R, a<b, f(a) > f(b)$ alors $f(b) < m = \frac{f(a)+f(b)}{2} < f(a)$. Soit $F=\{x \in [a,b], f(x) = m\}$ et $c = max(F)$. La fonction $f$ est continue donc elle coupe la droite $y=m$ au moins une fois entre les points $a$ et $b$, par cons\'equent l'ensemble $F$ n'est pas vide.   

On a $\forall x \in ]c,b], f(x) < m$ car $m > f(b)$ et si la fonction $f$ coupait la droite $y=m$ apr\`es $c$, cela contredirait la d\'efinition de $c$. Donc $c$ est le plus petit majorant de $E$, car aucun point sup\'erieur ou \'egal \`a $c$ n'est dans E. Donc $c$ est la borne sup\'erieure.

\subsubsection*{Exercice 2.2.b}
$f$ est continue et $x \in E$, donc $f(x) > m$. Posons $h = c-(x+\epsilon)$. On a $f(x+h) = f(c-\epsilon) > f(c) = m$. \\

$m = f(c) \notin E$ voir pr\'ec\'edent (2.2.a).

\subsubsection*{Exercice 2.2.c}
Soit $F_1=\{x \in [a,c[, f(x) = m\}$ et $c_1 = max(\{a\} \cup F_1)$.
Posons la suite $h_n = \frac{c-c_1}{2n}$, on a $h_n > 0$ car $c>c_1$. On a $\forall x \in ]c_1,c[, f(x) > m$ car la fonction $f$ est continue. Donc $f(c-h_n) > f(c) = m$ et $f(x-h_n) \in E$ et $\lim_{n \to +\infty} h_n = 0$.

\subsubsection*{Exercice 2.2.d}
Voir pr\'ec\'edent (2.2.a).

\subsubsection*{Exercice 2.2.e}
Voir pr\'ec\'edent (2.2.a).

\subsubsection*{Exercice 2.2.f}
On a $f(c+h_n) < f(c-h_n)$ car $f(c-h_n) \in E$ et $f(c+h_n) \notin E$. (voir pr\'ec\'edent 2.2.c et 2.2.e). Donc $f(c+h_n)- f(c-h_n) < 0$. Par cons\'equent, 
$$\lim_{n \to +\infty} \frac{f(c+h_n)- f(c-h_n)}{2h_n} < 0$$

Donc
$$ \lim_{n \to +\infty} \frac{f(c+h_n)- f(c-h_n)}{2h_n} =
\lim_{h \to 0+} \frac{g_\alpha(x+h)- g_\alpha(x-h)}{2h} = \tilde{f} < 0$$


\subsubsection*{Exercice 2.3.a}
$$\tilde{g_\alpha}(x) = \lim_{h \to 0} \frac{g_\alpha(x+h)- g_\alpha(x-h)}{2h} = \lim_{h \to 0} \frac{f(x+h) + \alpha (x+h)  - f(x-h) - \alpha (x-h)}{2h}$$
$$=\lim_{h \to 0} \frac{f(x+h) - f(x-h)}{2h} + \lim_{h \to 0} \frac{2 \alpha h}{2h} = \tilde{f}(x) + \alpha$$

\subsubsection*{Exercice 2.3.b}
On a $\alpha > 0$, $\tilde{f}(x) \geq 0$, donc  $\tilde{g_\alpha}(x) = \tilde{f}(x) + \alpha > 0$. En 2.1 on a montr\'e que si $\tilde{f}(x) \geq \alpha' > 0$ alors la fonction $f$ est croissante. Comme $\tilde{g_\alpha}(x) > \alpha > 0$ alors $g_{\alpha}(x)$ est croissante. 
$$\forall x, y \in \R, x \leq y, g(x) \geq g(y)$$
Donc
$$\forall x, y \in \R, x \leq y, f(x) + \alpha x \geq f(y) + \alpha y$$

\subsubsection*{Exercice 2.3.c}

La fonction $f	$ est croissante s\^urement mais pourquoi?.\\

QED.

\end{document}

