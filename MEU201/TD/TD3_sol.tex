\documentclass[]{book}

%These tell TeX which packages to use.
\usepackage{array,epsfig}
\usepackage{amsmath}
\usepackage{amsfonts}
\usepackage{amssymb}
\usepackage{amsxtra}
\usepackage{amsthm}
\usepackage{mathrsfs}
\usepackage{color}

%Here I define some theorem styles and shortcut commands for symbols I use often
\theoremstyle{definition}
\newtheorem{defn}{Definition}
\newtheorem{thm}{Theorem}
\newtheorem{cor}{Corollary}
\newtheorem*{rmk}{Remark}
\newtheorem{lem}{Lemma}
\newtheorem*{joke}{Joke}
\newtheorem{ex}{Example}
\newtheorem*{soln}{Solution}
\newtheorem{prop}{Proposition}

\newcommand{\lra}{\longrightarrow}
\newcommand{\ra}{\rightarrow}
\newcommand{\surj}{\twoheadrightarrow}
\newcommand{\graph}{\mathrm{graph}}
\newcommand{\bb}[1]{\mathbb{#1}}
\newcommand{\Z}{\bb{Z}}
\newcommand{\Q}{\bb{Q}}
\newcommand{\R}{\bb{R}}
\newcommand{\C}{\bb{C}}
\newcommand{\N}{\bb{N}}
\newcommand{\M}{\mathbf{M}}
\newcommand{\m}{\mathbf{m}}
\newcommand{\MM}{\mathscr{M}}
\newcommand{\HH}{\mathscr{H}}
\newcommand{\Om}{\Omega}
\newcommand{\Ho}{\in\HH(\Om)}
\newcommand{\bd}{\partial}
\newcommand{\del}{\partial}
\newcommand{\bardel}{\overline\partial}
\newcommand{\textdf}[1]{\textbf{\textsf{#1}}\index{#1}}
\newcommand{\img}{\mathrm{img}}
\newcommand{\ip}[2]{\left\langle{#1},{#2}\right\rangle}
\newcommand{\inter}[1]{\mathrm{int}{#1}}
\newcommand{\exter}[1]{\mathrm{ext}{#1}}
\newcommand{\cl}[1]{\mathrm{cl}{#1}}
\newcommand{\ds}{\displaystyle}
\newcommand{\vol}{\mathrm{vol}}
\newcommand{\cnt}{\mathrm{ct}}
\newcommand{\osc}{\mathrm{osc}}
\newcommand{\LL}{\mathbf{L}}
\newcommand{\UU}{\mathbf{U}}
\newcommand{\support}{\mathrm{support}}
\newcommand{\AND}{\;\wedge\;}
\newcommand{\OR}{\;\vee\;}
\newcommand{\Oset}{\varnothing}
\newcommand{\st}{\ni}
\newcommand{\wh}{\widehat}

%Pagination stuff.
\setlength{\topmargin}{-.3 in}
\setlength{\oddsidemargin}{0in}
\setlength{\evensidemargin}{0in}
\setlength{\textheight}{9.in}
\setlength{\textwidth}{6.5in}
\pagestyle{empty}



\begin{document}

\subsection*{Rappel de cours}
Croissance ompar\'ee. On a $\forall n_, n_2, n_3 \in \R^{*+}, \forall x \to \infty$:
$$(\ln x)^{n_1} << x^{n_2} << e^{n_3 x}$$


Donc, pour tous r\'eel $a,b > 0$, on a :
\begin{itemize}
\item $\lim_{x \to +\infty}{\frac{e^ax}{x^b}} = +\infty$. Par exemple, $\lim_{x \to +\infty}{\frac{e^x}{x}} = +\infty$  
\item $\lim_{x \to -\infty}{|x|^{b}e^ax} = 0$. Par exemple, $\lim_{x \to -\infty}{xe^x} = 0$  
\item $\lim_{x \to +\infty}{\frac{\ln(x)^a}{x^b}} = 0$. Par exemple, $\lim_{x \to -\infty}{\frac{\ln(x)}{x}} = 0$  
\item $\lim_{x \to 0^{+}}{x^{b}|\ln(x)|^ax} = 0$. Par exemple, $\lim_{x \to 0^{+}}{x\ln(x)} = 0$  
\end{itemize}

Quelques r\'esultats connus, sur les s\'eries de terme g\'en\'eral:
\begin{itemize}
\item  $\frac{c}{n^s}$ converge pour $s > 1$ et diverge pour $0 \leq s \leq 1$
\item si le terme g\'en\'eral ne tend pas vers 0, la s\'erie diverge.
\end{itemize}


\newpage
\subsection*{Exercice 3}
\subsection*{Exercice 3-e}
On a 
$$e_n = \frac{n!}{n^n} = \frac{1.2.3 \ldots (n-1).n}{1.2.3 \ldots (n-1).n} < \frac{1}{n}.\frac{2}{n}.1.1\ldots 1 = \frac{2}{n^2}$$
On sait que la s\'erie de terme g\'en\'eral $\frac{1}{n^2}$ converge ($\sum_{0}^{\infty}{\frac{1}{n^2}}$ converge). En appliquant le Th\'eor\`eme de Riemann, on d\'eduit que la suite de terme g\'en\'eral $e_n$ converge.

\subsection*{Exercice 3-f}
On a 
$$f_n = \frac{1}{n^2 \ln n} < \frac{1}{n^3}$$
On sait que la s\'erie de terme g\'en\'eral $\frac{1}{n^3}$ converge ($\sum_{0}^{\infty}{\frac{1}{n^3}}$ converge). En appliquant le Th\'eor\`eme de Riemann, on d\'eduit que la suite de terme g\'en\'eral $f_n$ converge.

\subsection*{Exercice 3-g}
On a 
$$f_n = \frac{2^n+3^n+n^4}{n5^n+7n^7+2}$$

\subsection*{Exercice 3-h}
On a 
$$f_n = \left( \frac{n-1}{2n+1} \right)^n = \left( \frac{1-\frac{1}{n}}{2+\frac{1}{n}} \right)^n < \left( \frac{1}{2} \right)^n = \frac{1}{2^n}$$
C'est une s\'erie g\'eome\'etrique de raison $\frac{1}{2}$ et de premier terme -1. Donc $\sum_{0}^{\infty}\frac{1}{2^n} = -\frac{1-\frac{1}{2^n}}{1-\frac{1}{2}} = -2(1-\frac{1}{2^n})$. Donc la suite de terme g\'en\'eral $h_n$ converge.

\subsection*{Exercice 4}
\subsection*{Exercice 4-i}
On a 
$$i_n = \left( \frac{3n+1}{n+5} \right)^n = \left( \frac{3+\frac{1}{n}}{1+\frac{5}{n}} \right)^n < 3^n = $$
C'est une s\'erie g\'eome\'etrique de raison $3$ et de premier terme $\frac{1}{5}$. Donc $\sum_{0}^{\infty} 3^n = \frac{1}{5}\frac{1-3^n}{1-3} = -\frac{1}{5}\frac{1-3^n}{2}$. Donc la suite de terme g\'en\'eral $i_n$ diverge.


QED

\end{document}

