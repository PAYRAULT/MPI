\documentclass[]{book}

%These tell TeX which packages to use.
\usepackage{array,epsfig}
\usepackage{amsmath}
\usepackage{amsfonts}
\usepackage{amssymb}
\usepackage{amsxtra}
\usepackage{amsthm}
\usepackage{mathrsfs}
\usepackage{color}

%Here I define some theorem styles and shortcut commands for symbols I use often
\theoremstyle{definition}
\newtheorem{defn}{Definition}
\newtheorem{thm}{Theorem}
\newtheorem{cor}{Corollary}
\newtheorem*{rmk}{Remark}
\newtheorem{lem}{Lemma}
\newtheorem*{joke}{Joke}
\newtheorem{ex}{Example}
\newtheorem*{soln}{Solution}
\newtheorem{prop}{Proposition}

\newcommand{\lra}{\longrightarrow}
\newcommand{\ra}{\rightarrow}
\newcommand{\surj}{\twoheadrightarrow}
\newcommand{\graph}{\mathrm{graph}}
\newcommand{\bb}[1]{\mathbb{#1}}
\newcommand{\Z}{\bb{Z}}
\newcommand{\Q}{\bb{Q}}
\newcommand{\R}{\bb{R}}
\newcommand{\C}{\bb{C}}
\newcommand{\N}{\bb{N}}
\newcommand{\M}{\mathbf{M}}
\newcommand{\m}{\mathbf{m}}
\newcommand{\MM}{\mathscr{M}}
\newcommand{\HH}{\mathscr{H}}
\newcommand{\Om}{\Omega}
\newcommand{\Ho}{\in\HH(\Om)}
\newcommand{\bd}{\partial}
\newcommand{\del}{\partial}
\newcommand{\bardel}{\overline\partial}
\newcommand{\textdf}[1]{\textbf{\textsf{#1}}\index{#1}}
\newcommand{\img}{\mathrm{img}}
\newcommand{\ip}[2]{\left\langle{#1},{#2}\right\rangle}
\newcommand{\inter}[1]{\mathrm{int}{#1}}
\newcommand{\exter}[1]{\mathrm{ext}{#1}}
\newcommand{\cl}[1]{\mathrm{cl}{#1}}
\newcommand{\ds}{\displaystyle}
\newcommand{\vol}{\mathrm{vol}}
\newcommand{\cnt}{\mathrm{ct}}
\newcommand{\osc}{\mathrm{osc}}
\newcommand{\LL}{\mathbf{L}}
\newcommand{\UU}{\mathbf{U}}
\newcommand{\support}{\mathrm{support}}
\newcommand{\AND}{\;\wedge\;}
\newcommand{\OR}{\;\vee\;}
\newcommand{\Oset}{\varnothing}
\newcommand{\st}{\ni}
\newcommand{\wh}{\widehat}

%Pagination stuff.
\setlength{\topmargin}{-.3 in}
\setlength{\oddsidemargin}{0in}
\setlength{\evensidemargin}{0in}
\setlength{\textheight}{9.in}
\setlength{\textwidth}{6.5in}
\pagestyle{empty}



\begin{document}

\subsection*{Rappel de cours}

\begin{defn}
Soit $f(x)$ et $g(x)$ deux fonctions r\'eelles d\'efinies au voisinage de $+\infty$ avec $g(x)$ qui ne s'annule pas en $+\infty$. Lorsque 	
$$\lim_{x \to +\infty}\frac{f(x)}{g(x)} = 0$$
On \'ecrit
$$f(x) = o_{x \to +\infty}(g(x))$$
Alors on dit que $f(x)$ est n\'egligeable pas rapport \`a $g(x)$.\\
\end{defn}


\begin{thm}
Th\'eor\`eme des croissances compar\'ees : Pour tous r\'eel $\alpha,\beta, \gamma, \lambda >0$
\begin{itemize}
\item Si $\alpha < \beta$ on a $x^{\alpha} = o_{x \to +\infty}(x^{\beta})$
\item on a $1 = o_{x \to +\infty}((\ln x)^{\gamma})$
\item on a $(\ln x)^{\gamma} = o_{x \to +\infty}(x^{\beta})$
\item on a $x^{\beta} = o_{x \to +\infty}(e^{\lambda x^{\alpha}})$
\end{itemize}
\end{thm}


\begin{thm}
On peut g\'en\'eraliser le th\'eor\`eme des croissances compar\'ees aux suites (avec $n$ un entier positif).
\begin{itemize}
\item on a $1 = o_{n \to +\infty}((\ln n)^{\gamma})$
\item on a $(\ln n)^{\gamma} = o_{n \to +\infty}(n^{\beta})$
\item on a $n^{\beta} = o_{n \to +\infty}(e^{\lambda n^{\alpha}})$
\item on a $e^{\lambda n^{\alpha}} = o_{n \to +\infty}(n!)$
\end{itemize}
\end{thm}

\newpage
\subsection*{Exercice 1}
\subsubsection*{$a_n$}
Il y a 4 cas, selon la valeur de $c$ :
\begin{itemize}
\item $|c| < 1$, $\lim_{n \to +\infty}{c^n} =0$
\item $c = 1$, $\lim_{n \to +\infty}{c^n} = 1$
\item $c \leq -1$, $\lim_{n \to +\infty}{c^n}$ n'existe pas
\item $c > 1$, $\lim_{n \to +\infty}{c^n} = +\infty$
\end{itemize}

\subsubsection*{$b_n$}
$$b_n = \frac{n^n}{n!} = \frac{n.n.n.n \ldots n.n}{1.2.3.4 \ldots (n-1)n} = n.\frac{n}{2}.\frac{n}{3}.\frac{n}{4} \ldots \frac{n}{n-1}.1$$
$$\lim_{n \to +\infty}{b_n} = +\infty$$
car $n>1, \frac{n}{2}>1, \frac{n}{3}>1,\ldots \frac{n-1}{n} >1$.

\subsubsection*{$c_n$}
$$c_n = \frac{n!}{2^n} = \frac{1.2.3.4 \ldots n}{2.2.2.2 \ldots 2} = \frac{1}{2}.1.\frac{3}{2}.\frac{4}{2} \ldots \frac{n}{2}$$
$$\lim_{n \to +\infty}\frac{3}{2}.\frac{4}{2} \ldots \frac{n}{2} = +\infty $$
car $\frac{3}{2}>1, \frac{4}{2}>1,\ldots \frac{n}{2} >1$ .
$$\lim_{n \to +\infty}{c_n} = \frac{1}{2}\lim_{n \to +\infty} \frac{3}{2}.\frac{4}{2} \ldots \frac{n}{2} = +\infty$$

\subsubsection*{$d_n$} 
Il y a 3 cas, selon la valeur de $c$ :
\begin{itemize}
\item $|c| < 1$, $\lim_{n \to +\infty}{c^n} =0$, donc $\lim_{n \to +\infty}{d_n} = 0$
\item $c = 1$, $\lim_{n \to +\infty}{c^n} = 1$, donc $\lim_{n \to +\infty}{d_n} = 0$ 
\item $|c| > 1$,  
$$d_n = \frac{c^n}{n!} = \frac{c.c.c.c \ldots c}{1.2.3.4 \ldots n} = c.\frac{c}{2}.\frac{c}{3}.\frac{c}{4} \ldots \frac{c}{c-1}.\frac{c}{c}.\frac{c}{c+1} \ldots \frac{c}{n}$$
On a $c << n$, donc il y a plus de nombres $<0$ que de nombres $>0$.
$$\lim_{n \to +\infty}{d_n} = 0$$
\end{itemize}

\subsubsection*{$e_n$}
$$\lim_{n \to +\infty}{\frac{-1^n}{n}} = 0$$ 
$$\lim_{n \to +\infty}{e_n} = \sqrt{4+\lim_{n \to +\infty}{\frac{-1^n}{n}}} = \sqrt{4} = 2$$

\subsubsection*{$f_n$}
$$\lim_{n \to +\infty}{\frac{(-1)^nn^2+5}{n^2+8}} = \lim_{n \to +\infty}{\frac{(-1)^nn^2}{n^2}} = \lim_{n \to +\infty}{(-1)^n}$$
car $5,8 << n^2$. Donc $\lim_{n \to +\infty}{f_n}$ n'existe pas.

\subsubsection*{$g_n$}
$1 \leq \lim_{n \to +\infty}{\cos n} \leq 1$ et $1 \leq \lim_{n \to +\infty}{\sin n} \leq 1$
$$\lim_{n \to +\infty}{\frac{3n^2-n+\cos n}{n^2 - \sin n}} = \lim_{n \to +\infty}{\frac{3n^2-n}{n^2}} = \lim_{n \to +\infty}{3-\frac{1}{n}}$$
car $\cos n << n^2$ et $\sin n << n^2$
$$\lim_{n \to +\infty}{g_n} = 3 $$

\subsubsection*{$h_n$}
$\lim_{n \to +\infty}{2^{-n}} = 0$ et $\lim_{n \to +\infty}{\sin(2^{-n})} = \sin(0) = 0$
$$\lim_{n \to +\infty}{h_n} = 0.0 =0$$

\subsubsection*{$i_n$}
$$\lim_{n \to +\infty}{\frac{2n+3}{4n+5}} = \lim_{n \to +\infty}{\frac{2n}{4n}} = \frac{2}{4} = \frac{1}{2}$$
On a $3,5 << n$ car $1$ est n\'egligeable devant $(\ln n)^{\gamma}$ et $(\ln n)^{\gamma}$ est n\'egligeable devant $n^{\beta}$, donc $1$ est n\'egligeable devant $n^{\beta}$ (Th\'eor\`eme des croissances compar\'ees)

\subsubsection*{$j_n$}
$$\lim_{n \to +\infty}{\frac{2n^3+3-7}{e^n+n^8}} = \lim_{n \to +\infty}{\frac{2n^3}{e^n+n^8}}$$
car $-4 << n$\\
$n^{\beta}$ est n\'egligeable devant $(e^{\lambda n^{\alpha}})$ (Th\'eor\`eme des croissances compar\'ees). Donc
$$\lim_{n \to +\infty}{\frac{2n^3}{e^n+n^8}} = \lim_{n \to +\infty}{\frac{2n^3}{e^n}} = 0$$

\subsubsection*{$k_n$}
$$\lim_{n \to +\infty}{\ln(k_n)} = \lim_{n \to +\infty}{\ln(n^{1/\ln(n)})} = \lim_{n \to +\infty}{\frac{1}{\ln(n)}.\ln(n)} = 1$$
Donc
$$\lim_{n \to +\infty}{k_n} = e^1 = e$$

\subsubsection*{$l_n$}
$$\lim_{n \to +\infty}{e^{m_n}} = \lim_{n \to +\infty}{e^{\ln(n)^{1/n}}} = \lim_{n \to +\infty}{e^{n.\frac{1}{n}}} = e$$
Donc
$$\lim_{n \to +\infty}{m_n} = 1$$

\subsubsection*{$m_n$}
$$\lim_{n \to +\infty}{\ln(m_n)} = \lim_{n \to +\infty}{\ln(n^{1/n}}) = \lim_{n \to +\infty}{\frac{\ln(n)}{n}} = 0$$
car $(\ln n)^{\gamma}$ est n\'egligeable devant $n^{\beta}$ (Th\'eor\`eme des croissances compar\'ees). Donc
$$\lim_{n \to +\infty}{m_n} = e^0 = 1$$
 

\subsubsection*{$o_n$}
$$\lim_{n \to +\infty}{\frac{n^n}{(n!)^{\frac{1}{2}}}} = \lim_{n \to +\infty}{\frac{n^{2n}}{n!}} = \lim_{n \to +\infty}{\frac{n^n}{n!}.n^n} = +\infty$$
Voir $b_n$.

\subsubsection*{$p_n$}
$$\lim_{n \to +\infty}{\frac{\ln(n^2+1}{n+1}} = \lim_{n \to +\infty}{\frac{\ln(n^2}{n}} = \lim_{n \to +\infty}{2\frac{\ln(n}{n}} = 2.0 = 0$$
Car $1 << n$.
$$\lim_{n \to +\infty}{p_n} = 0$$

\subsubsection*{$q_n$}
$(\ln n)^{\gamma}$ est n\'egligeable devant $n^{\beta}$ donc
$$\lim_{n \to +\infty}{n + 3\ln n} = \lim_{n \to +\infty}{n}$$

Et
$$\lim_{n \to +\infty}{e^{n-1}} = \lim_{n \to +\infty}{e^{n}}$$

$n^{\beta}$ est n\'egligeable devant $e^{\lambda n^{\alpha}}$ donc
$$\lim_{n \to +\infty}{\frac{n}{e^{n}}} = 0$$
$$\lim_{n \to +\infty}{q_n} = 0$$


\subsection*{Exercice 2}
\subsubsection*{$a_n$}
$$a_n = \frac{1}{n-1}-\frac{1}{n+1} = \frac{1}{n}-\frac{1}{n} = 0$$
car $1 = o_{n \to +\infty}n$.

\subsubsection*{$b_n$}
$$a_n = \sqrt{n+1}-\sqrt{n-1} = \sqrt{n}-\sqrt{n} = 0$$
car $1 = o_{n \to +\infty}n$.

\subsubsection*{$c_n$}
$$a_n = \sqrt{\ln(n+1)-\ln(n)} = \sqrt{\ln(n)-\ln(n)} = 0$$
car $1 = o_{n \to +\infty}n$.

\subsubsection*{$d_n$}
$n+1 ~ n$ car $1 = o_{n \to +\infty}n$.
$$\frac{1}{\sqrt{n+1}} = \frac{1}{\sqrt{n}} = \sqrt{\frac{1}{n}} = \sqrt{0} = 0$$

\subsubsection*{$e_n$}
$$\ln \left( \sin \frac{1}{n} \right) = \ln \left( \sin 0 \right) = \ln(0)$$
car $1 = o_{n \to +\infty}n$.
??

\subsubsection*{$f_n$}
Changement de variable $x=\frac{1}{n}$ et d\'eveloppement limit\'e de $\cos x$.
$$1-\cos x = 1 - (1-\frac{x^2}{2!}+o(x) = \frac{x^2}{2!}+o(x) = 0$$
Premier terme non nul est 0.

\subsubsection*{$g_n$}
??

\subsubsection*{$h_n$}
??

\subsection*{Exercice 3}
\subsubsection*{$a_n$}
$$\lim_{n \to +\infty} n \sin(\frac{1}{n}) = \lim_{n \to +\infty} \frac{\sin(\frac{1}{n})}{\frac{1}{n}}$$
Changement de variable $x=\frac{1}{n}$ et d\'eveloppement limit\'e de $\sin x$.
$$\lim_{x \to 0} \frac{\sin(x)}{x} = \lim_{x \to 0} \frac{x-\frac{x^3}{3!}+\frac{x^5}{5!}+o(x)}{x} = \lim_{x \to 0}{1-\frac{x^2}{3!}+\frac{x^4}{5!}+o(x)} = 1$$
$$\lim_{n \to +\infty} a_n = 1$$


\subsubsection*{$b_n$}
$$\lim_{n \to +\infty} n \sin(\frac{1}{n}) = \lim_{n \to +\infty} \frac{\sin(\frac{1}{n})}{\frac{1}{n}}$$
Changement de variable $x=\frac{1}{n}$ et d\'eveloppement limit\'e de $\sin x$.
$$\lim_{x \to 0} \frac{\sin(x)}{x} = \lim_{x \to 0} \frac{x-\frac{x^3}{3!}+o(x)}{x} = \lim_{x \to 0}{1-\frac{x^2}{3!}+o(x)}$$
$$\lim_{n \to +\infty} b_n = \lim_{x \to 0} {\left(1-\frac{x^2}{3!}+o(x)\right)^{\frac{1}{x^2}}}$$

$$\lim_{n \to +\infty} \ln( b_n ) = \lim_{x \to 0} {\ln \left(1-\frac{x^2}{3!}+o(x)\right)} = \lim_{x \to 0} {{\frac{1}{x^2}} \ln \left(1-\frac{x^2}{3!}+o(x)\right)}$$
D\'eveloppement limit\'e de $\ln(1-x)$.
$$\lim_{x \to 0} {{\frac{1}{x^2}}(-\frac{x^2}{3!}-\frac{x^4}{2.(3!)^2}+o(x))} = \lim_{x \to 0} {-\frac{1}{6}-\frac{x^2}{2.(3!)^2}+o(x)} = -\frac{1}{6}$$
$$\lim_{n \to +\infty} b_n  = e^{-\frac{1}{6}}$$

\subsubsection*{$c_n$}


QED

\end{document}

