\documentclass[]{book}

%These tell TeX which packages to use.
\usepackage{array,epsfig}
\usepackage{amsmath}
\usepackage{amsfonts}
\usepackage{amssymb}
\usepackage{amsxtra}
\usepackage{amsthm}
\usepackage{mathrsfs}
\usepackage{color}

%Here I define some theorem styles and shortcut commands for symbols I use often
\theoremstyle{definition}
\newtheorem{defn}{Definition}
\newtheorem{thm}{Theorem}
\newtheorem{cor}{Corollary}
\newtheorem*{rmk}{Remark}
\newtheorem{lem}{Lemma}
\newtheorem*{joke}{Joke}
\newtheorem{ex}{Example}
\newtheorem*{soln}{Solution}
\newtheorem{prop}{Proposition}

\newcommand{\lra}{\longrightarrow}
\newcommand{\ra}{\rightarrow}
\newcommand{\surj}{\twoheadrightarrow}
\newcommand{\graph}{\mathrm{graph}}
\newcommand{\bb}[1]{\mathbb{#1}}
\newcommand{\Z}{\bb{Z}}
\newcommand{\Q}{\bb{Q}}
\newcommand{\R}{\bb{R}}
\newcommand{\C}{\bb{C}}
\newcommand{\N}{\bb{N}}
\newcommand{\M}{\mathbf{M}}
\newcommand{\m}{\mathbf{m}}
\newcommand{\MM}{\mathscr{M}}
\newcommand{\HH}{\mathscr{H}}
\newcommand{\Om}{\Omega}
\newcommand{\Ho}{\in\HH(\Om)}
\newcommand{\bd}{\partial}
\newcommand{\del}{\partial}
\newcommand{\bardel}{\overline\partial}
\newcommand{\textdf}[1]{\textbf{\textsf{#1}}\index{#1}}
\newcommand{\img}{\mathrm{img}}
\newcommand{\ip}[2]{\left\langle{#1},{#2}\right\rangle}
\newcommand{\inter}[1]{\mathrm{int}{#1}}
\newcommand{\exter}[1]{\mathrm{ext}{#1}}
\newcommand{\cl}[1]{\mathrm{cl}{#1}}
\newcommand{\ds}{\displaystyle}
\newcommand{\vol}{\mathrm{vol}}
\newcommand{\cnt}{\mathrm{ct}}
\newcommand{\osc}{\mathrm{osc}}
\newcommand{\LL}{\mathbf{L}}
\newcommand{\UU}{\mathbf{U}}
\newcommand{\support}{\mathrm{support}}
\newcommand{\AND}{\;\wedge\;}
\newcommand{\OR}{\;\vee\;}
\newcommand{\Oset}{\varnothing}
\newcommand{\st}{\ni}
\newcommand{\wh}{\widehat}

%Pagination stuff.
\setlength{\topmargin}{-.3 in}
\setlength{\oddsidemargin}{0in}
\setlength{\evensidemargin}{0in}
\setlength{\textheight}{9.in}
\setlength{\textwidth}{6.5in}
\pagestyle{empty}



\begin{document}

\subsection*{Rappel de cours}
\begin{defn}
Deux suites $(u_n)_{n\geq0}$ et $(v_n)_{n\geq0}$ sont adjacentes ssi:
\begin{itemize}
\item $(u_n)_{n\geq0}$ est croisssante et $(v_n)_{n\geq0}$ est d\'ecroissante
\item $\forall n \in \N, u_n \leq v_n$
\item $\lim_{n \to \infty}(v_n-u_n)_{n\geq0} = 0$
\end{itemize}  
\end{defn}

\begin{defn}
Le coefficient binomial est donn\'ee par
$\begin{pmatrix}n\\k\end{pmatrix} = k\frac{n!}{k!(n-k)!}$.
\end{defn}

\begin{defn}
Quelques propri\'et\'es
\begin{itemize}
\item $\begin{pmatrix}n\\0\end{pmatrix} = 1$
\item $\begin{pmatrix}n\\n\end{pmatrix} = 1$
\item $\begin{pmatrix}n\\k\end{pmatrix} = \begin{pmatrix}n\\n-k\end{pmatrix}$ 
\item $\begin{pmatrix}n+1\\k\end{pmatrix} = \begin{pmatrix}n\\k\end{pmatrix} +\begin{pmatrix}n\\k-1\end{pmatrix}$ 
\item $(x+y)^n = \sum_{k=0}^{n}\begin{pmatrix}n\\k\end{pmatrix}x^{n-k}y^{k} = \begin{pmatrix}n\\0\end{pmatrix}x^n + \begin{pmatrix}n\\1\end{pmatrix}x^{n-1}y+ \begin{pmatrix}n\\2\end{pmatrix}x^{n-2}y^2+ \ldots +\begin{pmatrix}n\\n-1\end{pmatrix}xy^{n-1} + \begin{pmatrix}n\\n\end{pmatrix}y^n$
\item $\sum_{k=0}^{n}\begin{pmatrix}n\\k\end{pmatrix} = 2^{n}$
\end{itemize}  
\end{defn}


\newpage
\subsection*{Exercice 3}
Pour que $\sum{c_nz^n}$ converge, il suffit de montrer, par le crit\`ere d'Abel, que $\exists M, \forall n, |\sum_{k=0}^{n}{z^k}| \leq M$. 
On a 
$$|\sum_{k=0}^{n}{z^k}| = \left|1.\frac{1-z^{n+1}}{1-z}\right| \leq \frac{1 + |z^{n+1}|}{|1-z|} < \frac{2}{|1-z|} $$
car $|z| \leq 1$ et $|z^{n}| \leq 1$. On a trouv\'e un $M = \frac{2}{1-|z|}$, ce qui permet de montrer que $\sum{c_nz^n}$ converge. 


\subsection*{Exercice 4}
\subsubsection*{Exercice 4.1.a}
Calculons
$$
\frac{v_n}{u_n} = \frac{\frac{(-1)^n}{\sqrt{n}}}{\frac{(-1)^n}{\sqrt{n}-(-1)^n}} = \frac{\sqrt{n}-(-1)^n}{\sqrt{n}} = 1 - \frac{(-1)^n}{\sqrt{n}}
$$
On a $\lim_{n \to \infty} \frac{v_n}{u_n} = 1$ donc $u_n ~_{n\to \infty} v_n$ 

\subsubsection*{Exercice 4.1.b}
$v_n = \frac{(-1)^n}{\sqrt{n}}$ converge? \\

\begin{enumerate}
\item Y a-t-il Convergence absolue? $\sum{|\frac{(-1)^n}{\sqrt{n}}}| = \sum{\left|\frac{1}{\sqrt{n}}\right|}$ Cette suite diverge. Donc il n'y a pas de convergence absolue.
\item Cas Special S\'erie Altern\'ee? La s\'erie est altern\'ee car $(-1)^n$ est altern\'ee et $\frac{1}{\sqrt{n}}$ est positif. Il faut montrer que $\frac{1}{\sqrt{n}}$ converge vers 0. Ce qui est vrai quand $n \to \infty$. Donc, la s\'erie de terme g\'en\'eral $v_n = \frac{(-1)^n}{\sqrt{n}}$ converge.
\end{enumerate} 

\subsubsection*{Exercice 4.2}

\subsubsection*{Exercice 4.3}

\subsubsection*{Exercice 4.4}

\subsection*{Exercice 5}
\subsubsection*{Exercice 5.1}
$$u_n - \frac{(-1)^n}{n} = \frac{1}{\ln(n)+(-1)^{n}n} - \frac{(-1)^n}{n} = \frac{n}{n(\ln(n)+(-1)^{n}n)} - \frac{(-1)^n(\ln(n)+(-1)^{n}n)}{n(\ln(n)+(-1)^{n}n)}$$
$$=\frac{n-((-1)^n(\ln(n)+(-1)^{n}n))}{n(\ln(n)+(-1)^{n}n)} = \frac{n-(-1)^n\ln(n)-(-1)^{n}(-1)^{n}n}{n(\ln(n)+(-1)^{n}n)} = \frac{-(-1)^n\ln(n)}{n(\ln(n)+(-1)^{n}n)}$$
$$=\frac{-\ln(n)}{(-1)^{n}n\ln(n)+n^2)} = \frac{\ln(n)}{n}\frac{-1}{(-1)^{n}\ln(n)+n}$$
??

\subsubsection*{Exercice 5.2}
On a 
$$u_n = \left( u_n - \frac{(-1)^n}{n} \right) - \frac{(-1)^n}{n}$$
Avec $u_n - \frac{(-1)^n}{n}$ qui converge et $\frac{(-1)^n}{n}$ qui converge aussi (C.S.S.A avec $v_n = \frac{1}{n}$). Donc la s\'erie de terme g\'en\'eral $u_n$ converge (somme de 2 s\'eries qui convergent). 


\subsection*{Exercice 6}
\subsubsection*{Exercice 6.a}
$$a_n = \sum_{k=n}^{2n}\frac{1}{n+k} = \sum_{k=n}^{2n}\frac{1}{n}\frac{1}{1+\frac{k}{n}}$$
Prenons $x=\frac{k}{n}$, on a $dx=\frac{1}{n}$ donc
$$\sum_{k=n}^{2n}\frac{1}{n}\frac{1}{1+\frac{k}{n}} = \int_{1}^{2} \frac{1}{1+x}dx = \left[\ln(|1+x|\right]_{1}^{2} = \ln(3)-\ln(2) = \ln(\frac{3}{2})$$

\subsubsection*{Exercice 6.b}
$$b_n = \sqrt[n]{\frac{(2n)!}{n!n^n}}$$
Calcul de 
$$\ln\left(\sqrt[n]{\frac{(2n)!}{n!n^n}}\right) = \frac{1}{n}\ln\left(\frac{(2n)!}{n!n^n}\right) = \frac{1}{n}(\ln((2n)!)-\ln(n!)-n\ln(n))$$

On a 
$$\ln(n!) = \ln(1*2*3*\ldots*n) = \ln(1)+\ln(2)+\ln(3)+\ldots+\ln(n) = \sum_{k=1}^n\ln(k)$$
$$\ln(2n!) = \ln(1*2*3*\ldots*2n) = \ln(1)+\ln(2)+\ln(3)+\ldots+\ln(2n) = \sum_{k=1}^{2n}\ln(k)$$
Donc
$$\ln((2n)!)-\ln(n!) = \sum_{k=n+1}^{2n}\ln(k)$$
$$\ln((2n)!)-\ln(n!)-n\ln(n) = \sum_{k=n+1}^{2n}\ln(k) - \sum_{k=1}^{n}{\ln(n)} = \sum_{k=n+1}^{2n}\ln(k) - \sum_{k=n+1}^{2n}{\ln(n)} = \sum_{k=n+1}^{2n}{\ln(k)-\ln(n)} = \sum_{k=n+1}^{2n}{\ln\left(\frac{k}{n}\right)}$$
donc
$$\frac{1}{n}(\ln((2n)!)-\ln(n!)-n\ln(n)) = \frac{1}{n} \sum_{k=n+1}^{2n}{\ln\left(\frac{k}{n}\right)} = \int_{\frac{n+1}{n}}^{2}{\ln(x)} = \left[ x(\ln(x)-1)\right]_{\frac{n+1}{n}}^{2}$$

\subsection*{Exercice 7}
\subsubsection*{Exercice 7.a}
$$a_n = \sum_{k=0}^{n}\frac{1}{n+k} = \sum_{k=0}^{n}\frac{1}{n}\frac{1}{1+\frac{k}{n}}$$
Prenons $x=\frac{k}{n}$, on a $dx=\frac{1}{n}$ donc
$$\sum_{k=0}^{n}\frac{1}{n}\frac{1}{1+\frac{k}{n}} = \int_{0}^{1} \frac{1}{1+x}dx = \left[\ln(|1+x|\right]_{0}^{1} = \ln(2)-\ln(1) = \ln(2)$$

\subsubsection*{Exercice 7.b}
$$b_n = \sum_{k=0}^{n}\frac{n}{n^2+k^2} = \sum_{k=0}^{n}\frac{n}{n^2}\frac{1}{1+\frac{k^2}{n^2}}$$
Prenons $x=\frac{k}{n}$, on a $dx=\frac{1}{n}$ donc
$$\sum_{k=0}^{n}\frac{1}{n}\frac{1}{1+\left(\frac{k}{n}\right)^2} = \int_{0}^{1} \frac{1}{1+x^2}dx = \left[\arctan(x)\right]_{0}^{1} = \arctan(1)-\arctan(0) = \arctan(1)$$

\subsubsection*{Exercice 7.c}
$$c_n = \frac{1}{n^2}\prod_{k=1}^{n}{(n^2+k^2)^{1/n}}$$


\subsection*{Exercice 8}
\subsubsection*{Exercice 8.a}
$$k\begin{pmatrix}n\\k\end{pmatrix} = k\frac{n!}{k!(n-k)!} = k\frac{n!}{k(k-1)!(n-k)!}= \frac{n(n-1)!}{(k-1)!(n-k)!} = n\frac{(n-1)!}{(k-1)!((n-1)-(k-1))!}
= n \begin{pmatrix}n-1\\k-1\end{pmatrix}$$

\subsubsection*{Exercice 8.b}
$$k(k-1)\begin{pmatrix}n\\k\end{pmatrix} = k(k-1)\frac{n!}{k!(n-k)!} = k(k-1)\frac{n!}{k(k-1)(k-2)!((n-2)-(k-2))!} $$
$$= \frac{n(n-1)(n-2)!}{(k-2)!((n-2)-(k-2))!} = n(n-1)\begin{pmatrix}n-2\\k-2\end{pmatrix}$$

\subsection*{Exercice 9}
\subsubsection*{Exercice 9.1}
On sait que $(x+y)^n = \sum_{k=0}^{n}{\begin{pmatrix}n\\k\end{pmatrix}x^ky^{n-k}}$. En prenant $y = 1-x$, on a
$$\sum_{k=0}^{n}{\begin{pmatrix}n\\k\end{pmatrix}x^k(1-x)^{n-k}} = (x+(1-x))^n = 1^n = 1$$

\subsubsection*{Exercice 9.2}
Le premier terme est toujours \'egal \`a 0.
$$E(X) = \sum_{k=0}^{n}{kp(X=k)} = \sum_{k=1}^{n}{kp(X=k)} = \sum_{k=1}^{n}{k\begin{pmatrix}n\\k\end{pmatrix}p^k(1-p)^{n-k}} = \sum_{k=1}^{n}{n\begin{pmatrix}n-1\\k-1\end{pmatrix}p^k(1-p)^{n-k}} =$$
$$=\sum_{k=1}^{n}{n\begin{pmatrix}n-1\\k-1\end{pmatrix}pp^{(k-1)}(1-p)^{n-k}} = np\sum_{k=1}^{n}{\begin{pmatrix}n-1\\k-1\end{pmatrix}p^{(k-1)}(1-p)^{(n-1)-(k-1)}}$$
Changement de variables $j=k-1$ et $m=n-1$ on a 
$$np\sum_{j=0}^{m}{\begin{pmatrix}m\\j\end{pmatrix}p^{j}(1-p)^{(m-j)}} = np.1 = np$$


\subsection*{Exercice 10}
$$V(X) = \sum_{k=0}^{n}{(k-np)^2P(X=k)} = \sum_{k=0}^{n}{(k^2-2knp+n^2p^2)P(X=k)} $$
$$= \sum_{k=0}^{n}{k^2P(X=k)}-2np\sum_{k=0}^{n}{kP(X=k)}+n^2p^2\sum_{k=0}^{n}{P(X=k)} = \sum_{k=0}^{n}{k^2P(X=k)}-2npnp+n^2p^2 = \sum_{k=0}^{n}{k^2P(X=k)}-n^2p^2$$
On repart de 
$$ \sum_{k=0}^{n}{k^2P(X=k)} =  \sum_{k=0}^{n}{k.k.P(X=k)} = \sum_{k=0}^{n}{k.k.\begin{pmatrix}n\\k\end{pmatrix}p^k(1-p)^{n-k}} = \sum_{k=0}^{n}{k.n.\begin{pmatrix}n-1\\k-1\end{pmatrix}p^k(1-p)^{n-k}} $$
$$= \sum_{k=0}^{n}{k.n.\begin{pmatrix}n-1\\k-1\end{pmatrix}pp^{k-1}(1-p)^{(n-1)-(k-1)}} = np\sum_{k=0}^{n}{k.\begin{pmatrix}n-1\\k-1\end{pmatrix}p^{k-1}(1-p)^{(n-1)-(k-1)}}$$
Le premier terme est \'egal \`a 0.
$$np\sum_{k=1}^{n}{k.\begin{pmatrix}n-1\\k-1\end{pmatrix}p^{k-1}(1-p)^{(n-1)-(k-1)}}$$
Changement de variables $j=k-1$, $m=n-1$
$$np\sum_{j=0}^{m}{(j+1).\begin{pmatrix}m\\j\end{pmatrix}p^{j}(1-p)^{m-j}} = np\left(\sum_{j=0}^{m}{j\begin{pmatrix}m\\j\end{pmatrix}p^{j}(1-p)^{m-j}}+\sum_{j=0}^{m}{\begin{pmatrix}m\\j\end{pmatrix}p^{j}(1-p)^{m-j}}\right)$$
$$=np(mp+1) = np((n-1)p+1) = n^2p^2-np^2+np = n^2p^2+np(1-p)$$
Donc
$$V(X)= \sum_{k=0}^{n}{k^2P(X=k)}-n^2p^2 = n^2p^2+np(1-p)-n^2p^2 = np(1-p)$$

\subsection*{Exercice 11}
On a
$$\binom{n+1}{k} = \binom{n}{k} + \binom{n}{k-1}$$

V\'erifions pour $n=1$, $\sum_{k=1}^{1}\frac{(-1)^{k+1}}{k}\begin{pmatrix}1\\k\end{pmatrix} = 1 = \sum_{k=1}^{1} \frac{1}{k}$. Supposons $\sum_{k=1}^{n}\frac{(-1)^{k+1}}{k}\begin{pmatrix}n\\k\end{pmatrix} = \sum_{k=1}^{n}\frac{1}{k}$ vrai au rang $n$, calculons
$$\sum_{k=1}^{n+1}\frac{(-1)^{k+1}}{k}\begin{pmatrix}n+1\\k\end{pmatrix} = \sum_{k=1}^{n+1}\left(\frac{(-1)^{k+1}}{k}\binom{n}{k}+\binom{n}{k-1}\right)$$
$$\sum_{k=1}^{n+1}\frac{(-1)^{k+1}}{k}\binom{n}{k}+\sum_{k=1}^{n+1}\frac{(-1)^{k+1}}{k}\binom{n}{k-1} $$
Premiere partie
$$\sum_{k=1}^{n+1}\frac{(-1)^{k+1}}{k}\binom{n}{k} = \sum_{k=1}^{n}\frac{(-1)^{k+1}}{k}\binom{n}{k} + \frac{(-1)^{k+2}}{n+1}\binom{n}{n+1} = \sum_{k=1}^{n}\frac{(-1)^{k+1}}{k}\binom{n}{k} = \sum_{k=1}^{n}\frac{1}{k}$$
Car $\binom{n}{n+1} = 0$ par d\'efinition et par hypoth\`ese de r\'ecurrence.\\

Seconde partie, commencons par calculer
$$\sum_{k=1}^{n+1}\frac{(-1)^{k+1}}{k}\binom{n}{k-1} = \sum_{k=1}^{n+1}\frac{(-1)^{k+1}}{k}\binom{(n+1)-1}{k-1} = \sum_{k=1}^{n+1}\frac{(-1)^{k+1}}{n+1}\binom{n+1}{k} = \frac{1}{n+1}\sum_{k=1}^{n+1}(-1)^{k+1}\binom{n+1}{k}$$

On a $(x+y)^n = \sum_{k=0}^{n}x^ny^{n-k}\binom{n}{k}$, en prenant $x=-1$ et $y=1$ on a 
$$((-1)+1)^n = \sum_{k=0}^{n}(-1)^k 1^{n-k}\binom{n}{k} = \sum_{k=0}^{n}(-1)^{k}\binom{n}{k} = 1 + \sum_{k=1}^{n}(-1)^{k}\binom{n}{k}$$
Donc $\sum_{k=1}^{n}(-1)^{k}\binom{n}{k} = -1$ ce qui fait $\sum_{k=1}^{n}(-1)^{k+1}\binom{n}{k} = 1$
On a pour finir
$$\sum_{k=1}^{n+1}\frac{(-1)^{k+1}}{k}\binom{n}{k}+\sum_{k=1}^{n+1}\frac{(-1)^{k+1}}{k}\binom{n}{k-1} = \sum_{k=1}^{n}\frac{1}{k} + \frac{1}{n+1} = \sum_{k=1}^{n+1}\frac{1}{k}$$
Fini.

\subsection*{Exercice 12}
$$\sum_{k=p}^{n+1}\binom{k}{p} = \sum_{k=p}^{n}\binom{k}{p} + \binom{n+1}{p} = \binom{n+1}{p+1} + \binom{n+1}{p}$$
On a 
$$\binom{n+1}{p+1} = \frac{(n+1)!}{(p+1)!(n-p)!} = \frac{(n+1)!(n-p+1)}{(p+1)!(n-p+1)!}$$
$$\binom{n+1}{p} = \frac{(n+1)!}{(p)!(n+1-p)!} = \frac{(n+1)!(p+1)}{(p+1)!(n+1-p)!}$$
Donc
$$\binom{n+1}{p+1} + \binom{n+1}{p} = \frac{(n+1)!(n+1-p)}{(p+1)!(n+1-p)!} + \frac{(n+1)!(p+1)}{(p+1)!(n+1-p)!} = \frac{(n+1)!(n-p+1+p+1)}{(p+1)!(n+1-p)!}$$
$$= \frac{(n+2)!}{(p+1)!((n+2)-(p+1))!} = \binom{n+2}{p+1}$$
 
\subsection*{Exercice 13}
\subsection*{Exercice 13.1}$$\sum_{k=0}^{m}\binom{2m}{2k} - \sum_{k=0}^{m-1}\binom{2m}{2k+1} = \binom{2m}{0} + \binom{2m}{2} + \ldots + \binom{2m}{2m} - (\binom{2m}{1} + \binom{2m}{3} + \ \ldots + \binom{2m}{2m-1}) $$
$$\binom{2m}{0} - \binom{2m}{1} + \binom{2m}{2} - \binom{2m}{3} \ldots -\binom{2m}{2m-1} + \binom{2m}{2m}=  \sum_{k=0}^{2m} (-1)^k\binom{2m}{k} = 0$$

Pour $n=2m$ ($n$ pair), on a 
$$S_n = \sum_{k=0}^{m}\binom{2m}{2k}, T_n = \sum_{k=0}^{m-1}\binom{2m}{2k+1}$$

Calculons $S_n - T_n$
$$\sum_{k=0}^{m}\binom{2m}{2k} - \sum_{k=0}^{m-1}\binom{2m}{2k+1} = \binom{2m}{0} + \binom{2m}{2} + \ldots + \binom{2m}{2m} - \left(\binom{2m}{1} + \binom{2m}{3} + \ \ldots + \binom{2m}{2m-1}\right) $$
$$\binom{2m}{0} - \binom{2m}{1} + \binom{2m}{2} - \binom{2m}{3} \ldots -\binom{2m}{2m-1} + \binom{2m}{2m}=  \sum_{k=0}^{2m} (-1)^k\binom{2m}{k} = 0$$

Pour $n=2m+1$ ($n$ impair) on a 
$$S_n = \sum_{k=0}^{m}\binom{2m+1}{2k}, T_n = \sum_{k=0}^{m}\binom{2m+1}{2k+1}$$
Calculons $S_n - T_n$
$$\sum_{k=0}^{m}\binom{m}{2k} - \sum_{k=0}^{m}\binom{2m+1}{2k+1} = \binom{2m+1}{0} + \binom{2m+1}{2} + \ldots + \binom{2m+1}{2m} - \left(\binom{2m+1}{1} + \binom{2m+1}{3} + \ \ldots + \binom{2m+1}{2m+1}\right) $$
$$\binom{2m+1}{0} - \binom{2m+1}{1} + \binom{2m+1}{2} - \binom{2m+1}{3} \ldots + \binom{2m+1}{2m} -\binom{2m+1}{2m+1}=  \sum_{k=0}^{2m+1} (-1)^k\binom{2m+1}{k} = 0$$


\subsection*{Exercice 13.2}
Calculons $S_n + T_n$ pour $n=2m$

$$\sum_{k=0}^{m}\binom{2m}{2k} + \sum_{k=0}^{m-1}\binom{2m}{2k+1} = \binom{2m}{0} + \binom{2m}{2} + \ldots + \binom{2m}{2m} + \left(\binom{2m}{1} + \binom{2m}{3} + \ \ldots + \binom{2m}{2m-1}\right) $$
$$\binom{2m}{0} + \binom{2m}{1} + \binom{2m}{2} + \binom{2m}{3} \ldots + \binom{2m}{2m-1} + \binom{2m}{2m}=  \sum_{k=0}^{2m} \binom{2m}{k} = 2^{2m} = 2^n$$
car $(x+y)^n = \sum_{k=0}^{n}\binom{n}{k}x^ny^{n-k}$ et en prenant $x=y=1$ on a 
$$(1+1)^n = \sum_{k=0}^{n}\binom{n}{k}1^n1^{n-k} = \sum_{k=0}^{n}\binom{n}{k}$$

Calculons $S_n + T_n$ pour $n=2m+1$
$$\sum_{k=0}^{m}\binom{m}{2k} + \sum_{k=0}^{m}\binom{2m+1}{2k+1} = \binom{2m+1}{0} + \binom{2m+1}{2} + \ldots + \binom{2m+1}{2m} + \left(\binom{2m+1}{1} + \binom{2m+1}{3} + \ \ldots + \binom{2m+1}{2m+1}\right) $$
$$\binom{2m+1}{0} + \binom{2m+1}{1} + \binom{2m+1}{2} + \binom{2m+1}{3} \ldots + \binom{2m+1}{2m} +\binom{2m+1}{2m+1}=  \sum_{k=0}^{2m+1} \binom{2m+1}{k} = 2^{2m+1} = 2^{n}$$

\subsection*{Exercice 13.3}
On a $S_n - T_n = 0$ et $S_n + T_n = 2^n$, donc $2T_n = 2^n$, $T_n = 2^{n-1}$ et $S_n = T_n = 2^{n-1}$.\\


QED

\end{document}

