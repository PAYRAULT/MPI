\documentclass[]{book}

%These tell TeX which packages to use.
\usepackage{array,epsfig}
\usepackage{amsmath}
\usepackage{amsfonts}
\usepackage{amssymb}
\usepackage{amsxtra}
\usepackage{amsthm}
\usepackage{mathrsfs}
\usepackage{color}

%Here I define some theorem styles and shortcut commands for symbols I use often
\theoremstyle{definition}
\newtheorem{defn}{Definition}
\newtheorem{thm}{Theorem}
\newtheorem{cor}{Corollary}
\newtheorem*{rmk}{Remark}
\newtheorem{lem}{Lemma}
\newtheorem*{joke}{Joke}
\newtheorem{ex}{Example}
\newtheorem*{soln}{Solution}
\newtheorem{prop}{Proposition}

\newcommand{\lra}{\longrightarrow}
\newcommand{\ra}{\rightarrow}
\newcommand{\surj}{\twoheadrightarrow}
\newcommand{\graph}{\mathrm{graph}}
\newcommand{\bb}[1]{\mathbb{#1}}
\newcommand{\Z}{\bb{Z}}
\newcommand{\Q}{\bb{Q}}
\newcommand{\R}{\bb{R}}
\newcommand{\C}{\bb{C}}
\newcommand{\N}{\bb{N}}
\newcommand{\M}{\mathbf{M}}
\newcommand{\m}{\mathbf{m}}
\newcommand{\MM}{\mathscr{M}}
\newcommand{\HH}{\mathscr{H}}
\newcommand{\Om}{\Omega}
\newcommand{\Ho}{\in\HH(\Om)}
\newcommand{\bd}{\partial}
\newcommand{\del}{\partial}
\newcommand{\bardel}{\overline\partial}
\newcommand{\textdf}[1]{\textbf{\textsf{#1}}\index{#1}}
\newcommand{\img}{\mathrm{img}}
\newcommand{\ip}[2]{\left\langle{#1},{#2}\right\rangle}
\newcommand{\inter}[1]{\mathrm{int}{#1}}
\newcommand{\exter}[1]{\mathrm{ext}{#1}}
\newcommand{\cl}[1]{\mathrm{cl}{#1}}
\newcommand{\ds}{\displaystyle}
\newcommand{\vol}{\mathrm{vol}}
\newcommand{\cnt}{\mathrm{ct}}
\newcommand{\osc}{\mathrm{osc}}
\newcommand{\LL}{\mathbf{L}}
\newcommand{\UU}{\mathbf{U}}
\newcommand{\support}{\mathrm{support}}
\newcommand{\AND}{\;\wedge\;}
\newcommand{\OR}{\;\vee\;}
\newcommand{\Oset}{\varnothing}
\newcommand{\st}{\ni}
\newcommand{\wh}{\widehat}

%Pagination stuff.
\setlength{\topmargin}{-.3 in}
\setlength{\oddsidemargin}{0in}
\setlength{\evensidemargin}{0in}
\setlength{\textheight}{9.in}
\setlength{\textwidth}{6.5in}
\pagestyle{empty}



\begin{document}

\subsection*{Rappel de cours}
\begin{defn}
Deux suites $(u_n)_{n\geq0}$ et $(v_n)_{n\geq0}$ sont adjacentes ssi:
\begin{itemize}
\item $(u_n)_{n\geq0}$ est croisssante et $(v_n)_{n\geq0}$ est d\'ecroissante
\item $\forall n \in \N, u_n \leq v_n$
\item $\lim_{n \to \infty}(v_n-u_n)_{n\geq0} = 0$
\end{itemize}  
\end{defn}


\newpage
\subsection*{Exercice 3}
Pour que $\sum{c_nz^n}$ converge, il suffit de montrer, par le crit\`ere d'Abel, que $\exists M, \forall n, |\sum_{k=0}^{n}{z^k}| \leq M$. 
On a 
$$|\sum_{k=0}^{n}{z^k}| = \left|1.\frac{1-z^{n+1}}{1-z}\right| \leq \frac{1 + |z^{n+1}|}{|1-z|} < \frac{2}{|1-z|} $$
car $|z| \leq 1$ et $|z^{n}| \leq 1$. On a trouv\'e un $M = \frac{2}{1-|z|}$, ce qui permet de montrer que $\sum{c_nz^n}$ converge. 


\subsection*{Exercice 4}
\subsubsection*{Exercice 4.1.a}
Calculons
$$
\frac{v_n}{u_n} = \frac{\frac{(-1)^n}{\sqrt{n}}}{\frac{(-1)^n}{\sqrt{n}-(-1)^n}} = \frac{\sqrt{n}-(-1)^n}{\sqrt{n}} = 1 - \frac{(-1)^n}{\sqrt{n}}
$$
On a $\lim_{n \to \infty} \frac{v_n}{u_n} = 1$ donc $u_n ~_{n\to \infty} v_n$ 

\subsubsection*{Exercice 4.1.b}
$v_n = \frac{(-1)^n}{\sqrt{n}}$ converge? \\

\begin{enumerate}
\item Y a-t-il Convergence absolue? $\sum{|\frac{(-1)^n}{\sqrt{n}}}| = \sum{\left|\frac{1}{\sqrt{n}}\right|}$ Cette suite diverge. Donc il n'y a pas de convergence absolue.
\item Cas Special S\'erie Altern\'ee? La s\'erie est altern\'ee car $(-1)^n$ est altern\'ee et $\frac{1}{\sqrt{n}}$ est positif. Il faut montrer que $\frac{1}{\sqrt{n}}$ converge vers 0. Ce qui est vrai quand $n \to \infty$. Donc, la s\'erie de terme g\'en\'eral $v_n = \frac{(-1)^n}{\sqrt{n}}$ converge.
\end{enumerate} 

\subsubsection*{Exercice 4.2}

\subsubsection*{Exercice 4.3}

\subsubsection*{Exercice 4.4}

\subsection*{Exercice 5}
\subsubsection*{Exercice 5.1}
$$u_n - \frac{(-1)^n}{n} = \frac{1}{\ln(n)+(-1)^{n}n} - \frac{(-1)^n}{n} = \frac{n}{n(\ln(n)+(-1)^{n}n)} - \frac{(-1)^n(\ln(n)+(-1)^{n}n)}{n(\ln(n)+(-1)^{n}n)}$$
$$=\frac{n-((-1)^n(\ln(n)+(-1)^{n}n))}{n(\ln(n)+(-1)^{n}n)} = \frac{n-(-1)^n\ln(n)-(-1)^{n}(-1)^{n}n}{n(\ln(n)+(-1)^{n}n)} = \frac{-(-1)^n\ln(n)}{n(\ln(n)+(-1)^{n}n)}$$
$$=\frac{-\ln(n)}{(-1)^{n}n\ln(n)+n^2)} = \frac{\ln(n)}{n}\frac{-1}{(-1)^{n}\ln(n)+n}$$
??

\subsubsection*{Exercice 5.2}
On a 
$$u_n = \left( u_n - \frac{(-1)^n}{n} \right) - \frac{(-1)^n}{n}$$
Avec $u_n - \frac{(-1)^n}{n}$ qui converge et $\frac{(-1)^n}{n}$ qui converge aussi (C.S.S.A avec $v_n = \frac{1}{n}$). Donc la s\'erie de terme g\'en\'eral $u_n$ converge (somme de 2 s\'eries qui convergent). 


\subsection*{Exercice 6}
\subsubsection*{Exercice 6.a}
$$a_n = \sum_{k=n}^{2n}\frac{1}{n+k} = \sum_{k=n}^{2n}\frac{1}{n}\frac{1}{1+\frac{k}{n}}$$
Prenons $x=\frac{k}{n}$, on a $dx=\frac{1}{n}$ donc
$$\sum_{k=n}^{2n}\frac{1}{n}\frac{1}{1+\frac{k}{n}} = \int_{1}^{2} \frac{1}{1+x}dx = \left[\ln(|1+x|\right]_{1}^{2} = \ln(3)-\ln(2) = \ln(\frac{3}{2})$$

\subsubsection*{Exercice 6.b}
$$b_n = \sqrt[n]{\frac{(2n)!}{n!n^n}}$$


\subsection*{Exercice 7}
\subsubsection*{Exercice 7.a}
$$a_n = \sum_{k=0}^{n}\frac{1}{n+k} = \sum_{k=0}^{n}\frac{1}{n}\frac{1}{1+\frac{k}{n}}$$
Prenons $x=\frac{k}{n}$, on a $dx=\frac{1}{n}$ donc
$$\sum_{k=0}^{n}\frac{1}{n}\frac{1}{1+\frac{k}{n}} = \int_{0}^{1} \frac{1}{1+x}dx = \left[\ln(|1+x|\right]_{0}^{1} = \ln(2)-\ln(1) = \ln(2)$$

\subsubsection*{Exercice 7.b}
$$b_n = \sum_{k=0}^{n}\frac{n}{n^2+k^2} = \sum_{k=0}^{n}\frac{n}{n^2}\frac{1}{1+\frac{k^2}{n^2}}$$
Prenons $x=\frac{k}{n}$, on a $dx=\frac{1}{n}$ donc
$$\sum_{k=0}^{n}\frac{1}{n}\frac{1}{1+\left(\frac{k}{n}\right)^2} = \int_{0}^{1} \frac{1}{1+x^2}dx = \left[\arctan(x)\right]_{0}^{1} = \arctan(1)-\arctan(0) = \arctan(1)$$

\subsubsection*{Exercice 7.c}
$$c_n = \frac{1}{n^2}\prod_{k=1}^{n}{(n^2+k^2)^{1/n}}$$

QED

\end{document}

