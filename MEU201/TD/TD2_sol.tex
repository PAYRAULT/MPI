\documentclass[]{book}

%These tell TeX which packages to use.
\usepackage{array,epsfig}
\usepackage{amsmath}
\usepackage{amsfonts}
\usepackage{amssymb}
\usepackage{amsxtra}
\usepackage{amsthm}
\usepackage{mathrsfs}
\usepackage{color}

%Here I define some theorem styles and shortcut commands for symbols I use often
\theoremstyle{definition}
\newtheorem{defn}{Definition}
\newtheorem{thm}{Theorem}
\newtheorem{cor}{Corollary}
\newtheorem*{rmk}{Remark}
\newtheorem{lem}{Lemma}
\newtheorem*{joke}{Joke}
\newtheorem{ex}{Example}
\newtheorem*{soln}{Solution}
\newtheorem{prop}{Proposition}

\newcommand{\lra}{\longrightarrow}
\newcommand{\ra}{\rightarrow}
\newcommand{\surj}{\twoheadrightarrow}
\newcommand{\graph}{\mathrm{graph}}
\newcommand{\bb}[1]{\mathbb{#1}}
\newcommand{\Z}{\bb{Z}}
\newcommand{\Q}{\bb{Q}}
\newcommand{\R}{\bb{R}}
\newcommand{\C}{\bb{C}}
\newcommand{\N}{\bb{N}}
\newcommand{\M}{\mathbf{M}}
\newcommand{\m}{\mathbf{m}}
\newcommand{\MM}{\mathscr{M}}
\newcommand{\HH}{\mathscr{H}}
\newcommand{\Om}{\Omega}
\newcommand{\Ho}{\in\HH(\Om)}
\newcommand{\bd}{\partial}
\newcommand{\del}{\partial}
\newcommand{\bardel}{\overline\partial}
\newcommand{\textdf}[1]{\textbf{\textsf{#1}}\index{#1}}
\newcommand{\img}{\mathrm{img}}
\newcommand{\ip}[2]{\left\langle{#1},{#2}\right\rangle}
\newcommand{\inter}[1]{\mathrm{int}{#1}}
\newcommand{\exter}[1]{\mathrm{ext}{#1}}
\newcommand{\cl}[1]{\mathrm{cl}{#1}}
\newcommand{\ds}{\displaystyle}
\newcommand{\vol}{\mathrm{vol}}
\newcommand{\cnt}{\mathrm{ct}}
\newcommand{\osc}{\mathrm{osc}}
\newcommand{\LL}{\mathbf{L}}
\newcommand{\UU}{\mathbf{U}}
\newcommand{\support}{\mathrm{support}}
\newcommand{\AND}{\;\wedge\;}
\newcommand{\OR}{\;\vee\;}
\newcommand{\Oset}{\varnothing}
\newcommand{\st}{\ni}
\newcommand{\wh}{\widehat}

%Pagination stuff.
\setlength{\topmargin}{-.3 in}
\setlength{\oddsidemargin}{0in}
\setlength{\evensidemargin}{0in}
\setlength{\textheight}{9.in}
\setlength{\textwidth}{6.5in}
\pagestyle{empty}



\begin{document}

\subsection*{Rappel de cours}
\begin{defn}
Deux suites $(u_n)_{n\geq0}$ et $(v_n)_{n\geq0}$ sont adjacentes ssi:
\begin{itemize}
\item $(u_n)_{n\geq0}$ est croisssante et $(v_n)_{n\geq0}$ est d\'ecroissante
\item $\forall n \in \N, u_n \leq v_n$
\item $\lim_{n \to \infty}(v_n-u_n)_{n\geq0} = 0$
\end{itemize}  
\end{defn}


\newpage
\subsection*{Exercice 1}
Montrons que $(u_n)_{n\geq1}$ est croissante. 
$$u_{n+1} - u_{n} = \sum_{k=1}^{n+1}\frac{1}{k^3} - \sum_{k=1}^{n}\frac{1}{k^3} = \sum_{k=1}^{n}\frac{1}{k^3} + \frac{1}{(n+1)^3} - \sum_{k=1}^{n}\frac{1}{k^3} =  \frac{1}{(n+1)^3}$$
$\forall n, u_{n+1} - u_{n} > 0$ donc la suite $(u_n)_{n\geq1}$ est croisssante.\\

Montrons que $(v_n)_{n\geq1}$ est d\'croissante. 
$$v_{n+1} - v_{n} = u_{n+1}+\frac{1}{(n+1)^2} - (u_{n}+\frac{1}{n^2}) = \frac{1}{(n+1)^2} + \frac{1}{(n+1)^3} - \frac{1}{n^2} = \frac{n+2}{(n+1)^3} - \frac{1}{n^2}$$
$$\frac{-3n^2-n-1}{n^2(n+1)^3}$$
$\forall n, v_{n+1} - v_{n} < 0$ donc la suite $(v_n)_{n\geq1}$ est d\'ecroisssante.\\

Montrons que $\forall n \in \N, n \geq 1, u_n \leq v_n$
$$u_n \leq v_n, u_n \leq u_n + \frac{1}{n^2}$$
Vrai car $\frac{1}{n^2}$ est positif pour $n \geq 1$.\\

Montrons que $\lim_{n \to \infty}(v_n-u_n)_{n\geq0} = 0$
$$\lim_{n \to \infty}(v_n-u_n)_{n\geq1} = \lim_{n \to \infty}(u_n+\frac{1}{n^2}-u_n)_{n\geq1} = \lim_{n \to \infty}\frac{1}{n^2} = 0$$
Vrai\\

Donc les deux suites $(u_n)_{n\geq1}$ et $(v_n)_{n\geq1}$ sont adjacentes.


\subsection*{Exercice 2}
Montrons que $(u_n)_{n\geq1}$ est croissante. 
$$u_{n+1} - u_{n} = \sum_{k=1}^{n+1}\frac{1}{k!} - \sum_{k=1}^{n}\frac{1}{k!} = \sum_{k=1}^{n}\frac{1}{k!} + \frac{1}{(n+1)!} - \sum_{k=1}^{n}\frac{1}{k!} =  \frac{1}{(n+1)!}$$
$\forall n, u_{n+1} - u_{n} > 0$ donc la suite $(u_n)_{n\geq1}$ est croisssante.\\

Montrons que $(v_n)_{n\geq1}$ est d\'croissante. 
$$v_{n+1} - v_{n} = u_{n+1}+\frac{1}{(n+1)!} - (u_{n}+\frac{1}{n!}) = \frac{1}{(n+1)!} + \frac{1}{(n+1)!} - \frac{1}{n!} = \frac{2}{(n+1)!} - \frac{1}{n!}$$
$$\frac{1-n}{(n+1)!} $$
$\forall n, v_{n+1} - v_{n} \leq 0$ donc la suite $(v_n)_{n\geq1}$ est d\'ecroisssante.\\

Montrons que $\forall n \in \N, n \geq 1, u_n \leq v_n$
$$u_n \leq v_n, u_n \leq u_n + \frac{1}{n!}$$
Vrai car $\frac{1}{n!}$ est positif pour $n \geq 1$.\\

Montrons que $\lim_{n \to \infty}(v_n-u_n)_{n\geq0} = 0$
$$\lim_{n \to \infty}(v_n-u_n)_{n\geq1} = \lim_{n \to \infty}(u_n+\frac{1}{n!}-u_n)_{n\geq1} = \lim_{n \to \infty}\frac{1}{n!} = 0$$
Vrai\\

Donc les deux suites $(u_n)_{n\geq1}$ et $(v_n)_{n\geq1}$ sont adjacentes.


\subsection*{Exercice 3}
\subsubsection*{$3.1$}
$\forall n\geq 1, u_n > 0$ car $u_n$ est une somme de nombres tous positifs.\\
$\forall n\geq 1, u_n \leq 0$, ???


\subsubsection*{$3.2$}
Montrons que $(u_n)_{n\geq1}$ est croissante. 
$$u_{n+1} - u_{n} = \sum_{k=1}^{n+1}\frac{1}{k+n} - \sum_{k=1}^{n}\frac{1}{k+n} = \frac{1}{n+1+n}$$ 
$\forall n, u_{n+1} - u_{n} > 0$ donc la suite $(u_n)_{n\geq1}$ est croisssante.\\


\subsubsection*{$3.3$}
La suite $u_n$ est born\'e et croissante donc elle converge. Calculons sa limite.
$$\lim_{n\to\infty} u_n = \lim_{n\to\infty} \sum_{k=1}^{n} \frac{1}{n+k} =  \lim_{n\to\infty} (\sum_{k=1}^{2n}\frac{1}{k} - \sum_{k=1}^{n} \frac{1}{k}) = \lim_{n\to\infty} \sum_{k=1}^{2n}\frac{1}{k} - \lim_{n\to\infty} \sum_{k=1}^{n} \frac{1}{k} = \ln(2n) - \ln(n) = \ln(\frac{2n}{n}) = \ln(2)$$


\subsection*{Exercice 4}
\subsubsection*{$4.1$}
$$u_{n+1}-u_{n} = \sum_{k=1}^{n+1}\frac{k}{k^2+1} - \sum_{k=1}^{n}\frac{k}{k^2+1} = \frac{n+1}{(n+1)^2+1}$$
$u_{n+1}-n_{n} > 0$ donc la suite $u_n$ est croissante.

\subsubsection*{$4.2$}
$$u_{2n}-n_{n} = \sum_{k=1}^{2n}\frac{k}{k^2+1} - \sum_{k=1}^{n}\frac{k}{k^2+1} = \sum_{k=1}^{n}{\frac{n+k}{(n+k)^2+1}}$$
Preuve pae r\'ecurrence, pour $n=1$, on a $u_2-u_1=2/5=0.4$. Supposons que $u_{2n}-u_{n} \geq 1/4$, que vaut $u_{2(n+1)} - u_{(n+1)}$?
$$u_{2(n+1)} - u_{(n+1)} = \sum_{k=1}^{n+1}{\frac{n+k}{(n+1+k)^2+1}} = \sum_{k=1}^{n}{\frac{n+k}{(n+1+k)^2+1}} + \frac{n+1}{(2n+2)^2+1} = u_n + \frac{n+1}{2(n+1)}$$


QED

\end{document}

