\documentclass[]{book}

%These tell TeX which packages to use.
\usepackage{array,epsfig}
\usepackage{amsmath}
\usepackage{amsfonts}
\usepackage{amssymb}
\usepackage{amsxtra}
\usepackage{amsthm}
\usepackage{mathrsfs}
\usepackage{color}

%Here I define some theorem styles and shortcut commands for symbols I use often
\theoremstyle{definition}
\newtheorem{defn}{Definition}
\newtheorem{thm}{Theorem}
\newtheorem{cor}{Corollary}
\newtheorem*{rmk}{Remark}
\newtheorem{lem}{Lemma}
\newtheorem*{joke}{Joke}
\newtheorem{ex}{Example}
\newtheorem*{soln}{Solution}
\newtheorem{prop}{Proposition}

\newcommand{\lra}{\longrightarrow}
\newcommand{\ra}{\rightarrow}
\newcommand{\surj}{\twoheadrightarrow}
\newcommand{\graph}{\mathrm{graph}}
\newcommand{\bb}[1]{\mathbb{#1}}
\newcommand{\Z}{\bb{Z}}
\newcommand{\Q}{\bb{Q}}
\newcommand{\R}{\bb{R}}
\newcommand{\C}{\bb{C}}
\newcommand{\N}{\bb{N}}
\newcommand{\M}{\mathbf{M}}
\newcommand{\m}{\mathbf{m}}
\newcommand{\MM}{\mathscr{M}}
\newcommand{\HH}{\mathscr{H}}
\newcommand{\Om}{\Omega}
\newcommand{\Ho}{\in\HH(\Om)}
\newcommand{\bd}{\partial}
\newcommand{\del}{\partial}
\newcommand{\bardel}{\overline\partial}
\newcommand{\textdf}[1]{\textbf{\textsf{#1}}\index{#1}}
\newcommand{\img}{\mathrm{img}}
\newcommand{\ip}[2]{\left\langle{#1},{#2}\right\rangle}
\newcommand{\inter}[1]{\mathrm{int}{#1}}
\newcommand{\exter}[1]{\mathrm{ext}{#1}}
\newcommand{\cl}[1]{\mathrm{cl}{#1}}
\newcommand{\ds}{\displaystyle}
\newcommand{\vol}{\mathrm{vol}}
\newcommand{\cnt}{\mathrm{ct}}
\newcommand{\osc}{\mathrm{osc}}
\newcommand{\LL}{\mathbf{L}}
\newcommand{\UU}{\mathbf{U}}
\newcommand{\support}{\mathrm{support}}
\newcommand{\AND}{\;\wedge\;}
\newcommand{\OR}{\;\vee\;}
\newcommand{\Oset}{\varnothing}
\newcommand{\st}{\ni}
\newcommand{\wh}{\widehat}

%Pagination stuff.
\setlength{\topmargin}{-.3 in}
\setlength{\oddsidemargin}{0in}
\setlength{\evensidemargin}{0in}
\setlength{\textheight}{9.in}
\setlength{\textwidth}{6.5in}
\pagestyle{empty}



\begin{document}

\subsection*{Rappel de cours}

\begin{defn}
Soit $f(x)$ et $g(x)$ deux fonctions r\'eelles d\'efinies au voisinage de $+\infty$ avec $g(x)$ qui ne s'annule pas en $+\infty$. Lorsque 	
$$\lim_{x \to +\infty}\frac{f(x)}{g(x)} = 0$$
On \'ecrit
$$f(x) = o_{x \to +\infty}(g(x))$$
Alors on dit que $f(x)$ est n\'egligeable pas rapport \`a $g(x)$.\\
\end{defn}


\begin{thm}
Th\'eor\`eme des croissances compar\'ees : Pour tous r\'eel $\alpha,\beta, \gamma, \lambda >0$
\begin{itemize}
\item Si $\alpha < \beta$ on a $x^{\alpha} = o_{x \to +\infty}(x^{\beta})$
\item on a $1 = o_{x \to +\infty}((\ln x)^{\gamma})$
\item on a $(\ln x)^{\gamma} = o_{x \to +\infty}(x^{\beta})$
\item on a $x^{\beta} = o_{x \to +\infty}(e^{\lambda x^{\alpha}})$
\end{itemize}
\end{thm}


\begin{thm}
On peut g\'en\'eraliser le th\'eor\`eme des croissances compar\'ees aux suites (avec $n$ un entier positif).
\begin{itemize}
\item on a $1 = o_{n \to +\infty}((\ln n)^{\gamma})$
\item on a $(\ln x)^{\gamma} = o_{n \to +\infty}(n^{\beta})$
\item on a $n^{\beta} = o_{n \to +\infty}(e^{\lambda n^{\alpha}})$
\item on a $e^{\lambda n^{\alpha}} = o_{n \to +\infty}(n!)$
\end{itemize}
\end{thm}

\newpage
\subsection*{Exercice Cauchy}



QED

\end{document}

