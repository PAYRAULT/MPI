\documentclass[]{book}

%These tell TeX which packages to use.
\usepackage{array,epsfig}
\usepackage{amsmath}
\usepackage{amsfonts}
\usepackage{amssymb}
\usepackage{amsxtra}
\usepackage{amsthm}
\usepackage{mathrsfs}
\usepackage{color}

%Here I define some theorem styles and shortcut commands for symbols I use often
\theoremstyle{definition}
\newtheorem{defn}{Definition}
\newtheorem{thm}{Theorem}
\newtheorem{cor}{Corollary}
\newtheorem*{rmk}{Remark}
\newtheorem{lem}{Lemma}
\newtheorem*{joke}{Joke}
\newtheorem{ex}{Example}
\newtheorem*{soln}{Solution}
\newtheorem{prop}{Proposition}

\newcommand{\lra}{\longrightarrow}
\newcommand{\ra}{\rightarrow}
\newcommand{\surj}{\twoheadrightarrow}
\newcommand{\graph}{\mathrm{graph}}
\newcommand{\bb}[1]{\mathbb{#1}}
\newcommand{\Z}{\bb{Z}}
\newcommand{\Q}{\bb{Q}}
\newcommand{\R}{\bb{R}}
\newcommand{\C}{\bb{C}}
\newcommand{\N}{\bb{N}}
\newcommand{\M}{\mathbf{M}}
\newcommand{\E}{\mathscr{E}}
\newcommand{\m}{\mathbf{m}}
\newcommand{\MM}{\mathscr{M}}
\newcommand{\HH}{\mathscr{H}}
\newcommand{\Om}{\Omega}
\newcommand{\Ho}{\in\HH(\Om)}
\newcommand{\bd}{\partial}
\newcommand{\del}{\partial}
\newcommand{\bardel}{\overline\partial}
\newcommand{\textdf}[1]{\textbf{\textsf{#1}}\index{#1}}
\newcommand{\img}{\mathrm{img}}
\newcommand{\ip}[2]{\left\langle{#1},{#2}\right\rangle}
\newcommand{\inter}[1]{\mathrm{int}{#1}}
\newcommand{\exter}[1]{\mathrm{ext}{#1}}
\newcommand{\cl}[1]{\mathrm{cl}{#1}}
\newcommand{\ds}{\displaystyle}
\newcommand{\vol}{\mathrm{vol}}
\newcommand{\cnt}{\mathrm{ct}}
\newcommand{\osc}{\mathrm{osc}}
\newcommand{\LL}{\mathbf{L}}
\newcommand{\UU}{\mathbf{U}}
\newcommand{\support}{\mathrm{support}}
\newcommand{\AND}{\;\wedge\;}
\newcommand{\OR}{\;\vee\;}
\newcommand{\Oset}{\varnothing}
\newcommand{\st}{\ni}
\newcommand{\wh}{\widehat}

%Pagination stuff.
\setlength{\topmargin}{-.3 in}
\setlength{\oddsidemargin}{0in}
\setlength{\evensidemargin}{0in}
\setlength{\textheight}{9.in}
\setlength{\textwidth}{6.5in}
\pagestyle{empty}



\begin{document}

\subsection*{Rappel de cours}

\begin{defn}
Soit $\E$ un K-espace vectoriel. Une partie F de $\E$ est appelée un sous-espace vectoriel si :
\begin{itemize}
\item $0_E \in F$,
\item $u + v \in F$ pour tous $u, v \in F$,
\item $\lambda.u \in F$ pour tout $\lambda \in K$ et tout $u \in F$.
\end{itemize}
\end{defn}


\begin{defn}
Une famille $\{v_1, v_2, \ldots , v_p\}$ de $E$ est une famille libre ou lin\'eairement ind\'ependante si toute combinaison lin\'eaire nulle
$$\lambda_1v_1 + \lambda_2v_2 + ··· + \lambda_p v_p = 0$$
est telle que tous ses coefficients sont nuls, c’est-\`a-dire
$$\lambda_1 = 0, \lambda_2 = 0, \ldots, \lambda_p = 0$$
\end{defn}



\begin{defn}
Soient $v_1, \ldots, v_p$ des vecteurs de $E$. La famille $\{v_1, \ldots, v_p\}$ est une famille g\'en\'eratrice de l'espace vectoriel E si tout vecteur de E est une combinaison lin\'eaire des vecteurs $v_1, \ldots, v_p$. Ce qui peut s'\'ecrire aussi :
$$\forall v \in E, \exists \lambda_1, \ldots, \lambda_p, v = \lambda_1 v_1 + \ldots + \lambda_p v_p$$
\end{defn}


\begin{defn}
Si $f : E \to F$ est une application lin\'eaire, on d\'efinit le noyau de $f$ comme \'etant le sous-espace vectoriel (sev) de $E$ d\'efini par
$$ker(f) := \{x \in E | f(x) = 0\}$$
On d\'efinit l'image de f comme \'etant le sev de F d\'efini par
$$Im(f) = \{y \in F| \exists x \in E, f(x) = y\}$$
La dimension de $Im(f)$ s'appelle aussi le rang de f et se note $rg(f)$.
\end{defn}

\begin{defn}
Soit $\E$ un espace vectoriel de dimension $n$, et $B, B'$ deux bases de $\E$. On appelle la \emph{matrice de passage} de la base $B$ vers la base $B'$, not\'ee $P_{b,B'}$, la matrice dont la j-i\`eme colonne est form\'ee des coordonn\'ees du j-i\`eme vecteur de la base $B'$ par rapport la base $B$. \\
Dans le cas particulier o\`u $B$ est la base canonique de $\E$, alors la matrice de passe est form\'ee des vecteurs de $B'$.
\end{defn}

\begin{defn}
Soit 
\begin{itemize}
\item $f: \E \to \E$, ume application lin\'eaire
\item $B,B'$ deux bases de $\E$
\item $P_{B,B'}$, la matrice de passage de $B$ vers $B'$
\item $A=Mat_{B}(f)$, la matrice de l'application $f$ dans la base $B$
\item $B=Mat_{B'}(f)$, la matrice de l'application $f$ dans la base $B'$
\end{itemize}
Alors $$B = P^{-1}AP$$
\end{defn}



\newpage
\subsection*{Exercice 1}

Soit $U=\{(x_1,x_2,x_3,x_4) \in \R^4 | x_1+x_2-2x_3+4x_4 = 0\}$ et $V=\{(x_1,x_2,x_3,x_4) \in \R^4 | 2x_1+3x_2+x_3+5x_4 = 0\}$. Il faut calculer $U \cap V$.\\
Trouvons une base pour l'espace vectoriel $U$. 
$$U=\begin{pmatrix} -x_2+2x_3-4x_4 \\ x_2 \\ x_3 \\ x_4 \end{pmatrix} = ((-1,1,0,0),(2,0,1,0),(-4,0,0,1))$$ 

Trouvons une base pour l'espace vectoriel $V$. 
$$U=\begin{pmatrix} x_1 \\ x_2 \\ -2x_1-3x_2-5x_4 \\ x_4 \end{pmatrix} = ((1,0,-2,0),(0,1,-3,0),(0,0,-5,1))$$ 

Pour qu'un vecteur $v$ appartienne \`a $U \cap V$, il est n\'ecessaire et suffisant d'avoir $v$ comme une combinaison lin\'eaire des 2 bases de $U$ et $V$, donc
$$
v= 
a_1\begin{pmatrix}-1\\1\\0\\0\end{pmatrix}+a_2\begin{pmatrix}2\\0\\1\\0\end{pmatrix}+a_3\begin{pmatrix}-4\\0\\0\\1\end{pmatrix} 
= b_1\begin{pmatrix}1\\0\\-2\\0\end{pmatrix}+b_2\begin{pmatrix}0\\1\\-3\\0\end{pmatrix}+b_3\begin{pmatrix}0\\0\\-5\\1\end{pmatrix}
$$
Donc
$$
\begin{vmatrix} -1&2&-4&1&0&0 \\ 1&0&0&0&1&0 \\ 0&1&0&-2&-3&-5 \\ 0&0&1&0&0&1 \end{vmatrix} = 
\begin{vmatrix} 1&0&0&0&1&0 \\ 0&1&0&0&-\frac{1}{5}&\frac{3}{5} \\ 0&0&1&0&0&1 \\ 0&0&0&1&\frac{7}{5}&\frac{14}{5} \end{vmatrix}
$$
D'o\`u
$$
v= 
a_1\begin{pmatrix}1\\0\\0\\0\end{pmatrix}+a_2\begin{pmatrix}0\\1\\0\\0\end{pmatrix}+a_3\begin{pmatrix}0\\0\\1\\0\end{pmatrix}
= b_1\begin{pmatrix}0\\0\\0\\1\end{pmatrix}+b_2\begin{pmatrix}1\\-\frac{1}{5}\\0\\\frac{7}{5}\end{pmatrix}+b_3\begin{pmatrix}0\\\frac{3}{5}\\1\\\frac{14}{5}\end{pmatrix}
$$
On obtient le syst\`eme d'\'equations
$$
\left\{ 
\begin{array}{l l}
a_1 =  & b_2\\
a_2 = & -\frac{1}{5}b_2+\frac{3}{5}b_3\\
a_3 = & b_3\\
b_1 = & \frac{7}{5}b_2 + \frac{14}{5}b_3\\
\end{array}
\right. 
$$ 

On peut maintenant calculer $v$
$$
v= 
b_2\begin{pmatrix}-1\\1\\0\\0\end{pmatrix}+(-\frac{1}{5}b_2+\frac{3}{5}b_3)\begin{pmatrix}2\\0\\1\\0\end{pmatrix}+b_3\begin{pmatrix}-4\\0\\0\\1\end{pmatrix}
$$

La base de $U \cap V$ est 
$$((-1-\frac{1}{5}.2, 1-\frac{1}{5}.0,0-\frac{1}{5}.1,0-\frac{1}{5}.0),(\frac{3}{5}.2-4, \frac{3}{5}.0+0, \frac{3}{5}.1+0, \frac{3}{5}.0+1))$$
$$((-\frac{7}{5},1,-\frac{1}{5},0),(-\frac{14}{5}, 0, \frac{3}{5}, 1))$$  

\subsection*{Exercice 2}
Le rang est la dimension de l'image de $f$. Commencons par calculer l'image de $f$; $Im(f) = \{y \in \R^3| \exists x \in \R^3, f(x) = y\}$.
$$f(x_1, x_2, x_3) = (x_1 + 4x_2 + 3x_3, 3x_1 + x_2 + x_3, 5x_1 + 2x_2 + 3x_3) = \begin{vmatrix} 1 & 4 & 3\\ 3 &1 & 1\\5 &2 &3 \end{vmatrix}$$
$$Im f = Vect(\begin{pmatrix}1\\3\\5\end{pmatrix}, \begin{pmatrix}4\\1\\2\end{pmatrix}, \begin{pmatrix}3\\1\\3\end{pmatrix})$$
Il faut v\'erifier si ces vecteurs sont libres ou pas? On \'ecrit une relation de liaison entre les vecteurs.
$$a.\begin{pmatrix}1\\3\\5\end{pmatrix} + b.\begin{pmatrix}4\\1\\2\end{pmatrix} + c.\begin{pmatrix}3\\1\\3\end{pmatrix} = 0$$
On \'echelonne la matrice et on obtient
$$
a.\begin{pmatrix} 1 \\ 0 \\ 0 \end{pmatrix} + b.\begin{pmatrix} 0 \\ 1 \\ 0  \end{pmatrix} + c.\begin{pmatrix} 0 \\ 0 \\ 1 \end{pmatrix} = 0$$

Il y a une unique solution; $a,b,c = 0$. Donc $Im f$ est lin\'eairement ind\'ependent et est une base.
La dimension de $Im f = 3$, donc le rang de $f = 3$.

\subsection*{Exercice 3}
\subsection*{Exercice 3.1 - $Ker A$}

Le noyau $Ker A = \{X \in \R^3, A.X = 0\}$. Prenons $X = \begin{pmatrix} x \\ y \\ z \end{pmatrix}$
$$A.X = 0 \Leftrightarrow \begin{vmatrix} 1 & 2 & 1 \\ 2 & 3 & 1 \\ 1 & 1 & 0 \end{vmatrix} . \begin{pmatrix} x \\ y \\ z \end{pmatrix} = 0$$

$$ 
\left\{ 
\begin{array}{l l}
x + 2y + z & = 0\\
2x + 3y + z & = 0\\
x + y  & = 0\\
\end{array}
\right. 
$$ 
Apres \'echelonnage, on a 
$$ 
\left\{ 
\begin{array}{l l}
x + 2y + z & = 0\\
y + z & = 0\\
\end{array}
\right. 
$$ 

En prenant $z$ comme param\`etre secondaire on obtient:
$$Ker A = \{ \begin{pmatrix} z \\ -z \\ z \end{pmatrix}, z \in \R\} = \{ \begin{pmatrix} 1 \\ -1 \\ 1 \end{pmatrix}.z, z \in \R\} = Vect(\begin{pmatrix} 1 \\ -1 \\ 1 \end{pmatrix})$$

\subsection*{Exercice 3.1 - $Im A$}
L'image de $A$ est l'espace vectoriel engendr\'e par la famille g\'en\'eratrice des colonnes de $A$.
$$Im A =  Vect(\begin{pmatrix} 1 \\ 2 \\ 1 \end{pmatrix},\begin{pmatrix} 2 \\ 3 \\ 1 \end{pmatrix},\begin{pmatrix} 1 \\ 1 \\ 0 \end{pmatrix})$$
Il faut maintenant trouver une base de $Im A$.
Il faut v\'erifier si ces vecteurs sont libres ou pas? On \'ecrit une relation de liaison entre les vecteurs.
$$a.\begin{pmatrix}1\\2\\1\end{pmatrix} + b.\begin{pmatrix}2\\3\\1\end{pmatrix} + c.\begin{pmatrix}1\\1\\0\end{pmatrix} = 0$$
Ceci est $Ker A$, comme $Ker A \neq \{0\}$, la famille n'est pas libre. Prenons les 2 premiers vecteurs comme base et v\'erifions si ils sont libres.
Base de $Im A = vect(\begin{pmatrix}1\\2\\1\end{pmatrix} + b.\begin{pmatrix}2\\3\\1\end{pmatrix})$. Ils sont libres car lin\'eairement ind\'ependent. C'est une base de $Im A$.



\subsection*{Exercice 4}
La proposition est fausse. Soit $F_1,F_2,F_3$ 3 sous-espaces vectoriels de $\R^2$, $F_1=\{(0,x): x \in \R\}, F_2=\{(y,0): y \in \R\}, F_3=\{(x,x): x \in \R\}$. On a $F_1 \cap F_3 = \{(0,0)\}$ et $F_2 \cap F_3 = \{(0,0)\}$. On a $(F_1 \cap F_3) + (F_2 \cap F_3) = \{0_{\R^2}\} + \{0_{\R^2}\} = \{0_{\R^2}\}$. On a $(F_1 + F_2) = \{(x,y):x,y \in \R\}$ donc $(F_1 + F_2) \cap F_3 = {(x,x):x \in \R} = F_3$.



\subsection*{Exercice 5}
La matrice de passage $P_{B,B'}$ est 
$$
P_{B,B'} = \begin{pmatrix}1&0&-3\\2&-1&-1\\0&0&1\end{pmatrix}
$$


\subsection*{Exercice 6}
$ker f \oplus Im f = \R^n \Leftrightarrow ker f \cap Im f = \{0\}$. Trouvons un endomorphisme $f$ tel que $ker f \cap Im f \neq \{0\}$\\

Prenons $f=\begin{vmatrix} 0 & 1 \\ 0 & 0\end{vmatrix}$ un endomorphisme de $\R^2$. Calculons $ker f$ et $im f$.
$$
a.\begin{pmatrix}0 \\ 0 \end{pmatrix} + b.\begin{pmatrix}1 \\ 0 \end{pmatrix} = 0
$$ 
Donc $ker f = Vect((z,0))$\\

L'image de $f$ est l'espace vectoriel engendr\'e par la famille g\'en\'eratrice des colonnes de $f$.
$$Im A =  Vect(\begin{pmatrix} 0 \\ 0 \end{pmatrix},\begin{pmatrix} 1 \\ 0 \end{pmatrix})$$
Il faut maintenant trouver une base de $Im f$. $Vect((z,0))$ est une famille de $f$.

$$ker f + Im f = Vect((z,0)) \neq \R^2$$
de plus $ker f \cap Im f \neq \{0\}$

EN fait ceci est vrai lorsque $f$ est un 


\subsection*{Exercice 7}
Les matrices $A$ et $B$ sont semblables si $\exists P, A = PBP^{-1}$. 
Prenons $B$ \'egale \`a la matrice identit\'e $I_{d}$. On a $PBP^{-1} = PI_{d}P^{-1} = PP^{-1} = I_{d}$. Prenons $A$ une matrice de $M_2(\R)$ de rang 2 et diff\'erente de $I_d$. On a $\not\exists P, A = PI_{d}P^{-1}$, donc la proposition est fausse.


\subsection*{Exercice 8}
$$
\begin{vmatrix} 0 & a & b & 0\\ a & 0 & 0 & b\\ c & 0 & 0 & d\\ 0 & c & d & 0\end{vmatrix} =
-a.\begin{vmatrix} a & 0 & b\\ c & 0 & d\\ 0 & d & 0\end{vmatrix} + b.\begin{vmatrix} a & 0 & b\\ c & 0 & d\\ 0 & c & 0\end{vmatrix} =
a.d.\begin{vmatrix} a & b\\ c & d\end{vmatrix} - b.c.\begin{vmatrix} a & b\\ c & d\end{vmatrix}
$$
On a $\begin{vmatrix} a & b\\ c & d\end{vmatrix} = ad-bc$. Donc 
$$
\begin{vmatrix} 0 & a & b & 0\\ a & 0 & 0 & b\\ c & 0 & 0 & d\\ 0 & c & d & 0\end{vmatrix} = a.d.(ad-bc) - b.c.(ad-bc) = (ad-bc)^2
$$ 

QED

\end{document}

