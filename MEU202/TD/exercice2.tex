\documentclass[]{book}

%These tell TeX which packages to use.
\usepackage{array,epsfig}
\usepackage{amsmath}
\usepackage{amsfonts}
\usepackage{amssymb}
\usepackage{amsxtra}
\usepackage{amsthm}
\usepackage{mathrsfs}
\usepackage{color}

%Here I define some theorem styles and shortcut commands for symbols I use often
\theoremstyle{definition}
\newtheorem{defn}{Definition}
\newtheorem{thm}{Theorem}
\newtheorem{cor}{Corollary}
\newtheorem*{rmk}{Remark}
\newtheorem{lem}{Lemma}
\newtheorem*{joke}{Joke}
\newtheorem{ex}{Example}
\newtheorem*{soln}{Solution}
\newtheorem{prop}{Proposition}

\newcommand{\lra}{\longrightarrow}
\newcommand{\ra}{\rightarrow}
\newcommand{\surj}{\twoheadrightarrow}
\newcommand{\graph}{\mathrm{graph}}
\newcommand{\bb}[1]{\mathbb{#1}}
\newcommand{\Z}{\bb{Z}}
\newcommand{\Q}{\bb{Q}}
\newcommand{\R}{\bb{R}}
\newcommand{\C}{\bb{C}}
\newcommand{\N}{\bb{N}}
\newcommand{\M}{\mathbf{M}}
\newcommand{\m}{\mathbf{m}}
\newcommand{\MM}{\mathscr{M}}
\newcommand{\HH}{\mathscr{H}}
\newcommand{\Om}{\Omega}
\newcommand{\Ho}{\in\HH(\Om)}
\newcommand{\bd}{\partial}
\newcommand{\del}{\partial}
\newcommand{\bardel}{\overline\partial}
\newcommand{\textdf}[1]{\textbf{\textsf{#1}}\index{#1}}
\newcommand{\img}{\mathrm{img}}
\newcommand{\ip}[2]{\left\langle{#1},{#2}\right\rangle}
\newcommand{\inter}[1]{\mathrm{int}{#1}}
\newcommand{\exter}[1]{\mathrm{ext}{#1}}
\newcommand{\cl}[1]{\mathrm{cl}{#1}}
\newcommand{\ds}{\displaystyle}
\newcommand{\vol}{\mathrm{vol}}
\newcommand{\cnt}{\mathrm{ct}}
\newcommand{\osc}{\mathrm{osc}}
\newcommand{\LL}{\mathbf{L}}
\newcommand{\UU}{\mathbf{U}}
\newcommand{\support}{\mathrm{support}}
\newcommand{\AND}{\;\wedge\;}
\newcommand{\OR}{\;\vee\;}
\newcommand{\Oset}{\varnothing}
\newcommand{\st}{\ni}
\newcommand{\wh}{\widehat}

%Pagination stuff.
\setlength{\topmargin}{-.3 in}
\setlength{\oddsidemargin}{0in}
\setlength{\evensidemargin}{0in}
\setlength{\textheight}{9.in}
\setlength{\textwidth}{6.5in}
\pagestyle{empty}



\begin{document}

\subsection*{Rappel de cours}
Une matrice $n \times n$ $A$ est diagonalisable ($A = PDP^{-1}$0 si:
\begin{itemize}
\item Elle a n vecteurs propres lin\'eairement ind\'ependants, condition pour avoir une matrice $P$ form\'ee des vecteurs propores en colonne qui est inversible. 
\item Elle a n valeurs propres distinctes, car n valeurs propres g\'en\`erent n vecteurs propres lin\'eairement ind\'ependants
\item $\sum{dim\ E_{sp_n}(A)} = n$
\item pour chaque valeur propre $sp$, on a $dim\ E_{sp}(A) = multiplicite\ sp$. La multiplicit\'e de $sp$ le nombre de racine de $sp$.
\item si $\chi_{A}(X) = P(X)$ et $P(X)$ est un polynome scind\'e (ie $P(X) = C(X-A_1)(X-A_2)\ldots(X-A_{m-1})(X-A_m)$).
\item si $\chi_{A}(X) = P(X)$ et $P(A)=0$.
\end{itemize}


\newpage
\subsection*{Exercice 2}
\subsubsection*{Exercice 2.1}
On a $\chi_{A}(X) = (X-1)^2(X-2)^2$ avec $A^2-A+I_4 = 0$. La matrice $A$ est diagonalisable ssi on a $\chi_{A}(A) = 0$.
$$\chi_{A}(A) = (A-1)^2(A-2)^2 = ((A-1)(A-2))^2=(A^2-3A-2I_4)^2 = 0$$
Donc la matrice $A$ est diagonalisable.\\
Les valeurs propres $Sp(A)=\{1,2\}$. Comme la matrice $A$ est diagonalisable on a $dim\ E_1(A) + dim\ E_2(A) = 4$ et la multiplicit\'e de chaque valeur propre est 2.(ie $(X-1)^2 = 0$ g\'en\`ere une racie double 1). Donc $dim\ E_1(A)=2$.


\subsubsection*{Exercice 2.2}
On a $\chi_{B}(X) = (X-1)^2(X-2)^2$ avec $B^2-B+I_2 \neq 0$. La matrice $A$ est diagonalisable ssi on a $\chi_{B}(B) = 0$.
$$\chi_{B}(B) = (B-1)^2(B-2)^2 = ((B-1)(B-2))^2=(B^2-3B-2I_2)^2 \neq 0$$
Donc la matrice $B$ n'est pas diagonalisable.\\

\subsubsection*{Exercice 2.3}
On a une matrice $C$ sym\'etrique (ie $C^T = C$) et $\chi_{C}(X) = (X-1)^2(X-2)^2 = \det(XI-C)$. Comme la matrice $C$ est sym\'etrique, elle est diagonalisable. Comme elle est diagonalisable on a $\chi_{C}(C) = 0$
$$\chi_{C}(C) = (C-1)^2(C-2)^2 = ((C-1)(C-2))^2=(C^2-3C-2I_4)^2 = 0$$
Donc $C^2-3C-2I_4 = 0$.

QED

\end{document}

