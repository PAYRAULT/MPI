\documentclass[]{book}

%These tell TeX which packages to use.
\usepackage{array,epsfig}
\usepackage{amsmath}
\usepackage{amsfonts}
\usepackage{amssymb}
\usepackage{amsxtra}
\usepackage{amsthm}
\usepackage{mathrsfs}
\usepackage{color}

%Here I define some theorem styles and shortcut commands for symbols I use often
\theoremstyle{definition}
\newtheorem{defn}{Definition}
\newtheorem{thm}{Theorem}
\newtheorem{cor}{Corollary}
\newtheorem*{rmk}{Remark}
\newtheorem{lem}{Lemma}
\newtheorem*{joke}{Joke}
\newtheorem{ex}{Example}
\newtheorem*{soln}{Solution}
\newtheorem{prop}{Proposition}

\newcommand{\lra}{\longrightarrow}
\newcommand{\ra}{\rightarrow}
\newcommand{\surj}{\twoheadrightarrow}
\newcommand{\graph}{\mathrm{graph}}
\newcommand{\bb}[1]{\mathbb{#1}}
\newcommand{\Z}{\bb{Z}}
\newcommand{\Q}{\bb{Q}}
\newcommand{\R}{\bb{R}}
\newcommand{\C}{\bb{C}}
\newcommand{\N}{\bb{N}}
\newcommand{\M}{\mathbf{M}}
\newcommand{\m}{\mathbf{m}}
\newcommand{\MM}{\mathscr{M}}
\newcommand{\HH}{\mathscr{H}}
\newcommand{\Om}{\Omega}
\newcommand{\Ho}{\in\HH(\Om)}
\newcommand{\bd}{\partial}
\newcommand{\del}{\partial}
\newcommand{\bardel}{\overline\partial}
\newcommand{\textdf}[1]{\textbf{\textsf{#1}}\index{#1}}
\newcommand{\img}{\mathrm{img}}
\newcommand{\ip}[2]{\left\langle{#1},{#2}\right\rangle}
\newcommand{\inter}[1]{\mathrm{int}{#1}}
\newcommand{\exter}[1]{\mathrm{ext}{#1}}
\newcommand{\cl}[1]{\mathrm{cl}{#1}}
\newcommand{\ds}{\displaystyle}
\newcommand{\vol}{\mathrm{vol}}
\newcommand{\cnt}{\mathrm{ct}}
\newcommand{\osc}{\mathrm{osc}}
\newcommand{\LL}{\mathbf{L}}
\newcommand{\UU}{\mathbf{U}}
\newcommand{\support}{\mathrm{support}}
\newcommand{\AND}{\;\wedge\;}
\newcommand{\OR}{\;\vee\;}
\newcommand{\Oset}{\varnothing}
\newcommand{\st}{\ni}
\newcommand{\wh}{\widehat}

%Pagination stuff.
\setlength{\topmargin}{-.3 in}
\setlength{\oddsidemargin}{0in}
\setlength{\evensidemargin}{0in}
\setlength{\textheight}{9.in}
\setlength{\textwidth}{6.5in}
\pagestyle{empty}



\begin{document}

\subsection*{Rappel de cours}

\begin{defn}
Soit E un K-espace vectoriel. Une partie F de E est appelée un sous-espace vectoriel si :
\begin{itemize}
\item $0_E \in F$,
\item $u + v \in F$ pour tous $u, v \in F$,
\item $\lambda.u \in F$ pour tout $\lambda \in K$ et tout $u \in F$.
\end{itemize}
\end{defn}


\newpage
\subsection*{Exercice 1}
\subsubsection*{1-a}

Il suffit de montrer les 3 conditions qui d\'efinissent un sous-espace vectoriel.
\begin{itemize}
\item $0_E \in F(\R,\R)$. Vrai la car fonction $0_E:x\to 0$ est continue sur $\R$
\item Pour tous $u, v \in C(\R,\R)$, $u + v \in C(\R,\R)$, car la somme de 2 fonctions continues est une fonction continue 
\item Pour tout $\lambda \in \R$ et tout $u \in C(\R,\R)$, $\lambda.u \in C(\R,\R)$ car la multiplication par une constante ne change pas la continuit\'e d'une fonction.
\end{itemize}
Donc $C(\R,\R)$ est un sous-espace vectoriel de $F(\R,\R)$.

\subsubsection*{1-b}
Notons $G(\R,\R)$ l'ensemble des fonctions $f \in C(\R,R)$ qui sont d\'erivables et telles que : $\forall x \in \R, f'(x)+xf(x) = 0$.
Il suffit de montrer les 3 conditions qui d\'efinissent un sous-espace vectoriel.
\begin{itemize}
\item $0_E \in G(\R,\R)$. Vrai car $0_E'(x) + x.0_E(x) = 0 + x.0 = 0$
\item Pour tous $u, v \in C(\R,\R)$, $u'(x) + x.u(x) = 0$ et $v'(x)+x.v(x) = 0$. On a $(u+v)'(x) + x.(u+v)(x) = u'(x) + v'(x) + x.u(x) + x.v(x) = 0 + 0 = 0$. Donc  $u+v \in G(\R,\R)$.
\item Pour tout $\lambda \in \R$ et tout $u \in G(\R,\R)$, $(\lambda.u(x))'+x.\lambda.u(x) = \lambda (u'(x) + x.u(x)) = \lambda.0 = 0$ donc $\lambda.u \in G(\R,\R)$.
\end{itemize}
Donc $G(\R,\R)$ est un sous-espace vectoriel de $C(\R,\R)$.

\subsubsection*{1-c}
Notons $H(\R,\R)$ l'ensemble des fonctions $f \in C(\R,R)$ telles que : $\forall x \in \R, 0 \leq f(x) \leq 1$.
Il suffit de montrer les 3 conditions qui d\'efinissent un sous-espace vectoriel.
\begin{itemize}
\item $0_E \in H(\R,\R)$. Vrai car $0 \leq 0_E(x) \leq 1$
\item Pour tous $u, v \in C(\R,\R)$, $0 \leq u(x) \leq 1$ et $0 \leq v(x) \leq 1$. On a $(u+v)(x) = u(x) + v(x) \geq 1$. Donc  $u+v \notin H(\R,\R)$.
\item Pour tout $\lambda \in \R$ et tout $u \in G(\R,\R)$, $\lambda.u(x) \geq 1$ lorsque $\lambda > 1$ donc $\lambda.u \notin H(\R,\R)$.
\end{itemize}
Donc $H(\R,\R)$ n'est pas un sous-espace vectoriel de $C(\R,\R)$.

QED

\end{document}

