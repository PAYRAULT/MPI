\documentclass[]{book}

%These tell TeX which packages to use.
\usepackage{array,epsfig}
\usepackage{amsmath}
\usepackage{amsfonts}
\usepackage{amssymb}
\usepackage{amsxtra}
\usepackage{amsthm}
\usepackage{mathrsfs}
\usepackage{color}

%Here I define some theorem styles and shortcut commands for symbols I use often
\theoremstyle{definition}
\newtheorem{defn}{Definition}
\newtheorem{thm}{Theorem}
\newtheorem{cor}{Corollary}
\newtheorem*{rmk}{Remark}
\newtheorem{lem}{Lemma}
\newtheorem*{joke}{Joke}
\newtheorem{ex}{Example}
\newtheorem*{soln}{Solution}
\newtheorem{prop}{Proposition}

\newcommand{\lra}{\longrightarrow}
\newcommand{\ra}{\rightarrow}
\newcommand{\surj}{\twoheadrightarrow}
\newcommand{\graph}{\mathrm{graph}}
\newcommand{\bb}[1]{\mathbb{#1}}
\newcommand{\Z}{\bb{Z}}
\newcommand{\Q}{\bb{Q}}
\newcommand{\R}{\bb{R}}
\newcommand{\C}{\bb{C}}
\newcommand{\N}{\bb{N}}
\newcommand{\M}{\mathbf{M}}
\newcommand{\m}{\mathbf{m}}
\newcommand{\MM}{\mathscr{M}}
\newcommand{\HH}{\mathscr{H}}
\newcommand{\Om}{\Omega}
\newcommand{\Ho}{\in\HH(\Om)}
\newcommand{\bd}{\partial}
\newcommand{\del}{\partial}
\newcommand{\bardel}{\overline\partial}
\newcommand{\textdf}[1]{\textbf{\textsf{#1}}\index{#1}}
\newcommand{\img}{\mathrm{img}}
\newcommand{\ip}[2]{\left\langle{#1},{#2}\right\rangle}
\newcommand{\inter}[1]{\mathrm{int}{#1}}
\newcommand{\exter}[1]{\mathrm{ext}{#1}}
\newcommand{\cl}[1]{\mathrm{cl}{#1}}
\newcommand{\ds}{\displaystyle}
\newcommand{\vol}{\mathrm{vol}}
\newcommand{\cnt}{\mathrm{ct}}
\newcommand{\osc}{\mathrm{osc}}
\newcommand{\LL}{\mathbf{L}}
\newcommand{\UU}{\mathbf{U}}
\newcommand{\support}{\mathrm{support}}
\newcommand{\AND}{\;\wedge\;}
\newcommand{\OR}{\;\vee\;}
\newcommand{\Oset}{\varnothing}
\newcommand{\st}{\ni}
\newcommand{\wh}{\widehat}

%Pagination stuff.
\setlength{\topmargin}{-.3 in}
\setlength{\oddsidemargin}{0in}
\setlength{\evensidemargin}{0in}
\setlength{\textheight}{9.in}
\setlength{\textwidth}{6.5in}
\pagestyle{empty}



\begin{document}

\subsection*{Rappel de cours}

\begin{defn}
Soit E un K-espace vectoriel. Une partie F de E est appelée un sous-espace vectoriel si :
\begin{itemize}
\item $0_E \in F$,
\item $u + v \in F$ pour tous $u, v \in F$,
\item $\lambda.u \in F$ pour tout $\lambda \in K$ et tout $u \in F$.
\end{itemize}
\end{defn}


\begin{defn}
Une famille $\{v_1, v_2, \ldots , v_p\}$ de $E$ est une famille libre ou lin\'eairement ind\'ependante si toute combinaison lin\'eaire nulle
$$\lambda_1v_1 + \lambda_2v_2 + ··· + \lambda_p v_p = 0$$
est telle que tous ses coefficients sont nuls, c’est-\`a-dire
$$\lambda_1 = 0, \lambda_2 = 0, \ldots, \lambda_p = 0$$
\end{defn}



\begin{defn}
Soient $v_1, \ldots, v_p$ des vecteurs de $E$. La famille $\{v_1, \ldots, v_p\}$ est une famille g\'en\'eratrice de l'espace vectoriel E si tout vecteur de E est une combinaison lin\'eaire des vecteurs $v_1, \ldots, v_p$. Ce qui peut s'\'ecrire aussi :
$$\forall v \in E, \exists \lambda_1, \ldots, \lambda_p, v = \lambda_1 v_1 + \ldots + \lambda_p v_p$$
\end{defn}


\newpage
\subsection*{Exercice 1}
\subsubsection*{1-a}

Il suffit de montrer les 3 conditions qui d\'efinissent un sous-espace vectoriel.
\begin{itemize}
\item $0_F \in F(\R,\R)$. La fonction $0_F:x\to 0$ est continue sur $\R$, donc $0_F \in C(\R,\R)$.
\item Pour tout $u, v \in C(\R,\R)$, $u + v \in C(\R,\R)$, car la somme de 2 fonctions continues est une fonction continue 
\item Pour tout $\lambda \in \R$ et tout $u \in C(\R,\R)$, $\lambda.u \in C(\R,\R)$ car la multiplication par une constante ne change pas la continuit\'e d'une fonction.
\end{itemize}
Donc $C(\R,\R)$ est un sous-espace vectoriel de $F(\R,\R)$.

\subsubsection*{1-b}
Notons $G(\R,\R)$ l'ensemble des fonctions $f \in C(\R,R)$ qui sont d\'erivables et telles que : $\forall x \in \R, f'(x)+xf(x) = 0$.
Il suffit de montrer les 3 conditions qui d\'efinissent un sous-espace vectoriel.
\begin{itemize}
\item $0_C \in C(\R,\R)$. $0_C'(x) + x.0_C(x) = 0 + x.0 = 0$ donc $0_C \in G(\R,\R)$ 
\item Pour tout $u, v \in G(\R,\R)$, $u'(x) + x.u(x) = 0$ et $v'(x)+x.v(x) = 0$. On a $(u+v)'(x) + x.(u+v)(x) = u'(x) + v'(x) + x.u(x) + x.v(x) = 0 + 0 = 0$. Donc  $u+v \in G(\R,\R)$.
\item Pour tout $\lambda \in \R$ et tout $u \in G(\R,\R)$, $(\lambda.u(x))'+x.(\lambda.u(x)) = \lambda.u'(x) + \lambda.x.u(x) = \lambda (u'(x) + x.u(x)) = \lambda.0 = 0$ donc $\lambda.u \in G(\R,\R)$.
\end{itemize}
Donc $G(\R,\R)$ est un sous-espace vectoriel de $C(\R,\R)$.

\subsubsection*{1-c}
Notons $H(\R,\R)$ l'ensemble des fonctions $f \in C(\R,R)$ telles que : $\forall x \in \R, 0 \leq f(x) \leq 1$.
Il suffit de montrer les 3 conditions qui d\'efinissent un sous-espace vectoriel.
\begin{itemize}
\item $0_C \in C(\R,\R)$. $0 \leq 0_E(x) \leq 1$
\item Pour tout $u, v \in C(\R,\R)$, $0 \leq u(x) \leq 1$ et $0 \leq v(x) \leq 1$. On a $(u+v)(x) = u(x) + v(x) \geq 1$. Donc  $u+v \notin H(\R,\R)$.
\item Pour tout $\lambda \in \R$ et tout $u \in G(\R,\R)$, $\lambda.u(x) \geq 1$ lorsque $\lambda > 1$ donc $\lambda.u \notin H(\R,\R)$.
\end{itemize}
Donc $H(\R,\R)$ n'est pas un sous-espace vectoriel de $C(\R,\R)$.


\subsection*{Exercice 2}
\subsubsection*{2-a}
Famille libre? \\
$\lambda_1 v_1 + \lambda_2 v_2 = \lambda_1 (3,5) + \lambda_2 (7,-3) = 0$
$$
\left\{ 
\begin{array}{l l}
3 \lambda_1 + 7 \lambda_2 = 0 & (1)\\
5 \lambda_1 - 3 \lambda_2 = 0 & (2)\\
\end{array}
\right. 
$$ 

$$
\left\{ 
\begin{array}{l l}
3 \lambda_1 + 7 \lambda_2 = 0 & \\
0 \lambda_1 + 44 \lambda_2 = 0 & 5(1) - 3(2)\\
\end{array}
\right. 
$$ 

On a $\lambda_2 = 0$ et $\lambda_1 = 0$. Donc, la famille est libre. \\

Famille g\'en\'eratrice?
$$\forall v = (x,y) \in E, \exists \lambda_1, \lambda_2, v = \lambda_1 v_1 + \lambda_2 v_2 = \lambda_1 (3,5) + \lambda_2 (7,-3)$$


$$
\left\{ 
\begin{array}{l l}
3 \lambda_1 + 7 \lambda_2 = x & (1)\\
5 \lambda_1 - 3 \lambda_2 = y & (2)\\
\end{array}
\right. 
$$ 

$$
\left\{ 
\begin{array}{l l}
3 \lambda_1 + 7 \lambda_2 = x & (1)\\
0 \lambda_1 +44 \lambda_2 = 5x -3y & 5(1) -3(2)\\
\end{array}
\right. 
$$ 

Il existe $\lambda_2 = \frac{5x -3y}{44}$ et $\lambda_1 = \frac{3x + y}{44}$. Donc, la famille est g\'en\'eratrice.

\subsubsection*{2-c}
Famille libre? \\
$\lambda_1 v_1 + \lambda_2 v_2 + \lambda_3 v_3 = \lambda_1 (1,2,3) + \lambda_2 (4,5,6) + \lambda_3 (2,1,7)  = 0$
$$
\left\{ 
\begin{array}{l l}
1 \lambda_1 + 4 \lambda_2 + 2 \lambda_3 = 0 & (1)\\
2 \lambda_1 + 5 \lambda_2 + 1 \lambda_3 = 0 & (2)\\
3 \lambda_1 + 6 \lambda_2 + 7 \lambda_3 = 0 & (3)\\
\end{array}
\right. 
$$ 

$$
\left\{ 
\begin{array}{l l}
1 \lambda_1 + 4 \lambda_2 + 2 \lambda_3 = 0 & (1)\\
0 \lambda_1 - 3 \lambda_2 - 3 \lambda_3 = 0 & (2) - 2(1) = (4)\\
0 \lambda_1 - 6 \lambda_2 + 1 \lambda_3 = 0 & (3) - 3(1) = (5)\\
\end{array}
\right. 
$$ 

$$
\left\{ 
\begin{array}{l l}
1 \lambda_1 + 4 \lambda_2 + 2 \lambda_3 = 0 & (1)\\
0 \lambda_1 - 3 \lambda_2 - 3 \lambda_3 = 0 & (2) - 2(1) = (4)\\
0 \lambda_1 - 33 \lambda_2 + 0 \lambda_3 = 0 & 3(5) + (4)\\
\end{array}
\right. 
$$ 


On a $\lambda_2 = 0$, $\lambda_3 = 0$ et $\lambda_1 = 0$. Donc, la famille est libre. \\

Famille g\'en\'eratrice?
$$\forall v = (x,y,z) \in E, \exists \lambda_1, \lambda_2, \lambda_3, v = \lambda_1 v_1 + \lambda_2 v_2 + \lambda_3 v_3 = \lambda_1 (1,2,3) + \lambda_2 (4,5,6) + \lambda_2 (2,1,7)$$

$$
\left\{ 
\begin{array}{l l}
1 \lambda_1 + 4 \lambda_2 + 2 \lambda_3 = x & (1)\\
2 \lambda_1 + 5 \lambda_2 + 1 \lambda_3 = y & (2)\\
3 \lambda_1 + 6 \lambda_2 + 7 \lambda_3 = z & (3)\\
\end{array}
\right. 
$$ 

$$
\left\{ 
\begin{array}{l l}
1 \lambda_1 + 4 \lambda_2 + 2 \lambda_3 = x & (1)\\
0 \lambda_1 - 3 \lambda_2 - 3 \lambda_3 = y-2x & (2) - 2(1) = (4)\\
0 \lambda_1 - 6 \lambda_2 + 1 \lambda_3 = z-3x & (3) - 3(1) = (5)\\
\end{array}
\right. 
$$ 

$$
\left\{ 
\begin{array}{l l}
1 \lambda_1 + 4 \lambda_2 + 2 \lambda_3 = x & (1)\\
0 \lambda_1 - 3 \lambda_2 - 3 \lambda_3 = y-2x & (4)\\
0 \lambda_1 - 33 \lambda_2 + 0 \lambda_3 = 3(z-3x)+y-2x = 3z-11x+y & 3(5) + (4)\\
\end{array}
\right. 
$$ 


Il existe $\lambda_2 = -\frac{3z-11x+y}{33}$, $\lambda_3 = \frac{5x -3y}{44}$ et $\lambda_1 = \frac{3x + y}{44}$. Donc, la famille est g\'en\'eratrice.

\subsubsection*{2-d}
Famille libre? \\
$$\lambda_1 v_1 + \lambda_2 v_2 + \lambda_3 v_3 = \lambda_1 (1,2,3,4) + \lambda_2 (2,3,4,5) + \lambda_3 (4,5,6,7)  = 0$$

$$
\left\{ 
\begin{array}{l l}
1 \lambda_1 + 2 \lambda_2 + 4 \lambda_3 = 0 & (1)\\
2 \lambda_1 + 3 \lambda_2 + 5 \lambda_3 = 0 & (2)\\
3 \lambda_1 + 4 \lambda_2 + 6 \lambda_3 = 0 & (3)\\
4 \lambda_1 + 5 \lambda_2 + 7 \lambda_3 = 0 & (4)\\
\end{array}
\right. 
$$ 


Famille g\'en\'eratrice?
$$\forall v = (x,y,z) \in E, \exists \lambda_1, \lambda_2, \lambda_3, v = \lambda_1 v_1 + \lambda_2 v_2 + \lambda_3 v_3 = \lambda_1 (1,2,3,4) + \lambda_2 (2,3,4,5) + \lambda_3 (4,5,6,7)$$

$$
\left\{ 
\begin{array}{l l}
1 \lambda_1 + 2 \lambda_2 + 4 \lambda_3 = x & (1)\\
2 \lambda_1 + 3 \lambda_2 + 5 \lambda_3 = y & (2)\\
3 \lambda_1 + 4 \lambda_2 + 6 \lambda_3 = z & (3)\\
4 \lambda_1 + 5 \lambda_2 + 7 \lambda_3 = w & (4)\\
\end{array}
\right. 
$$ 


\subsection*{Exercice 3}
\subsubsection*{3-a}
$$v = (1,-1)= \lambda_1.(1,2) + \lambda_2.(5,3) = (\lambda_1+5\lambda_2,2\lambda_1+3\lambda_2)$$

$$
\left\{ 
\begin{array}{l l}
\lambda_1+5\lambda_2 = 1 & (1)\\
2\lambda_1+3\lambda_2 = -1 & (2)\\
\end{array}
\right. 
$$ 

$$
\left\{ 
\begin{array}{l l}
\lambda_1 = 1 - 5\lambda_2 & (1)\\
2-10\lambda_2+3\lambda_2 = -1 & (2)\\
\end{array}
\right. 
$$ 

$$
\left\{ 
\begin{array}{l l}
\lambda_1 = 1 - 5\lambda_2 & (1)\\
\lambda_2 = \frac{3}{7} & (2)\\
\end{array}
\right. 
$$ 

$$
\left\{ 
\begin{array}{l l}
\lambda_1 = 1 - 5\frac{3}{7} = \frac{7}{7}- \frac{15}{7} = \frac{-8}{7}& (1)\\
\lambda_2 = \frac{3}{7} & (2)\\
\end{array}
\right. 
$$ 

Dans la base $\mathcal{B}$, $v=(\frac{-8}{7},\frac{3}{7})$.

\subsubsection*{3-b}
$$v=(1+X)^3 = 1+3X+3X^2+X^3 = \lambda_1 + X\lambda_2 + X^2\lambda_3 + X^3\lambda_4$$
Dans la base $\mathcal{B}$, $v=(1,3,3,1)$???

\subsubsection*{3-c}
$$v=X^2 = \lambda_1 + (X+1)\lambda_2 + (X+1)^2\lambda_3 = \lambda_1+\lambda_2+\lambda_3 +X(\lambda_2+2\lambda_3) + X^2\lambda_3$$
Dans la base $\mathcal{B}$, $v=(1,-2,1)$ ???

\subsubsection*{3-d}
$$v=\cos^2 = \frac{1}{2}(1+\cos_2) = \lambda_1 + \lambda_2 \cos + \lambda_3 \sin + \lambda_4 \cos_2 + \lambda_5 \sin_2 $$
Dans la base $\mathcal{B}$, $v=(\frac{1}{2},0,0,\frac{1}{2},0)$ ???


\subsection*{Exercice 4}
On fixe $x_2, x_3,x_4$ et on regarde comment $x_1$ est impact\'e. Soit la base $( (x_{11},1,0,0), (x_{12},0,1,0), (x_{13},0,0,1))$, pour v\'erifier $2x_1+3x_2 -x_3+x_4 = 0$ il faut $x_{11}=\frac{-3}{2}$, $x_{12}=\frac{1}{2}$ et $x_{13}=-\frac{1}{2}$.\\
Ceci est m\'ecamiquement une base car les 3 vecteurs sont mutuellement ind\'ependants.



\subsection*{Exercice 8}
$$E + F = \{v \in \R^3: v = \lambda_{1e}u_1 + \lambda_{2e}u_2 + \lambda_{3e}u_3 + \lambda_{1f}u_4 + \lambda_{2f}u_5\}$$  

Il faut r\'esoudre:
$$
\left\{ 
\begin{array}{l l}
\lambda_{1e} + \lambda_{2e} + 3\lambda_{3e} + \lambda_{1f} + \lambda_{2f} = 0 \\
2\lambda_{1e} + 0\lambda_{2e} + 2\lambda_{3e} + \lambda_{1f} + 2\lambda_{2f} = 0\\
3\lambda_{1e} - \lambda_{2e} + 1\lambda_{3e} + \lambda_{1f} + 2\lambda_{2f} = 0\\
\end{array}
\right. 
$$ 


$$
\begin{vmatrix}
1 & 1  & 3 & 1 & 1 & 0 \\
2 & 0  & 2 & 1 & 2 & 0 \\
3 & -1 & 1 & 1 & 2 & 0 \\
\end{vmatrix}
\Leftrightarrow
\begin{vmatrix}
1 & 0  & 1 & \frac{1}{2} & 0 & 0 \\
0 & 1  & 2 & \frac{1}{2} & 0 & 0 \\
0 & 0 & 0 & 0 & 1 & 0 \\
\end{vmatrix}
$$


$$
\left\{ 
\begin{array}{l l}
\lambda_{1e} + \lambda_{3e} + \frac{1}{2} \lambda_{1f} = 0 \\
2\lambda_{2e} + 2\lambda_{3e} + \frac{1}{2}\lambda_{1f} = 0 \\
\lambda_{2f}= 0 \\
\end{array}
\right. 
$$ 


$\lambda_{3e}$, et $\lambda_{1f}$ sont ind\'etermin\'es, donc les vecteurs $u_3$ et $u_4$ peuvent s'exprimer en fonction des 3 autres vecteurs ($u_3 = u_1 + 2u_2$ et $u_4 = 1/2.(u_1 + u_2$). Donc la base de $E+F$ est $\{u_1, u_2, u_5\}$.



$$E \cap F = \{v \in \R^3: v = \lambda_{1e}u_1 + \lambda_{2e}u_2 + \lambda_{3e}u_3 \land v = \lambda_{1f}u_4 + \lambda_{2f}u_5\}$$  
$$E \cap F = \{v \in \R^3: \lambda_{1e}u_1 + \lambda_{2e}u_2 + \lambda_{3e}u_3 - \lambda_{1f}u_4 - \lambda_{2f}u_5 = 0\}$$  

$$
\left\{ 
\begin{array}{l l}
\lambda_{1e} + \lambda_{2e} + 3\lambda_{3e} - \lambda_{1f} - \lambda_{2f} = 0 & (1)\\
2\lambda_{1e} + 0\lambda_{2e} + 2\lambda_{3e} - \lambda_{1f} - 2\lambda_{2f} = 0 & (2)\\
3\lambda_{1e} - \lambda_{2e} + 1\lambda_{3e} - \lambda_{1f} - 2\lambda_{2f} = 0 & (3)\\
\end{array}
\right. 
$$ 

$$
\begin{vmatrix}
1 & 1  & 3 & -1 & -1 & 0\\
2 & 0  & 2 & -1 & -2 & 0\\
3 & -1 & 1 & -1 & -2 & 0\\
\end{vmatrix}
\Leftrightarrow
\begin{vmatrix}
1 & 1  & 3 & -1 & -1  & 0\\
0 & 1  & 2 & -\frac{1}{2} & 0 & 0\\
0 & 0 & 0 & 0 & 1 & 0\\
\end{vmatrix}
$$

$$
\left\{ 
\begin{array}{l l}
\lambda_{1e} = -\lambda_{2e} - 3\lambda_{3e} + \lambda_{1f} + \lambda_{2f} & (1)\\
\lambda_{2e} = -2\lambda_{3e} + \frac{1}{2}\lambda_{1f} = 0 & (2)\\
\lambda_{2f} = 0 & (3)\\
\end{array}
\right. 
$$ 

$$
\left\{ 
\begin{array}{l l}
\lambda_{1e} = -\lambda_{2e} - 3\lambda_{3e} + (4\lambda_{3e} + 2\lambda_{2e} = \lambda_{2e} + \lambda_{3e} & (1)\\
\lambda_{1f} = 4\lambda_{3e} + 2\lambda_{2e} & (2)\\
\lambda_{2f} = 0 & (3)\\
\end{array}
\right. 
$$ 


La base est $\{x \in \R^3: (\lambda_{2e} + \lambda_{3e})u_1 + \lambda_{2e}u_2 + \lambda_{2e}u_3 \}$, $\{x \in \R^3: (2\lambda_2 + 4\lambda4, 2\lambda_2 + 4\lambda4, 2\lambda_2 + 4\lambda4) \} = \{x \in \R^3: (1,1,1) \}$.

QED

\end{document}

