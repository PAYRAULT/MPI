\documentclass[]{book}

%These tell TeX which packages to use.
\usepackage{array,epsfig}
\usepackage{amsmath}
\usepackage{amsfonts}
\usepackage{amssymb}
\usepackage{amsxtra}
\usepackage{amsthm}
\usepackage{mathrsfs}
\usepackage{color}

%Here I define some theorem styles and shortcut commands for symbols I use often
\theoremstyle{definition}
\newtheorem{defn}{Definition}
\newtheorem{thm}{Theorem}
\newtheorem{cor}{Corollary}
\newtheorem*{rmk}{Remark}
\newtheorem{lem}{Lemma}
\newtheorem*{joke}{Joke}
\newtheorem{ex}{Example}
\newtheorem*{soln}{Solution}
\newtheorem{prop}{Proposition}

\newcommand{\lra}{\longrightarrow}
\newcommand{\ra}{\rightarrow}
\newcommand{\surj}{\twoheadrightarrow}
\newcommand{\graph}{\mathrm{graph}}
\newcommand{\bb}[1]{\mathbb{#1}}
\newcommand{\Z}{\bb{Z}}
\newcommand{\Q}{\bb{Q}}
\newcommand{\R}{\bb{R}}
\newcommand{\C}{\bb{C}}
\newcommand{\N}{\bb{N}}
\newcommand{\M}{\mathbf{M}}
\newcommand{\E}{\mathscr{E}}
\newcommand{\m}{\mathbf{m}}
\newcommand{\MM}{\mathscr{M}}
\newcommand{\HH}{\mathscr{H}}
\newcommand{\Om}{\Omega}
\newcommand{\Ho}{\in\HH(\Om)}
\newcommand{\bd}{\partial}
\newcommand{\del}{\partial}
\newcommand{\bardel}{\overline\partial}
\newcommand{\textdf}[1]{\textbf{\textsf{#1}}\index{#1}}
\newcommand{\img}{\mathrm{img}}
\newcommand{\ip}[2]{\left\langle{#1},{#2}\right\rangle}
\newcommand{\inter}[1]{\mathrm{int}{#1}}
\newcommand{\exter}[1]{\mathrm{ext}{#1}}
\newcommand{\cl}[1]{\mathrm{cl}{#1}}
\newcommand{\ds}{\displaystyle}
\newcommand{\vol}{\mathrm{vol}}
\newcommand{\cnt}{\mathrm{ct}}
\newcommand{\osc}{\mathrm{osc}}
\newcommand{\LL}{\mathbf{L}}
\newcommand{\UU}{\mathbf{U}}
\newcommand{\support}{\mathrm{support}}
\newcommand{\AND}{\;\wedge\;}
\newcommand{\OR}{\;\vee\;}
\newcommand{\Oset}{\varnothing}
\newcommand{\st}{\ni}
\newcommand{\wh}{\widehat}

%Pagination stuff.
\setlength{\topmargin}{-.3 in}
\setlength{\oddsidemargin}{0in}
\setlength{\evensidemargin}{0in}
\setlength{\textheight}{9.in}
\setlength{\textwidth}{6.5in}
\pagestyle{empty}


\pdfinfo{
   /Author (Myself)
   /Title  (DM1 - MEU202)
}

\begin{document}

\subsection*{Rappel de cours}k



\newpage
\subsection*{Exercice 1}
\subsubsection*{Exercice 1.1 pour A}
On a 
$$\det(A -\lambda I) = \det\left(\begin{vmatrix} 0-\lambda & 0 & 1 \\ 2 & i-\lambda & 2i \\ -1 & 0 & 0-\lambda \end{vmatrix}\right) = (i-\lambda)(\lambda^2+1)$$

Calculons $\det(A -\lambda I) = 0$, $(i-\lambda)(\lambda^2+1)=0$, donc $sp(A)=\{i\}$. 

\subsubsection*{Exercice 1.2 pour A}
Calculons
$$E_{i}(A) = ker(A -iI) = ker\left(\begin{vmatrix} 0-i & 0 & 1 \\ 2 & i-i & 2i \\ -1 & 0 & 0-i \end{vmatrix}\right)$$
Cherchons $\lambda_1,\lambda_2,\lambda_3$  tel que
$$\begin{vmatrix} -i & 0 & 1 \\ 2 & 0 & 2i \\ -1 & 0 & -i \end{vmatrix}.\begin{vmatrix} \lambda_1 & \lambda_2 & \lambda_3 \end{vmatrix} = 0$$

$$
\left\{ 
\begin{array}{l l}
-i\lambda_{1} + 2\lambda_{2} -\lambda_{3} & = 0\\
0\lambda_{1} + 0\lambda_{2} + 0\lambda_{3} & = 0\\
-\lambda_{1} + 2i\lambda_{2} -i \lambda_{3} & = 0\\
\end{array}
\right. 
$$ 
Echelonnage
$$
\left\{ 
\begin{array}{l l}
-i\lambda_{1} + 2\lambda_{2} -\lambda_{3} & = 0\\
0\lambda_{1} + 0\lambda_{2} + 0\lambda_{3} & = 0\\
0\lambda_{1} + 0\lambda_{2} + 0\lambda_{3} & = 0  (L_3=L_1-iL_3)\\
\end{array}
\right. 
$$ 

Donc, en fixant $\lambda_1 = c_1$ et $\lambda_2=c_2$ on a $\lambda_3 = -ic_1+2c_2$
$$ker(A-iI) = \{(1,0,-i), (0,1,2)\}$$ 
et $\dim(ker(A-iI)) = 2$.

\subsubsection*{Exercice 1.1 pour B}
On a 
$$\det(A -\lambda I) = \det\left(\begin{vmatrix} i-\lambda & 0 & 2 \\ 0 & 0-\lambda & 1 \\ 0 & -1 & 0-\lambda \end{vmatrix}\right) = (i-\lambda)(\lambda^2+1)$$

Calculons $\det(A -\lambda I) = 0$, $(i-\lambda)(\lambda^2+1)=0$, donc $sp(A)=\{i\}$. 

\subsubsection*{Exercice 1.2}
Calculons
$$E_{i}(B) = ker(B -iI) = ker\left(\begin{vmatrix} i-i & 0 & 2 \\ 0 & 0-i & 1 \\ 0 & -1 & 0-i \end{vmatrix}\right)$$
Cherchons $\lambda_1,\lambda_2,\lambda_3$  tel que
$$\begin{vmatrix} 0 & 0 & 2 \\ 0 & -i & 1 \\ 0 & -1 & -i \end{vmatrix}.\begin{vmatrix} \lambda_1 & \lambda_2 & \lambda_3 \end{vmatrix} = 0$$

$$
\left\{ 
\begin{array}{l l}
0\lambda_{1} + 0\lambda_{2} + 0\lambda_{3} & = 0\\
0\lambda_{1} - i\lambda_{2} + \lambda_{3} & = 0\\
2\lambda_{1} + \lambda_{2} - i\lambda_{3} & = 0\\
\end{array}
\right. 
$$ 
Echelonnage
$$
\left\{ 
\begin{array}{l l}
2\lambda_{1} + \lambda_{2} - i\lambda_{3} & = 0 (L_1=L_3)\\
0\lambda_{1} - i\lambda_{2} + \lambda_{3} & = 0\\
0\lambda_{1} + 0\lambda_{2} + 0\lambda_{3} & = 0 (L_3=L_1)\\
\end{array}
\right. 
,
\left\{ 
\begin{array}{l l}
2\lambda_{1} + \lambda_{2} - i\lambda_{3} & = 0\\
\lambda_{3} & = i\lambda_{2}\\
\end{array}
\right. 
,
\left\{ 
\begin{array}{l l}
2\lambda_{1} + \lambda_{2} - i(i\lambda_{2}) & = 0\\
\lambda_{3} & = i\lambda_{2}\\
\end{array}
\right. 
$$ 

Donc, en fixant $\lambda_2 = c_2$ on a $\lambda_1 = -c_2$, $\lambda_3 = ic_2$
$$ker(B-iI) = \{(-1,1,i)\}$$ 
et $\dim(ker(B-iI)) = 1$.


QED

\end{document}

