\documentclass[]{book}

%These tell TeX which packages to use.
\usepackage{array,epsfig}
\usepackage{amsmath}
\usepackage{amsfonts}
\usepackage{amssymb}
\usepackage{amsxtra}
\usepackage{amsthm}
\usepackage{mathrsfs}
\usepackage{color}

%Here I define some theorem styles and shortcut commands for symbols I use often
\theoremstyle{definition}
\newtheorem{defn}{Definition}
\newtheorem{thm}{Theorem}
\newtheorem{cor}{Corollary}
\newtheorem*{rmk}{Remark}
\newtheorem{lem}{Lemma}
\newtheorem*{joke}{Joke}
\newtheorem{ex}{Example}
\newtheorem*{soln}{Solution}
\newtheorem{prop}{Proposition}

\newcommand{\lra}{\longrightarrow}
\newcommand{\ra}{\rightarrow}
\newcommand{\surj}{\twoheadrightarrow}
\newcommand{\graph}{\mathrm{graph}}
\newcommand{\bb}[1]{\mathbb{#1}}
\newcommand{\Z}{\bb{Z}}
\newcommand{\Q}{\bb{Q}}
\newcommand{\R}{\bb{R}}
\newcommand{\C}{\bb{C}}
\newcommand{\N}{\bb{N}}
\newcommand{\M}{\mathbf{M}}
\newcommand{\E}{\mathscr{E}}
\newcommand{\m}{\mathbf{m}}
\newcommand{\MM}{\mathscr{M}}
\newcommand{\HH}{\mathscr{H}}
\newcommand{\Om}{\Omega}
\newcommand{\Ho}{\in\HH(\Om)}
\newcommand{\bd}{\partial}
\newcommand{\del}{\partial}
\newcommand{\bardel}{\overline\partial}
\newcommand{\textdf}[1]{\textbf{\textsf{#1}}\index{#1}}
\newcommand{\img}{\mathrm{img}}
\newcommand{\ip}[2]{\left\langle{#1},{#2}\right\rangle}
\newcommand{\inter}[1]{\mathrm{int}{#1}}
\newcommand{\exter}[1]{\mathrm{ext}{#1}}
\newcommand{\cl}[1]{\mathrm{cl}{#1}}
\newcommand{\ds}{\displaystyle}
\newcommand{\vol}{\mathrm{vol}}
\newcommand{\cnt}{\mathrm{ct}}
\newcommand{\osc}{\mathrm{osc}}
\newcommand{\LL}{\mathbf{L}}
\newcommand{\UU}{\mathbf{U}}
\newcommand{\support}{\mathrm{support}}
\newcommand{\AND}{\;\wedge\;}
\newcommand{\OR}{\;\vee\;}
\newcommand{\Oset}{\varnothing}
\newcommand{\st}{\ni}
\newcommand{\wh}{\widehat}

%Pagination stuff.
\setlength{\topmargin}{-.3 in}
\setlength{\oddsidemargin}{0in}
\setlength{\evensidemargin}{0in}
\setlength{\textheight}{9.in}
\setlength{\textwidth}{6.5in}
\pagestyle{empty}


\pdfinfo{
   /Author (Myself)
   /Title  (DM1 - MEU202)
}

\begin{document}

\subsection*{Rappel de cours}k



\newpage
\subsection*{Exercice 1}
\subsubsection*{Exercice 1.1 pour A}
On a 
$$\det(A -\lambda I) = \det\left(\begin{vmatrix} 0-\lambda & 0 & 1 \\ 2 & i-\lambda & 2i \\ -1 & 0 & 0-\lambda \end{vmatrix}\right) = (i-\lambda)(\lambda^2+1)$$

Calculons $\det(A -\lambda I) = 0$, $(i-\lambda)(\lambda^2+1)=0$, donc $sp(A)=\{i, -i\}$. 

\subsubsection*{Exercice 1.2 pour A}
Calculons
$$E_{i}(A) = ker(A -iI) = ker\left(\begin{vmatrix} 0-i & 0 & 1 \\ 2 & i-i & 2i \\ -1 & 0 & 0-i \end{vmatrix}\right)$$
Cherchons $\lambda_1,\lambda_2,\lambda_3$  tel que
$$\begin{vmatrix} -i & 0 & 1 \\ 2 & 0 & 2i \\ -1 & 0 & -i \end{vmatrix}.\begin{vmatrix} \lambda_1 \\ \lambda_2 \\ \lambda_3 \end{vmatrix} = 0$$

$$
\left\{ 
\begin{array}{l l}
-i\lambda_{1} + 0\lambda_{2} + \lambda_{3} & = 0\\
2\lambda_{1} + 0\lambda_{2} + 2i\lambda_{3} & = 0\\
-\lambda_{1} + 0\lambda_{2} -i \lambda_{3} & = 0\\
\end{array}
\right. 
$$ 
Echelonnage
$$
\left\{ 
\begin{array}{l l}
\lambda_{1} + 0\lambda_{2} + i\lambda_{3} & = 0 (L_1=L_1*-i)\\
0\lambda_{1} + 0\lambda_{2} + 0\lambda_{3} & = 0 (L_2 = L_2+2iL_2)\\
0\lambda_{1} + 0\lambda_{2} + 0\lambda_{3} & = 0 (L_3 = L_3+iL_1)\\
\end{array}
\right. 
$$ 

Donc, en fixant $\lambda_3 = c_3$ et $\lambda_2=c_2$ on a $\lambda_1 = -ic_3$
$$ker(A-iI) = \{(-i,0,1), (0,1,0)\}$$ 
et $\dim(E_i(A) = \dim(ker(A-iI)) = 2$.

Calculons
$$E_{-i}(A) = ker(A +iI) = ker\left(\begin{vmatrix} 0+i & 0 & 1 \\ 2 & i+i & 2i \\ -1 & 0 & 0+i \end{vmatrix}\right)$$
Cherchons $\lambda_1,\lambda_2,\lambda_3$  tel que
$$\begin{vmatrix} i & 0 & 1 \\ 2 & 2i & 2i \\ -1 & 0 & i \end{vmatrix}.\begin{vmatrix} \lambda_1 \\ \lambda_2 \\ \lambda_3 \end{vmatrix} = 0$$

$$
\left\{ 
\begin{array}{l l}
i\lambda_{1} + 0\lambda_{2} + \lambda_{3} & = 0\\
2\lambda_{1} + 2i\lambda_{2} + 2i\lambda_{3} & = 0\\
-\lambda_{1} + 0\lambda_{2} + i\lambda_{3} & = 0\\
\end{array}
\right. 
$$ 
Echelonnage
$$
\left\{ 
\begin{array}{l l}
-\lambda_{1} + 0\lambda_{2} + i\lambda_{3} & = 0 (L_1=L_1.i)\\
0\lambda_{1} + 2i\lambda_{2} + 4i\lambda_{3} & = 0 (L_2=L_2+2iL_1)\\
0\lambda_{1} + 0\lambda_{2} + 0\lambda_{3} & = 0 (L_3=L_1+iL_3)\\
\end{array}
\right. 
$$ 

Donc, en fixant $\lambda_3 = c_3$ on a $\lambda_1 = ic_3$ et $\lambda_2 = -2c_3$
$$ker(A+iI) = \{(i,-2,1)\}$$ 
et $\dim(E_{-i}(A)) = \dim(ker(A+iI)) = 1$.

0

\subsubsection*{Exercice 1.3 pour A}
Calculons $dim(A)$.
$$\begin{vmatrix} 0 & 0 & 1 \\ 2 & i & 2i \\ -1 & 0 & 0 \end{vmatrix}$$
Echelonnage
$$\begin{vmatrix} 1 & 0 & 0  \\ 0 & i & 2i \\ 0 & 0 & 1 \end{vmatrix}$$
Donc $dim(A)=3$ et $dim(E_{i}(A))+dim(E_{-i}(A)) = 3$ donc diagonisable.

On a 
$$P = \begin{vmatrix} i & -i & 0 \\ -2 & 0 & 1 \\ 1 & 1 & 0 \end{vmatrix}$$
$$P^{-1} = \begin{vmatrix} -\frac{i}{2} & 0 & \frac{1}{2} \\ \frac{i}{2} & 0 & \frac{1}{2} \\ -i & 1 & 1 \end{vmatrix}$$
$$P^{-1}AP = \begin{vmatrix} -i & 0 & 0 \\ 0 & i & 0 \\ 0 & 0 & i \end{vmatrix}$$



\subsubsection*{Exercice 1.1 pour B}
On a 
$$\det(B -\lambda I) = \det\left(\begin{vmatrix} i-\lambda & 0 & 2 \\ 0 & 0-\lambda & 1 \\ 0 & -1 & 0-\lambda \end{vmatrix}\right) = (i-\lambda)(\lambda^2+1)$$

Calculons $\det(B -\lambda I) = 0$, $(i-\lambda)(\lambda^2+1)=0$, donc $sp(B)=\{i, -i\}$. 

\subsubsection*{Exercice 1.2 pour B}
Calculons
$$E_{i}(B) = ker(B -iI) = ker\left(\begin{vmatrix} i-i & 0 & 2 \\ 0 & 0-i & 1 \\ 0 & -1 & 0-i \end{vmatrix}\right)$$
Cherchons $\lambda_1,\lambda_2,\lambda_3$  tel que
$$\begin{vmatrix} 0 & 0 & 2 \\ 0 & -i & 1 \\ 0 & -1 & -i \end{vmatrix}.\begin{vmatrix} \lambda_1 & \lambda_2 & \lambda_3 \end{vmatrix} = 0$$

$$
\left\{ 
\begin{array}{l l}
0\lambda_{1} + 0\lambda_{2} + 0\lambda_{3} & = 0\\
0\lambda_{1} - i\lambda_{2} - \lambda_{3} & = 0\\
2\lambda_{1} + \lambda_{2} - i\lambda_{3} & = 0\\
\end{array}
\right. 
$$ 
Echelonnage
$$
\left\{ 
\begin{array}{l l}
2\lambda_{1} + \lambda_{2} - i\lambda_{3} & = 0 (L_1=L_3)\\
0\lambda_{1} - i\lambda_{2} - \lambda_{3} & = 0\\
0\lambda_{1} + 0\lambda_{2} + 0\lambda_{3} & = 0 (L_3=L_1)\\
\end{array}
\right. 
,
\left\{ 
\begin{array}{l l}
2\lambda_{1} + \lambda_{2} - i\lambda_{3} & = 0\\
\lambda_{3} & = -i\lambda_{2}\\
\end{array}
\right. 
,
\left\{ 
\begin{array}{l l}
2\lambda_{1} + \lambda_{2} - i(-i\lambda_{2}) & = 0\\
\lambda_{3} & = -i\lambda_{2}\\
\end{array}
\right. 
$$ 

Donc, en fixant $\lambda_2 = c_2$ on a $\lambda_1 = 0$, $\lambda_3 = ic_2$
$$ker(B-iI) = \{(0,1,i)\}$$ 
et $\dim(E_{i}(B))= \dim(ker(B-iI)) = 1$.

Calculons
$$E_{-i}(B) = ker(B +iI) = ker\left(\begin{vmatrix} i+i & 0 & 2 \\ 0 & 0+i & 1 \\ 0 & -1 & 0+i \end{vmatrix}\right)$$
Cherchons $\lambda_1,\lambda_2,\lambda_3$  tel que
$$\begin{vmatrix} 2i & 0 & 2 \\ 0 & i & 1 \\ 0 & -1 & i \end{vmatrix}.\begin{vmatrix} \lambda_1 & \lambda_2 & \lambda_3 \end{vmatrix} = 0$$

$$
\left\{ 
\begin{array}{l l}
2i\lambda_{1} + 0\lambda_{2} + 0\lambda_{3} & = 0\\
0\lambda_{1} + i\lambda_{2} - \lambda_{3} & = 0\\
2\lambda_{1} + \lambda_{2} + i\lambda_{3} & = 0\\
\end{array}
\right. 
$$ 

$$
\left\{ 
\begin{array}{l l}
\lambda_{1}  & = 0\\
i\lambda_{2}  & = \lambda_{3}\\
\lambda_{2}  & = - i\lambda_{3}\\
\end{array}
\right. 
$$ 

Donc, en fixant $\lambda_2 = c_2$ on a $\lambda_1 = 0$, $\lambda_3 = ic_2$
$$ker(B-iI) = \{(0,1,i)\}$$ 
et $\dim(E_{-i}(B))= \dim(ker(B+iI)) = 1$.


\subsubsection*{Exercice 1.3 pour B}
Calculons $dim(B)$.
$$\begin{vmatrix} -i & 0 & 2 \\ 0 & 0 & 1 \\ 0 & -1 & 0 \end{vmatrix}$$
Echelonnage
$$\begin{vmatrix} -i & 0 & 2  \\ 0 & 1 & 0 \\ 0 & 0 & 1 \end{vmatrix}$$
Donc $dim(B)=3$ et $dim(E_{i}(B))+dim(E_{-i}(B)) = 2$ donc pas diagonisable. 


\subsection*{Exercice 2}
Si $k$ est une valeur propre d'un endomorphisme $f:E \to E$ alors $\exists v \in E, v \neq 0_E, f(v) = k.v$. L'endomorphisme $f$ est surjective donc $im(f) = E$, et $dim(im\ f) = dim(E)\ [1]$. Preuve par l'absurde. Supposons que 0 soit une valeur propre de $f$, $\exists v \in E, v \neq 0_E, f(v) = 0.v = 0_E$. Donc $v \in ker(f)$. On a $dim(ker\ f) + dim(im\ f) = dim(E)$, donc par $[1]$, on a $dim(ker\ f) = 0$ ce qui contredit l'existence de $v$. 


\subsection*{Exercice 3}
\subsubsection*{Exercice 3.1}
Non, la matrice nulle est diagonale mais n'est pas inversible car son d\'eterminant est nul.

\subsubsection*{Exercice 3.2}
On a $Sp(A)=\{\lambda \in \C, \det(A-\lambda.Id_n) = 0\}$ et la matrice $A$ est inversible si son d\'eterminant est non nul. Le d\'eterminant d'une matrice diagonale est \'egal \`a $\prod_{i=1}^{n}{a_{ii}}$. La matrice $A-\lambda.Id_n$ est diagonale donc pour avoir son d\'eterminant \'egale \`a nul il faut que $\prod_{i=1}^{n}{a_{ii}-\lambda}$, donc $Sp(A)= \{a_{ii}\}$. D'un autre cot\'e, pour que $A$ soit inversible on a $\prod_{i=1}^{n}{a_{ii}} \neq 0$ (ie. $\forall i, a_{ii} \neq 0$). Donc, il faut que $0 \not \in Sp(A)$ .


\subsubsection*{Exercice 3.3}
Si la matrice $A$ est diagonalisable alors il existe une matrice $P$ tel que $P^{-1}AP$ soit diagonale. Donc d'apr\`es l'exercice pre\'ec\'edent, il faut que 


\subsection*{Exercice 4}
\subsubsection*{Exercice 4.1}
La base $\mathscr{B}$ a $n+1$ vecteurs ind\'ependents, dont sa dimension est $n+1$.

\subsubsection*{Exercice 4.2}
Soit 2 polynomes $P_a = \sum_{i=0}^{n}{a_iX^i}$ et $P_b= \sum_{i=0}^{n}{b_iX^i}$, on a $P_a+P_b = \sum_{i=0}^{n}{a_iX^i} + \sum_{i=0}^{n}{b_iX^i} = \sum_{i=0}^{n}{(a_i+b_i)X^i}$. On a $c.P_a = c.\sum_{i=0}^{n}{a_iX^i} = \sum_{i=0}^{n}{c.a_iX^i}$. On a $f(P) = X.\frac{d}{dX}P = X.\sum_{i=1}^{n}{ia_iX^{i-1}}$

On doit montrer que  
\begin{itemize}
\item $f(P_a+P_b)=f(P_a)+f(P_b)?$, $f(P_a+P_b) = X.\sum_{i=1}^{n}{i(a_i+b_i)X^{i-1}} = X.(\sum_{i=1}^{n}{i.a_iX^{i-1}} + \sum_{i=1}^{n}{i.b_iX^{i-1}}) = X.(\sum_{i=1}^{n}{i.a_iX^{i-1}}) + X.(\sum_{i=1}^{n}{i.b_iX^{i-1}}) = f(P_a)+f(P_b)$
\item $f(c.P_a) = c.f(P_a)?$, $f(c.P_a) = X.\sum_{i=1}^{n}{i(c.a_i)X^{i-1}} = c.X.\sum_{i=1}^{n}{i(a_i)X^{i-1}} = c.f(P_a)$
\end{itemize}
Donc, $f$ est une application lin\'eaire.

\subsubsection*{Exercice 4.3}
On a $f(P) = X.\sum_{i=1}^{n}{i.a_i.X^{i-1}} = \sum_{i=1}^{n}{i.a_i.X^{i}}$, donc

$$f(P) = [f]_{\mathscr{B}}. \begin{vmatrix} 1 \\ X \\ X^2 \\ \vdots \\ X^{n-1} \\ X^{n} \end{vmatrix} = \begin{vmatrix} 0 & 0 & 0 & \ldots & 0 & 0 \\ 0 & a_1 & 0 & \ldots & 0 & 0 \\ 0 & 0 & 2a_2 & \ldots & 0 & 0 \\ \vdots & \vdots & \vdots & \vdots & \vdots & \vdots \\ 0 & 0 & 0 & \ldots & (n-1).a_{n-1} & 0\\ 0 & 0 & 0 & \ldots & 0 & n.a_n  \end{vmatrix}.\begin{vmatrix} 1 \\ X \\ X^2 \\ \vdots \\ X^{n-1} \\ X^{n} \end{vmatrix}$$

La matrice $[f]_{\mathscr{B}}$ est une matrice diagonale dont $f$ est diagonalisable.

\subsubsection*{Exercice 4.4}
On a $Sp(f) = \{\lambda \in \C, \det([f]_{\mathscr{B}} -\lambda Id_n) = 0 \} $. Donc

$$[f]_{\mathscr{B}} -\lambda Id_n = \begin{vmatrix} -\lambda & 0 & 0 & \ldots & 0 & 0 \\ 0 & a_1-\lambda & 0 & \ldots & 0 & 0 \\ 0 & 0 & 2a_2-\lambda & \ldots & 0 & 0 \\ \vdots & \vdots & \vdots & \vdots & \vdots & \vdots \\ 0 & 0 & 0 & \ldots & (n-1).a_{n-1}-\lambda & 0\\ 0 & 0 & 0 & \ldots & 0 & n.a_n-\lambda  \end{vmatrix}$$

$$\det([f]_{\mathscr{B}} -\lambda Id_n) = \prod_{i=0}^{n} i.a_i.X^i-\lambda = \prod_{i=0}^{n}(i.a_i.X^i) -(n+1)\lambda$$ 

Mais $\prod_{i=0}^{n}(i.a_i.X^i) = 0$, donc $Sp(f) = \emptyset$. 

QED

\end{document}

