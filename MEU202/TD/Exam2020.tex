\documentclass[]{book}

%These tell TeX which packages to use.
\usepackage{array,epsfig}
\usepackage{amsmath}
\usepackage{amsfonts}
\usepackage{amssymb}
\usepackage{amsxtra}
\usepackage{amsthm}
\usepackage{mathrsfs}
\usepackage{color}

%Here I define some theorem styles and shortcut commands for symbols I use often
\theoremstyle{definition}
\newtheorem{defn}{Definition}
\newtheorem{thm}{Theorem}
\newtheorem{cor}{Corollary}
\newtheorem*{rmk}{Remark}
\newtheorem{lem}{Lemma}
\newtheorem*{joke}{Joke}
\newtheorem{ex}{Example}
\newtheorem*{soln}{Solution}
\newtheorem{prop}{Proposition}

\newcommand{\lra}{\longrightarrow}
\newcommand{\ra}{\rightarrow}
\newcommand{\surj}{\twoheadrightarrow}
\newcommand{\graph}{\mathrm{graph}}
\newcommand{\bb}[1]{\mathbb{#1}}
\newcommand{\Z}{\bb{Z}}
\newcommand{\Q}{\bb{Q}}
\newcommand{\R}{\bb{R}}
\newcommand{\C}{\bb{C}}
\newcommand{\N}{\bb{N}}
\newcommand{\M}{\mathbf{M}}
\newcommand{\E}{\mathscr{E}}
\newcommand{\m}{\mathbf{m}}
\newcommand{\MM}{\mathscr{M}}
\newcommand{\HH}{\mathscr{H}}
\newcommand{\Om}{\Omega}
\newcommand{\Ho}{\in\HH(\Om)}
\newcommand{\bd}{\partial}
\newcommand{\del}{\partial}
\newcommand{\bardel}{\overline\partial}
\newcommand{\textdf}[1]{\textbf{\textsf{#1}}\index{#1}}
\newcommand{\img}{\mathrm{img}}
\newcommand{\ip}[2]{\left\langle{#1},{#2}\right\rangle}
\newcommand{\inter}[1]{\mathrm{int}{#1}}
\newcommand{\exter}[1]{\mathrm{ext}{#1}}
\newcommand{\cl}[1]{\mathrm{cl}{#1}}
\newcommand{\ds}{\displaystyle}
\newcommand{\vol}{\mathrm{vol}}
\newcommand{\cnt}{\mathrm{ct}}
\newcommand{\osc}{\mathrm{osc}}
\newcommand{\LL}{\mathbf{L}}
\newcommand{\UU}{\mathbf{U}}
\newcommand{\support}{\mathrm{support}}
\newcommand{\AND}{\;\wedge\;}
\newcommand{\OR}{\;\vee\;}
\newcommand{\Oset}{\varnothing}
\newcommand{\st}{\ni}
\newcommand{\wh}{\widehat}

%Pagination stuff.
\setlength{\topmargin}{-.3 in}
\setlength{\oddsidemargin}{0in}
\setlength{\evensidemargin}{0in}
\setlength{\textheight}{9.in}
\setlength{\textwidth}{6.5in}
\pagestyle{empty}


\pdfinfo{
   /Author (Myself)
   /Title  (Exam 2020 - MEU202)
}

\begin{document}

\subsection*{Rappel de cours}



\newpage

\subsection*{Exercice 3}
\subsubsection*{Exercice 3.1}
Les matrices A et B commutent donc $A.B = B.A$. Calculons $B.A^2$
$$
B.A^2 = B.A.A = (B.A).A = (A.B).A = A.B.A = A.(B.A) = A.(A.B) = A.A.B = A^2B
$$
Comme $B.A^2 = A^2.B$ les matrices $A^@$ et $B$ commutent.

\subsubsection*{Exercice 3.2}
$$
X_A(\lambda) = (a-\lambda)(d-\lambda)-bc = \lambda^2 -\lambda(a+d) + ad-bc
$$

\subsubsection*{Exercice 3.3}
$$
A^2 = \begin{vmatrix} a^2+bc & ab+bd \\ ac+cd & bc+d^2 \end{vmatrix}
$$

$A \in vect(A^2, I^2)$ si il existe $\lambda_1$ et $\lambda_2$ tel que $A = \lambda_1A^2 + \lambda_2Id^2$
4 \'equations a v\'erifier
$$
\left\{
\begin{array}{l l}
a = \lambda_1(a^2+bc) + \lambda_2 & \\
b = \lambda_1(ab+bd) & \implies \lambda_1(a+d) = 1\\
c = \lambda_1(ac+cd) & \implies \lambda_1(a+d) = 1\\
d = \lambda_1(bc+d^2) + \lambda_2 & \\
\end{array}
\right.
$$

On a $\lambda_1 = \frac{1}{a+d}$ car $a+d \neq 0$. Calculons $\lambda_2$
$$
a = \frac{1}{a+d}(a^2+bc) + \lambda_2
$$
$$
\frac{a^2+ad}{a+d} = \frac{a^2+bc}{a+d} + \frac{\lambda_2(a+d)}{a+d}
$$
$$
a^2+ad = a^2+bc + \lambda_2(a+d)
$$
$$
\lambda_2 = \frac{ad - bc}{a+d}
$$

V\'erifions $\lambda_2$
$$
d = \lambda_1(bc+d^2) + \lambda_2 = \frac{bc+d^2}{a+d} + \frac{ad - bc}{a+d} = \frac{d(a+d)}{a+d} = d
$$

$\lambda_1$ et $\lambda_2$ existent, donc $A \in vect(A^2, I^2)$.

\subsubsection*{Exercice 3.4}
$$
AB = (\lambda_1A^2 + \lambda_2Id^2)B = \lambda_1A^2B + \lambda_2Id^2B = \lambda_1BA^2 + \lambda_2Id^2B = \lambda_1BA^2 + \lambda_2BId^2 = B(\lambda_1A^2 + \lambda_2Id^2) = BA
$$

\subsection*{Exercice 5}
\subsubsection*{Exercice 5.1}
La dimension de la matrice $\M_2(\C)$ est $d=4$ car sa base est:
$$U_1 = \begin{vmatrix} 1 & 0 \\ 0 & 0 \end{vmatrix} . U_2 = \begin{vmatrix} 0 & 1 \\ 0 & 0 \end{vmatrix} . U_3 = \begin{vmatrix} 0 & 0 \\ 1 & 0 \end{vmatrix} . U_4 = \begin{vmatrix} 0 & 0 \\ 0 & 1 \end{vmatrix}$$

Par d\'efinition, la dimension du $\R-ev$ est $2d = 8$.

\subsubsection*{Exercice 5.1}
Par d\'efinition la base dans l'espace vectoriel $\R-ev$ est $(U_1, iU_1, \ldots, U_4, iU_4)$.
Donc la base est
$$U_1 = \begin{vmatrix} 1 & 0 \\ 0 & 0 \end{vmatrix} . U_2 = \begin{vmatrix} 0 & 1 \\ 0 & 0 \end{vmatrix} . U_3 = \begin{vmatrix} 0 & 0 \\ 1 & 0 \end{vmatrix} . U_4 = \begin{vmatrix} 0 & 0 \\ 0 & 1 \end{vmatrix}$$
$$iU_1 = \begin{vmatrix} i & 0 \\ 0 & 0 \end{vmatrix} . iU_2 = \begin{vmatrix} 0 & i \\ 0 & 0 \end{vmatrix} . iU_3 = \begin{vmatrix} 0 & 0 \\ i & 0 \end{vmatrix} . iU_4 = \begin{vmatrix} 0 & 0 \\ 0 & i \end{vmatrix}$$

\subsection*{Exercice 6}
\subsubsection*{Exercice 6.1}
On a H de la forme
$$
H = \begin{vmatrix} a_{11} + ib_{11} & a_{12} + ib_{12} \\ a_{21} + ib_{21} & -a_{11} - ib_{11} \end{vmatrix}
$$

Une base de H est 
$$U_1 = \begin{vmatrix} 1 & 0 \\ 0 & -1 \end{vmatrix} . U_2 = \begin{vmatrix} 0 & 1 \\ 0 & 0 \end{vmatrix} . U_3 = \begin{vmatrix} 0 & 0 \\ 1 & 0 \end{vmatrix}$$

La dimension de H est 3.

\subsubsection*{Exercice 6.2}
On a 
$$E_{11} = \begin{vmatrix} 1 & 0 \\ 0 & 0 \end{vmatrix} . E_{12} = \begin{vmatrix} 0 & 1 \\ 0 & 0 \end{vmatrix} . E_{21} = \begin{vmatrix} 0 & 0 \\ 1 & 0 \end{vmatrix}. E_{22} = \begin{vmatrix} 0 & 0 \\ 0 & 1 \end{vmatrix}$$

Donc
$$
E_1 = E_{11}-E_{22} = \begin{vmatrix} 1 & 0 \\ 0 & 0 \end{vmatrix} - \begin{vmatrix} 0 & 0 \\ 0 & 1 \end{vmatrix} =  \begin{vmatrix} 1 & 0 \\ 0 & -1 \end{vmatrix}
$$
et $Tr(E_{11}-E_{22}) = 1 + (-1) = 0$, matrice diagonale.

$$
E_2 = E_{12}+E_{21} = \begin{vmatrix} 0 & 1 \\ 0 & 0 \end{vmatrix} - \begin{vmatrix} 0 & 0 \\ 1 & 0 \end{vmatrix} =  \begin{vmatrix} 0 & 1 \\ 1 & 0 \end{vmatrix} = U_2 + U_3
$$
et $Tr(E_{12}+E_{21}) = 0 + 0 = 0$, $X_m(\lambda) = \lambda^2-1$, $S_p(E_{12}+E_{21}) = \{1, -1\}$.... 

$$
E_3 = E_{11}-E_{22}+E_{12} = \begin{vmatrix} 1 & 0 \\ 0 & 0 \end{vmatrix} - \begin{vmatrix} 0 & 0 \\ 0 & 1 \end{vmatrix} + \begin{vmatrix} 0 & 1 \\ 0 & 0 \end{vmatrix} =  \begin{vmatrix} 1 & 1 \\ 0 & -1 \end{vmatrix} 
$$
et $Tr(E_{11}-E_{22}+E_{12}) = 1 + (-1) = 0$, $X_m(\lambda) = \lambda^2-1$, $S_p(E_{11}-E_{22}+E_{12}) = \{1, -1\}$.... 

Donc les 3 matrices sont dans H.

\subsubsection*{Exercice 6.3}
Calcul si $(E_1, E_2, E_3)$ est une base de H. 

\subsection*{Exercice 7}
\subsubsection*{Exercice 7.1}
On a $A^2 = A$ La matrice $A$ est diagonalisable si il existe deux matrices $T$ et $D$ tel que $A=TDT^{-1}$. Il faut trouver les valeurs propres $\lambda$ tel que $Ax = \lambda x$.
$$
Ax = A^2x = A(Ax) = A(\lambda x) = \lambda A(x) = \lambda^2 x = \lambda x
$$

Les 2 solutions sont $\lambda = 0$ ou $\lambda = 1$. Les espaces propres sont $E_0 = \{x \in M_n(\R)|Ax = 0\}$ et $E_1 = \{x \in M_n(\R)|Ax = x\}$. Il maintenant trouver la dimension de $E_0$ et $E_1$ et montrer que leur somme est $n$ ou que 



\subsection*{Exercice 8}
Si $a$ et $b$ sont premiers entre eux, que vaux $\gcd(a+b, a-b)$?\\

On a $\gcd(a,b) = 1$. Supposons $d = \gcd(a+b, a-b)$. Donc $d | a+b$ et $d|a-b$. Aussi, $a+b = k_1d$ et $a-b = k_2d$. On obtient 2 \'equations en additionnant et soustrayant. $2a = k_1d+k_2d$ et $2b = k_1d-k_2d$. Donc $d$ divise $2a$ et $2b$. D'un autre cot\'e, on a $\gcd(2a,2b) = 2 \gcd(a,b) = 2$. Par cons\'equent, le prlus grand diviseur commun de $2a$ et $2b$ est 2 et $d$ est un diviseur de $2a$ et $2b$. Les seules valeurs possible pour $d$ sont 1 et 2.
$$\gcd(a,b) = 1 \implies \gcd(a+b, a-b) = 1 \text{ ou } 2$$

Si $a$ et $b$ sont tous les 2 pairs (ou impairs) alors $\gcd(a+b, a-b) = 2$ car $a+b$ et $a-b$ sont pairs, si $a$ et $b$ ne sont pas tous les 2 pairs (ou impairs) alors $\gcd(a+b, a-b) = 1$ car $a+b$ et $a-b$ sont pair/impair ou impair/pair.



\end{document}

