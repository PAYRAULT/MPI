\documentclass[]{book}

%These tell TeX which packages to use.
\usepackage{array,epsfig}
\usepackage{amsmath}
\usepackage{amsfonts}
\usepackage{amssymb}
\usepackage{amsxtra}
\usepackage{amsthm}
\usepackage{mathrsfs}
\usepackage{color}

%Here I define some theorem styles and shortcut commands for symbols I use often
\theoremstyle{definition}
\newtheorem{defn}{Definition}
\newtheorem{thm}{Theorem}
\newtheorem{cor}{Corollary}
\newtheorem*{rmk}{Remark}
\newtheorem{lem}{Lemma}
\newtheorem*{joke}{Joke}
\newtheorem{ex}{Example}
\newtheorem*{soln}{Solution}
\newtheorem{prop}{Proposition}

\newcommand{\lra}{\longrightarrow}
\newcommand{\ra}{\rightarrow}
\newcommand{\surj}{\twoheadrightarrow}
\newcommand{\graph}{\mathrm{graph}}
\newcommand{\bb}[1]{\mathbb{#1}}
\newcommand{\Z}{\bb{Z}}
\newcommand{\Q}{\bb{Q}}
\newcommand{\R}{\bb{R}}
\newcommand{\C}{\bb{C}}
\newcommand{\N}{\bb{N}}
\newcommand{\M}{\mathbf{M}}
\newcommand{\m}{\mathbf{m}}
\newcommand{\MM}{\mathscr{M}}
\newcommand{\HH}{\mathscr{H}}
\newcommand{\Om}{\Omega}
\newcommand{\Ho}{\in\HH(\Om)}
\newcommand{\bd}{\partial}
\newcommand{\del}{\partial}
\newcommand{\bardel}{\overline\partial}
\newcommand{\textdf}[1]{\textbf{\textsf{#1}}\index{#1}}
\newcommand{\img}{\mathrm{img}}
\newcommand{\ip}[2]{\left\langle{#1},{#2}\right\rangle}
\newcommand{\inter}[1]{\mathrm{int}{#1}}
\newcommand{\exter}[1]{\mathrm{ext}{#1}}
\newcommand{\cl}[1]{\mathrm{cl}{#1}}
\newcommand{\ds}{\displaystyle}
\newcommand{\vol}{\mathrm{vol}}
\newcommand{\cnt}{\mathrm{ct}}
\newcommand{\osc}{\mathrm{osc}}
\newcommand{\LL}{\mathbf{L}}
\newcommand{\UU}{\mathbf{U}}
\newcommand{\support}{\mathrm{support}}
\newcommand{\AND}{\;\wedge\;}
\newcommand{\OR}{\;\vee\;}
\newcommand{\Oset}{\varnothing}
\newcommand{\st}{\ni}
\newcommand{\wh}{\widehat}

%Pagination stuff.
\setlength{\topmargin}{-.3 in}
\setlength{\oddsidemargin}{0in}
\setlength{\evensidemargin}{0in}
\setlength{\textheight}{9.in}
\setlength{\textwidth}{6.5in}
\pagestyle{empty}



\begin{document}

\subsection*{Rappel de cours}
Une matrice $n \times n$ $A$ est diagonalisable ($A = PDP^{-1}$0 si:
\begin{itemize}
\item Elle a n vecteurs propres lin\'eairement ind\'ependants, condition pour avoir une matrice $P$ form\'ee des vecteurs propores en colonne qui est inversible. 
\item Elle a n valeurs propres distinctes, car n valeurs propres g\'en\`erent n vecteurs propres lin\'eairement ind\'ependants
\item $\sum{dim\ E_{sp_n}(A)} = n$
\item pour chaque valeur propre $sp$, on a $dim\ E_{sp}(A) = multiplicite\ sp$. La multiplicit\'e de $sp$ le nombre de racine de $sp$.
\item si $\chi_{A}(X) = P(X)$ et $P(X)$ est un polynome scind\'e (ie $P(X) = C(X-A_1)(X-A_2)\ldots(X-A_{m-1})(X-A_m)$).
\item si $\chi_{A}(X) = P(X)$ et $P(A)=0$.
\end{itemize}


\newpage
\subsection*{Exercice 4}

On cherche les $\lambda$ tel que
$$ \begin{vmatrix} 1 & 1 & 1 & \ldots & 1 \\ 1 & 1 & 1 & \ldots & 1 \\ 1 & 1 & 1 & \ldots & 1 \\ \vdots & \vdots & \vdots & \ldots & \vdots \\ 1 & 1 & 1 & \ldots & 1 \end{vmatrix} . \begin{vmatrix} x_1 \\ x_2 \\ x_3 \\ \vdots \\ x_n \end{vmatrix} = \begin{vmatrix} \lambda x_1 \\ \lambda x_2 \\ \lambda x_3 \\ \vdots \\ \lambda x_n \end{vmatrix}$$

$$
\left\{
\begin{array}{ll}
x_1 + x_2 + x_3 + \ldots + x_n & = \lambda x_1 \\
x_1 + x_2 + x_3 + \ldots + x_n & = \lambda x_2 \\
x_1 + x_2 + x_3 + \ldots + x_n & = \lambda x_3 \\
\ldots  & = \lambda x_i \\
x_1 + x_2 + x_3 + \ldots + x_n & = \lambda x_n \\
\end{array}
\right.
$$

Tous les $x_i$ ne sont pas \'egale \`a 0. Donc une premi\`ere solution est $\lambda = 0$ avec une multiplicit\'e de $n-1$, car cela correspond \`a un syst\`eme de $n$ \'equations et $n$ inconnues.\\
La seconde solution est lorsque tous les $x_i$ sont \'egaux alors on a $n$ \'equations $n x_i = \lambda x_i$, d'o\`u $\lambda = n$.\\

Par cons\'equent, $E_0(A) = \{(-1,1,0,0,\ldots,0),(-1,0,1,0,\ldots,0), (-1,0,0,1,\ldots,0), \dots (-1,0,0,0,\ldots,1)\}$ et $E_1(A) = \{(1,1,1,1,\dots,1)\}$.\\
Donc $dim\ E_0(A) = n-1$ et $dim\ E_1(A) = 1$. La matrice est diagonalisable. La matrice $D$ la matrice diagonale compos\'e des valeurs propres de $A$.

$$D=\begin{vmatrix} n & 0 & 0 & \ldots & 0 & 0 \\ 0 & 0 & 0 & \ldots & 0 & 0 \\ 0 & 0 & 0 & \ldots & 0 & 0 \\ \vdots & \vdots & \vdots & \ldots & \vdots & \vdots \\ 0 & 0 & 0 & \ldots & 0 & 0 \end{vmatrix}$$

On a $D=B$ et $A=PDP^{-1}$ donc les matrcies $A$ et $B$ sont semblables (ie. $A=PBP^{-1}$).

\subsection*{Exercice 7}
\subsubsection*{Exercice 7.1}
On cherche les $\lambda$ tel que
$$ \begin{vmatrix} t & 1 & 1 & \ldots & 1 \\ 1 & t & 1 & \ldots & 1 \\ 1 & 1 & t & \ldots & 1 \\ \vdots & \vdots & \vdots & \ldots & \vdots \\ 1 & 1 & 1 & \ldots & t \end{vmatrix} . \begin{vmatrix} x_1 \\ x_2 \\ x_3 \\ \vdots \\ x_n \end{vmatrix} = \begin{vmatrix} \lambda x_1 \\ \lambda x_2 \\ \lambda x_3 \\ \vdots \\ \lambda x_n \end{vmatrix}$$

$$
\left\{
\begin{array}{ll}
tx_1 + x_2 + x_3 + \ldots + x_n & = \lambda x_1 \\
x_1 + tx_2 + x_3 + \ldots + x_n & = \lambda x_2 \\
x_1 + x_2 + tx_3 + \ldots + x_n & = \lambda x_3 \\
\ldots  & = \lambda x_i \\
x_1 + x_2 + x_3 + \ldots + tx_n & = \lambda x_n \\
\end{array}
\right.
$$

En prenant $\lambda = t-1$ on a 
$$
\left\{
\begin{array}{ll}
tx_1 + x_2 + x_3 + \ldots + x_n & = (t-1) x_1 \\
x_1 + tx_2 + x_3 + \ldots + x_n & = (t-1) x_2 \\
x_1 + x_2 + tx_3 + \ldots + x_n & = (t-1) x_3 \\
\ldots  & = \lambda (t-1)x_i \\
x_1 + x_2 + x_3 + \ldots + tx_n & = (t-1) x_n \\
\end{array}
\right.
$$

$$
\left\{
\begin{array}{ll}
x_1 + x_2 + x_3 + \ldots + x_n & = 0 \\
x_1 + x_2 + x_3 + \ldots + x_n & = 0 \\
x_1 + x_2 + x_3 + \ldots + x_n & = 0 \\
\ldots  & = 0 \\
x_1 + x_2 + x_3 + \ldots + x_n & = 0 \\
\end{array}
\right.
$$

Syst\`eme de $n$ \'equations \`a $n$ inconnues qui a donc $n$ solutions. La multiplicit\'e de $\lambda$ est $n$. On a $E_{t-1}(A) = \{(-1,1,0,0,\ldots,0),(-1,0,1,0,\ldots,0), (-1,0,0,1,\ldots,0), \dots (-1,0,0,0,\ldots,1)\}$ et $dim\ E_{t-1}(A) = n$. 

\subsubsection*{Exercice 7.2}
$$Tr(a)=\sum_{i=1}^{n}a_{ii} = n.t$$ 
Relation avec ls spectre ???

\subsubsection*{Exercice 7.3}
Oui car $dim\ E_{t-1}(A) = n$.

\subsubsection*{Exercice 7.4}
On a $A.A^(-1) = I_n$ et $A$ diagonalisable. Ceci fait $PDP^{-1}A^{-1} = I_n$ donc $A^{-1} = P^{-1}DP$. Donc $A$ inversible si $D$ est inversible.
 
La matrice $D$ est la matrice diagonale compos\'e des valeurs propres de $A$. 
$$D=\begin{vmatrix} t-1 & 0 & 0 & \ldots & 0 & 0 \\ 0 & t-1 & 0 & \ldots & 0 & 0 \\ 0 & 0 & t-1 & \ldots & 0 & 0 \\ \vdots & \vdots & \vdots & \ldots & \vdots & \vdots \\ 0 & 0 & 0 & \ldots & t-1 & 0 \\ 0 & 0 & 0 & \ldots & 0 & t-1 \end{vmatrix}$$

Matrice $D$ est inversible si $a_{ii}$ sont tous diff\'erents de 0. Donc il faut que $t \neq 1$.


\subsection*{Exercice 8}
\subsubsection*{Exercice 8.a}
On a 2 endomorphismes $u$ et $v$ qui commutent (ie. $u \circ v = c \circ u$). $\lambda$ une valeur propre de $v$ et $E_{\lambda}(v) = \{p \in E, v(p) = \lambda p\}$. Donc calculons $\lambda u(p)$.  $\lambda u(p) = u(\lambda p)$ car $u$ est un endomorphime. $\lambda u(p) = u(v(p))$ car $p$ est un vecteur propre de $v$ de valeur propre $\lambda$. $\lambda u(p) = v(u(p))$. On en d\'eduit que $u(p) \in E_{\lambda}(v)$. Ce qui montre que $E_{\lambda}(v)$ est stable par $u$ (ie. $\forall p \in E_{\lambda}(v), u(p) \in E_{\lambda}(v)$).


\subsubsection*{Exercice 8.b}
Pas compris la question

\subsubsection*{Exercice 8.c}
Si $u$ et $v$ sont diagonalisables donc ilv existe $E_{\lambda_{u}}(u) = \{p_u, u(p_u) = \lambda_{u} p_u\}$ et $E_{\lambda_{v}}(v) = \{p_v, v(p_v) = \lambda_{v} p_v\}$. Donc
$$E_{\lambda u}(u) = \{p_u, u(p_u) = \lambda p_u\}$$
$$E_{\lambda u}(u) = \{p_u, \lambda_{v} u(p_u) = \lambda_{v} \lambda_{u} p_u\}$$
$$E_{\lambda u}(u) = \{p_u, u(\lambda_{v} p_u) = \lambda_{v} \lambda_{u} p_u\}$$
???



QED

\end{document}

