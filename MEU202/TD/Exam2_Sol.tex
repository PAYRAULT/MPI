\documentclass[]{book}

%These tell TeX which packages to use.
\usepackage{array,epsfig}
\usepackage{amsmath}
\usepackage{amsfonts}
\usepackage{amssymb}
\usepackage{amsxtra}
\usepackage{amsthm}
\usepackage{mathrsfs}
\usepackage{color}

%Here I define some theorem styles and shortcut commands for symbols I use often
\theoremstyle{definition}
\newtheorem{defn}{Definition}
\newtheorem{thm}{Theorem}
\newtheorem{cor}{Corollary}
\newtheorem*{rmk}{Remark}
\newtheorem{lem}{Lemma}
\newtheorem*{joke}{Joke}
\newtheorem{ex}{Example}
\newtheorem*{soln}{Solution}
\newtheorem{prop}{Proposition}

\newcommand{\lra}{\longrightarrow}
\newcommand{\ra}{\rightarrow}
\newcommand{\surj}{\twoheadrightarrow}
\newcommand{\graph}{\mathrm{graph}}
\newcommand{\bb}[1]{\mathbb{#1}}
\newcommand{\Z}{\bb{Z}}
\newcommand{\Q}{\bb{Q}}
\newcommand{\R}{\bb{R}}
\newcommand{\C}{\bb{C}}
\newcommand{\N}{\bb{N}}
\newcommand{\M}{\mathbf{M}}
\newcommand{\E}{\mathscr{E}}
\newcommand{\m}{\mathbf{m}}
\newcommand{\MM}{\mathscr{M}}
\newcommand{\HH}{\mathscr{H}}
\newcommand{\Om}{\Omega}
\newcommand{\Ho}{\in\HH(\Om)}
\newcommand{\bd}{\partial}
\newcommand{\del}{\partial}
\newcommand{\bardel}{\overline\partial}
\newcommand{\textdf}[1]{\textbf{\textsf{#1}}\index{#1}}
\newcommand{\img}{\mathrm{img}}
\newcommand{\ip}[2]{\left\langle{#1},{#2}\right\rangle}
\newcommand{\inter}[1]{\mathrm{int}{#1}}
\newcommand{\exter}[1]{\mathrm{ext}{#1}}
\newcommand{\cl}[1]{\mathrm{cl}{#1}}
\newcommand{\ds}{\displaystyle}
\newcommand{\vol}{\mathrm{vol}}
\newcommand{\cnt}{\mathrm{ct}}
\newcommand{\osc}{\mathrm{osc}}
\newcommand{\LL}{\mathbf{L}}
\newcommand{\UU}{\mathbf{U}}
\newcommand{\support}{\mathrm{support}}
\newcommand{\AND}{\;\wedge\;}
\newcommand{\OR}{\;\vee\;}
\newcommand{\Oset}{\varnothing}
\newcommand{\st}{\ni}
\newcommand{\wh}{\widehat}

%Pagination stuff.
\setlength{\topmargin}{-.3 in}
\setlength{\oddsidemargin}{0in}
\setlength{\evensidemargin}{0in}
\setlength{\textheight}{9.in}
\setlength{\textwidth}{6.5in}
\pagestyle{empty}


\pdfinfo{
   /Author (Myself)
   /Title  (DM1 - MEU202)
}

\begin{document}

\subsection*{Rappel de cours}



\newpage
\subsection*{Exercice 1}
\subsubsection*{Exercice 1.1}
Calculons $\det(A-\lambda.I)$
$$\det \left( \begin{vmatrix} 1 - \lambda & 0 & 0 \\ a & -2 - \lambda & 3 \\  1 & -1 & 2 -\lambda \end{vmatrix}\right) $$
$$= (-2-\lambda)((2-\lambda)(1-\lambda)-0*1) - (-1)((1-\lambda)*3 -0*a) = -(4-\lambda^2)(1-\lambda)+3(1-\lambda) = (1-\lambda)(-1+\lambda^2)$$
donc
$$Sp(A) = \{1, -1\}$$

\subsubsection*{Exercice 1.2}
D\'eterminons les vecteurs propres de $A$.\\
Calculons 
$$E_{1}(A)=ker(A-I) = ker\left( \begin{vmatrix} 1 - 1 & 0 & 0 \\ a & -2 - 1 & 3 \\  1 & -1 & 2 - 1 \end{vmatrix} \right)$$

Cherchons $x,y,z$ tel que
$$\begin{vmatrix} 0 & 0 & 0 \\ a & -3 & 3 \\  1 & -1 & 1 \end{vmatrix} . \begin{vmatrix} x \\ y \\  z \end{vmatrix} = 0$$

$$
\left\{ 
\begin{array}{l l}
0x + 0y + 0z & = 0\\
ax - 3y + 3z & = 0\\
x - y + z & = 0\\
\end{array}
\right. 
$$ 

$$
\left\{ 
\begin{array}{l l}
x  + z & = y\\
ax - 3x -3z + 3z & = 0\\
0x + 0y + 0z & = 0\\
\end{array}
\right. 
$$ 

$$
\left\{ 
\begin{array}{l l}
x  + z & = y\\
x(a - 3) & = 0\\
0x + 0y + 0z & = 0\\
\end{array}
\right. 
$$ 

En fixant $x=c_1$ et $z=c_2$, on a
$$
E_{1}(A)=
\left\{ 
\begin{array}{l l}
\{(1,1,0),(0,1,1)\} & a = 3\\
\{0,1,1) & a \neq 3\\
\end{array}
\right. 
$$ 

On a $dim\ E_{1}(A)= 2$ lorsque $a=3$ et $dim\ E_{1}(A)= 1$ lorsque $a \neq 3$.\\


Calculons 
$$E_{-1}(A)=ker(A+I) = ker\left( \begin{vmatrix} 1 + 1 & 0 & 0 \\ a & -2 + 1 & 3 \\  1 & -1 & 2 + 1 \end{vmatrix} \right)$$

Cherchons $x,y,z$ tel que
$$\begin{vmatrix} 2 & 0 & 0 \\ a & -1 & 3 \\  1 & -1 & 3 \end{vmatrix} . \begin{vmatrix} x \\ y \\  z \end{vmatrix} = 0$$

$$
\left\{ 
\begin{array}{l l}
2x + 0y + 0z & = 0\\
ax - y + 3z & = 0\\
x - y + 3z & = 0\\
\end{array}
\right. 
$$ 

$$
\left\{ 
\begin{array}{l l}
x  & = 0\\
3z & = y\\
\end{array}
\right. 
$$ 

En fixant $x=0$ et $z=c_2$, on a
$$
E_{-1}(A)=\{(0,3,1)\}
$$ 

On a $dim\ E_{-1}(A)= 1$.\\

Pour que $A$ soit diagonalisable il faut que $dim\ A = dim\ E_{1} + dim\ E_{-1}$, donc on a $a = 3$.

La matrice
$$P = \begin{vmatrix} 1 & 0 & 0 \\ 1 & 1 & 3 \\  0 & 1 & 1 \end{vmatrix} $$

$$P^{-1} = \begin{vmatrix}1&0&0\\ \:\:\:\frac{1}{2}&-\frac{1}{2}&\frac{3}{2}\\ \:\:\:-\frac{1}{2}&\frac{1}{2}&-\frac{1}{2}\end{vmatrix}$$

$$D = P^{-1}AP = \begin{vmatrix}1&0&0\\ \:\:\:\frac{1}{2}&-\frac{1}{2}&\frac{3}{2}\\ \:\:\:-\frac{1}{2}&\frac{1}{2}&-\frac{1}{2}\end{vmatrix}.\begin{vmatrix}1&0&0\\ \:3&-2&3\\ \:1&-1&2\end{vmatrix}. \begin{vmatrix} 1 & 0 & 0 \\ 1 & 1 & 3 \\  0 & 1 & 1 \end{vmatrix} = \begin{vmatrix}1&0&0\\ 0&1&0\\ 0&0&-1\end{vmatrix} $$

QED


\subsection*{Exercice 2}
\subsubsection*{Exercice 2.1}
Calculons $\det(A-\lambda.I)$
$$\det \left( \begin{vmatrix} 2 - \lambda & 0 & 0 \\ 0 & 0 - \lambda & -1 \\  0 & 1 & 0 -\lambda \end{vmatrix}\right) $$
$$= (-\lambda)((2-\lambda)(-\lambda)-0*0) - (1)((2-\lambda)(-1) -0*0) = (2-\lambda)(\lambda^2+1)$$
donc
$$Sp(A) = \{2\}\ dans\ \R, et\  Sp(A) = \{2,i,-i\}\ dans\ \C,$$

\subsubsection*{Exercice 2.2}
D\'eterminons les vecteurs propres de $A$ dans $\R$.\\
Calculons 
$$E_{2}(A)=ker(A-2I) = ker\left( \begin{vmatrix} 2 - 2 & 0 & 0 \\ 0 & 0-2  & -1 \\  0 & 1 & 0 - 2 \end{vmatrix} \right)$$

Cherchons $x,y,z$ tel que
$$\begin{vmatrix} 0 & 0 & 0 \\ 0 & -2 & -1 \\  0 & 1 & -2 \end{vmatrix} . \begin{vmatrix} x \\ y \\  z \end{vmatrix} = 0$$

$$
\left\{ 
\begin{array}{l l}
0x  + 0y + 0z & = 0\\
0x  -2y - z & = 0\\
0x +y -2z & = 0\\
\end{array}
\right. 
$$ 

$$
\left\{ 
\begin{array}{l l}
0x  + 0y + 0z & = 0\\
z & = 2y\\
-3y & = 0\\
\end{array}
\right. 
$$ 

En fixant $x=c_1$ on a 
$$E_{2}(A)=ker(A-2I) = \{(1,0,0)\}$$
Donc dans $\R$, $dim\ Sp(A) = 1$, et $dim\ A = 3$ donc la matrice n'est pas diagonalisable dans $\R$.


\subsubsection*{Exercice 2.2}
D\'eterminons les vecteurs propres de $A$ dans $\C$.\\
De l'exerce pr\'ec\'edent, on a $E_{2}(A)=ker(A-2I) = \{(1,0,0\}$.\\
Calculons 
$$E_{i}(A)=ker(A-i.I) = ker\left( \begin{vmatrix} 2 - i & 0 & 0 \\ 0 & 0-i  & 1 \\  0 & -1 & 0 - i \end{vmatrix} \right)$$
Cherchons $x,y,z$ tel que
$$\begin{vmatrix} 2-i & 0 & 0 \\ 0 & -i & 1 \\  0 & -1 & -i \end{vmatrix} . \begin{vmatrix} x \\ y \\  z \end{vmatrix} = 0$$

$$
\left\{ 
\begin{array}{l l}
(2-i)x  + 0y + 0z & = 0\\
0x  -iy - z & = 0\\
0x +y -iz & = 0\\
\end{array}
\right. 
$$ 

$$
\left\{ 
\begin{array}{l l}
x & = 0\\
z & = -iy \\
z & = \frac{y}{i} = -iy\\
\end{array}
\right. 
$$ 

En fixant $x=0$ et $y=c_1$ on a
$$E_{i}(A)=ker(A-2I) = \{(0,1,-i)\}$$

Calculons 
$$E_{-i}(A)=ker(A+i.I) = ker\left( \begin{vmatrix} 2 + i & 0 & 0 \\ 0 & 0+i  & 1 \\  0 & -1 & 0 + i \end{vmatrix} \right)$$
Cherchons $x,y,z$ tel que
$$\begin{vmatrix} 2+i & 0 & 0 \\ 0 & i & 1 \\  0 & -1 & i \end{vmatrix} . \begin{vmatrix} x \\ y \\  z \end{vmatrix} = 0$$

$$
\left\{ 
\begin{array}{l l}
(2+i)x  + 0y + 0z & = 0\\
0x  + iy - z & = 0\\
0x +y + iz & = 0\\
\end{array}
\right. 
$$ 

$$
\left\{ 
\begin{array}{l l}
x & = 0\\
z & = iy \\
z & = -\frac{y}{i} = iy\\
\end{array}
\right. 
$$ 

En fixant $x=0$ et $z=c_1$ on a
$$E_{-i}(A)=ker(A-2I) = \{(0,1,i)\}$$

$$Sp(A) = \{(1,0,0), (0,1,-i), (0,1,i)\}$$

Donc dans $\C$, $dim\ Sp(A) = 3$, et $dim\ A = 3$ donc la matrice est diagonalisable dans $\R$.

\subsection*{Exercice 3}
Si les valeurs propres de $A$, $Sp(A) = \{1,2\}$, on a $(A-I) = 0$ et $(A-2I) = 0$. On a 
$$A^2 = 3A - 2I, A^2-3A+2I = 0, (A-I)(A-2I) = 0$$

Ce qui est vrai.

\subsection*{Exercice 4}
\subsubsection*{Exercice 4.1}
$u$ est un vecteur propre associ\'e \`a la valeur $\lambda$ de $B$, donc $B.u = \lambda.u$ et $B=Q^{-1}.A.Q$.
On cherche $\lambda_1$ tel que $A.(Q.u) = \lambda_1.Q.u$.  

Donc 
$$B.u = \lambda.u$$
$$Q^{-1}.A.Q.u = \lambda.u$$
$$Q.Q^{-1}.A.Q.u = Q.\lambda.u$$
$$A.(Q.u) = \lambda.Q.u$$  

Donc, en prenant $\lambda_1 = \lambda$, on a $Q.u$ est un vecteur propre de $A$ associ\'e \`a la valeur propre $\lambda$. On peut en d\'eduire que $dim\ E_{\lambda}(A) =  dim\ E_{\lambda}(B)$ car pour tous les \'el\'ement de $u \in E_{\lambda}(B)$ on peut associer un \'element $Q.u \in E_{\lambda}(A)$. 

\subsubsection*{Exercice 4.2}
Calculons les valeurs propres de $A$ et $B$.\\
Calculons $\det(A-\lambda_a.I)$
$$\det \left( \begin{vmatrix} -1 - \lambda_a & 0 & 1 \\ 0 & 3 - \lambda_a & 0 \\  0 & 0 & -1 -\lambda_a \end{vmatrix}\right) $$
$$= (-1-\lambda_a)(3-\lambda_a)(-1-\lambda_a) = -(1+\lambda_a)^2(3-\lambda_a)$$
donc
$$Sp(A) = \{-1,3\}$$

Calculons $\det(B-\lambda_b.I)$
$$\det \left( \begin{vmatrix} -1 - \lambda_b & 1 & 0 \\ 0 & 3 - \lambda_b & 0 \\  0 & 0 & -1 -\lambda_b \end{vmatrix}\right) $$
$$= (-1-\lambda_b)(3-\lambda_b)(-1-\lambda_b) = -(1+\lambda_b)^2(3-\lambda_b)$$
donc
$$Sp(B) = \{-1,3\}$$

Calculons les vecteurs propres associ\'es \`a 1, $E_{-1}(A)$ et $E_{-1}(B)$
$$E_{-1}(A)=ker(A+i.I) = ker\left( \begin{vmatrix} -1 + 1 & 0 & 1 \\ 0 & 3+1  & 0 \\  0 & 0 & -1 + 1 \end{vmatrix} \right)$$
Cherchons $x,y,z$ tel que
$$\begin{vmatrix} 0 & 0 & 1 \\ 0 & 4 & 0 \\  0 & 0 & 0 \end{vmatrix} . \begin{vmatrix} x \\ y \\  z \end{vmatrix} = 0$$

$$
\left\{ 
\begin{array}{l l}
0x  + 0y + z & = 0\\
0x  + 4y + 0z & = 0\\
0x - y + 0z & = 0\\
\end{array}
\right. 
$$ 

En fixant $x=c_1$, $y=z=0$ on a $E_{-1}(A) = \{(1,0,0)\}$ 

$$E_{-1}(B)=ker(B+i.I) = ker\left( \begin{vmatrix} -1 + 1 & 1 & 0 \\ 0 & 3+1  & 0 \\  0 & 0 & -1 + 1 \end{vmatrix} \right)$$
Cherchons $x,y,z$ tel que
$$\begin{vmatrix} 0 & 1 & 0 \\ 0 & 4 & 0 \\  0 & 0 & 0 \end{vmatrix} . \begin{vmatrix} x \\ y \\  z \end{vmatrix} = 0$$

$$
\left\{ 
\begin{array}{l l}
0x  + y + 0z & = 0\\
0x  + 4y + 0z & = 0\\
0x - y + 0z & = 0\\
\end{array}
\right. 
$$ 

En fixant $x=c_1$, $z=c_2$, $y=0$ on a $E_{-1}(B) = \{(1,0,0), (0,0,1)\}$ 

Les matrices $A$ et $B$ ne sont pas semblables car $dim\ E_{-1}(A) \neq dim\ E_{-1}(B)$.

\subsection*{Exercice 5}
Comme la matrice $A$ est diagonalisable, il existe une matrice $P$ inversible et une matrice diagonale $D$ tel que $A = P.D.P^{-1}$
Donc 
$$A^3+4A-16I_n = (P.D.P^{-1})^3 + 4(P.D.P^{-1}) -16I_n = P.D^3.P^{-1} + P.4D.P^{-1} - 16P.I_n.P^{-1} = P.(D^3+4D-16I_n).P^{-1} = 0_n$$
La matrice $P$ est inversible donc $P \neq 0_n$. donc on cherche une matrice diagonale $D$ tel que 
$$D^3+4D-16I_n = (D-2I_n)(D^2+2D+8) = 0_n$$
Cette \'equation a une seule solution dans $\R$, $D=2I_n$.\\
Donc $A=\{P.D.P^{-1}, P\ inversible\} = \{P.(2I_n).P^{-1}, P\ inversible\} = \{2P.I_n.P^{-1}, P\ inversible\} = \{2I_n\}$.



\end{document}

