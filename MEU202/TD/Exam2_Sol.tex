\documentclass[]{book}

%These tell TeX which packages to use.
\usepackage{array,epsfig}
\usepackage{amsmath}
\usepackage{amsfonts}
\usepackage{amssymb}
\usepackage{amsxtra}
\usepackage{amsthm}
\usepackage{mathrsfs}
\usepackage{color}

%Here I define some theorem styles and shortcut commands for symbols I use often
\theoremstyle{definition}
\newtheorem{defn}{Definition}
\newtheorem{thm}{Theorem}
\newtheorem{cor}{Corollary}
\newtheorem*{rmk}{Remark}
\newtheorem{lem}{Lemma}
\newtheorem*{joke}{Joke}
\newtheorem{ex}{Example}
\newtheorem*{soln}{Solution}
\newtheorem{prop}{Proposition}

\newcommand{\lra}{\longrightarrow}
\newcommand{\ra}{\rightarrow}
\newcommand{\surj}{\twoheadrightarrow}
\newcommand{\graph}{\mathrm{graph}}
\newcommand{\bb}[1]{\mathbb{#1}}
\newcommand{\Z}{\bb{Z}}
\newcommand{\Q}{\bb{Q}}
\newcommand{\R}{\bb{R}}
\newcommand{\C}{\bb{C}}
\newcommand{\N}{\bb{N}}
\newcommand{\M}{\mathbf{M}}
\newcommand{\E}{\mathscr{E}}
\newcommand{\m}{\mathbf{m}}
\newcommand{\MM}{\mathscr{M}}
\newcommand{\HH}{\mathscr{H}}
\newcommand{\Om}{\Omega}
\newcommand{\Ho}{\in\HH(\Om)}
\newcommand{\bd}{\partial}
\newcommand{\del}{\partial}
\newcommand{\bardel}{\overline\partial}
\newcommand{\textdf}[1]{\textbf{\textsf{#1}}\index{#1}}
\newcommand{\img}{\mathrm{img}}
\newcommand{\ip}[2]{\left\langle{#1},{#2}\right\rangle}
\newcommand{\inter}[1]{\mathrm{int}{#1}}
\newcommand{\exter}[1]{\mathrm{ext}{#1}}
\newcommand{\cl}[1]{\mathrm{cl}{#1}}
\newcommand{\ds}{\displaystyle}
\newcommand{\vol}{\mathrm{vol}}
\newcommand{\cnt}{\mathrm{ct}}
\newcommand{\osc}{\mathrm{osc}}
\newcommand{\LL}{\mathbf{L}}
\newcommand{\UU}{\mathbf{U}}
\newcommand{\support}{\mathrm{support}}
\newcommand{\AND}{\;\wedge\;}
\newcommand{\OR}{\;\vee\;}
\newcommand{\Oset}{\varnothing}
\newcommand{\st}{\ni}
\newcommand{\wh}{\widehat}

%Pagination stuff.
\setlength{\topmargin}{-.3 in}
\setlength{\oddsidemargin}{0in}
\setlength{\evensidemargin}{0in}
\setlength{\textheight}{9.in}
\setlength{\textwidth}{6.5in}
\pagestyle{empty}


\pdfinfo{
   /Author (Myself)
   /Title  (DM1 - MEU202)
}

\begin{document}

\subsection*{Rappel de cours}



\newpage
\subsection*{Exercice 1}
\subsubsection*{Exercice 1.1}
Calculons $\det(A-\lambda.I)$
$$\det \left( \begin{vmatrix} 1 - \lambda & 0 & 0 \\ a & -2 - \lambda & 3 \\  1 & -1 & 2 -\lambda \end{vmatrix}\right) $$
$$= (-2-\lambda)((2-\lambda)(1-\lambda)-0*1) - (-1)((1-\lambda)*3 -0*a) = -(4-\lambda^2)(1-\lambda)+3(1-\lambda) = (1-\lambda)(-1+\lambda^2)$$
donc
$$Sp(A) = \{1, -1\}$$

\subsubsection*{Exercice 1.2}
D\'eterminons les vecteurs propres de $A$.\\
Calculons 
$$E_{1}(A)=ker(A-I) = ker\left( \begin{vmatrix} 1 - 1 & 0 & 0 \\ a & -2 - 1 & 3 \\  1 & -1 & 2 - 1 \end{vmatrix} \right)$$

Cherchons $x,y,z$ tel que
$$\begin{vmatrix} 0 & 0 & 0 \\ a & -3 & 3 \\  1 & -1 & 1 \end{vmatrix} . \begin{vmatrix} x \\ y \\  z \end{vmatrix} = 0$$

$$
\left\{ 
\begin{array}{l l}
0x + 0y + 0z & = 0\\
ax - 3y + 3z & = 0\\
x - y + z & = 0\\
\end{array}
\right. 
$$ 

$$
\left\{ 
\begin{array}{l l}
x  + z & = y\\
ax - 3x -3z + 3z & = 0\\
0x + 0y + 0z & = 0\\
\end{array}
\right. 
$$ 

$$
\left\{ 
\begin{array}{l l}
x  + z & = y\\
x(a - 3) & = 0\\
0x + 0y + 0z & = 0\\
\end{array}
\right. 
$$ 

En fixant $x=c_1$ et $z=c_2$, on a
$$
E_{1}(A)=
\left\{ 
\begin{array}{l l}
\{(1,1,0),(0,1,1)\} & a = 3\\
\{0,1,1) & a \neq 3\\
\end{array}
\right. 
$$ 

On a $dim\ E_{1}(A)= 2$ lorsque $a=3$ et $dim\ E_{1}(A)= 1$ lorsque $a \neq 3$.\\


Calculons 
$$E_{-1}(A)=ker(A+I) = ker\left( \begin{vmatrix} 1 + 1 & 0 & 0 \\ a & -2 + 1 & 3 \\  1 & -1 & 2 + 1 \end{vmatrix} \right)$$

Cherchons $x,y,z$ tel que
$$\begin{vmatrix} 2 & 0 & 0 \\ a & -1 & 3 \\  1 & -1 & 3 \end{vmatrix} . \begin{vmatrix} x \\ y \\  z \end{vmatrix} = 0$$

$$
\left\{ 
\begin{array}{l l}
2x + 0y + 0z & = 0\\
ax - y + 3z & = 0\\
x - y + 3z & = 0\\
\end{array}
\right. 
$$ 

$$
\left\{ 
\begin{array}{l l}
x  & = 0\\
3z & = y\\
\end{array}
\right. 
$$ 

En fixant $x=0$ et $z=c_2$, on a
$$
E_{-1}(A)=\{(0,3,1)\}
$$ 

On a $dim\ E_{-1}(A)= 1$.\\

Pour que $A$ soit diagonalisable il faut que $dim\ A = dim\ E_{1} + dim\ E_{-1}$, donc on a $a = 3$.

La matrice
$$P = \begin{vmatrix} 1 & 0 & 0 \\ 1 & 1 & 3 \\  0 & 1 & 1 \end{vmatrix} $$

QED


\end{document}

