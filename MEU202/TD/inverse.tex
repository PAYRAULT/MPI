\documentclass[]{book}

%These tell TeX which packages to use.
\usepackage{array,epsfig}
\usepackage{amsmath}
\usepackage{amsfonts}
\usepackage{amssymb}
\usepackage{amsxtra}
\usepackage{amsthm}
\usepackage{mathrsfs}
\usepackage{color}

%Here I define some theorem styles and shortcut commands for symbols I use often
\theoremstyle{definition}
\newtheorem{defn}{Definition}
\newtheorem{thm}{Theorem}
\newtheorem{cor}{Corollary}
\newtheorem*{rmk}{Remark}
\newtheorem{lem}{Lemma}
\newtheorem*{joke}{Joke}
\newtheorem{ex}{Example}
\newtheorem*{soln}{Solution}
\newtheorem{prop}{Proposition}

\newcommand{\lra}{\longrightarrow}
\newcommand{\ra}{\rightarrow}
\newcommand{\surj}{\twoheadrightarrow}
\newcommand{\graph}{\mathrm{graph}}
\newcommand{\bb}[1]{\mathbb{#1}}
\newcommand{\Z}{\bb{Z}}
\newcommand{\Q}{\bb{Q}}
\newcommand{\R}{\bb{R}}
\newcommand{\C}{\bb{C}}
\newcommand{\N}{\bb{N}}
\newcommand{\M}{\mathbf{M}}
\newcommand{\m}{\mathbf{m}}
\newcommand{\MM}{\mathscr{M}}
\newcommand{\HH}{\mathscr{H}}
\newcommand{\Om}{\Omega}
\newcommand{\Ho}{\in\HH(\Om)}
\newcommand{\bd}{\partial}
\newcommand{\del}{\partial}
\newcommand{\bardel}{\overline\partial}
\newcommand{\textdf}[1]{\textbf{\textsf{#1}}\index{#1}}
\newcommand{\img}{\mathrm{img}}
\newcommand{\ip}[2]{\left\langle{#1},{#2}\right\rangle}
\newcommand{\inter}[1]{\mathrm{int}{#1}}
\newcommand{\exter}[1]{\mathrm{ext}{#1}}
\newcommand{\cl}[1]{\mathrm{cl}{#1}}
\newcommand{\ds}{\displaystyle}
\newcommand{\vol}{\mathrm{vol}}
\newcommand{\cnt}{\mathrm{ct}}
\newcommand{\osc}{\mathrm{osc}}
\newcommand{\LL}{\mathbf{L}}
\newcommand{\UU}{\mathbf{U}}
\newcommand{\support}{\mathrm{support}}
\newcommand{\AND}{\;\wedge\;}
\newcommand{\OR}{\;\vee\;}
\newcommand{\Oset}{\varnothing}
\newcommand{\st}{\ni}
\newcommand{\wh}{\widehat}

%Pagination stuff.
\setlength{\topmargin}{-.3 in}
\setlength{\oddsidemargin}{0in}
\setlength{\evensidemargin}{0in}
\setlength{\textheight}{9.in}
\setlength{\textwidth}{6.5in}
\pagestyle{empty}



\begin{document}

\subsection*{Rappel de cours}



\newpage
\subsection*{Exercice 1}
Trouver $17^{-1}$ dans $\Z/20\Z$.\\
On a $pgcd(17,20) = 1$ donc l'inverse existe. Il faut trouver $x$ tel que $17.x = 1 (\mod 20)$ (ou $17.x = 1 + k.20$). Ou bien $17.x + (-k).20 = 1 = pgcd(17,20)$. Rebonjour Mr Bezout. Il 
faut trouver $x$ et $k$. Donc, n steps. Trouver le reste de $20/17$.
$$20 = 17.(1) + 3$$
Trouver le reste de $17/3$
$$17=3.(5) + 2$$
Trouver le reste de $3/2$
$$3=2.(1)+1$$
Le reste est 1 on arr\^ete. Ensuite on inverse
$$20 + 17.(-1) = 3$$
$$17 + 3.(-5) = 2$$
$$3+2.(-1)=1$$
Et on remonte et on remplace
$$3+2.(-1)=1$$
$$3+(17 + 3.(-5)).(-1)= 3.(6) + 17(-1) =1$$
$$(20 + 17.(-1)).(6) + 17(-1) = 20.(6) + 17(-7) = 1$$
On a trouv\'e $k=6$ et $x=-7$. Mais l'inverse doit \^etre positif. on calcule $-7 (\mod 20) = 13$ donc $17^{-1} = 13 (\mod 20)$. En effet, $17.13 = 221 = 1 (\mod 20)$.


QED

\end{document}

