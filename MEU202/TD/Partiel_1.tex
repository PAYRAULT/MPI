\documentclass[]{book}

%These tell TeX which packages to use.
\usepackage{array,epsfig}
\usepackage{amsmath}
\usepackage{amsfonts}
\usepackage{amssymb}
\usepackage{amsxtra}
\usepackage{amsthm}
\usepackage{mathrsfs}
\usepackage{color}

%Here I define some theorem styles and shortcut commands for symbols I use often
\theoremstyle{definition}
\newtheorem{defn}{Definition}
\newtheorem{thm}{Theorem}
\newtheorem{cor}{Corollary}
\newtheorem*{rmk}{Remark}
\newtheorem{lem}{Lemma}
\newtheorem*{joke}{Joke}
\newtheorem{ex}{Example}
\newtheorem*{soln}{Solution}
\newtheorem{prop}{Proposition}

\newcommand{\lra}{\longrightarrow}
\newcommand{\ra}{\rightarrow}
\newcommand{\surj}{\twoheadrightarrow}
\newcommand{\graph}{\mathrm{graph}}
\newcommand{\bb}[1]{\mathbb{#1}}
\newcommand{\Z}{\bb{Z}}
\newcommand{\Q}{\bb{Q}}
\newcommand{\R}{\bb{R}}
\newcommand{\C}{\bb{C}}
\newcommand{\N}{\bb{N}}
\newcommand{\M}{\mathbf{M}}
\newcommand{\m}{\mathbf{m}}
\newcommand{\MM}{\mathscr{M}}
\newcommand{\HH}{\mathscr{H}}
\newcommand{\Om}{\Omega}
\newcommand{\Ho}{\in\HH(\Om)}
\newcommand{\bd}{\partial}
\newcommand{\del}{\partial}
\newcommand{\bardel}{\overline\partial}
\newcommand{\textdf}[1]{\textbf{\textsf{#1}}\index{#1}}
\newcommand{\img}{\mathrm{img}}
\newcommand{\ip}[2]{\left\langle{#1},{#2}\right\rangle}
\newcommand{\inter}[1]{\mathrm{int}{#1}}
\newcommand{\exter}[1]{\mathrm{ext}{#1}}
\newcommand{\cl}[1]{\mathrm{cl}{#1}}
\newcommand{\ds}{\displaystyle}
\newcommand{\vol}{\mathrm{vol}}
\newcommand{\cnt}{\mathrm{ct}}
\newcommand{\osc}{\mathrm{osc}}
\newcommand{\LL}{\mathbf{L}}
\newcommand{\UU}{\mathbf{U}}
\newcommand{\support}{\mathrm{support}}
\newcommand{\AND}{\;\wedge\;}
\newcommand{\OR}{\;\vee\;}
\newcommand{\Oset}{\varnothing}
\newcommand{\st}{\ni}
\newcommand{\wh}{\widehat}

%Pagination stuff.
\setlength{\topmargin}{-.3 in}
\setlength{\oddsidemargin}{0in}
\setlength{\evensidemargin}{0in}
\setlength{\textheight}{9.in}
\setlength{\textwidth}{6.5in}
\pagestyle{empty}



\begin{document}


\subsection*{Exercice 1}
\subsubsection*{Exercice 1.1}
On a $f: \R^2 \to \R^3$ et $dim(\R^3) = 3$. $f$ est surjective si $dim(Im f) = dim(\R^3) = 3$


\subsubsection*{Exercice 1.2}


\subsection*{Exercice 2}
Non. Car une application lin\'eaire doit satisfaire $f(\lambda x) = \lambda f(x)$. On sait que $f(\lambda x)$ et $f(x)$ sont dans le cercle $C(0,1)$. Donc pour un $\lambda > 1$, on ne peut pas avoir $f(\lambda x) = \lambda f(x)$. 



\subsection*{Exercice 3}
Famille libre? Il faut r\'esoudre:
$$
\left\{ 
\begin{array}{l l}
\lambda_{1} + 2\lambda_{2} + 0\lambda_{3} & = 0\\
\lambda_{1} + \lambda_{2} - 1\lambda_{3} & = 0\\
-1\lambda_{1} + 3\lambda_{2} + 5\lambda_{3} & = 0\\
\end{array}
\right. 
$$ 


$$
\begin{vmatrix}
1 & 2  & 0 & 0 \\
1 & 1  & -1 & 0 \\
-1 & 3 & 5 & 0 \\
\end{vmatrix}
\Leftrightarrow
\begin{vmatrix}
1 & 0 & -2 & 0 \\
0 & 1 & 1 & 0  \\
0 & 0 & 0 & 0  \\
\end{vmatrix}
$$

$$
\left\{ 
\begin{array}{l l}
\lambda_{1} + -2\lambda_{3} & = 0\\
\lambda_{2} + \lambda_{3} & = 0\\
\end{array}
\right. 
$$ 

Famille non libre car on peut $\lambda_{3}$ n'est pas \'egal \`a 0.\\

Famille g\'en\'eratrice? Non car du r\'esultat pr\'ec\'edent, on peut exprimer $\lambda_1$ et $\lambda_2$ en fonction de $\lambda_3$.\\

On a $\lambda_1 = -2 \lambda_2$. Donc, $\forall k \in \R, v_3 = -2k.v_1 + kv_2$. 


\subsection*{Exercice 4}
\subsubsection*{Exercice 4.1}
Soit $a,b \in \mathscr{H}_n$, on a $\sum_{k=1}^{n}a_{kk} = 0$ et $\sum_{k=1}^{n}b_{kk} = 0$.\\

V\'erifions que $(a+\lambda b) \in \mathscr{H}_n$? La diagonale de la matrice $(a+\lambda b)$ sont les termes $a_{kk} + \lambda b_{kk}$. Donc $\sum_{k=1}^{n}(a_{kk} + \lambda b`_{kk}) = \sum_{k=1}^{n}a`_{kk} + \lambda \sum_{k=1}^{n}b`_{kk} = 0+ \lambda 0 = 0$. Comme la somme des coefficients diagonaux est nulle, alors $(a,b) \in \mathscr{H}_n$ et $\mathscr{H}_n$ est un sous espace vectoriel de $M_n(\C)$.\\


\subsubsection*{Exercice 4.2}
Soit 
$$m_1= 
\begin{vmatrix}
1 & 0   \\
0 & -1 \\
\end{vmatrix}
, m_2= 
\begin{vmatrix}
0 & 1 \\
0 & 0 \\
\end{vmatrix}
, m_3= 
\begin{vmatrix}
0 & 0 \\
1 & 0 \\
\end{vmatrix}
$$

Montrons que $B = (m_1, m_2 , m_3)$ est une base de $\mathscr{H}_2$.
B est g\'en\'eratrice? Tout \'el\'ement de $\mathscr{H}_2$ est de la forme $m = \begin{vmatrix}
a & b \\
c & -a \\
\end{vmatrix}
$.
Montrons qu'il existe $\lambda_1, \lambda_2, \lambda_3$, tel que $\lambda_1 m_1 + \lambda_2 m_2 + \lambda_3 m_3 = m$. Vrai en prenant, $\lambda_1 = a$, $\lambda_2 = b$ et $\lambda_3 = c$

B est libre? Montrons que si $\lambda_1 m_1 + \lambda_2 m_2 + \lambda_3 m_3 = 0$ alors $\lambda_1 = \lambda_2 = \lambda_3 = 0$.
$$
\left\{ 
\begin{array}{l l}
\lambda_{1} + 0\lambda_{2} + 0\lambda_{3} & = 0\\
0\lambda_{1} + \lambda_{2} + 0\lambda_{3} & = 0\\
0\lambda_{1} + 0\lambda_{2} + 1\lambda_{3} & = 0\\
-1\lambda_{1} + 0\lambda_{2} + 0\lambda_{3} & = 0\\
\end{array}
\right. 
$$ 

Vrai. Donc famille libre et $B$ est une base de $\mathscr{H}_2$.

La dimension de $\mathscr{H}_2$ est 3.

\subsubsection*{Exercice 4.3}
???

\subsubsection*{Exercice 4.4}
???


\subsection*{Exercice 5}
\subsubsection*{Exercice 5.1}
Pour que $\mathscr{F}$ soit une base, il faut qu'il soit libre.
$$
\left\{ 
\begin{array}{l l}
2\lambda_{1} + 8\lambda_{2} + 7\lambda_{3} & = 0\\
0\lambda_{1} + -4\lambda_{2} + 2\lambda_{3} & = 0\\
0\lambda_{1} + 0\lambda_{2} + \alpha \lambda_{3} & = 0\\
\end{array}
\right. 
$$ 
Pour que la famille $\mathscr{F}$ soit libre, il faut que $\alpha \neq 0$.

\subsubsection*{Exercice 5.2}


\subsection*{Exercice 6}
\subsubsection*{Exercice 6.1}
Pour $n=1$, $A_1 = |c|$ et $-A_1 = |-c|$, donc $det(A_1) = - det(-A_1)$, pour $n=2$, $A_2 = \begin{vmatrix} a & b \\ c & d \end{vmatrix}$ et $-A_2 = \begin{vmatrix} -a & -b \\ -c & -d \end{vmatrix}$, donc $det(A_1) = det(-A_1)$. Supposons que pour $n$ impair, $det(M_n) = - det(M_n)$ et $n$ pair, $det(M_n) = det(M_n)$. Faisons une preuve par r\'ecurrence de cette hypoth\`ese. \\

Cas $n+1$ impair, montrons que $det(A_{n+1} = det(-A_{n+1})$.
$$det(-A_{n+1}) = (-a_11).det(-A_{11}) - (-a_12).det(-A_{11}) + \ldots + (-a_1(n)).det(-A_{1(n)}) - (-a_1(n+1)).det(-A_{1(n+1)})$$
par hypoth\'ese de r\'ecurrence on a
$$det(-A_{n+1}) = (-a_11).det(A_{11}) - (-a_12).det(A_{11}) + \ldots + (-a_1(n)).det(A_{1(n)}) - (-a_1(n+1)).det(A_{1(n+1)}) = -(a_11.det(A_{11}) - a_12.det(A_{11}) + \ldots + a_1(n).det(A_{1(n)}) - a_1(n+1).det(A_{1(n+1)}) = -det(A_{n+1})$$

Cas $n+1$ pair, montrons que $det(A_{n+1} = det(-A_{n+1})$.
$$det(-A_{n+1}) = (-a_11).det(-A_{11}) - (-a_12).det(-A_{11}) + \ldots - (-a_1(n)).det(-A_{1(n)}) + (-a_1(n+1)).det(-A_{1(n+1)})$$
par hypoth\'ese de r\'ecurrence on a
$$det(-A_{n+1}) = (-a_11).-det(A_{11}) - (-a_12).-det(A_{11}) + \ldots - (-a_1(n)).-det(A_{1(n)}) + (-a_1(n+1)).-det(A_{1(n+1)}) = a_11.det(A_{11}) - a_12.det(A_{11}) + \ldots + a_1(n).det(A_{1(n)}) - a_1(n+1).det(A_{1(n+1)} = det(A_{n+1})$$

V\'erifi\'e.

\subsubsection*{Exercice 6.2}
$n$ est impair donc $det(A) = -det(-A)$, une matrice et sa transpos\'ee ont le m\^eme d\'eterminant donc $det(^{t}A) = det(A)$. On a $^{t}A = - A$ donc $det(^{t}A) = det(- A)$. Ceux qui fait $det(-A) = det(^{t}A) = det(A) = -det(-A)$. Donc la seule valeur possible pour $det(-A)$ est 0. Donc $det(A) = -det(-A) = -0 = 0$.  


\subsection*{Exercice 7}
\subsubsection*{Exercice 7.1}
$$det(A_1) = a_1.x - (-1)a_0 = a_1.x + a_0$$
$$det(A_2) = a_2.det\left( \begin{vmatrix} x & 0 \\ -1 & x \\ \end{vmatrix} \right) - a_1.det\left( \begin{vmatrix} -1 & 0 \\ 0 & x \\ \end{vmatrix}\right) + a_0.det\left( \begin{vmatrix} -1 & x \\ 0 & -1 \\ \end{vmatrix}\right) = a_2.x^2 + a1.x + a_0$$

\subsubsection*{Exercice 7.2}
Preuve par r\'ecurrence. Supposons que $det(A_n) = a_nx^n+ \ldots + a_1x + a_0$, montrons que $det(A_{n+1}) = a_{n+1}x^{n+1} +  a_nx^n+ \ldots + a_1x + a_0$ 

$$det(A_{n+1}) = a_{n+1}.\begin{vmatrix} x & 0 & 0 & \ldots & 0 \\ -1 & x & 0 & \ldots & 0 \\ 0 & -1 & x & \ldots & 0 \\ \vdots & \vdots & \vdots & -1 & x \\ \end{vmatrix} -(-1)det(A_n) = a_{n+1}x^{n+1} + det(A_n)$$


$\square$

\end{document}

