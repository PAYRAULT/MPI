\documentclass[]{book}

%These tell TeX which packages to use.
\usepackage{array,epsfig}
\usepackage{amsmath}
\usepackage{amsfonts}
\usepackage{amssymb}
\usepackage{amsxtra}
\usepackage{amsthm}
\usepackage{mathrsfs}
\usepackage{color}

%Here I define some theorem styles and shortcut commands for symbols I use often
\theoremstyle{definition}
\newtheorem{defn}{Definition}
\newtheorem{thm}{Theorem}
\newtheorem{cor}{Corollary}
\newtheorem*{rmk}{Remark}
\newtheorem{lem}{Lemma}
\newtheorem*{joke}{Joke}
\newtheorem{ex}{Example}
\newtheorem*{soln}{Solution}
\newtheorem{prop}{Proposition}

\newcommand{\lra}{\longrightarrow}
\newcommand{\ra}{\rightarrow}
\newcommand{\surj}{\twoheadrightarrow}
\newcommand{\graph}{\mathrm{graph}}
\newcommand{\bb}[1]{\mathbb{#1}}
\newcommand{\Z}{\bb{Z}}
\newcommand{\Q}{\bb{Q}}
\newcommand{\R}{\bb{R}}
\newcommand{\C}{\bb{C}}
\newcommand{\N}{\bb{N}}
\newcommand{\M}{\mathbf{M}}
\newcommand{\E}{\mathscr{E}}
\newcommand{\m}{\mathbf{m}}
\newcommand{\MM}{\mathscr{M}}
\newcommand{\HH}{\mathscr{H}}
\newcommand{\Om}{\Omega}
\newcommand{\Ho}{\in\HH(\Om)}
\newcommand{\bd}{\partial}
\newcommand{\del}{\partial}
\newcommand{\bardel}{\overline\partial}
\newcommand{\textdf}[1]{\textbf{\textsf{#1}}\index{#1}}
\newcommand{\img}{\mathrm{img}}
\newcommand{\ip}[2]{\left\langle{#1},{#2}\right\rangle}
\newcommand{\inter}[1]{\mathrm{int}{#1}}
\newcommand{\exter}[1]{\mathrm{ext}{#1}}
\newcommand{\cl}[1]{\mathrm{cl}{#1}}
\newcommand{\ds}{\displaystyle}
\newcommand{\vol}{\mathrm{vol}}
\newcommand{\cnt}{\mathrm{ct}}
\newcommand{\osc}{\mathrm{osc}}
\newcommand{\LL}{\mathbf{L}}
\newcommand{\UU}{\mathbf{U}}
\newcommand{\support}{\mathrm{support}}
\newcommand{\AND}{\;\wedge\;}
\newcommand{\OR}{\;\vee\;}
\newcommand{\Oset}{\varnothing}
\newcommand{\st}{\ni}
\newcommand{\wh}{\widehat}

%Pagination stuff.
\setlength{\topmargin}{-.3 in}
\setlength{\oddsidemargin}{0in}
\setlength{\evensidemargin}{0in}
\setlength{\textheight}{9.in}
\setlength{\textwidth}{6.5in}
\pagestyle{empty}



\begin{document}

\subsection*{Rappel de cours}

\begin{defn}
Soit $\E$ un K-espace vectoriel. Une partie F de $\E$ est appelée un sous-espace vectoriel si :
\begin{itemize}
\item $0_E \in F$,
\item $u + v \in F$ pour tous $u, v \in F$,
\item $\lambda.u \in F$ pour tout $\lambda \in K$ et tout $u \in F$.
\end{itemize}
\end{defn}


\begin{defn}
Une famille $\{v_1, v_2, \ldots , v_p\}$ de $E$ est une famille libre ou lin\'eairement ind\'ependante si toute combinaison lin\'eaire nulle
$$\lambda_1v_1 + \lambda_2v_2 + ··· + \lambda_p v_p = 0$$
est telle que tous ses coefficients sont nuls, c’est-\`a-dire
$$\lambda_1 = 0, \lambda_2 = 0, \ldots, \lambda_p = 0$$
\end{defn}



\begin{defn}
Soient $v_1, \ldots, v_p$ des vecteurs de $E$. La famille $\{v_1, \ldots, v_p\}$ est une famille g\'en\'eratrice de l'espace vectoriel E si tout vecteur de E est une combinaison lin\'eaire des vecteurs $v_1, \ldots, v_p$. Ce qui peut s'\'ecrire aussi :
$$\forall v \in E, \exists \lambda_1, \ldots, \lambda_p, v = \lambda_1 v_1 + \ldots + \lambda_p v_p$$
\end{defn}


\begin{defn}
Si $f : E \to F$ est une application lin\'eaire, on d\'efinit le noyau de $f$ comme \'etant le sous-espace vectoriel (sev) de $E$ d\'efini par
$$ker(f) := \{x \in E | f(x) = 0\}$$
On d\'efinit l'image de f comme \'etant le sev de F d\'efini par
$$Im(f) = \{y \in F| \exists x \in E, f(x) = y\}$$
La dimension de $Im(f)$ s'appelle aussi le rang de f et se note $rg(f)$.
\end{defn}

\begin{defn}
Soit $\E$ un espace vectoriel de dimension $n$, et $B, B'$ deux bases de $\E$. On appelle la \emph{matrice de passage} de la base $B$ vers la base $B'$, not\'ee $P_{b,B'}$, la matrice dont la j-i\`eme colonne est form\'ee des coordonn\'ees du j-i\`eme vecteur de la base $B'$ par rapport la base $B$. \\
Dans le cas particulier o\`u $B$ est la base canonique de $\E$, alors la matrice de passe est form\'ee des vecteurs de $B'$.
\end{defn}

\begin{defn}
Soit 
\begin{itemize}
\item $f: \E \to \E$, ume application lin\'eaire
\item $B,B'$ deux bases de $\E$
\item $P_{B,B'}$, la matrice de passage de $B$ vers $B'$
\item $A=Mat_{B}(f)$, la matrice de l'application $f$ dans la base $B$
\item $B=Mat_{B'}(f)$, la matrice de l'application $f$ dans la base $B'$
\end{itemize}
Alors $$B = P^{-1}AP$$
\end{defn}



\newpage
\subsection*{Exercice 1}
\subsubsection*{Exercice 1.1}
Il faut montrer que $\forall A, B \in S_n, A + B \in S_n$ et $\forall \lambda \in \R, \forall A \in S_n, \lambda A \in S_n$. On a $A = ^{t}A$ et $B = ^{t}B$, donc
$$
A + B = \begin{vmatrix} a_{11} & a_{12} & \ldots & a_{1n} \\ a_{12} & a_{22} & \ldots & a_{2n} \\ \vdots & & & \\ a_{1n} & a_{2n} & \ldots & a_{nn} \end{vmatrix} +
\begin{vmatrix} b_{11} & b_{12} & \ldots & b_{1n} \\ b_{12} & b_{22} & \ldots & b_{2n} \\ \vdots & & & \\ b_{1n} & b_{2n} & \ldots & b_{nn} \end{vmatrix} =
\begin{vmatrix} a_{11} + b_{11} & a_{12} + b_{12} & \ldots & a_{1n} + b_{1n} \\ a_{12} + b_{12} & a_{22} + b_{22} & \ldots & a_{2n} + b_{2n} \\ \vdots & & & \\ a_{1n} + b_{1n}& a_{2n} + b_{2n}& \ldots & a_{nn} + b_{nn} \end{vmatrix}
= ^{t}(A+B)
$$ 
et
$$
\lambda A = \lambda ^{t}A = \begin{vmatrix} \lambda a_{11} & \lambda a_{12} & \ldots & \lambda a_{1n} \\ \lambda a_{12} & \lambda a_{22} & \ldots & \lambda a_{2n} \\ \vdots & & & \\ \lambda a_{1n} & \lambda a_{2n} & \ldots & \lambda a_{nn} \end{vmatrix} = ^{t}(\lambda A)
$$

$S_n$ est bien un sev de $M_n(\R)$.\\


Il faut montrer que $\forall A, B \in A_n, A + B \in A_n$ et $\forall \lambda \in \R, \forall A \in A_n, \lambda A \in A_n$. On a $-A = ^{t}A$ et $-B = ^{t}B$, donc
$$
A + B = \begin{vmatrix} 0 & a_{12} & \ldots & a_{1n} \\ -a_{12} & 0 & \ldots & a_{2n} \\ \vdots & & & \\ -a_{1n} & -a_{2n} & \ldots & 0 \end{vmatrix} +
\begin{vmatrix} 0 & b_{12} & \ldots & b_{1n} \\ -b_{12} & 0 & \ldots & b_{2n} \\ \vdots & & & \\ -b_{1n} & -b_{2n} & \ldots & 0 \end{vmatrix} =
\begin{vmatrix} 0 + 0 & a_{12} + b_{12} & \ldots & a_{1n} + b_{1n} \\ -(a_{12} + b_{12}) & 0 + 0 & \ldots & a_{2n} + b_{2n} \\ \vdots & & & \\ -(a_{1n} + b_{1n})& -(a_{2n} + b_{2n})& \ldots & 0 + 0 \end{vmatrix}
= -^{t}(A+B)
$$ 
et
$$
\lambda A = \lambda ^{t}A = \begin{vmatrix} 0 & \lambda a_{12} & \ldots & \lambda a_{1n} \\ -\lambda a_{12} & 0 & \ldots & \lambda a_{2n} \\ \vdots & & & \\ -\lambda a_{1n} & -\lambda a_{2n} & \ldots & 0 \end{vmatrix} = -^{t}(\lambda A)
$$

$A_n$ est bien un sev de $M_n(\R)$.

\subsubsection*{Exercice 1.2}
La famille $F = Vect(E_{11},E_{22},E_{12+21})$ est libre ?
	
$$
\lambda_1 E_{11} + \lambda_2 E_{22} + \lambda_3 E_{12+21} = \begin{vmatrix} \lambda_1 & \lambda_3 \\ \lambda_3  & \lambda_2 \end{vmatrix} = 0
$$
Vrai uniquement si $\lambda_1 = \lambda_2 = \lambda_3  = 0$. Donc la famille est libre.\\

La famille $F = Vect(E_{11},E_{22},E_{12+21})$ est g\'en\'eratrice ?
	
$$
\forall a \in S_n,
\lambda_1 E_{11} + \lambda_2 E_{22} + \lambda_3 E_{12+21} = \begin{vmatrix} \lambda_1 & \lambda_3 \\ \lambda_3  & \lambda_2 \end{vmatrix} = 
\begin{vmatrix} a_{11} & a_{12} \\ a_{12}  & a_{22} \end{vmatrix}
$$
Vrai pour $\lambda_1 = a_{11}, \lambda_2 = a_{22}, \lambda_3  = a_{12}$. Donc la famille est g\'en\'eratrice.\\


La famille $G = Vect(E_{12-21}))$ est libre ?
	
$$
\lambda_1 E_{12-21} = \begin{vmatrix} 0 & \lambda_1 \\ -\lambda_1  & 0 \end{vmatrix} = 0
$$
Vrai uniquement si $\lambda_1 = 0$. Donc la famille est libre.\\

La famille $G = Vect(E_{12-21})$ est g\'en\'eratrice ?
	
$$
\forall a \in A_n,
\lambda_1 E_{12-21} = \begin{vmatrix} 0 & \lambda_1 \\ -\lambda_1  & 0 \end{vmatrix} = 
\begin{vmatrix} 0 & a_{12} \\ -a_{12} & 0 \end{vmatrix}
$$
Vrai pour $\lambda_1 = a{12}$. Donc la famille est g\'en\'eratrice.

\subsubsection*{Exercice 1.3} 
Trouver une famille g\'en\'eratrice de $S_n$.
	
$$
\forall a \in A_n,
\sum_{i=1}^{k_1+k_2}\lambda_i E_{i} =  
\begin{vmatrix} a_{11} & a_{12} & \ldots & a_{1n} \\ a_{12} & a_{22} & \ldots & a_{2n} \\ \vdots & \vdots & & \vdots \\ a_{1n} & a_{2n} & \ldots & a_{nn} \end{vmatrix}
$$

Si on prends toutes les matrices avec que des 0 except\'e $\forall i, j, 1 \leq i \leq n, 1 \leq j \leq n, i \neq j, a_{ij} = a_{ji}$. Il y a $k1=\frac{(n-1)n}{2}$ matrices diff\'erentes (sommes des \'el\'ements au dessus de la diagonale) et les matrices avec que des 0 except\'e $\forall i, 1 \leq i \leq n, a_{ii} = c$. Il y a $k_2 = n$ matrices diff\'erentes (nombre d'\'el\'emets de la diagonale). Cela est une famille g\'eneratrice.
Cette famille est g\'en\'eratrice, elle est \'egalement libre car car coefficient n'apparait que sur une cellule, donc il faut qu'il soit tous nul pour avoir $\sum_{i=1}^{k_1+k_2}\lambda_i E_{i} =  0$.

 

Trouver une famille g\'en\'eratrice de $A_n$.
	
$$
\forall a \in A_n,
\sum_{i=1}^{k}\lambda_i E_{i} =  
\begin{vmatrix} 0 & a_{12} & \ldots & a_{1n} \\ -a_{12} & 0 & \ldots & a_{2n} \\ \vdots & \vdots & & \vdots \\ -a_{1n} & -a_{2n} & \ldots & 0 \end{vmatrix}
$$
Si on prends les matrices avec que des 0 et $a_{ij} = - a_{ji}$. Il y a $k=\frac{(n-1)n}{2}$ matrices diff\'erentes (sommes des \'el\'ements au dessus de la diagonale). En prenant, $\forall i, j, 1 \leq i \leq n, 1 \leq j \leq n, i \neq j, a_{ij}=-a{ji}$.
Cette famille est g\'en\'eratrice, elle est \'egalement libre car car coefficient n'apparait que sur une cellule, donc il faut qu'il soit tous nul pour avoir $\sum_{i=1}^{k}\lambda_i E_{i} =  0$.
 

\subsubsection*{Exercice 1.4}
On a $\forall A \in \M_n(\R), A+^{t}A \in S_n$. En effet,
$$
A + ^{t}A = 
\begin{vmatrix} a_{11} & a_{12} & \ldots & a_{1n} \\ a_{21} & a_{22} & \ldots & a_{2n} \\ \vdots & \vdots & & \vdots \\ a_{n1} & a_{n2} & \ldots & a_{nn} \end{vmatrix} +
\begin{vmatrix} a_{11} & a_{21} & \ldots & a_{n1} \\ a_{12} & a_{22} & \ldots & a_{n2} \\ \vdots & \vdots & & \vdots \\ a_{1n} & a_{2n} & \ldots & a_{nn} \end{vmatrix} =
\begin{vmatrix} a_{11} + a_{11} & a_{12} + a_{21} & \ldots & a_{1n} + a_{n1} \\ a_{12} + a_{21} & a_{22} + a_{22} & \ldots & a_{2n} + a_{n2} \\ \vdots & \vdots & & \vdots \\ a_{1n} + a_{n1}& a_{2n} + a_{n2}& \ldots & a_{nn} + a_{nn} \end{vmatrix}
$$

On a $\forall A \in M_n(\R), A-^{t}A \in A_n$. En effet,
$$
A - ^{t}A = 
\begin{vmatrix} a_{11} & a_{12} & \ldots & a_{1n} \\ a_{21} & a_{22} & \ldots & a_{2n} \\ \vdots & \vdots & & \vdots \\ a_{n1} & a_{n2} & \ldots & a_{nn} \end{vmatrix} +
\begin{vmatrix} a_{11} & a_{21} & \ldots & a_{n1} \\ a_{12} & a_{22} & \ldots & a_{n2} \\ \vdots & \vdots & & \vdots \\ a_{1n} & a_{2n} & \ldots & a_{nn} \end{vmatrix} =
\begin{vmatrix} 0 & a_{12} - a_{21} & \ldots & a_{1n} - a_{n1} \\ -(a_{12} - a_{21}) & 0 & \ldots & a_{2n} - a_{n2} \\ \vdots & \vdots & & \vdots \\ -(a_{1n} - a_{n1}) & -(a_{2n} - a_{n2})& \ldots & 0 \end{vmatrix}
$$

et $S_n \cap A_n = \{0\}$. En effet, $A \in S_n \Leftrightarrow A = ^{t}A$ et $A \in A_n \Leftrightarrow -A = ^{t}A$. Donc $S_n \cap A_n = \{A \in M_n(|R), A=-A\}$. La seule matrice qui v\'erifie est $0$.

On a 
$$
\forall A \in M_n(\R), A = \frac{A+^{t}A}{2} + \frac{A-^{t}A}{2}
$$

Pour r\'esumer, $\forall A \in M_n(\R), A = B + C$ avec $B=\frac{A+^{t}A}{2} \in S_n$ et $C=\frac{A-^{t}A}{2} \in A_n$ et $S_n \cap A_n = \{0\}$ donc $S_n \oplus A_n = M_n(\R)$.


\subsection*{Exercice 2}
On a $dim (ker) f = 1$ et $rg f = dim (Im f) = 2$ (l'image de $f$ est un plan). D'apr\`es le th\'eor\`eme du rang, on a $dim(ker f) + rg f = dim(\R^4) = 4$ ce qui est faux car $1+3 \neq 4$.

\subsection*{Exercice 3}

Une application lin\'eaire $v:\R^2 \to \R$ est injective ssi son noyau est r\'eduit \`a $\{0\}$. D'apr\`es le th\'eor\`eme du rang, on a $dim(ker v) + rg v = dim(\R^2) = 2$, mais $rg v \leq dim(\R) = 1$ donc $dim(ker v) = dim(\R^2) - rg v \geq  2 - 1 = 1$ Donc $ker v \neq \{0\}$.


\subsection*{Exercice 4}
Simple

\subsection*{Exercice 5}
Si le d\'terminant de la matrice est \'egal \`a 0, alors une seule solution $0_4$, sinon il y a une infinit\'e de solutions. Calculons le d\'eterminant de la matrice.
$$
Det = \begin{vmatrix} 1 & 1 & 1 & 1 \\ x & a & 0 & 0 \\ x & 0 & b & 0 \\ x & 0 & 0 * c\end{vmatrix} = 
-1.\begin{vmatrix}  x & 0 & 0 \\ x & b & 0 \\ x & 0 & c \end{vmatrix} + a.\begin{vmatrix}  1 & 1 & 1 \\ x & b & 0 \\ x & 0 & c \end{vmatrix}
$$
$$
= -1.c\begin{vmatrix}  x & 0 \\ x & b \end{vmatrix} + a.\left(-1.\begin{vmatrix}  x & 0 \\ x & c \end{vmatrix} + b.\begin{vmatrix}  1 & 1 \\ x & c \end{vmatrix}\right) =
-cbx - axc + abc -abx
$$

On a $Det=0$ lorsque
$$
x = \frac{abc}{ab+ac+bc} = \frac{1}{\frac{1}{a}+\frac{1}{b}+\frac{1}{c}}
$$

\subsection*{Exercice 6}
Les vecteurs ne forment pas une famille si ils ne sont pas libre.
$$
a_1\begin{pmatrix}2\\\mu\\1\end{pmatrix} + a_2\begin{pmatrix}\lambda \\ 0 \\1 \end{pmatrix} + a_3\begin{pmatrix}\mu\\\lambda\\0\end{pmatrix} = 0
\Rightarrow a_1 \neq 0 \lor a_2 \neq 0 \lor a_3 \neq 0$$

$$ 
\left\{ 
\begin{array}{l l}
2a_1 + \lambda a_2 + \mu a_3 & = 0\\
\mu a_1 + \lambda a_3 & = 0\\
a_1 + a_2  & = 0\\
\end{array}
\right. 
$$ 

$$ 
\left\{ 
\begin{array}{l l}
a_1  & = - a_2\\
-\frac{\mu}{\lambda}a_1  & =  a_3\\
2a_1 - \lambda a_1 & =  -\mu a_3\\
\end{array}
\right. 
$$ 

$$ 
\left\{ 
\begin{array}{l l}
a_1  & = - a_2\\
-\frac{\mu}{\lambda}a_1  & =  a_3\\
2a_1 - \lambda a_1 & =  \frac{\mu^2}{\lambda}a_1\\
\end{array}
\right. 
$$ 

$$a_1(2\lambda -\lambda^2 - \mu^2) = 0 $$

On cherche $a_1 \neq 0$, donc il faut $2\lambda = \lambda^2 + \mu^2$. C'est le cercle de centre $(1,0)$ et de rayon 1.
En effet, en faisant un changement de variable $\lambda'= \lambda-1$ on a 
$$
2(\lambda'+1) = (\lambda'+1)^2+\mu^2 = \lambda'^2+2\lambda'+1+\mu^2$$
$$1=\lambda'^2+\mu^2$$
qui est le cercle de rayon 1 de centre $\mathscr{O}$.

QED

\end{document}

