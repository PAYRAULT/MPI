\documentclass[]{book}

%These tell TeX which packages to use.
\usepackage{array,epsfig}
\usepackage{amsmath}
\usepackage{amsfonts}
\usepackage{amssymb}
\usepackage{amsxtra}
\usepackage{amsthm}
\usepackage{mathrsfs}
\usepackage{color}
\usepackage{tikz}
\usepackage{fancyhdr}
\usepackage{enumerate}

%Here I define some theorem styles and shortcut commands for symbols I use often
\theoremstyle{definition}
\newtheorem{defn}{Definition}
\newtheorem{thm}{Theorem}
\newtheorem{cor}{Corollary}
\newtheorem*{rmk}{Remark}
\newtheorem{lem}{Lemma}
\newtheorem*{joke}{Joke}
\newtheorem{ex}{Example}
\newtheorem*{soln}{Solution}
\newtheorem{prop}{Proposition}

\newcommand{\lra}{\longrightarrow}
\newcommand{\ra}{\rightarrow}
\newcommand{\surj}{\twoheadrightarrow}
\newcommand{\graph}{\mathrm{graph}}
\newcommand{\bb}[1]{\mathbb{#1}}
\newcommand{\Z}{\bb{Z}}
\newcommand{\Q}{\bb{Q}}
\newcommand{\R}{\bb{R}}
\newcommand{\C}{\bb{C}}
\newcommand{\N}{\bb{N}}
\newcommand{\M}{\mathbf{M}}
\newcommand{\m}{\mathbf{m}}
\newcommand{\MM}{\mathscr{M}}
\newcommand{\HH}{\mathscr{H}}
\newcommand{\Om}{\Omega}
\newcommand{\Ho}{\in\HH(\Om)}
\newcommand{\bd}{\partial}
\newcommand{\del}{\partial}
\newcommand{\bardel}{\overline\partial}
\newcommand{\textdf}[1]{\textbf{\textsf{#1}}\index{#1}}
\newcommand{\img}{\mathrm{img}}
\newcommand{\ip}[2]{\left\langle{#1},{#2}\right\rangle}
\newcommand{\inter}[1]{\mathrm{int}{#1}}
\newcommand{\exter}[1]{\mathrm{ext}{#1}}
\newcommand{\cl}[1]{\mathrm{cl}{#1}}
\newcommand{\ds}{\displaystyle}
\newcommand{\vol}{\mathrm{vol}}
\newcommand{\cnt}{\mathrm{ct}}
\newcommand{\osc}{\mathrm{osc}}
\newcommand{\LL}{\mathbf{L}}
\newcommand{\UU}{\mathbf{U}}
\newcommand{\support}{\mathrm{support}}
\newcommand{\AND}{\;\wedge\;}
\newcommand{\OR}{\;\vee\;}
\newcommand{\Oset}{\varnothing}
\newcommand{\st}{\ni}
\newcommand{\wh}{\widehat}

%Pagination stuff.
\setlength{\topmargin}{-.3 in}
\setlength{\oddsidemargin}{0in}
\setlength{\evensidemargin}{0in}
\setlength{\textheight}{9.in}
\setlength{\textwidth}{6.5in}
\pagestyle{fancy}
\fancyhf{}
\rhead{TD1}
\lhead{M\'ethodologie}
\rfoot{Page \thepage}

\begin{document}


\subsection*{Rappel de cours}
\subsubsection*{Travail}

\begin{itemize}
\item "pour tout", $\forall$
\item "il existe", $\exists$
\item "non", $\lnot$
\item "ou", $\lor$
\item "et", $\land$
\end{itemize}


\subsection*{TD1}
\subsubsection*{Exo 1}

\begin{tabular}{l l }
$x \in \R, x^2 = 4$ car $x = 2$ & $x \in \R, x^2 = 4 \Leftarrow x = 2$ \\
$z \in \Z, z = \bar{z}$ donc $z \in \R$ & $z \in \Z, z = \bar{z} \Leftrightarrow z \in \R$ \\
$x \in \R, x = \pi$ donc $e^{2ix}= 1$ & $x \in \R, x = \pi \Rightarrow e^{2ix}= 1$ \\
\end{tabular}

\subsubsection*{Exo 2}
\begin{enumerate}
\item $\forall x \in \R, x^2 > 0$
\item $\exists x \in \R, x > x^2$
\item $\lnot \exists x \in \R, \forall y \in \R, x > y$ ou $\forall x \in \R, \exists y \in \R, x > y$
\item $\exists x \in \R, \forall n \in \N, \forall m \in \N^{*} x, \neq \frac{n}{m}$
\item $\exists x \in \N, \forall n \in \N, \exists m \in \N, n = x . m$
\item $\forall x_1, x_2 \in \R, x_1 < x_2, \exists x, x_1 < x < x_2$
\item $\forall x_1, x_2, x_3 \in \R, x_1.x_2 \geq 0 \lor x_2.x_3 \geq 0 \lor x_1.x_3 \geq 0$
\end{enumerate}

\subsubsection*{Exo 4}
\begin{enumerate}
\item non( P et Q) = (non P) ou (non Q) $\neq$ (non P) et (non Q). Non elles ne sont pas la n\'egation l'une de l'autre.
\item non( P ou Q) = (non P) et (non Q). Oui elles sont la n\'egation l'une de l'autre.
\item non( P $\Rightarrow$ Q) = non Q $\Rightarrow$ non P (contrapos\'e) $\neq$ non P $\Rightarrow$ non Q. Non elles ne sont pas la n\'egation l'une de l'autre.
\end{enumerate}

\subsubsection*{Exo 6}
\begin{enumerate}
\item La contrapos\'ee de $A \Rightarrow B$ est $non B \Rightarrow non A$
\item P:"L'entier $(n^2 - 1)$ n'est pas divisible par 8" donc "l'entier n est pair" ou $\forall m \in \N, (n^2-1) \neq 8m \Rightarrow \exists x \in \N, n = 2x$. La contrapos\'ee de P est "l'entier n n'est pas pair (n est impair)" donc "l'entier $(n^2 - 1)$ est divisible par 8" ou $\forall x \in N, n \neq 2x \Rightarrow \exists m \in \N, (n^2-1) = 8m$.
\item un entier $n$ impair est de la forme $n = 2x+1$. Deux cas possibles, soit $x$ est pair, soit $x$ est impair. Donc $n=2(2k)+1=4k+1$ ou $n=2(2k+1)+1 = 4k+3$. Par cons\'equent $n = 4k +\{1,3\}$
\item $\forall x \in N, n \neq 2x \Rightarrow \exists m \in \N, (n^2-1) = 8m$. Donc $\exists k \in \N, n = 4k +\{1,3\} \Rightarrow \exists m \in \N, (n^2-1) = 8m$. Deux cas: $(4k+1)^2 -1 = 16k^2 + 8k = 8(2k^2+k)$ ou $(4k+3)^2 -1 = 16k^2+24k+8 = 8(2k^2+3k+1)$. Dans les 2 cas, $n$ est divisible par 8.
\item Oui, car la d\'emonstration de P est faite car nous avons montr\'e la contrapos\'ee de P.
\end{enumerate}

\subsubsection*{Exo 7}
\begin{enumerate}
\item P:$\exists i \in \{1..n\}, x_i - x_{i-1} < \frac{1}{n}$
\item $\lnot P = \lnot(\exists i \in \{1..n\}, x_i - x_{i-1} < \frac{1}{n}) = \forall i \in \{1..n\}, x_i - x_{i-1} > \frac{1}{n} = (x_1 - x_0)+(x_2 - x_1)...(x_{n}-x_{n-1}) > \frac{1}{n} + \frac{1}{n}...+ \frac{1}{n} = x_n - x_0 > 1 \Rightarrow faux$
\item  $(\lnot P \Rightarrow faux) \Leftrightarrow P$. Donc la propri\'et\'e $P$ est v\'erifi\'ee.
\end{enumerate}

\subsubsection*{Exo 9}
\begin{enumerate}
\item
\begin{enumerate}
\item $\forall a, mange(moi,a) \Rightarrow aime(moi,a)$
\item $\exists a, \lnot aime(moi, a) \land mange(moi, a)$
\item $\forall p, \lnot aime(p, legume) \Rightarrow \forall a, \lnot mange(p, a)$
\item $(\forall p, \exists a \lnot aime(p, a) \Rightarrow mange(p,a)) \Rightarrow mange(moi, legume)$
\end{enumerate}
\item
\begin{enumerate}
\item Toute personne qui aime quelque chose, le mange
\item Il existe quelque chose que tout le monde aime et mange
\end{enumerate}
\end{enumerate}

\subsubsection*{Exo 10}

\subsubsection*{Exo 11}
\begin{enumerate}
\item Il existe une voiture qui n'est pas rouge. 
\begin{itemize}
\item P:$\forall v \in voitures, rouge(v)$
\item (non P):$\lnot(\forall v \in voitures, rouge(v)) = \exists v \in voitures, \lnot rouge(v)$
\end{itemize}
\item 
\begin{itemize}
\item P:$\exists m \in moutons, ecossais(m) \land cote(m, noir)$
\item (non P):$\lnot(\exists m \in moutons, ecossais(m) \land cote(m, noir)) = \forall m \in moutons, \lnot ecossais(m) \lor \lnot cote(m. noir)$.
\end{itemize}
\item Il existe une \'ecurie avec un cheval non blanc. 
\begin{itemize}
\item P:$\forall e \in ecuries, \forall c \in chevaux, dans(c,e) \Rightarrow couleur(c,blanc)$,
\item (non P): $\lnot(\forall e \in ecuries, \forall c \in chevaux, dans(c,e) \Rightarrow couleur(c,blanc)) = \exists e \in ecurie, \exists c \in cheval, dans(c,e) \land \lnot couleur(blanc,c)$
\end{itemize}
\item Il existe un \'etudiant qui se reveille tous les jours de la semaines apr\`es 8 heures. 
\begin{itemize}
\item P:$\forall e \in etudiants, \exists j \in jours, \forall h \in heures, reveil(e, j, h) => h < 8$
\item (non P): $\lnot(\forall e \in etudiants, \exists j \in jours, \exists h \in heures, reveil(e, j, h) => h < 8) = \exists e \in etudiants, \forall j \in jours, \exists h \in heures, reveil(e, j, h) \land h > 8$.
\end{itemize}
\item Il existe une prison avec un prisonnier qui aime un des gardiens.
\begin{itemize}
\item P:$\forall p \in prisons, \forall d \in detenus, \forall g \in gardiens, \lnot aime(d, g)$
\item (non P):$\lnot(\forall p \in prisons, \forall d \in detenus, \forall g \in gardiens, \lnot aime(d, g)) = \exists p \in prisons, \exists d \in detenus, \exists g \in gardiens, aime(d, g)$
\end{itemize}
\item Il existe une personne habitant Rue du havre ayant les yeux bleus qui ne gagnera pas au loto ou qui ne prendra pas sa retraite avant 50 ans.
\begin{itemize}
\item P:$\forall p \in personnes, habite(p, "Rue du Havre") \land yeux(p, bleus) \Rightarrow (gagnant(p, loto) \land retraite\_avant(p, 50)$
\item (non P):$\lnot(\forall p \in personnes, habite(p, "Rue du Havre") \land yeux(p, bleus) \Rightarrow (gagnant(p, loto) \land retraite\_avant(p, 50)) = \exists p \in personne habite(p, "Rue du Havre") \land yeux(p, bleus) \land \lnot(gagnant(p, loto) \lor \lnot retraite\_avant(p, 50)$
\end{itemize}
\end{enumerate}

\subsubsection*{Exo 12}
\begin{enumerate}
\item P et Q
\item non P et non Q
\item P et non Q
\item non( Q et non P)
\item Q et non( Q et P)
\item non P ou non Q
\item non(Q et P)
\item non(Q ou P)
\end{enumerate}

\subsubsection*{Exo 13}
\begin{enumerate}
\item $\forall p \in poules,\, OntDesDents(p) \Rightarrow Mamifere(p) = \forall p \in poules,\, \lnot Mamifere(p) \Rightarrow \lnot OntDesDents(p)$ par contrapos\'ee et $\forall p \in poules,\, \lnot Mamifere(p)$ donc par modus ponens $\lnot OntDesDents(p)$. Raisonnement valide.  
\item P1:"assiste et non bavarde et ecoute $\Rightarrow$ reussi\_cours", P2:"ecoute $\Rightarrow$ assiste et non bavarde", P3:"ecoute". Donc par Modus Ponens sur P2 et P3 on a P4:"assiste et non bavarde"et Modus Ponens sur P3/P4 et P1, on a "reussi\_cours". Donc "ecoute $\Rightarrow$ reussi\_cours". Raisonnement valide.
\item P1:"viens\_fete(Pierre) $\Rightarrow$ triste(Marie)", P2:"triste(Marie) $\Rightarrow$ non viens\_fete(Jean)", P3: "non viens\_fete(Jean) $\Rightarrow$ non viens\_fete(Pierre)" , P1 et P2 par transitivit\'e donne P4:"viens\_fete(Pierre) $\Rightarrow$ non viens\_fete(Jean))", P3 et P4 par transitivit\'e donne P5:"viens\_fete(Pierre) $\Rightarrow$ non viens\_fete(Pierre))". Faux par contradiction. Raisonnement invalide.
\end{enumerate}


\subsubsection*{Exo 14}
\begin{itemize}
\item bois(j)
\item dors(j)
\item mange(j)
\item content(j)
\end{itemize}
\begin{enumerate}[P1:]
\item "non bois(j) et dors(j) $\Rightarrow$ non content(j)"
\item "bois(j) $\Rightarrow$ non content(j) et dors(j)"
\item "non mange(j) $\Rightarrow$ non content(j) ou dors(j) ou (non content(j) et dors(j))"
\item "mange(j) $\Rightarrow$ content(j) ou bois(j) ou (content(j) et bois(j))"
\item "content(aujourdhui)"
\end{enumerate}


\begin{itemize}
\item Contradiction P1 et P5 donne par Modus Ponens P6:"bois(aujourdhui) ou non dors(aujourdhui)". 
\item Soit bois(aujourdhui) est vrai. Bois(aujourdhui) et P2 donne par Modus Ponens "non content(aujourdhui) et dors(aujourdhui)". Contradiction donc bois(aujourdhui) est faux par l'absurde.
\item Soit non dors(aujourdhui) est vrai. COntrapos\'ee de P3 et P5 et P6 donne mange"aujourdhui)
\end{itemize}
Donc, aujourd'hui, il a mang\'e et il n'a pas dormi.

\subsubsection*{Exo 15}
\begin{enumerate}[P1:]
\item Si sur le lieu(N) alors Coupable. "lieu(N) $\Rightarrow$ C"
\item sur le lieu W. "lieu(W)"
\end{enumerate}

Cela fait "faux $\Rightarrow$ C" donc on ne peux pas dire si il est coupable ou non.\\
Il faudrait \'ecrire "lieu(N) $\Leftrightarrow$ C". Il est coupable si et seulement si il \'etait sur le lieu N.

\subsubsection*{Exo 16}
\begin{itemize}
\item Les deux proposition ne sont pas compl\'ementaires donc d'autre possibilit\'es existent
\item P1: "non mange(soupe) $\Rightarrow$ prison" et P2:"mange(soupe)". Cela fait "faux $\Rightarrow$ prison". Donc on ne peux pas conclure car faux implique vrai ou faux.
\item ??
\item P2" gagnant $\Rightarrow$ jouer" et P2:"jouer". En prenant la contrapos\'ee de P1 et P2 on a "Faux $\Rightarrow$ non gagnant". Donc on ne peux pas conclure car faux implique vrai ou faux.
\item Le titre est une g\'en\'eralisation de la r\'eponse sans mentionner la r\'ef\'erence \`a la situation actuelle. Dans une autre situation, la r\'eponse pourrait \^etre diff\'erente. ???
\end{itemize}

\subsubsection*{Exo 17}
\begin{itemize}
\item oasis(P) est vraie si la piste P m\`ene \`a un oasis
\end{itemize}
 
\begin{enumerate}[P1:]
\item oasis(droite) ou oasis(gauche)
\item non oasis(droite)
\item (P1 et P2) ou (non P1 et non P2)
\end{enumerate}

\begin{itemize}
\item Premier cas: "P1 et P2" "non oasis(droite) et (oasis(droite) ou oasis(gauche))" par distribution "(non oasis(droite) et oasis(droite)) ou (non oasis(droite) et oasis(gauche))",
\item Second cas: "non P1 et non P2" "non(non oasis(droite)) et non (oasis(droite) ou oasis(gauche)))", "oasis(droite) et non oasis(droite) et non oasis(gauche)", "Faux"
\end{itemize}
donc il faut prendre la piste de gauche.

\subsubsection*{Exo 18}
\begin{itemize}
\item coffre(P) est vraie si le portrait est dans le coffre P.
\end{itemize}

\begin{enumerate}[P1:]
\item coffre(1)
\item non coffre(2)
\item non coffre(1)
\item (P1 et non P2 et non P3) ou (non P1 et P2 et non P3) ou (P1 et non P2 et non P3)
\end{enumerate}

3 cas:
\begin{itemize}
\item (P1 et non P2 et non P3). "coffre(1) et non non coffre(2) et non non coffre(1)", "coffre(2) et coffre(1)", pas possible car un seul portrait.
\item (non P1 et P2 et non P3). "non coffre(1) et non coffre(2) et non non coffre(1)", "Faux"
\item (non P1 et non P2 et P3). "non coffre(1) et non non coffre(2) et non coffre(1)", "non coffre(1) et coffre(2)".
\end{itemize}

Le portrait est dans le coffre 2.

\end{document}

