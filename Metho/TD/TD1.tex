\documentclass[]{book}

%These tell TeX which packages to use.
\usepackage{array,epsfig}
\usepackage{amsmath}
\usepackage{amsfonts}
\usepackage{amssymb}
\usepackage{amsxtra}
\usepackage{amsthm}
\usepackage{mathrsfs}
\usepackage{color}
\usepackage{tikz}
\usepackage{fancyhdr}


%Here I define some theorem styles and shortcut commands for symbols I use often
\theoremstyle{definition}
\newtheorem{defn}{Definition}
\newtheorem{thm}{Theorem}
\newtheorem{cor}{Corollary}
\newtheorem*{rmk}{Remark}
\newtheorem{lem}{Lemma}
\newtheorem*{joke}{Joke}
\newtheorem{ex}{Example}
\newtheorem*{soln}{Solution}
\newtheorem{prop}{Proposition}

\newcommand{\lra}{\longrightarrow}
\newcommand{\ra}{\rightarrow}
\newcommand{\surj}{\twoheadrightarrow}
\newcommand{\graph}{\mathrm{graph}}
\newcommand{\bb}[1]{\mathbb{#1}}
\newcommand{\Z}{\bb{Z}}
\newcommand{\Q}{\bb{Q}}
\newcommand{\R}{\bb{R}}
\newcommand{\C}{\bb{C}}
\newcommand{\N}{\bb{N}}
\newcommand{\M}{\mathbf{M}}
\newcommand{\m}{\mathbf{m}}
\newcommand{\MM}{\mathscr{M}}
\newcommand{\HH}{\mathscr{H}}
\newcommand{\Om}{\Omega}
\newcommand{\Ho}{\in\HH(\Om)}
\newcommand{\bd}{\partial}
\newcommand{\del}{\partial}
\newcommand{\bardel}{\overline\partial}
\newcommand{\textdf}[1]{\textbf{\textsf{#1}}\index{#1}}
\newcommand{\img}{\mathrm{img}}
\newcommand{\ip}[2]{\left\langle{#1},{#2}\right\rangle}
\newcommand{\inter}[1]{\mathrm{int}{#1}}
\newcommand{\exter}[1]{\mathrm{ext}{#1}}
\newcommand{\cl}[1]{\mathrm{cl}{#1}}
\newcommand{\ds}{\displaystyle}
\newcommand{\vol}{\mathrm{vol}}
\newcommand{\cnt}{\mathrm{ct}}
\newcommand{\osc}{\mathrm{osc}}
\newcommand{\LL}{\mathbf{L}}
\newcommand{\UU}{\mathbf{U}}
\newcommand{\support}{\mathrm{support}}
\newcommand{\AND}{\;\wedge\;}
\newcommand{\OR}{\;\vee\;}
\newcommand{\Oset}{\varnothing}
\newcommand{\st}{\ni}
\newcommand{\wh}{\widehat}

%Pagination stuff.
\setlength{\topmargin}{-.3 in}
\setlength{\oddsidemargin}{0in}
\setlength{\evensidemargin}{0in}
\setlength{\textheight}{9.in}
\setlength{\textwidth}{6.5in}
\pagestyle{fancy}
\fancyhf{}
\rhead{Phys\_103}
\lhead{CCR1}
\rfoot{Page \thepage}

\begin{document}


\subsection*{Rappel de cours}
\subsubsection*{Travail}

\begin{itemize}
\item "pour tout", $\forall$
\item "il existe", $\exists$
\item "non", $\lnot$
\item "ou", $\lor$
\item "et", $\land$
\end{itemize}


\subsection*{TD1}
\subsubsection*{Exo 1}

\begin{tabular}{l l }
$x \in \R, x^2 = 4$ car $x = 2$ & $x \in \R, x^2 = 4 \Leftarrow x = 2$ \\
$z \in \Z, z = \bar{z}$ donc $z \in \R$ & $z \in \Z, z = \bar{z} \Leftrightarrow z \in \R$ \\
$x \in \R, x = \pi$ donc $e^{2ix}= 1$ & $x \in \R, x = \pi \Rightarrow e^{2ix}= 1$ \\
\end{tabular}

\subsubsection*{Exo 2}
\begin{enumerate}
\item $\forall x \in \R, x^2 > 0$
\item $\exists x \in \R, x > x^2$
\item $\lnot \exists x \in \R, \forall y \in \R, x > y$ ou $\forall x \in \R, \exists y \in \R, x > y$
\item $\exists x \in \R, \forall n \in \N, \forall m \in \N^{*} x, \neq \frac{n}{m}$
\item $\exists x \in \N, \forall n \in \N, \exists m \in \N, n = x . m$
\item $\forall x_1, x_2 \in \R, x_1 < x_2, \exists x, x_1 < x < x_2$
\item $\forall x_1, x_2, x_3 \in \R, x_1.x_2 \geq 0 \lor x_2.x_3 \geq 0 \lor x_1.x_3 \geq 0$
\end{enumerate}

\subsubsection*{Exo 4}
\begin{enumerate}
\item non( P et Q) = (non P) ou (non Q) $\neq$ (non P) et (non Q). Non elles ne sont pas la n\'egation l'une de l'autre.
\item non( P ou Q) = (non P) et (non Q). Oui elles sont la n\'egation l'une de l'autre.
\item non( P $\Rightarrow$ Q) = non Q $\Rightarrow$ non P (contrapos\'e) $\neq$ non P $\Rightarrow$ non Q. Non elles ne sont pas la n\'egation l'une de l'autre.
\end{enumerate}

\subsubsection*{Exo 6}
\begin{enumerate}
\item La contrapos\'ee de $A \Rightarrow B$ est $non B \Rightarrow non A$
\item P:"L'entier $(n^2 - 1)$ n'est pas divisible par 8" donc "l'entier n est pair" ou $\forall m \in \N, (n^2-1) \neq 8m \Rightarrow \exists x \in \N, n = 2x$. La contrapos\'ee de P est "l'entier n n'est pas pair (n est impair)" donc "l'entier $(n^2 - 1)$ est divisible par 8" ou $\forall x \in N, n \neq 2x \Rightarrow \exists m \in \N, (n^2-1) = 8m$.
\item un entier $n$ impair est de la forme $n = 2x+1$. Deux cas possibles, soit $x$ est pair, soit $x$ est impair. Donc $n=2(2k)+1=4k+1$ ou $n=2(2k+1)+1 = 4k+3$. Par cons\'equent $n = 4k +\{1,3\}$
\item $\forall x \in N, n \neq 2x \Rightarrow \exists m \in \N, (n^2-1) = 8m$. Donc $\exists k \in \N, n = 4k +\{1,3\} \Rightarrow \exists m \in \N, (n^2-1) = 8m$. Deux cas: $(4k+1)^2 -1 = 16k^2 + 8k = 8(2k^2+k)$ ou $(4k+3)^2 -1 = 16k^2+24k+8 = 8(2k^2+3k+1)$. Dans les 2 cas, $n$ est divisible par 8.
\item Oui, car la d\'emonstration de P est faite car nous avons montr\'e la contrapos\'ee de P.
\end{enumerate}

\subsubsection*{Exo 7}
\begin{enumerate}
\item P:$\exists i \in \{1..n\}, x_i - x_{i-1} < \frac{1}{n}$
\item $\lnot P = \lnot(\exists i \in \{1..n\}, x_i - x_{i-1} < \frac{1}{n}) = \forall i \in \{1..n\}, x_i - x_{i-1} > \frac{1}{n} = (x_1 - x_0)+(x_2 - x_1)...(x_{n}-x_{n-1}) > \frac{1}{n} + \frac{1}{n}...+ \frac{1}{n} = x_n - x_0 > 1 \Rightarrow faux$
\item  $(\lnot P \Rightarrow faux) \Leftrightarrow P$. Donc la propri\'et\'e $P$ est v\'erifi\'ee.
\end{enumerate}

\subsubsection*{Exo 9}
\begin{enumerate}
\item
\begin{enumerate}
\item $\forall a, mange(moi,a) \Rightarrow aime(moi,a)$
\item $\exists a, \lnot aime(moi, a) \land mange(moi, a)$
\item $\forall p, \lnot aime(p, legume) \Rightarrow forall a, \lnot mange(p, a)$
\end{enumerate} 
\end{enumerate}

\subsubsection*{Exo 10}

\subsubsection*{Exo 11}

\subsubsection*{Exo 12}

\subsubsection*{Exo 14}

\subsubsection*{Exo 15}

\subsubsection*{Exo 16}

\subsubsection*{Exo 17}

\subsubsection*{Exo 18}



\end{document}

