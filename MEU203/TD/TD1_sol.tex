\documentclass[]{book}

%These tell TeX which packages to use.
\usepackage{array,epsfig}
\usepackage{amsmath}
\usepackage{amsfonts}
\usepackage{amssymb}
\usepackage{amsxtra}
\usepackage{amsthm}
\usepackage{mathrsfs}
\usepackage{color}
\usepackage{pgfplots}

%Here I define some theorem styles and shortcut commands for symbols I use often
\theoremstyle{definition}
\newtheorem{defn}{Definition}
\newtheorem{thm}{Theorem}
\newtheorem{cor}{Corollary}
\newtheorem*{rmk}{Remark}
\newtheorem{lem}{Lemma}
\newtheorem*{joke}{Joke}
\newtheorem{ex}{Example}
\newtheorem*{soln}{Solution}
\newtheorem{prop}{Proposition}

\newcommand{\lra}{\longrightarrow}
\newcommand{\ra}{\rightarrow}
\newcommand{\surj}{\twoheadrightarrow}
\newcommand{\graph}{\mathrm{graph}}
\newcommand{\bb}[1]{\mathbb{#1}}
\newcommand{\Z}{\bb{Z}}
\newcommand{\Q}{\bb{Q}}
\newcommand{\R}{\bb{R}}
\newcommand{\C}{\bb{C}}
\newcommand{\N}{\bb{N}}
\newcommand{\M}{\mathbf{M}}
\newcommand{\m}{\mathbf{m}}
\newcommand{\MM}{\mathscr{M}}
\newcommand{\HH}{\mathscr{H}}
\newcommand{\Om}{\Omega}
\newcommand{\Ho}{\in\HH(\Om)}
\newcommand{\bd}{\partial}
\newcommand{\del}{\partial}
\newcommand{\bardel}{\overline\partial}
\newcommand{\textdf}[1]{\textbf{\textsf{#1}}\index{#1}}
\newcommand{\img}{\mathrm{img}}
\newcommand{\ip}[2]{\left\langle{#1},{#2}\right\rangle}
\newcommand{\inter}[1]{\mathrm{int}{#1}}
\newcommand{\exter}[1]{\mathrm{ext}{#1}}
\newcommand{\cl}[1]{\mathrm{cl}{#1}}
\newcommand{\ds}{\displaystyle}
\newcommand{\vol}{\mathrm{vol}}
\newcommand{\cnt}{\mathrm{ct}}
\newcommand{\osc}{\mathrm{osc}}
\newcommand{\LL}{\mathbf{L}}
\newcommand{\UU}{\mathbf{U}}
\newcommand{\support}{\mathrm{support}}
\newcommand{\AND}{\;\wedge\;}
\newcommand{\OR}{\;\vee\;}
\newcommand{\Oset}{\varnothing}
\newcommand{\st}{\ni}
\newcommand{\wh}{\widehat}

%Pagination stuff.
\setlength{\topmargin}{-.3 in}
\setlength{\oddsidemargin}{0in}
\setlength{\evensidemargin}{0in}
\setlength{\textheight}{9.in}
\setlength{\textwidth}{6.5in}
\pagestyle{empty}



\begin{document}

\subsection*{Rappel de cours}

\begin{defn}
Soit $(x_n)_n$ une suite r\'eelle. On dit que $(x_n)_n$ est une suite de Cauchy si elle v\'erifie la propri\'et\'e suivante :
$$\forall \epsilon > 0; \exists N \text{ tel que }, \forall p q \geq N, q \geq N \text{, on a } |x_p - x_q| < \epsilon$$
\end{defn}

\begin{defn}
Soit $E \subset \R$ une partie. On dit que $E$ est dense dans $\R$ si pour tout $x \in \R$, et pour tout $r > 0$, il existe $e \in E$ tel que $e$ est r-proche de $x$, soit $|x - e| < r$.
\end{defn}


\newpage
\subsection*{Exercice 1.1}


\begin{tikzpicture}
\draw (0,0) -- (10,0);
\node [label=b] at (1,0) {\textbullet};
\node [label=c] at (8,0) {\textbullet};
\node [label=a] at (4.5,0) {$\circ$};
\draw [<->,label=r] (1,-0.5) -- node[above] {r} ++ (3.5,0);
\end{tikzpicture}

Les points $b$ et $c$ co\"incident avec les bornes d'un intervalle ouverts $]b,c[$. L'intervalle \'equivalent de centre $a$ et de rayon $r$ est d\'efini par  le centre $a$ au milieu des 2 points $b$ et $c$, donc $a = \frac{b+c}{2}$ et le rayon soit la distance entre les points $a$ et $b$ (resp. $c$), donc $r=a-b=\frac{b+c}{2}-b = \frac{c-b}{2}$. 


\subsection*{Exercice 1.2}
$$I=[0+(-1),1+1] = [-1,2]$$
On a $s in [0,1] \equiv 0 \leq s \leq 1$ et $t in [-1,1] \equiv -1 \leq t \leq 1$. La borne inf\'erieure de l'intervalle $s+t$ est \'egale \`a $min(0+(-1), 0+1, 1+(-1), 1+1)$ car l'addition est une fonction strictement croissante. Mais on sait que $0+(-1)<0+1$,$0+(-1)<1+(-1)$, $0+(-1)<1+1$ car $0<1$ donc $min(\ldots) = 0+(-1)$. M\^eme raisonnement pour la borne sup\'erieur. Pour les bornes, les deux intervalles ont des bornes ferm\'ees donc la valeurs des bornes sont dans l'intervalles. par cons\'equent l'intervalle $s+t$ est \'egelement un intervalle ferm\'e Donc $I = [-1,2]$.    
$$I'=]0+(-1),1+1[ = ]-1,2[$$
Mais raisonnement sur la valeur des bornes. Maintenant l'intervalle $t$ est un intervalle ouvert. Donc ses bornes ne sont pas dans l'intervalle $t$. Par cons\'equent elles ne peuvent pas \^etre dans l'intervalle $s+t$. 

$$J=[3,8]$$
On a $s in [-2,-1] \equiv -2 \leq s \leq -1$ et $t in [-4,-3] \equiv -4 \leq t \leq -3$. La borne inf\'erieure de l'intervalle $s+t$ est \'egale \`a $min((-2)*(-4), (-2)*(-3), (-1)*(-4), (-1)*(-3)1+1)$ car la multiplication est une fonction strictement croissante. M\^eme raisonnement pour la borne sup\'erieur en prenant le max. Pour les bornes, les deux intervalles ont des bornes ferm\'ees donc la valeurs des bornes sont dans l'intervalles. par cons\'equent l'intervalle $s+t$ est \'egelement un intervalle ferm\'e Donc $I = [3,8]$.    
$$I'= ]3,8[$$
Mais raisonnement sur la valeur des bornes. Maintenant l'intervalle $t$ est un intervalle ouvert. Donc ses bornes ne sont pas dans l'intervalle $t$. Par cons\'equent elles ne peuvent pas \^etre dans l'intervalle $s+t$. 


\subsection*{Exercice 1.3}
\begin{tikzpicture}
\draw [<->,label=r] (1,0) -- node[above] {I} ++ (7,0);
\draw [<->,label=r] (4,-0.5) -- node[above] {J} ++ (5,0);
\end{tikzpicture}

On a $I \cap J = \emptyset$ donc $\exists a$ tel que $a \in I$ et $a \in J$. On a $a \leq \sup(I) (ou +\infty)$ et $\inf(J) (ou -\infty) \leq a$, donc $\inf(J) (ou -\infty) \leq a \leq \sup(I) (ou +\infty)$.\\

Prenons $a,b \in (I \cup J)^2, a < b$, a-t-on $[a,b] \subset I \cup J$? 
Plusieurs cas:
\begin{itemize}
\item $(a,b) \in I^2$, $I$ est un intervalle donc $[a,b] \subset I$ (caract\'erisation des intervalles) et $I \subset (I \cup J)$ (par d\'efinition, donc $[a,b] \subset (I \cup J)$.
\item $(a,b) \in J^2$, m\^eme raisonnement 
\item $a \in I, b \in J$, $I$ est un intervalle donc $[a,\sup(I)] \subset I$ (caract\'erisation des intervalles) et $[\inf(J),b] \subset J$, on a $\inf(J) \leq \sup(I)$ 
\item $a \in J, b \in I$, ???
\end{itemize}


\subsection*{Exercice 1.4}


\subsection*{Exercice 1.5} 
\subsubsection*{A}
L'intervalle $A$ est minor\'e par $-1$ ($A \subset [-1,+\infty]$) et major\'e par $2$ ($A \subset [-\infty; 2]$) donc l'intervalle $A$ est born\'ee. Le diam\`etre $d(A) = \sup(d(x,x'),x,x' \in A)$. Le diam\`etre de $A = 1-0 = 1$.

\subsubsection*{B}
L'intervalle $B$ est minor\'e par $-3$ ($B \subset [-3,+\infty]$) et major\'e par $4$ ($B \subset [-\infty; 4]$) donc l'intervalle $B$ est born\'ee. Le diam\`etre $d(B) = \sup(d(x,x'), \forall x,x' \in B)$. Au passage \`a la limite le diam\`etre de $B = \lim_{\epsilon \to 0} (3-\epsilon)-(-2+\epsilon) = 5$.

\subsubsection*{C}
La partie $C$ est born\'ee par $7$ car $\forall x \in C, |x| \leq 7$. Le diam\'etre de $C$ est $d(C) = \sup(d(x,x'), \forall x,x' \in C)$. Le diam\`etre de $C = \max(C) - \min(C) = 6 -4 = 2$.  

\subsubsection*{D}
La partie $D$ est born\'ee par $2$ car $\forall x \in D, |x| \leq 2$. Le diam\'etre de $D$ est $d(D) = \sup(d(x,x'), \forall x,x' \in D)$. Le diam\`etre de $C = max(C)-min(C) = \lim_{\epsilon \to 0} (1-\epsilon) - (0+\epsilon) = 1$ .  


\subsection*{Exercice 1.6}
\subsubsection*{1.6.1}
La partie $E$ est born\'ee donc $\exists M,m \in \R, E \subset [m,M]$, on a $F \subset E$, donc $F \subset [m,M]$ (car l'inclusion est transitive). $m$ est un minorant de $F$ et $M$ est un majorant de $F$ dons $F$ est born\'ee.

\subsubsection*{1.6.2}
Soit $E, F$ deux parties born\'ees. $\exists M_E,m_E \in \R, E \subset [m_E,M_E]$ et $\exists M_F,m_F \in \R, E \subset [m_F,M_F]$. On a $\forall x, x \in E et x \in F, x \in [m_E, M_E] et x \in [m_F, M_F]$ Donc $x \geq m_E et x \geq m_F \implies x \geq \max(m_E, m_F)$, de m\^eme pour le majorant. Donc $E \cup F \in [\max(m_E,m_F), \min(M_E, M_F)]$, $E \cup F$ est born\'ee.

\subsubsection*{1.6.3}
Soit $E, F$ deux parties born\'ees. $\exists M_E,m_E \in \R, E \subset [m_E,M_E]$ et $\exists M_F,m_F \in \R, E \subset [m_F,M_F]$. On a $\forall x, x \in E et x \in F, x \in [m_E, M_E] ou x \in [m_F, M_F]$ Donc $x \geq m_E ou x \geq m_F \implies x \geq \min(m_E, m_F)$, de m\^eme pour le majorant. Donc $E \cup F \in [\min(m_E,m_F), \max(M_E, M_F)]$, $E \cup F$ est born\'ee.

\subsection*{Exercice 1.7}
\subsubsection*{1.7.1}
Si $\R \setminus \{a\}$ est ouvert alors $\forall x \in \R \setminus \{a\}, \exists \epsilon > 0, ]x-\epsilon, x+\epsilon[ \subset \R \setminus \{a\}$. Soit un point $x=a+\alpha$ avec $\alpha>0$ quelconque. On a $x \in \R \setminus \{a\}$. Prenons $\epsilon = \frac{\alpha}{2}$. On a bien $]x-\epsilon, x+\epsilon[ \subset \R \setminus \{a\}$ car $a \notin ]x-\epsilon, x+\epsilon[$. Donc $\R \setminus \{a\}$ est ouvert. \\
$\{a\}$ est ferm\'e car $\R \setminus \{a\}$ est ouvert.
\subsubsection*{1.7.2}
De l'exercice pr\'ec\'edenton a $\forall n >1, \R \setminus \{a_n\}$ qui est ouvert. De plus, une r\'eunion quelconque d’ouverts de $\R$ est un ouvert de $\R$. Donc 
$\R \setminus \{a_1, a_2, \ldots a_n\}$ est ouvert. Par cons\'equent $\{a_1, a_2, \ldots a_n\}$ est ferm\'e.

\subsection*{Exercice 1.8}
\subsubsection*{1.8.1}
Si $[0,1]$ est ouvert alors $\forall x \in [0,1], \exists \epsilon > 0, ]x-\epsilon, x+\epsilon[ \subset [0,1]$. Prenons $x = 0$ (ou 1), on a $x \in [0,1]$ mais 
$x-\epsilon \notin [0,1]$. Donc $[0,1]$ n'est pas ouvert.\\

Si $]1,2[$ est ferm\'e alors $\R \setminus ]1,2[$ est ouvert. Donc $\forall x \in \R \setminus ]1,2[, \exists \epsilon > 0, ]x-\epsilon, x+\epsilon[ \subset \R \setminus ]1,2[$. Prenons $x = 1$ (ou 2), on a $x \in \R \setminus ]1,2[$ mais $x+\epsilon \notin \R \setminus ]1,2[, \forall \epsilon >0$. Donc $\R \setminus ]1,2[$ n'est pas ouvert et $]1,2[$ n'est pas ferm\'e.\\

$[-1,0[$, ni ouvert, ni ferm\'e. Tu fais.

\subsubsection*{1.8.2.a}
Si $\C$ est ouvert alors $\forall x \in \C, \exists \epsilon > 0, ]x-\epsilon, x+\epsilon[ \subset \C$. Prenons un nombre r\'eel $x$ dont le developpement decimal \'egale \`a 5 est le dernier chiffre. On a $x \in \C$. Il n'existe pas de $\not\exists \epsilon, ]x-\epsilon, x+\epsilon[ \subset \C$ car soit $n$ la puissance associ\'e a 5. Si $\epsilon \leq 10^{n}, x-\frac{\epsilon}{2} \notin \C$ et si $\epsilon > 10^{n}, x-\frac{10^{n}}{2} \in ]x-\epsilon,x+\epsilon[ mais \notin \C$ car il ne contient pas de 5 dans son developpement limit\'e.

\subsubsection*{1.8.2.b}
On a $x \in \C$, donc $x$ contient au moins un 5 dans son d\'eveloppement limit\'e et au moins un 5 n'est pas le dernier chiffre (car $x$ n'est pas d\'ecimal). Soit $n$ le rang du premier chiffre $\neq 0$ apr\'es le premier 5 (il existe car $x$ n'est pas d\'ecimal), prenons $\epsilon = 10^{n}$  on a bien $\in ]x-\epsilon,x+\epsilon[ \in \C$ car tous les nombre contienne le premier 5.


\subsection*{Exercice 1.9}
\subsubsection*{1.9.1}
Soit $E = [a,b]$ ferm\'e donc $a \leq x \leq b$, et $M = \sup(E)$. D\'emonstration par l'absurde. Si $M \not\in E$ alors prenons $M_1 = \frac{b+M}{2}$. On a $M_1 > b$ donc c'est un majorant de $E$ et $M_1 < M$ ceci contredit $M=\sup(E)$. Donc $M \in E$.  

\subsubsection*{1.9.2}
Soit $E = ]a,b[$ ouvert donc $a < x < b$, et $M = \sup(E)$. Si $M \in E$ alors $M < b$ donc $M$ n'est pas un majorant de $E$. Ceci contredit $M=\sup(E)$. Donc $M \not\in E$.  

\subsubsection*{1.9.3}
??


\subsection*{Exercice 1.10}
\subsubsection*{1.10.1}
Prenons $p > q$ et calculons $|x_p - x_q|$ 
$$|x_p - x_q| = |\frac{\cos(0)}{10^0}+\frac{\cos(1)}{10^1}+\ldots+\frac{\cos(p)}{10^p} - (\frac{\cos(0)}{10^0}+\frac{\cos(1)}{10^1}+\ldots+\frac{\cos(p)}{10^p}+\ldots+\frac{\cos(q)}{10^q})|$$
$$=\frac{\cos(p+1)}{10^{p+1}}+\ldots+\frac{\cos(q)}{10^q} < \frac{1}{10^{p+1}}+\ldots+\frac{1}{10^q} = \frac{1}{10^{p+1}}\left(1+\frac{1}{10^1}+\ldots+\frac{1}{10^{q-p-1}}\right) < \frac{2}{10^{p+1}}$$
car $1+\frac{1}{10^1}+\ldots+\frac{1}{10^{q-p-1}} < 2$. Donc si on prend $N, \epsilon > \frac{2}{10^N+1}$, pour tout $\epsilon$, on a trouv\'e un entier $N$ tel que $(x_n)_n$ v\'erifie la propri\'et\'e de Cauchy.

\subsubsection*{1.10.2}


\subsection*{Exercice 1.11}
\subsubsection*{1.11.1}
Calculons $|x_p - x_q|$ avec $p > q$. 
$$|x_p - x_q| = \left|\sum_{n=0}^{p}{\frac{\cos(n)}{10^{n}}} - \sum_{n=0}^{q}{\frac{\cos(n)}{10^{n}}}\right| = \left|\sum_{n=p+1}^{q}{\frac{\cos(n)}{10^{n}}}\right| = \frac{1}{10^{p+1}}\left|\sum_{n=p+1}^{q}{\frac{\cos(n)}{10^{n-(p+1)}}}\right| $$
$$= \frac{1}{10^{p+1}}\left|\cos(p+1)+\frac{\cos(p+2)}{10}+\frac{\cos(p+2)}{10^2}+ \ldots +\frac{\cos(q)}{10^{q-(p+1)}}\right|$$
On a $-1<\cos(n)<1$, 
$$= \frac{1}{10^{p+1}}\left|1+\frac{1}{10}+\frac{1}{10^2}+ \ldots +\frac{1}{10^{q-(p+1)}}\right| < \frac{2}{10^{p+1}}$$

Donc $\forall p, q > N, |x_p - x_q| < \frac{2}{10^N}$. Ceci d\'emontre que $x_n$ est une suite de Cauchy.

\subsubsection*{1.11.2}
On a $\lim_{x \to \infty}xf'(x) = 0 \implies \lim_{x \to \infty}f'(x) = 0$ car on a $f'(x)$ qui tend vers 0 plus rapidement que $n$ croit. (ou $n = o(\frac{1}{f'(x)})$. Donc $\lim_{x \to \infty}f'(x) = 0 \implies  \lim_{x \to \infty}f(x) = C$ donc la fonction $f$ converge vers $C$ donc $(f(x_n))_n$ est de Cauchy.




QED

\end{document}

