\documentclass[]{book}

%These tell TeX which packages to use.
\usepackage{array,epsfig}
\usepackage{amsmath}
\usepackage{amsfonts}
\usepackage{amssymb}
\usepackage{amsxtra}
\usepackage{amsthm}
\usepackage{mathrsfs}
\usepackage{color}
\usepackage{pgfplots}

%Here I define some theorem styles and shortcut commands for symbols I use often
\theoremstyle{definition}
\newtheorem{defn}{Definition}
\newtheorem{thm}{Theorem}
\newtheorem{cor}{Corollary}
\newtheorem*{rmk}{Remark}
\newtheorem{lem}{Lemma}
\newtheorem*{joke}{Joke}
\newtheorem{ex}{Example}
\newtheorem*{soln}{Solution}
\newtheorem{prop}{Proposition}

\newcommand{\lra}{\longrightarrow}
\newcommand{\ra}{\rightarrow}
\newcommand{\surj}{\twoheadrightarrow}
\newcommand{\graph}{\mathrm{graph}}
\newcommand{\bb}[1]{\mathbb{#1}}
\newcommand{\B}{\mathrm{B}}
\newcommand{\Z}{\bb{Z}}
\newcommand{\Q}{\bb{Q}}
\newcommand{\R}{\bb{R}}
\newcommand{\C}{\bb{C}}
\newcommand{\N}{\bb{N}}
\newcommand{\M}{\mathbf{M}}
\newcommand{\m}{\mathbf{m}}
\newcommand{\MM}{\mathscr{M}}
\newcommand{\HH}{\mathscr{H}}
\newcommand{\Om}{\Omega}
\newcommand{\Ho}{\in\HH(\Om)}
\newcommand{\bd}{\partial}
\newcommand{\del}{\partial}
\newcommand{\bardel}{\overline\partial}
\newcommand{\textdf}[1]{\textbf{\textsf{#1}}\index{#1}}
\newcommand{\img}{\mathrm{img}}
\newcommand{\ip}[2]{\left\langle{#1},{#2}\right\rangle}
\newcommand{\inter}[1]{\mathrm{int}{#1}}
\newcommand{\exter}[1]{\mathrm{ext}{#1}}
\newcommand{\cl}[1]{\mathrm{cl}{#1}}
\newcommand{\ds}{\displaystyle}
\newcommand{\vol}{\mathrm{vol}}
\newcommand{\cnt}{\mathrm{ct}}
\newcommand{\osc}{\mathrm{osc}}
\newcommand{\LL}{\mathbf{L}}
\newcommand{\UU}{\mathbf{U}}
\newcommand{\support}{\mathrm{support}}
\newcommand{\AND}{\;\wedge\;}
\newcommand{\OR}{\;\vee\;}
\newcommand{\Oset}{\varnothing}
\newcommand{\st}{\ni}
\newcommand{\wh}{\widehat}

%Pagination stuff
\setlength{\topmargin}{-.3 in}
\setlength{\oddsidemargin}{0in}
\setlength{\evensidemargin}{0in}
\setlength{\textheight}{9.in}
\setlength{\textwidth}{6.5in}
\pagestyle{empty}



\begin{document}

\newpage
\subsection*{Exercice 1}
\subsubsection*{Exercice 1.1}
$$
\frac{\partial }{\partial x}f(x,y) = 8x -4y -4
$$
$$
\frac{\partial }{\partial y}f(x,y) = 20y - 4x -16
$$

\subsubsection*{Exercice 1.2}
Point critique est le point o\`u les 2 d\'eriv\'ees s'annulent.
$$
\left\{
\begin{array}{l}
8x - 4y - 4 = 0 \\
20y - 4x - 16 = 0 \\
\end{array}
\right.
\implies
\left\{
\begin{array}{l}
2x - y - 1 = 0 \\
10y -2x - 8 = 0 \\
\end{array}
\right.
\implies
\left\{
\begin{array}{l}
9y - 9 = 0 \\
2x - y - 1 = 0 \\
\end{array}
\right.
\implies
\left\{
\begin{array}{l}
y = 1 \\
2x - 2 = 0 \\
\end{array}
\right.
$$

Le point critique est $A = (1,1)$.

\subsubsection*{Exercice 1.3.a}
$f(x,0) = 4x^2 - 4x + 11$ on a $f(x,0) < 4x^2 +11$
donc $f(x,0) > C$ pour tout $x > M$ avec $M = \sqrt{|C-11|/4}$

\subsubsection*{Exercice 1.3.b}
Le point $A$ est un maximum global si $\forall (x,y) \in \R^2, f(x,y) < f(A) = f(1,1)$. Il suffit de trouver un contre exemple, ie un point $(x,y)$ tel que $ f(x,y) > f(1,1)$. On a $f(1,1) = 1$, il suffit de prendre le point $(0,0)$ car $f(0,0) = 11$. 

\subsubsection*{Exercice 1.4.a}
$$
g(x,y) = f(x,y) - 4x^2 + 4xy +4x -y^2 -2y -1 = 9y^2 -18y +10 
$$

$$
\frac{\partial }{\partial x}g(x,y) = \frac{\partial }{\partial x}(9y^2 -18y +10) = 0
$$

\subsubsection*{Exercice 1.4.b}
$$
g(x,y) = (ay+b)^2 + 1 = a^2y^2+2aby+b^2+1 = 9y^2-18y+10
$$
Donc $a^2=9$, $2ab = -18$  et $b^2=9$. Ce qui fait $a=3,b=-3$ ou $a=-3,b=3$.

\subsubsection*{Exercice 1.4.b}
Le point $A$ est un minimum global si $\forall (x,y) \in \R^2, f(x,y) \ge f(A) = f(1,1)$.	On a $g(x,y) = f(x,y) - (2x-y-1)^2 = (ay+b)^2+1$. Donc $f(x,y) = (ay+b)^2+1 + (2x-y-1)^2$ et $f(x,y) = 1$. On a $(ay+b)^2 \ge 0$ et $(2x-y-1)^2 \ge 0$ donc $(ay+b)^2 + (2x-y-1)^2 +1 \ge 1$. Donc le point $A$ est un minimum global.


\subsection*{Exercice 2}
\subsubsection*{Exercice 2.1}
$$
\frac{\partial }{\partial x}f(x,y) = \frac{\partial }{\partial x} (1+x+\sin(y))e^{-x} = −(x+\sin(y))e^{−x}
$$
$$
\frac{\partial }{\partial y}f(x,y) = \frac{\partial }{\partial y} (1+x+\sin(y))e^{-x} = e^{-x}\cos(y)
$$


\subsection*{Exercice 3}
\subsubsection*{Exercice 3.1}
??
\subsubsection*{Exercice 3.2}
Gradient en tout point $(x,y) \in \R^2$.
$$
\nabla f : (x,y) \to \left\{
\begin{array}{l}
\frac{\partial }{\partial x}f(x,y) = 8x - 4y -4 \\
\frac{\partial }{\partial y}f(x,y) = 2y - 4x -2 \\
\end{array}
\right.
$$

\subsubsection*{Exercice 3.3}
$f(0,2) = 0$ donc le point $(0,2) \in \L$

Gradient au point $(0,2)$.
$$\nabla_{(0,2)} f = (-12,2)$$

Tangente en un point $(x_0,y_0)$ est
$$\{(x,y) \in \R^2, \frac{\partial }{\partial x}f(x_0,y_0)(x-x_0) + \frac{\partial }{\partial y}f(x_0,y_0)(y-y_0) = 0\}$$ 
$$\{(x,y) \in \R^2, -12(x-0) + 2 (y-2) = -12x + 2y -4 = 0\}$$
$$\{(x,y) \in \R^2,  y = 6x + 2\}$$

Coefficent directeur est 6.

\subsubsection*{Exercice 3.4.a}
$f(x,h_{+}(x)) = 0$?
$$f(x,h_{+}(x)) = 4x^2 -4x(1+2x+\sqrt{1+8x}) + (1+2x+\sqrt{1+8x})^2 - 4x -2(1+2x+\sqrt{1+8x})$$
$$ = 4x^2 -4x -8x^2 -4x\sqrt{1+8x} + 1 +2x + \sqrt{1+8x} + 2x +4x^2 +2x\sqrt{1+8x} + \sqrt{1+8x} +2x\sqrt{1+8x}+1+8x -4x -2 + 4x -2\sqrt{1+8x}= 0$$
ou
$$f(x,h_{+}(x)) = (2x-(1+2x+\sqrt{1+8x}))^2 - 4x -2(1+2x+\sqrt{1+8x}) = (-1-\sqrt{1+8x})^2 - 4x - 2 - 4x - 2\sqrt{1+8x}$$
$$= 1 + 2\sqrt{1+8x} + 1 +8x - 8x -2 - 2\sqrt{1+8x} = 0$$

$f(x,h_{-}(x)) = 0$?
$$f(x,h_{-}(x)) = 4x^2 -4x(1+2x-\sqrt{1+8x}) + (1+2x-\sqrt{1+8x})^2 - 4x -2(1-2x-\sqrt{1+8x})$$
$$ = 4x^2 -4x -8x^2 + 4x\sqrt{1+8x} + 1 +2x - \sqrt{1+8x} + 2x +4x^2 - 2x\sqrt{1+8x} - \sqrt{1+8x} -2x\sqrt{1+8x}+1+8x -4x -2 + 4x +2\sqrt{1+8x}= 0 $$
ou
$$f(x,h_{+}(x)) = (2x-(1+2x-\sqrt{1+8x}))^2 - 4x -2(1+2x-\sqrt{1+8x}) = (-1+\sqrt{1+8x})^2 - 4x - 2 - 4x + 2\sqrt{1+8x}$$
$$= 1 - 2\sqrt{1+8x} + 1 +8x - 8x -2 + 2\sqrt{1+8x} = 0$$

\subsubsection*{Exercice 3.4.b}
$$h'_{+}(x) = 2 + \frac{4}{\sqrt{1+8x}}$$
$$h'_{+}(0) = 2 + \frac{4}{\sqrt{1+8.0}} = 6$$

On a $h_{+}(0) = 2$, donc le point $(0,h_{+}(0)) = (0,2)$. On sait que $h_{+} \in \L$. Donc la ...


QED

\end{document}

