\documentclass[]{book}

%These tell TeX which packages to use.
\usepackage{array,epsfig}
\usepackage{amsmath}
\usepackage{amsfonts}
\usepackage{amssymb}
\usepackage{amsxtra}
\usepackage{amsthm}
\usepackage{mathrsfs}
\usepackage{color}

%Here I define some theorem styles and shortcut commands for symbols I use often
\theoremstyle{definition}
\newtheorem{defn}{Definition}
\newtheorem{thm}{Theorem}
\newtheorem{cor}{Corollary}
\newtheorem*{rmk}{Remark}
\newtheorem{lem}{Lemma}
\newtheorem*{joke}{Joke}
\newtheorem{ex}{Example}
\newtheorem*{soln}{Solution}
\newtheorem{prop}{Proposition}

\newcommand{\lra}{\longrightarrow}
\newcommand{\ra}{\rightarrow}
\newcommand{\surj}{\twoheadrightarrow}
\newcommand{\graph}{\mathrm{graph}}
\newcommand{\bb}[1]{\mathbb{#1}}
\newcommand{\Z}{\bb{Z}}
\newcommand{\Q}{\bb{Q}}
\newcommand{\R}{\bb{R}}
\newcommand{\C}{\bb{C}}
\newcommand{\N}{\bb{N}}
\newcommand{\M}{\mathbf{M}}
\newcommand{\m}{\mathbf{m}}
\newcommand{\MM}{\mathscr{M}}
\newcommand{\HH}{\mathscr{H}}
\newcommand{\Om}{\Omega}
\newcommand{\Ho}{\in\HH(\Om)}
\newcommand{\bd}{\partial}
\newcommand{\del}{\partial}
\newcommand{\bardel}{\overline\partial}
\newcommand{\textdf}[1]{\textbf{\textsf{#1}}\index{#1}}
\newcommand{\img}{\mathrm{img}}
\newcommand{\ip}[2]{\left\langle{#1},{#2}\right\rangle}
\newcommand{\inter}[1]{\mathrm{int}{#1}}
\newcommand{\exter}[1]{\mathrm{ext}{#1}}
\newcommand{\cl}[1]{\mathrm{cl}{#1}}
\newcommand{\ds}{\displaystyle}
\newcommand{\vol}{\mathrm{vol}}
\newcommand{\cnt}{\mathrm{ct}}
\newcommand{\osc}{\mathrm{osc}}
\newcommand{\LL}{\mathbf{L}}
\newcommand{\UU}{\mathbf{U}}
\newcommand{\support}{\mathrm{support}}
\newcommand{\AND}{\;\wedge\;}
\newcommand{\OR}{\;\vee\;}
\newcommand{\Oset}{\varnothing}
\newcommand{\st}{\ni}
\newcommand{\wh}{\widehat}

%Pagination stuff.
\setlength{\topmargin}{-.3 in}
\setlength{\oddsidemargin}{0in}
\setlength{\evensidemargin}{0in}
\setlength{\textheight}{9.in}
\setlength{\textwidth}{6.5in}
\pagestyle{empty}



\begin{document}


\subsection*{Exo 3.1.3}
\subsubsection*{Q1}
cours: La position d'\'equilibre de la masse est lorsque la force de rappel \'elastique $F_r=-k(l_e-l_0)$
est \'egale \`a la force gravitationnelle $F_g = m.g$. Donc,
$$F_e = F_g$$
$$m.g = -k(l_e-l_0)$$
$$l_e = \frac{k.l_o - m.g}{k}$$


\subsubsection*{Q2}
(a) La d\'eriv\'ee premi\`ere de $f(x) = sin(ax)$ est $f'(x) = a.cos(ax)$. 
La d\'eriv\'ee seconde de $f(x)$ est $f''(x) = -a^2.sin(ax)$. 

(b) Une solution de l'\'equation $f''(x) = -a^2f(x)$ est $f(x) = C.sin(ax)$. 
En effet, $f''(x) = -C.a^2.sin(ax) = -a^2.C.sin(ax) = -a^2 f(x)$.

Une autre solution de l'\'equation $f''(x) = -a^2f(x)$ est $f(x) = C.cos(ax)$. 
En effet, $f''(x) = -C.a^2.cos(ax) = -a^2.C.cos(ax) = -a^2 f(x)$.

La solution g\'en\'erale de de l'\'equation $f''(x) = -a^2f(x)$ est $f(x) = C_1.cos(ax) + C_2.sin(ax)$
En effet, $f''(x) = -C_1.a^2.cos(ax) - C_2.a^2.sin(ax) = -a^2 (C_1.cos(ax) + C_2.sin(ax)) = -a^2 f(x)$.


\subsubsection*{Q3}
\emph{Identification du syst\`eme} : Le syst\`eme \'etudi\'e est la masse
$m$ qui est accroch\'ee au ressort.


\emph{Bilan des forces} : La masse est soumise \`a trois forces; 
\begin{itemize}
\item la force de rappel du ressort qui est proportionnelle \`a l'allongement
du ressort $F_r = -k.(y(t) + y_0)\overrightarrow{i}$.
\end{itemize}


\emph{Composantes dans un rep\`ere cart\'esien} : Le mouvement est rectiligne.
Il a lieu le long de l'axe $O_y$. Il est avantageux de prendre la
position de la bille \`a l'\'equilibre pour l'origine O des coordonn\'ees sur
l'axe $O_y$. Donc, nous avons $y(t) = l(t) - (l_e)$ et $\overrightarrow{g}=g.\overrightarrow{i}$$


\emph{PFD} : La r\'esulante des forces est \'egale \`a 
$m.\overrightarrow{a}=m.\frac{d^2y(t)}{dt}\overrightarrow{i}$

$$m.g.\overrightarrow{i} -k.y(t).\overrightarrow{i} = $$


\emph{Solution}

\end{document}

