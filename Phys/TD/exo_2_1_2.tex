\documentclass[]{book}

%These tell TeX which packages to use.
\usepackage{array,epsfig}
\usepackage{amsmath}
\usepackage{amsfonts}
\usepackage{amssymb}
\usepackage{amsxtra}
\usepackage{amsthm}
\usepackage{mathrsfs}
\usepackage{color}

%Here I define some theorem styles and shortcut commands for symbols I use often
\theoremstyle{definition}
\newtheorem{defn}{Definition}
\newtheorem{thm}{Theorem}
\newtheorem{cor}{Corollary}
\newtheorem*{rmk}{Remark}
\newtheorem{lem}{Lemma}
\newtheorem*{joke}{Joke}
\newtheorem{ex}{Example}
\newtheorem*{soln}{Solution}
\newtheorem{prop}{Proposition}

\newcommand{\lra}{\longrightarrow}
\newcommand{\ra}{\rightarrow}
\newcommand{\surj}{\twoheadrightarrow}
\newcommand{\graph}{\mathrm{graph}}
\newcommand{\bb}[1]{\mathbb{#1}}
\newcommand{\Z}{\bb{Z}}
\newcommand{\Q}{\bb{Q}}
\newcommand{\R}{\bb{R}}
\newcommand{\C}{\bb{C}}
\newcommand{\N}{\bb{N}}
\newcommand{\M}{\mathbf{M}}
\newcommand{\m}{\mathbf{m}}
\newcommand{\MM}{\mathscr{M}}
\newcommand{\HH}{\mathscr{H}}
\newcommand{\Om}{\Omega}
\newcommand{\Ho}{\in\HH(\Om)}
\newcommand{\bd}{\partial}
\newcommand{\del}{\partial}
\newcommand{\bardel}{\overline\partial}
\newcommand{\textdf}[1]{\textbf{\textsf{#1}}\index{#1}}
\newcommand{\img}{\mathrm{img}}
\newcommand{\ip}[2]{\left\langle{#1},{#2}\right\rangle}
\newcommand{\inter}[1]{\mathrm{int}{#1}}
\newcommand{\exter}[1]{\mathrm{ext}{#1}}
\newcommand{\cl}[1]{\mathrm{cl}{#1}}
\newcommand{\ds}{\displaystyle}
\newcommand{\vol}{\mathrm{vol}}
\newcommand{\cnt}{\mathrm{ct}}
\newcommand{\osc}{\mathrm{osc}}
\newcommand{\LL}{\mathbf{L}}
\newcommand{\UU}{\mathbf{U}}
\newcommand{\support}{\mathrm{support}}
\newcommand{\AND}{\;\wedge\;}
\newcommand{\OR}{\;\vee\;}
\newcommand{\Oset}{\varnothing}
\newcommand{\st}{\ni}
\newcommand{\wh}{\widehat}

%Pagination stuff.
\setlength{\topmargin}{-.3 in}
\setlength{\oddsidemargin}{0in}
\setlength{\evensidemargin}{0in}
\setlength{\textheight}{9.in}
\setlength{\textwidth}{6.5in}
\pagestyle{empty}



\begin{document}


\subsection*{Exo 2.1.2}
\subsubsection*{Q1}
(a) $F(x)$ est la primitive de la fonction $f(x)$ si la d\'eriv\'ee de
la fonction $F(x)$est \'egale \`a la fonction $f(x)$.
$$F'(x) = f(x)$$

(b) Lorsque $f(x) = C$, les primitives de la fonction $f(x)$ sont
$F(x) = Cx + C_0$ avec $C_0$ une constante arbitraire.

(c) Lorsque $g(x) = Cx$, les primitives de la fonction $g(x)$ sont
$G(x) = \frac{Cx^2}{2} + C_0$ avec $C_0$ une constante arbitraire.

\subsubsection*{Q2}
La vitesse $v(t)$ est la primitive de l'acc\'el\'eration. Entre
les instants $t=0$ et $t=2$, l'acc\'el\'eration est constante et
\'egale \`a $2m/s^2$. Donc, $a(t) = 2$. Par cons\'equent l'\'equation
de la vitesse est $v(t) = 2t+C_0$. Comme le mobile est sans vitesse
initiale, donc $v(0)=0$, donc $C_0 = 0$.
\\
La distance $x(t)$ est la primitive de la vitesse. Entre
les instants $t=0$ et $t=2$, la vitesse est $v(t) = 2t$. Par cons\'equent l'\'equation
de la distance est $x(t) = t^2+C_0$. Comme le mobile est \`a l'origine
\`a l'instant 0.
initiale, donc $x(0)=0$, donc $C_0 = 0$.
$$v(t)=2t, x(t)=t^2$$


\subsubsection*{Q3}
La vitesse $v(t)$ est la primitive de l'acc\'el\'eration. Entre
les instants $t=2$ et $t=4$, l'acc\'el\'eration est constante et
\'egale \`a $-2m/s^2$. Donc, $a(t) = -2$. Par cons\'equent l'\'equation
de la vitesse est $v(t) = -2t+C_0$. \`A l'instant $t=2$, donc
$v(2)=2*2=4$, donc $C_0 = 8$.
\\
La distance $x(t)$ est la primitive de la vitesse. Entre
les instants $t=2$ et $t=4$, la vitesse est $v(t) = -2t+8$. Par cons\'equent l'\'equation
de la distance est $x(t) = -t^2+8t+C_0$. \`A l'instant $t=2$, donc $x(2)=2^2=4$, donc $C_0 = -8$.
$$v(t)=-2t+8, x(t)=-t^2+8t-8$$


La vitesse s'annule lorsque $v(t)=0$. $v(t) = 2t = 0 \implies t = 0$ et
$v(t) = -2t +8 = 0 \implies t = 4$. La position du mobile lorsque la
vitesse est nulle est $x(0) = 0$ et $x(4) = 8$.

\subsubsection*{Q4}
La vitesse moyenne entre les instants $t=0$ et $t=2$ est
$v_{moy}=\frac{x(2)-x(0)}{2-0}=2 m/s$

\end{document}

