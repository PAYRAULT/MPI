\documentclass[]{book}

%These tell TeX which packages to use.
\usepackage{array,epsfig}
\usepackage{amsmath}
\usepackage{amsfonts}
\usepackage{amssymb}
\usepackage{amsxtra}
\usepackage{amsthm}
\usepackage{mathrsfs}
\usepackage{color}

%Here I define some theorem styles and shortcut commands for symbols I use often
\theoremstyle{definition}
\newtheorem{defn}{Definition}
\newtheorem{thm}{Theorem}
\newtheorem{cor}{Corollary}
\newtheorem*{rmk}{Remark}
\newtheorem{lem}{Lemma}
\newtheorem*{joke}{Joke}
\newtheorem{ex}{Example}
\newtheorem*{soln}{Solution}
\newtheorem{prop}{Proposition}

\newcommand{\lra}{\longrightarrow}
\newcommand{\ra}{\rightarrow}
\newcommand{\surj}{\twoheadrightarrow}
\newcommand{\graph}{\mathrm{graph}}
\newcommand{\bb}[1]{\mathbb{#1}}
\newcommand{\Z}{\bb{Z}}
\newcommand{\Q}{\bb{Q}}
\newcommand{\R}{\bb{R}}
\newcommand{\C}{\bb{C}}
\newcommand{\N}{\bb{N}}
\newcommand{\M}{\mathbf{M}}
\newcommand{\m}{\mathbf{m}}
\newcommand{\MM}{\mathscr{M}}
\newcommand{\HH}{\mathscr{H}}
\newcommand{\Om}{\Omega}
\newcommand{\Ho}{\in\HH(\Om)}
\newcommand{\bd}{\partial}
\newcommand{\del}{\partial}
\newcommand{\bardel}{\overline\partial}
\newcommand{\textdf}[1]{\textbf{\textsf{#1}}\index{#1}}
\newcommand{\img}{\mathrm{img}}
\newcommand{\ip}[2]{\left\langle{#1},{#2}\right\rangle}
\newcommand{\inter}[1]{\mathrm{int}{#1}}
\newcommand{\exter}[1]{\mathrm{ext}{#1}}
\newcommand{\cl}[1]{\mathrm{cl}{#1}}
\newcommand{\ds}{\displaystyle}
\newcommand{\vol}{\mathrm{vol}}
\newcommand{\cnt}{\mathrm{ct}}
\newcommand{\osc}{\mathrm{osc}}
\newcommand{\LL}{\mathbf{L}}
\newcommand{\UU}{\mathbf{U}}
\newcommand{\support}{\mathrm{support}}
\newcommand{\AND}{\;\wedge\;}
\newcommand{\OR}{\;\vee\;}
\newcommand{\Oset}{\varnothing}
\newcommand{\st}{\ni}
\newcommand{\wh}{\widehat}

%Pagination stuff.
\setlength{\topmargin}{-.3 in}
\setlength{\oddsidemargin}{0in}
\setlength{\evensidemargin}{0in}
\setlength{\textheight}{9.in}
\setlength{\textwidth}{6.5in}
\pagestyle{empty}



\begin{document}


\subsection*{Exo 2.2.2}
\subsubsection*{Q1}



\subsubsection*{Q2}


\subsubsection*{Q3}
$x(t) = 2acos(2\omega t))$, et $\left(cos(f)\right)' = -f'sin(f)$ avec
$f=2\omega t$. Donc
$v_x(t) = x'(t) = 2a*-2\omega sin(2\omega t)$.


$y(t)=4asin(2\omega t)$ et $\left(sin(f)\right)' = f'cos(f)$ avec
$f=2\omega t$. Donc 
$v_x(t) = x'(t) = 4a*2\omega cos(2\omega t)$.


$$
v(t) = \left\{
\begin{array}{l} 
  v_x(t) = x'(t) = -4a\omega sin(2\omega t) \\
  v_y(t) = y'(t) =  8a \omega cos(2\omega t)\\
  v_z(t) = z'(t) =  0 \\
\end{array}
\right\}
$$

$a_x(t) = v_x(t)$ et $v_x(t) = -4a\omega sin(2\omega t)$. et $\left(sin(f)\right)' = f'cos(f)$ avec
$f=2\omega t$. Donc 
$a_x(t) = = -4a\omega * 2 \omega cos(2\omega t)$

$a_y(t) = v_y(t)$ et $v_y(t) = 8a \omega cos(2\omega t)$ et $\left(cos(f)\right)' = -f'sin(f)$ avec
$f=2\omega t$. Donc
$a_y(t) = 8a \omega *-2 \omega sin(2\omega t)$.


$$a(t) = \left\{
\begin{array}{l} 
  a_x(t) = v'_x(t) = -8a \omega^2 cos(2\omega t)\\
  a_y(t) = v'_y(t) = -16a \omega^2 sin(2 \omega t) \\
  a_z(t) = v'_z(t) = 0 \\
\end{array}
\right\}
$$


Calcul de $v(0)$
$$
v(t) = \left\{
\begin{array}{l} 
  v_x(0) = -4a\omega sin(2\omega * 0) = 0\\
  v_y(0) =  8a \omega cos(2\omega * 0) = 8a\omega = 8\\
  v_z(0) =  0 \\
\end{array}
\right\}
$$

Calcul de $v(\pi/2)$
$$
v(t) = \left\{
\begin{array}{l} 
  v_x(\pi/2) =  -4a\omega sin(2\omega * \pi/2) = 0\\
  v_y(\pi/2) =  8a \omega cos(2\omega * \pi/2) = -8a\\
  v_z(\pi/2) =  0 \\
\end{array}
\right\}
$$



\end{document}

