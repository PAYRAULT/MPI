\documentclass[]{book}

%These tell TeX which packages to use.
\usepackage{array,epsfig}
\usepackage{amsmath}
\usepackage{amsfonts}
\usepackage{amssymb}
\usepackage{amsxtra}
\usepackage{amsthm}
\usepackage{mathrsfs}
\usepackage{color}
\usepackage{pgfplots}
\usepackage{fancyhdr}

%Here I define some theorem styles and shortcut commands for symbols I use often
\theoremstyle{definition}
\newtheorem{defn}{Definition}
\newtheorem{thm}{Theorem}
\newtheorem{cor}{Corollary}
\newtheorem*{rmk}{Remark}
\newtheorem{lem}{Lemma}
\newtheorem*{joke}{Joke}
\newtheorem{ex}{Example}
\newtheorem*{soln}{Solution}
\newtheorem{prop}{Proposition}

\newcommand{\lra}{\longrightarrow}
\newcommand{\ra}{\rightarrow}
\newcommand{\surj}{\twoheadrightarrow}
\newcommand{\graph}{\mathrm{graph}}
\newcommand{\bb}[1]{\mathbb{#1}}
\newcommand{\Z}{\bb{Z}}
\newcommand{\Q}{\bb{Q}}
\newcommand{\R}{\bb{R}}
\newcommand{\C}{\bb{C}}
\newcommand{\N}{\bb{N}}
\newcommand{\M}{\mathbf{M}}
\newcommand{\m}{\mathbf{m}}
\newcommand{\MM}{\mathscr{M}}
\newcommand{\HH}{\mathscr{H}}
\newcommand{\Om}{\Omega}
\newcommand{\Ho}{\in\HH(\Om)}
\newcommand{\bd}{\partial}
\newcommand{\del}{\partial}
\newcommand{\bardel}{\overline\partial}
\newcommand{\textdf}[1]{\textbf{\textsf{#1}}\index{#1}}
\newcommand{\img}{\mathrm{img}}
\newcommand{\ip}[2]{\left\langle{#1},{#2}\right\rangle}
\newcommand{\inter}[1]{\mathrm{int}{#1}}
\newcommand{\exter}[1]{\mathrm{ext}{#1}}
\newcommand{\cl}[1]{\mathrm{cl}{#1}}
\newcommand{\ds}{\displaystyle}
\newcommand{\vol}{\mathrm{vol}}
\newcommand{\cnt}{\mathrm{ct}}
\newcommand{\osc}{\mathrm{osc}}
\newcommand{\LL}{\mathbf{L}}
\newcommand{\UU}{\mathbf{U}}
\newcommand{\support}{\mathrm{support}}
\newcommand{\AND}{\;\wedge\;}
\newcommand{\OR}{\;\vee\;}
\newcommand{\Oset}{\varnothing}
\newcommand{\st}{\ni}
\newcommand{\wh}{\widehat}

%Pagination stuff.
\setlength{\topmargin}{-.3 in}
\setlength{\oddsidemargin}{0in}
\setlength{\evensidemargin}{0in}
\setlength{\textheight}{9.in}
\setlength{\textwidth}{6.5in}
\pagestyle{fancy}
\fancyhf{}
\rhead{Math\_103}
\lhead{Exam\_1}
\rfoot{Page \thepage}



\begin{document}

\subsection*{Rappel de cours}

\begin{itemize}
\item 
\end{itemize}

\subsection*{Exercice 1}
R\'esoudre l'\'equation diff\'erentielle :
$$y'(t) - \frac{t}{t^2+1}y(t) = t$$

\subsubsection*{1.1}
Si l'\'equation homog\`ene est de la forme:
$$y'(t) = a(t)y(t)$$
Alors la solution est
$$y_0(t) = \lambda e^{A(t)}\; avec\; A'(t) = a(t)$$

Dans notre cas on a $a(t) = \frac{t}{t^2+1}$, on cherche $A(t) = \int{\frac{t}{t^2+1}dt}$.
Par substitution $u = t^2 +1$, ce qui fait $\frac{du}{dt} = 2t$, donc $dt = \frac{1}{2t}du$.
$$\int{\frac{t}{t^2+1}dt} = \int{\frac{t}{t^2+1}\frac{1}{2t}du} = \frac{1}{2}\int{\frac{1}{u}du} = \frac{1}{2}\ln(u) = \frac{1}{2}\ln(t^2+1)$$

Donc
$$y_0(t) = \lambda e^{\frac{1}{2}\ln(t^2+1)} = \lambda \sqrt{e^{\ln(t^2+1)}} = \lambda \sqrt{t^2+1)}$$

\subsubsection*{1.2}
Calculer une solution particuli\`ere par la methode de la variation de la constante.
Si l'\'equation homog\`ene est de la forme:
$$y'(t) = a(t)y(t) + b(t)$$
Alors la solution est
$$y_1(t) = \lambda(t) e^{A(t)}\; avec\; A'(t) = a(t)\; et\; \lambda'(t) = b(t)e^{-A(t)}$$

Dans notre cas $a(t) = \frac{t}{t^2+1}$, $A(t) = \frac{1}{2}\ln(t^2+1)$, $\lambda(t) = \int{t.e^{-A(t)}dt} = \int{\frac{t}{\sqrt{t^2+1}}dt}$.
Par substitution $u = t^2+1$, $\frac{du}{dt} = 2t$ donc $dt = \frac{1}{2t}du$

$$\int{\frac{t}{\sqrt{t^2+1}}dt} = \int{\frac{t}{\sqrt{t^2+1}}\frac{1}{2t}du} = \frac{1}{2}\int{\frac{1}{\sqrt{u}}du} = \frac{1}{2}.2\sqrt{u} = \sqrt{u} = \sqrt{t^2+1}$$

Donc
$$y_1(t) = \lambda(t) e^{A(t)} = \sqrt{t^2+1} . \sqrt{t^2+1} = t^2+1$$

\subsubsection*{1.3}
L'unique solution v\'erifiant $y(0)= 0$ est
$$y(t) = y_0(t) + y_1(t) = \lambda\sqrt{t^2+1} + t^2+1$$
$$y(0) = \lambda\sqrt{0^2+1} + 0^2+1 = 0,\; donc\; \lambda = -1$$

Donc
$$y(t) = -\sqrt{t^2+1} + t^2+1$$

\subsection*{Exercice 2}
R\'esoudre l'\'equation diff\'erentielle :
$$y''(t) - 2y'(t) -3y(t)= 3t^2 - 2t$$

\subsubsection*{2.1}
Si l'\'equation homog\`ene est de la forme:
$$y''(t) +b.y'(t) + a.y(t) = 0$$
Alors la solution est
$$y_0(t) = \lambda e^{r_1 t} + \mu e^{r_2 t},\; avec\; r_1, r_2\; les\; solutions\; de\; r^2 + br + a = 0$$

Dans notre cas $b = -2$ et $a = -3$. Les solution de l'\'equation $r^2 - 2r - 3 = 0$ sont
$$r_1 = \frac{-(-2)+\sqrt{(-2)^2-4.(1)(-3)}}{2.(1)} = 3,\; r_2 = \frac{-(-2)-\sqrt{(-2)^2-4.(1)(-3)}}{2.(1)} = -1$$

Donc
$$y_0(t) = \lambda e^{3t} + \mu e^{-t}$$

\subsubsection*{2.2}
En prenant $y(t) = at^2 + bt + c$, $y'(t) = 2at+b$ et $y''(t) = 2a$ on a
$$2a - 2(2at+b) -3(at^2+bt+c) = 3t^2 - 2t = 2a -4at -2b -3at^2 -3bt -3c = 3t^2 - 2t$$
$$(-3a - 3)t^2 + (-3b -4a +2)t + 2a - 3c = 0$$

Il faut r\'esoudre
$$
\left\{ 
\begin{array}{l}
 2a - 3c = 0\\
 -3b -4a + 2 = 0\\
 -3a - 3 = 0\\
\end{array}
\right. 
$$
Donc $a=-1$, $b=2$ et $c=-\frac{2}{3}$.

Donc 
$$y_1(t) = -t^2 + 2t -\frac{2}{3}$$

\subsubsection*{2.3}
La solution de $F$ est
$$y(t) = y_0(t) + y_1(t) = \lambda e^{3t} + \mu e^{-t} -t^2 + 2t -\frac{2}{3}$$

\subsubsection*{2.3}
Avec $y'(0) = -4$, $y(0) = 0$
$$y(0) = \lambda e^{3.0} + \mu e^{-0} -0^2 + 2.0 -\frac{2}{3} = \lambda + \mu - \frac{2}{3} = 0$$
$$y'(0) = 3\lambda e^{3.0} - \mu e^{-0} -2.0 + 2 = 3\lambda - \mu + 2 = -4$$

Donc $4\mu - 4 = 4$, donc $\mu = 2$ et $\lambda = -\frac{4}{3}$
Donc
$$y(t) = -\frac{4}{3}e^{3t} + 2 e^{-t} -t^2 + 2t -\frac{2}{3}$$



\subsection*{Exercice 3}
La formule de Taylor pour le developpement limit\'e d'ordre $n$ est
$$f(b) = \sum_{k=0}^n{\frac{(b-a)^k}{k!}f^{(k)}(a)} + \int_a^b{\frac{(b-t)^n}{n!}f^{(n+1)}(t)dt}$$

\subsubsection*{3.1}
Dans notre cas, $f(x) = \sin(x)$, $a = 0$, $b = x$ et $n = 2$.
Donc 
$$\sin(x) = \sum_{k=0}^2{\frac{(x-0)^k}{k!}\sin^{(k)}(0)} + \int_0^x{\frac{(x-t)^2}{2!}\sin^{(2+1)}(t)dt}$$
$$\sin(x) = \sin(0) + x\cos(0) - \frac{x^2}{2}\sin(0) + \int_0^x{\frac{(x-t)^2}{2}(-\cos(t))dt} = x - \int_0^x{\frac{(x-t)^2}{2}\cos(t)dt}$$

\subsubsection*{3.2}
Pour $x \in [0,\frac{\pi}{2}]$, on a $\frac{(x-t)^2}{2} >0$ et $cos(x) \geq 0$. Donc, le signe du reste integral est n\'egatif.

\subsubsection*{3.3}
Pour $x \in [0,\frac{\pi}{2}]$, on a $cos(x) \leq 1$.
En utilisant l'in\'egalit\'e de Taylor-Lagrance on a 
$$|f(b) - \sum_{k=0}^2{\frac{(b-a)^k}{k!}f^{(k)}(a)}| \leq M\frac{x^3}{3!}$$
$$|\sin(x) - x| \leq M\frac{x^3}{6}$$

Avec $M$ un majorant de $|-cos(t)|$. Donc $M=1$.
On a donc 
$$|\sin(x) - x| \leq \frac{x^3}{6}$$
On a $sin(x) - x < 0$ pour $x \in [0,\frac{\pi}{2}]$ Donc
$$-\sin(x) + x \leq \frac{x^3}{6}$$
$$x+\frac{x^3}{6} \leq \sin(x)$$

\subsubsection*{3.4}


\subsection*{Exercice 4}
$$y'(t) + t.y(t) = t^3.y(^3t)$$
\subsubsection*{4.1}
On a $y(t) = 0$ et $y'(t) = 0$.
Donc
$$0 + t.0 = t^3.0^3$$

Donc $(y) = 0$ est une solution.

\subsubsection*{4.2}
Si la fonction ne s'annule pas sur $\R$ donc $y^3(t) \neq 0$, donc on peut diviser les 2 cot\'es par $y^3(t)$. Donc si $y(t)$ est solution de $B$ alors $y(t)$ est solution de $\frac{y'(t)}{y^3(t)} + \frac{t.y(t)}{y^3(t)} = t^3.y(t)$. Et si $y(t)$ est solution de $\frac{y'(t)}{y^3(t)} + \frac{t.y(t)}{y^3(t)} = t^3.y(t)$ et $y(t) \neq 0$ alors $y(t)$ est solution de $B$ en multiplant les 2 cot\'es par $y^3(t)$.








\end{document}

