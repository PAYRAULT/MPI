\documentclass[]{book}

%These tell TeX which packages to use.
\usepackage{array,epsfig}
\usepackage{amsmath}
\usepackage{amsfonts}
\usepackage{amssymb}
\usepackage{amsxtra}
\usepackage{amsthm}
\usepackage{mathrsfs}
\usepackage{color}
\usepackage{pgfplots}
\usepackage{fancyhdr}

%Here I define some theorem styles and shortcut commands for symbols I use often
\theoremstyle{definition}
\newtheorem{defn}{Definition}
\newtheorem{thm}{Theorem}
\newtheorem{cor}{Corollary}
\newtheorem*{rmk}{Remark}
\newtheorem{lem}{Lemma}
\newtheorem*{joke}{Joke}
\newtheorem{ex}{Example}
\newtheorem*{soln}{Solution}
\newtheorem{prop}{Proposition}

\newcommand{\lra}{\longrightarrow}
\newcommand{\ra}{\rightarrow}
\newcommand{\surj}{\twoheadrightarrow}
\newcommand{\graph}{\mathrm{graph}}
\newcommand{\bb}[1]{\mathbb{#1}}
\newcommand{\Z}{\bb{Z}}
\newcommand{\Q}{\bb{Q}}
\newcommand{\R}{\bb{R}}
\newcommand{\C}{\bb{C}}
\newcommand{\N}{\bb{N}}
\newcommand{\M}{\mathbf{M}}
\newcommand{\m}{\mathbf{m}}
\newcommand{\MM}{\mathscr{M}}
\newcommand{\HH}{\mathscr{H}}
\newcommand{\Om}{\Omega}
\newcommand{\Ho}{\in\HH(\Om)}
\newcommand{\bd}{\partial}
\newcommand{\del}{\partial}
\newcommand{\bardel}{\overline\partial}
\newcommand{\textdf}[1]{\textbf{\textsf{#1}}\index{#1}}
\newcommand{\img}{\mathrm{img}}
\newcommand{\ip}[2]{\left\langle{#1},{#2}\right\rangle}
\newcommand{\inter}[1]{\mathrm{int}{#1}}
\newcommand{\exter}[1]{\mathrm{ext}{#1}}
\newcommand{\cl}[1]{\mathrm{cl}{#1}}
\newcommand{\ds}{\displaystyle}
\newcommand{\vol}{\mathrm{vol}}
\newcommand{\cnt}{\mathrm{ct}}
\newcommand{\osc}{\mathrm{osc}}
\newcommand{\LL}{\mathbf{L}}
\newcommand{\UU}{\mathbf{U}}
\newcommand{\support}{\mathrm{support}}
\newcommand{\AND}{\;\wedge\;}
\newcommand{\OR}{\;\vee\;}
\newcommand{\Oset}{\varnothing}
\newcommand{\st}{\ni}
\newcommand{\wh}{\widehat}

%Pagination stuff.
\setlength{\topmargin}{-.3 in}
\setlength{\oddsidemargin}{0in}
\setlength{\evensidemargin}{0in}
\setlength{\textheight}{9.in}
\setlength{\textwidth}{6.5in}
\pagestyle{fancy}
\fancyhf{}
\rhead{Math\_103}
\lhead{Exam\_1}
\rfoot{Page \thepage}



\begin{document}

\subsection*{Rappel de cours}

\begin{itemize}
\item 
\end{itemize}

\subsection*{Exercice 1}
R\'esoudre l'\'equation diff\'erentielle :
$$y'(t) - \frac{t}{t^2+1}y(t) = t$$

\subsubsection*{1.1}
Si l'\'equation homog\`ene est de la forme:
$$y'(t) = a(t)y(t)$$
Alors la solution est
$$y_0(t) = \lambda e^{A(t)}\; avec\; A'(t) = a(t)$$

Dans notre cas on a $a(t) = \frac{t}{t^2+1}$, on cherche $A(t) = \int{\frac{t}{t^2+1}dt}$.
Par substitution $u = t^2 +1$, ce qui fait $\frac{du}{dt} = 2t$, donc $dt = \frac{1}{2t}du$.
$$\int{\frac{t}{t^2+1}dt} = \int{\frac{t}{t^2+1}\frac{1}{2t}du} = \frac{1}{2}\int{\frac{1}{u}du} = \frac{1}{2}\ln(u) = \frac{1}{2}\ln(t^2+1)$$

Donc
$$y_0(t) = \lambda e^{\frac{1}{2}\ln(t^2+1)} = \lambda \sqrt{e^{\ln(t^2+1)}} = \lambda \sqrt{t^2+1)}$$

\subsubsection*{1.2}
Calculer une solution particuli\`ere par la methode de la variation de la constante.
Si l'\'equation homog\`ene est de la forme:
$$y'(t) = a(t)y(t) + b(t)$$
Alors la solution est
$$y_1(t) = \lambda(t) e^{A(t)}\; avec\; A'(t) = a(t)\; et\; \lambda'(t) = b(t)e^{-A(t)}$$

Dans notre cas $a(t) = \frac{t}{t^2+1}$, $A(t) = \frac{1}{2}\ln(t^2+1)$, $\lambda(t) = \int{t.e^{-A(t)}dt} = \int{\frac{t}{\sqrt{t^2+1}}dt}$.
Par substitution $u = t^2+1$, $\frac{du}{dt} = 2t$ donc $dt = \frac{1}{2t}du$

$$\int{\frac{t}{\sqrt{t^2+1}}dt} = \int{\frac{t}{\sqrt{t^2+1}}\frac{1}{2t}du} = \frac{1}{2}\int{\frac{1}{\sqrt{u}}du} = \frac{1}{2}.2\sqrt{u} = \sqrt{u} = \sqrt{t^2+1}$$

Donc
$$y_1(t) = \lambda(t) e^{A(t)} = \sqrt{t^2+1} . \sqrt{t^2+1} = t^2+1$$

\subsubsection*{1.3}
L'unique solution v\'erifiant $y(0)= 0$ est
$$y(t) = y_0(t) + y_1(t) = \lambda\sqrt{t^2+1} + t^2+1$$
$$y(0) = \lambda\sqrt{0^2+1} + 0^2+1 = 0,\; donc\; \lambda = -1$$

Donc
$$y(t) = -\sqrt{t^2+1} + t^2+1$$






\end{document}

