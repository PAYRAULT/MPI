\documentclass[]{book}

%These tell TeX which packages to use.
\usepackage{array,epsfig}
\usepackage{amsmath}
\usepackage{amsfonts}
\usepackage{amssymb}
\usepackage{amsxtra}
\usepackage{amsthm}
\usepackage{mathrsfs}
\usepackage{color}
\usepackage{pgfplots}
\usepackage{fancyhdr}

%Here I define some theorem styles and shortcut commands for symbols I use often
\theoremstyle{definition}
\newtheorem{defn}{Definition}
\newtheorem{thm}{Theorem}
\newtheorem{cor}{Corollary}
\newtheorem*{rmk}{Remark}
\newtheorem{lem}{Lemma}
\newtheorem*{joke}{Joke}
\newtheorem{ex}{Example}
\newtheorem*{soln}{Solution}
\newtheorem{prop}{Proposition}

\newcommand{\lra}{\longrightarrow}
\newcommand{\ra}{\rightarrow}
\newcommand{\surj}{\twoheadrightarrow}
\newcommand{\graph}{\mathrm{graph}}
\newcommand{\bb}[1]{\mathbb{#1}}
\newcommand{\Z}{\bb{Z}}
\newcommand{\Q}{\bb{Q}}
\newcommand{\R}{\bb{R}}
\newcommand{\C}{\bb{C}}
\newcommand{\N}{\bb{N}}
\newcommand{\M}{\mathbf{M}}
\newcommand{\m}{\mathbf{m}}
\newcommand{\MM}{\mathscr{M}}
\newcommand{\HH}{\mathscr{H}}
\newcommand{\Om}{\Omega}
\newcommand{\Ho}{\in\HH(\Om)}
\newcommand{\bd}{\partial}
\newcommand{\del}{\partial}
\newcommand{\bardel}{\overline\partial}
\newcommand{\textdf}[1]{\textbf{\textsf{#1}}\index{#1}}
\newcommand{\img}{\mathrm{img}}
\newcommand{\ip}[2]{\left\langle{#1},{#2}\right\rangle}
\newcommand{\inter}[1]{\mathrm{int}{#1}}
\newcommand{\exter}[1]{\mathrm{ext}{#1}}
\newcommand{\cl}[1]{\mathrm{cl}{#1}}
\newcommand{\ds}{\displaystyle}
\newcommand{\vol}{\mathrm{vol}}
\newcommand{\cnt}{\mathrm{ct}}
\newcommand{\osc}{\mathrm{osc}}
\newcommand{\LL}{\mathbf{L}}
\newcommand{\UU}{\mathbf{U}}
\newcommand{\support}{\mathrm{support}}
\newcommand{\AND}{\;\wedge\;}
\newcommand{\OR}{\;\vee\;}
\newcommand{\Oset}{\varnothing}
\newcommand{\st}{\ni}
\newcommand{\wh}{\widehat}

%Pagination stuff.
\setlength{\topmargin}{-.3 in}
\setlength{\oddsidemargin}{0in}
\setlength{\evensidemargin}{0in}
\setlength{\textheight}{9.in}
\setlength{\textwidth}{6.5in}
\pagestyle{fancy}
\fancyhf{}
\rhead{Math\_104}
\lhead{Exam}
\rfoot{Page \thepage}


\begin{document}

\subsection*{Rappel de cours}


\subsection*{Exercice 1}

\subsubsection*{1.1}
Forme Trigonometrique: $re^{j\theta} = r\cos(\theta) + j\sin(\theta)$, avec $r=\sqrt{x^2 + y^2}$ et $\theta = \tan^{-1}(\frac{y}{x})$.

Pour $1 + j\sqrt{3}$, $r = 2$, $\theta = tan^{-1}(\sqrt{3}) = 60$, $2e^{j60} = 2\cos(\frac{\pi}{3}) + j2\sin(\frac{\pi}{3})$.\\
Pour $1 + j$, $r = \sqrt{2}$, $\theta = tan^{-1}(1) = 45$, $\sqrt{2}e^{j45} = \sqrt{2}\cos(\frac{\pi}{4}) + j\sqrt{2}\sin(\frac{\pi}{4})$.


\subsubsection*{1.2}
$$Z=\frac{1+j\sqrt{3}}{1+j} = \frac{1+ j\sqrt{3})(1-j)}{(1+j)(1-j)} = \frac{1+\sqrt{3} + j(\sqrt{3}-1)}{2}$$

$$r_{Z} = \sqrt{\frac{(1+\sqrt{3})^2}{4} + \frac{(\sqrt{3}-1)^2}{4}} = \sqrt{2}$$
$$\theta_{Z} = \tan^{-1}(\frac{\sqrt{3}-1}{1+\sqrt{3}}) = \frac{\pi}{12}$$

\subsubsection*{1.3}
On a, $\frac{1+\sqrt{3} + j(\sqrt{3}-1)}{2} = \sqrt{2}\cos(\theta_{Z}) + j\sqrt{2}\sin(\theta_{Z})$. Donc\\
$\sqrt{2}\cos(\theta_{Z}) = \frac{1+\sqrt{3}}{2}$ et $\sqrt{2}\sin(\theta_{Z}) = \frac{\sqrt{3}-1}{2}$.

\subsubsection*{1.4}
$$Z^{1000} = (r_{Z}e^{j\theta_{Z}})^{1000} = r_{Z}^{1000}e^{j1000\theta_{Z}}$$

$$Z=r_{Z}^{1000}\cos(1000\theta_{Z}) + jr_{Z}^{1000}\sin(1000\theta_{Z}) = \sqrt{2}^{1000}\cos(\frac{4\pi}{3}) + j\sqrt{2}^{1000}\sin(\frac{4\pi}{3}) = 2^{500}\cos(\frac{4\pi}{3}) + j2^{500}\sin(\frac{4\pi}{3})$$


\end{document}

