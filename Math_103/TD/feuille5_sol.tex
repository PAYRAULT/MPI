\documentclass[]{book}

%These tell TeX which packages to use.
\usepackage{array,epsfig}
\usepackage{amsmath}
\usepackage{amsfonts}
\usepackage{amssymb}
\usepackage{amsxtra}
\usepackage{amsthm}
\usepackage{mathrsfs}
\usepackage{color}
\usepackage{fancyhdr}

%Here I define some theorem styles and shortcut commands for symbols I use often
\theoremstyle{definition}
\newtheorem{defn}{Definition}
\newtheorem{thm}{Theorem}
\newtheorem{cor}{Corollary}
\newtheorem*{rmk}{Remark}
\newtheorem{lem}{Lemma}
\newtheorem*{joke}{Joke}
\newtheorem{ex}{Example}
\newtheorem*{soln}{Solution}
\newtheorem{prop}{Proposition}

\newcommand{\lra}{\longrightarrow}
\newcommand{\ra}{\rightarrow}
\newcommand{\surj}{\twoheadrightarrow}
\newcommand{\graph}{\mathrm{graph}}
\newcommand{\bb}[1]{\mathbb{#1}}
\newcommand{\Z}{\bb{Z}}
\newcommand{\Q}{\bb{Q}}
\newcommand{\R}{\bb{R}}
\newcommand{\C}{\bb{C}}
\newcommand{\N}{\bb{N}}
\newcommand{\M}{\mathbf{M}}
\newcommand{\m}{\mathbf{m}}
\newcommand{\MM}{\mathscr{M}}
\newcommand{\HH}{\mathscr{H}}
\newcommand{\Om}{\Omega}
\newcommand{\Ho}{\in\HH(\Om)}
\newcommand{\bd}{\partial}
\newcommand{\del}{\partial}
\newcommand{\bardel}{\overline\partial}
\newcommand{\textdf}[1]{\textbf{\textsf{#1}}\index{#1}}
\newcommand{\img}{\mathrm{img}}
\newcommand{\ip}[2]{\left\langle{#1},{#2}\right\rangle}
\newcommand{\inter}[1]{\mathrm{int}{#1}}
\newcommand{\exter}[1]{\mathrm{ext}{#1}}
\newcommand{\cl}[1]{\mathrm{cl}{#1}}
\newcommand{\ds}{\displaystyle}
\newcommand{\vol}{\mathrm{vol}}
\newcommand{\cnt}{\mathrm{ct}}
\newcommand{\osc}{\mathrm{osc}}
\newcommand{\LL}{\mathbf{L}}
\newcommand{\UU}{\mathbf{U}}
\newcommand{\support}{\mathrm{support}}
\newcommand{\AND}{\;\wedge\;}
\newcommand{\OR}{\;\vee\;}
\newcommand{\Oset}{\varnothing}
\newcommand{\st}{\ni}
\newcommand{\wh}{\widehat}

%Pagination stuff.
\setlength{\topmargin}{-.3 in}
\setlength{\oddsidemargin}{0in}
\setlength{\evensidemargin}{0in}
\setlength{\textheight}{9.in}
\setlength{\textwidth}{6.5in}
\pagestyle{fancy}
\fancyhf{}
\rhead{Math\_103}
\lhead{Feuille\_4}
\rfoot{Page \thepage}



\begin{document}

Rappel de cours: 
\begin{itemize}
\item 
\end{itemize}

 

\subsection*{Exercice 5.1}
\subsubsection*{5.1.1.a}
La relation $\phi$ est lin\'eaire de $E$ si 
\begin{itemize}
\item $\forall A, B \in E, \phi(A+B) = \phi(A) + \phi(B)$  
\item $\forall A \in E, \forall \lambda \in \R, \phi(\lambda A) = \lambda \phi(A)$
\end{itemize}

Posons $M_1=\begin{vmatrix} 1 & 2 \\ 2 & 3 \end{vmatrix}$ et $M_2 = \begin{vmatrix} -3 & 2 \\ 2 & -1 \end{vmatrix}$.\\

En utilisant deux fois la distibitibit\'e par rapport \`a l'addition, on a:
$$
\phi(A+B) = M_1.(A+B).M_2 = M_1.(A.M_2 + B.M_2) = M_1.A.M_2 + M_1.B.M_2 = \phi(A) + \phi(B)
$$

On a $(\lambda A).B = \lambda (A.B) = A. (\lambda B)$ (voir cours).
$$
\phi(\lambda A) = M_1 . (\lambda A) . M_2 = M_1 . (\lambda (A . M_2)) = \lambda (M_1.A.M_2) = \lambda \phi(A)
$$

Donc la relation $\phi(M)$ est lin\'eaire.

\subsubsection*{5.1.1.b}
La matrice $A$ est un point fixe de la relation $\phi$ si $\phi(A) = A$, soit 
$$
\phi(A) =
\begin{vmatrix} 1 & 2 \\ 2 & 3 \end{vmatrix} . A . 
\begin{vmatrix} -3 & 2 \\ 2 & -1 \end{vmatrix} = 
A
$$

$$
\phi(A) =
\begin{vmatrix} 1 & 2 \\ 2 & 3 \end{vmatrix} .
\begin{vmatrix} 1 & 2 \\ 2 & 3 \end{vmatrix} . 
\begin{vmatrix} -3 & 2 \\ 2 & -1 \end{vmatrix} = 
\begin{vmatrix} 5 & 8 \\ 8 & 13 \end{vmatrix} .
\begin{vmatrix} -3 & 2 \\ 2 & -1 \end{vmatrix} =
\begin{vmatrix} 1 & 2 \\ 2 & 3 \end{vmatrix} =
A
$$

\subsubsection*{5.1.2.a}
$$
\phi(P_1+P_2) = ((P_1+P_2)(1),(P_1+P_2)'(2)) = ((P_1(1)+P_2(1)),(P_1'+P_2')(2)) = (P_1(1)+P_2(1),P_1'(2)+P_2'(2))
$$
$$
= (P_1(1), P_1'(2)) + (P_2(1),P_2'(2)) = \phi(P_1) + \phi(P_2)
$$

Et

$$
\phi(\lambda P) = (\lambda P(1), (\lambda P(2))') = (\lambda P(1), \lambda P'(2)) = \lambda (P(1), P'(2)) = \lambda \phi(P)
$$

La relation $\phi$ est lin\'eaire.

\subsubsection*{5.1.2.b}
$$
\phi(t - 1) = ((t - 1)(1), (t - 1)'(2)) = ((t - 1)(1), (1)(2)) = (0,1)
$$

$$
\phi((t - 2)^2) = ((t - 2)^2(1), ((t - 2)^2)'(2)) = ((t - 2)^2(1), (t^2 -4t +4)'(2)) = ((t - 2)^2(1), (2t -4)(2)) = (1,0)
$$

On cherche $P$, tel que $\phi(P) = (1,-1)$. Comme la relation $\phi$ est lin\'eaire on a
$$
(1,-1) = 1.(1,0) -1.(0,1) = 1.\phi((t - 2)^2) - 1.\phi(t-1) = \phi(1.(t-2)^2) + \phi(-(t-1)) = \phi((t-2)^2-(t-1)) = \phi(t^2-5t+5)
$$
Donc $P=t^2-5t+5$.

\subsubsection*{5.1.3.a}
$$
\phi(f_1+f_2) = \int_0^1{(f_1+f_2)e^t dt} = \int_0^1{f_1 e^t + f_2 e^t dt} = \int_0^1{f_1 e^t} + \int_0^1{f_2 e^t dt} = \phi(f_1)+\phi(f_2)
$$

$$
\phi(\lambda f) = \int_0^1{(\lambda f)e^t dt} = \int_0^1{ \lambda f e^t dt} =  \lambda \int_0^1{f e^t dt} = \lambda \phi(f)
$$

\subsubsection*{5.1.3.b}
Une fonction $f$ de $E \to F$ appartient au noyau de $\phi$ si $\phi(f) = 0_F$. Une fonction est dite affine si elle est de la forme $f(t) = at + b$. Donc on cherche une fonction $f=at+b$ tel que $\phi(f)=0$.

$$
\phi(at+b) = \int_0^1{(at+b)e^t dt} = \int_0^1{at e^t + b e^t dt} = a\int_0^1{t e^t dt} + b\int_0^1{e^t dt} = a + b(e-1) = 0
$$

car $\int{e^t dt} = e^t$ et $\int{t e^t dt} = (t-1)e^t$.\\

Il faut trouver $a$ et $b$ tel que $a+b(e-1) = 0$. Prenons par exemple, $b=0$ donc $a=0$, et $b=1$ donc $a = 1-e$.\\
Les 2 fonctions affines sont $f_0(t) = 0t+0 = 0$ et $f_1(t) = (1-e)t + 1$.\\





QED


\end{document}

