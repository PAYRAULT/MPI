\documentclass[]{book}

%These tell TeX which packages to use.
\usepackage{array,epsfig}
\usepackage{amsmath}
\usepackage{amsfonts}
\usepackage{amssymb}
\usepackage{amsxtra}
\usepackage{amsthm}
\usepackage{mathrsfs}
\usepackage{color}
\usepackage{fancyhdr}

%Here I define some theorem styles and shortcut commands for symbols I use often
\theoremstyle{definition}
\newtheorem{defn}{Definition}
\newtheorem{thm}{Theorem}
\newtheorem{cor}{Corollary}
\newtheorem*{rmk}{Remark}
\newtheorem{lem}{Lemma}
\newtheorem*{joke}{Joke}
\newtheorem{ex}{Example}
\newtheorem*{soln}{Solution}
\newtheorem{prop}{Proposition}

\newcommand{\lra}{\longrightarrow}
\newcommand{\ra}{\rightarrow}
\newcommand{\surj}{\twoheadrightarrow}
\newcommand{\graph}{\mathrm{graph}}
\newcommand{\bb}[1]{\mathbb{#1}}
\newcommand{\Z}{\bb{Z}}
\newcommand{\Q}{\bb{Q}}
\newcommand{\R}{\bb{R}}
\newcommand{\C}{\bb{C}}
\newcommand{\N}{\bb{N}}
\newcommand{\M}{\mathbf{M}}
\newcommand{\m}{\mathbf{m}}
\newcommand{\MM}{\mathscr{M}}
\newcommand{\HH}{\mathscr{H}}
\newcommand{\Om}{\Omega}
\newcommand{\Ho}{\in\HH(\Om)}
\newcommand{\bd}{\partial}
\newcommand{\del}{\partial}
\newcommand{\bardel}{\overline\partial}
\newcommand{\textdf}[1]{\textbf{\textsf{#1}}\index{#1}}
\newcommand{\img}{\mathrm{img}}
\newcommand{\ip}[2]{\left\langle{#1},{#2}\right\rangle}
\newcommand{\inter}[1]{\mathrm{int}{#1}}
\newcommand{\exter}[1]{\mathrm{ext}{#1}}
\newcommand{\cl}[1]{\mathrm{cl}{#1}}
\newcommand{\ds}{\displaystyle}
\newcommand{\vol}{\mathrm{vol}}
\newcommand{\cnt}{\mathrm{ct}}
\newcommand{\osc}{\mathrm{osc}}
\newcommand{\LL}{\mathbf{L}}
\newcommand{\UU}{\mathbf{U}}
\newcommand{\support}{\mathrm{support}}
\newcommand{\AND}{\;\wedge\;}
\newcommand{\OR}{\;\vee\;}
\newcommand{\Oset}{\varnothing}
\newcommand{\st}{\ni}
\newcommand{\wh}{\widehat}

%Pagination stuff.
\setlength{\topmargin}{-.3 in}
\setlength{\oddsidemargin}{0in}
\setlength{\evensidemargin}{0in}
\setlength{\textheight}{9.in}
\setlength{\textwidth}{6.5in}
\pagestyle{fancy}
\fancyhf{}
\rhead{Math\_103}
\lhead{Feuille\_5}
\rfoot{Page \thepage}



\begin{document}

Rappel de cours: 
\begin{itemize}
\item 
\end{itemize}

 

\subsection*{Exercice 5.1}
\subsubsection*{5.1.1.a}
La relation $\phi$ est lin\'eaire de $E$ si 
\begin{itemize}
\item $\forall A, B \in E, \phi(A+B) = \phi(A) + \phi(B)$  
\item $\forall A \in E, \forall \lambda \in \R, \phi(\lambda A) = \lambda \phi(A)$
\end{itemize}

Posons $M_1=\begin{vmatrix} 1 & 2 \\ 2 & 3 \end{vmatrix}$ et $M_2 = \begin{vmatrix} -3 & 2 \\ 2 & -1 \end{vmatrix}$.\\

En utilisant deux fois la distibitibit\'e par rapport \`a l'addition, on a:
$$
\phi(A+B) = M_1.(A+B).M_2 = M_1.(A.M_2 + B.M_2) = M_1.A.M_2 + M_1.B.M_2 = \phi(A) + \phi(B)
$$

On a $(\lambda A).B = \lambda (A.B) = A. (\lambda B)$ (voir cours).
$$
\phi(\lambda A) = M_1 . (\lambda A) . M_2 = M_1 . (\lambda (A . M_2)) = \lambda (M_1.A.M_2) = \lambda \phi(A)
$$

Donc la relation $\phi(M)$ est lin\'eaire.

\subsubsection*{5.1.1.b}
La matrice $A$ est un point fixe de la relation $\phi$ si $\phi(A) = A$, soit 
$$
\phi(A) =
\begin{vmatrix} 1 & 2 \\ 2 & 3 \end{vmatrix} . A . 
\begin{vmatrix} -3 & 2 \\ 2 & -1 \end{vmatrix} = 
A
$$

$$
\phi(A) =
\begin{vmatrix} 1 & 2 \\ 2 & 3 \end{vmatrix} .
\begin{vmatrix} 1 & 2 \\ 2 & 3 \end{vmatrix} . 
\begin{vmatrix} -3 & 2 \\ 2 & -1 \end{vmatrix} = 
\begin{vmatrix} 5 & 8 \\ 8 & 13 \end{vmatrix} .
\begin{vmatrix} -3 & 2 \\ 2 & -1 \end{vmatrix} =
\begin{vmatrix} 1 & 2 \\ 2 & 3 \end{vmatrix} =
A
$$

\subsubsection*{5.1.2.a}
$$
\phi(P_1+P_2) = ((P_1+P_2)(1),(P_1+P_2)'(2)) = ((P_1(1)+P_2(1)),(P_1'+P_2')(2)) = (P_1(1)+P_2(1),P_1'(2)+P_2'(2))
$$
$$
= (P_1(1), P_1'(2)) + (P_2(1),P_2'(2)) = \phi(P_1) + \phi(P_2)
$$

Et

$$
\phi(\lambda P) = (\lambda P(1), (\lambda P(2))') = (\lambda P(1), \lambda P'(2)) = \lambda (P(1), P'(2)) = \lambda \phi(P)
$$

La relation $\phi$ est lin\'eaire.

\subsubsection*{5.1.2.b}
$$
\phi(t - 1) = ((t - 1)(1), (t - 1)'(2)) = ((t - 1)(1), (1)(2)) = (0,1)
$$

$$
\phi((t - 2)^2) = ((t - 2)^2(1), ((t - 2)^2)'(2)) = ((t - 2)^2(1), (t^2 -4t +4)'(2)) = ((t - 2)^2(1), (2t -4)(2)) = (1,0)
$$

On cherche $P$, tel que $\phi(P) = (1,-1)$. Comme la relation $\phi$ est lin\'eaire on a
$$
(1,-1) = 1.(1,0) -1.(0,1) = 1.\phi((t - 2)^2) - 1.\phi(t-1) = \phi(1.(t-2)^2) + \phi(-(t-1)) = \phi((t-2)^2-(t-1)) = \phi(t^2-5t+5)
$$
Donc $P=t^2-5t+5$.

\subsubsection*{5.1.3.a}
$$
\phi(f_1+f_2) = \int_0^1{(f_1+f_2)e^t dt} = \int_0^1{f_1 e^t + f_2 e^t dt} = \int_0^1{f_1 e^t} + \int_0^1{f_2 e^t dt} = \phi(f_1)+\phi(f_2)
$$

$$
\phi(\lambda f) = \int_0^1{(\lambda f)e^t dt} = \int_0^1{ \lambda f e^t dt} =  \lambda \int_0^1{f e^t dt} = \lambda \phi(f)
$$

\subsubsection*{5.1.3.b}
Une fonction $f$ de $E \to F$ appartient au noyau de $\phi$ si $\phi(f) = 0_F$. Une fonction est dite affine si elle est de la forme $f(t) = at + b$. Donc on cherche une fonction $f=at+b$ tel que $\phi(f)=0$.

$$
\phi(at+b) = \int_0^1{(at+b)e^t dt} = \int_0^1{at e^t + b e^t dt} = a\int_0^1{t e^t dt} + b\int_0^1{e^t dt} = a + b(e-1) = 0
$$

car $\int{e^t dt} = e^t$ et $\int{t e^t dt} = (t-1)e^t$.\\

Il faut trouver $a$ et $b$ tel que $a+b(e-1) = 0$. Prenons par exemple, $b=0$ donc $a=0$, et $b=1$ donc $a = 1-e$.\\
Les 2 fonctions affines sont $f_0(t) = 0t+0 = 0$ et $f_1(t) = (1-e)t + 1$.\\


\subsection*{Exercice 5.2}
\subsubsection*{5.2.1}
$$
f(x,y,z,t)=
\begin{vmatrix} 1 & -1 & 3 & -1 \\ 2 & 1 & 3 & 4 \\ -1 & 2 & -4 & 3\end{vmatrix} .
\begin{vmatrix} x \\ y \\ z \\ t \end{vmatrix} = 
\begin{vmatrix} x -y +3z -t \\ 2x + y + 3z + 4t \\ -x + 2y -4z + 3t \end{vmatrix} 
$$

\subsubsection*{5.2.2.a}
$Ker(f) = \{ X \in \R^4, f(X) = 0_{\R^3}\}$.\\

$$
\left\{ 
\begin{array}{l l l}
x -y +3z -t & = 0 & l_1 \\
2x + y + 3z + 4t & = 0 & l_2 \\
-x + 2y -4z + 3t & = 0 & l_3 \\
\end{array}
\right. 
$$

$$
\left\{ 
\begin{array}{l l l}
x -y +3z -t & = 0 & l_1 \\
-3y + 3z - 6t & = 0 & 2l_1 - l_2 \\
y -z + 2t & = 0 & l_1 + l_3 \\
\end{array}
\right. 
$$

$$
\left\{ 
\begin{array}{l l l}
x -y +3z -t & = 0 & l_1 \\
y -z + 2t & = 0 & l_1 + l_3 \\
0 & = 0 & (2l_1 - l_2)+3(l_1 + l_3) \\
\end{array}
\right. 
$$

On a 2 variables primaires (x,y) et deux variables secondaires (z,t). Donc $Ker(f)=\{X \in \R^3, X=(2z-t,z-2t, z, t)\}$.


\subsubsection*{5.2.2.b}
$$
\left\{ 
\begin{array}{l l l}
x -y +3z -t & = x_b & l_1 \\
2x + y + 3z + 4t & = y_b & l_2 \\
-x + 2y -4z + 3t & = z_b & l_3 \\
\end{array}
\right. 
$$

$$
\left\{ 
\begin{array}{l l}
x -y +3z -t & = x_b \\
y -z + 2t & = x_b+z_b \\
-3y + 3z - 6t & =  2x_b - y_b \\
\end{array}
\right. 
$$

$$
\left\{ 
\begin{array}{l l}
x -y +3z -t & = x_b \\
y -z + 2t & = x_b+z_b \\
-3(x_b+z_b) & =  2x_b - y_b \\
\end{array}
\right. 
$$

$$
\left\{ 
\begin{array}{l l}
x -y +3z -t & = x_b \\
y -z + 2t & = x_b+z_b \\
-5x_b +y_b -3 z_b & =  0 \\
\end{array}
\right. 
$$


L'application $f$ est injective ssi $Ker(f) = 0_{\R^3}\}$. On a $Ker(f)= \{X \in \R^3, X=(4z-t,z-2t, z, t)\} \neq 0_{\R^3}$. L'application n'est pas injective.\\

L'application est surjective ssi $Im(f) = \R^3$. On a $Im(f) = B= (x_b,y_b,z_b) \neq \R^3$ car il y a une relation entre $x_b,y_b et z_b$. Donc l'application $f$ n'est pas surjective.

\subsubsection*{5.2.3.a}	
$$
\begin{array}{l}
u_1 = 3e_1-e_3 = 3(1,0,0,0)-(0,0,1,0) = (3,0,-1,0) \\
u_2 = e_2 -e_4 = (0,1,0,0) - (0,0,0,1) = (0,1,0,-1) \\
f(u_1) = (3 - 0 -3 - 0, 6 - 0 -3 -0, -3 + 0 +4 +0)= (0, 3 ,1) \\
f(u_2) = (0-1+0--1, 0 +1 + 0 -4, -0 + 2 -0 -3) = (0,-3,-1) \\
\end{array}
$$

On a $Vect(f(u_1), f(u_2)) = \{(0,3,1),(0,-3,-1)\}$. On remarque que $f(u_1)=-f(u_2)$, donc les 2 vecteurs sont colin\'eaires. Par cons\'equent $Vect(f(u_1), f(u_2))$ ne repr\'esente que les points de la droite de du plan $(y,z)$ et de vecteur directeur $(0,3,1)$.

\subsubsection*{5.2.3.b}
On a $Ker(f)=\{X \in \R^3, X=(-2z-t,z-2t, z, t)\}$,  les points de l'espace vectorielle sont $(x_f,y_f,z_f,t_f) = (-2z-t,z-2t, z, t)$.\\
$E=Vect((3,0,-1,0), (0,1,0,-1))$, les points de l'espace vectorielle $D$ sont $(x_e,y_e,z_e,t_e) = a.(3,0,-1,0) + b.(0,1,0,-1) = (3a,b,-a,-b)$.\\

$E \cap Ker(f)$ sont les points communs entre $E$ et $Ker(f)$. Donc

$$
\left\{ 
\begin{array}{l l}
3a & = -2z - t \\
b & = z-2t \\
-a & = z \\
-b & = t \\
\end{array}
\right. 
$$

La solution de ce syst\`eme est $b=a$. Donc $D=(3a,a,-a,-a)$ ou $D=(3,1,-1,-1)$.\\

$D'=(3,0,-1,0)$ et $D=(3,1,-1,-1)$ sont suppl\'ementaires dans E si $((3,0,-1,0),(3,1,-1,-1)) $ est une base de $E$. \\

Famille libre?
$$
\left\{ 
\begin{array}{l l}
3\lambda_1 + 3\lambda_2& = 0 \\
\lambda_2 & = 0 \\
-\lambda_1 -\lambda_2 & = 0 \\
-\lambda_2 & = 0 \\
\end{array}
\right. 
$$

$\lambda_1=\lambda_2=0$, la famille est libre. 

Et g\'en\'eratrice?
$$
\left\{ 
\begin{array}{l l}
3\lambda_1 + 3\lambda_2& = x \\
\lambda_2 & = y \\
-\lambda_1 -\lambda_2 & = z \\
-\lambda_2 & = t \\
\end{array}
\right. 
$$

$$
\left\{ 
\begin{array}{l l}
0 & = t + y\\
0 & = x +3z\\
\lambda_2 & = y \\
-\lambda_1 & = z + y \\
\end{array}
\right. 
$$

Oups la famille n'est pas g\'en\'eratrice. Probl\`eme.\\


\subsection*{5.3}	
\subsubsection*{5.3.1}	

$$
A = \begin{vmatrix} 3 & 4 & 2 & 1 \\ 1 & 2 & -1 & -1 \\  -1 & -1 & 3 & 2\\ 2 & -1 & 0 & 1 \end{vmatrix} 
$$

\subsubsection*{5.3.2.a}	

$$
f(1,0,1,-1) = (4, 1, 0, 1) 
$$

$$
f(0,-1,1,-2) = (-4, -1, 0, -1) 
$$

$$
\left\{ 
\begin{array}{l l}
4\lambda_1 - 4\lambda_2 & = 0 \\
\lambda_1 - \lambda_2 & = 0\\
0\lambda_1 - 0\lambda_2 & = 0 \\
\lambda_1 - \lambda_2 & = 0 \\
\end{array}
\right. 
$$

On a $\lambda_1 = \lambda_2$, la famille n'est pas libre.\\

Ou plus subtilement, on remarque que $f(1,0,1,-1) = -f(0,-1,1,-2)$ donc la famille est li\'ee.

\subsubsection*{5.3.2.b}	
L'application $f$ est injective si $Ker(f) = \{0\}$ avec $Ker(f) = \{X \in \R^4, f(X) = 0\}$.\\
De la question pr\'ec\'edente on a:
$$
f(1,0,1,-1) + f(0,-1,1,-2) = 0
$$
L'application $f$ est lin\'eaire, donc\\
$$
f(1,-1,2,-3) = 0
$$
Donc
$$
(1,-1,2,-3) \in Ker(f)
$$
Donc
$$
Ker(f) \neq \{0\}
$$
Donc l'application $f$ n'est pas injective.

\subsubsection*{5.3.2.c}	
Une application lin\'eaire entre deux espaces vectoriels de m\^eme dimension est injective si, et seulement si, elle est surjective. L'application $f$ est lin\'eaire, de m\^eme dimension et elle n'est pas injective donc elle n'est pas surjective.

\subsubsection*{5.3.3.a}	
On note les vecteurs colonnes de A: $C_1 = (3,1,-1,2)$, $C_2 = (4,2,-1,-1)$, $C_3 = (2,-1,3,0)$ et $C_4 = (1,-1,2,1)$. \\

Le deuxi\`eme vecteur-colonne est unee combinaison lin\'eaire des troisi\`eme et quatri\`eme vecteurs-colonnes si $\exists a,b \in \R, C_2 = aC_3+bC_4$.

$$
(4,2,-1,-1) = a(2,-1,3,0) + b(1,-1,2,1) = (2a+b,-a-b,3a+2b,b)
$$
Donc
$$
\left\{ 
\begin{array}{l l}
2a+b & = 4 \\
-a-b & = 2\\
3a+2b & = -1 \\
b & = -1 \\
\end{array}
\right. 
$$

$$
\left\{ 
\begin{array}{l l}
2a+b & \neq 4 \\
a & = -1\\
3a+2b & = -1 \\
b & = -1 \\
\end{array}
\right. 
$$

Contradiction. Donc non.

\subsubsection*{5.3.3.b}	
Voir cours

\subsubsection*{5.3.3.c}	


\subsection*{5.4}	
\subsubsection*{5.4.1}	
En utilisant la distributivit\'e par rapport \`a l'addition.
$$g(A+B) = (A+B)M - M(A+B) = AM+BM -MA -MB = AM -MA +BM -MB = g(A) + g(B)$$

$$g(\lambda A) = (\lambda A)M - M(\lambda A) = \lambda (AM) - \lambda(MA) = \lambda (AM-MA) = \lambda g(A)$$

Donc $g$ est lin\'eaire.

\subsubsection*{5.4.2}	
Par d\'efinition, $Ker(g) = \{ M \in \mathcal{M}_2(\R), g(M) = 0_{\mathcal{M}_2(\R)}\}$.\\
On a $g(M) = AM - MA = 0_{\mathcal{M}_2(\R)}$ pour tout $M \in \mathcal{M}_2(\R), AM=MA$ qui est la d\'efinition de $E$.\\

Donc $Ker(g) = E$. 

\subsubsection*{5.4.3}	
On a $M \in E, AM = MA$.\\
La matrice $A \in E$ car $AA =AA$.\\
La matrice $I_2 \in E$ car $I_2A = A = AI_2$.\\

On a trouv\'e 2 matrices qui appartiennent \`a $E$. Comme $Ker(g) = E$, on a $g(A) = 0_{\mathcal{M}_2(\R)}$ et $g(I_2) = 0_{\mathcal{M}_2(\R)}$. De plus, la relation $g$ est lin\'eaire donc $g(\lambda_1 A + \lambda_2 I_2) = \lambda_1g(A) + \lambda_2 g(I_2) = \lambda_1 0 + \lambda_2 0 = 0_{\mathcal{M}_2(\R)}$. Donc $\lambda_1 A + \lambda_2 I_2$ est dans $Ker(g)$. Par cons\'equent $Ker(g) = \{A,I_2,\lambda_1 A + \lambda_2 I_2\}$, et $dim\; Ker(g) \geq 2$.

\subsubsection*{5.4.4}	
$A$ et $I_2$ forment une base de $Ker(g)$ car la famille $(A,I_2)$ est libre et g\'en\'eratrice. 
$dim\, Ker(g)$ et $rang(g)$ voir cours.


\subsubsection*{5.4.5}	
On a $g(A^2) = A^2A - AA^2 = A^3 - A^3 = 0_{\mathcal{M}_2(\R)}$. Donc la matrice $A^2 \in Ker(g)$.\\

Comme les matrices $A$ et $I_2$ sont une base de $Ker(g)$ alors tous les \'el\'ement de l'ensemble $Ker(g)$ peuvent s'exprimer comme une combinaison lin\'eaire des matrices  $A$ et $I_2$.
Ou plus bestialement,
$$A^2 = 
\begin{vmatrix} 2 & 3 \\ 1 & 2 \end{vmatrix} . \begin{vmatrix} 2 & 3 \\ 1 & 2 \end{vmatrix} =
\begin{vmatrix} 7 & 12 \\ 4 & 7 \end{vmatrix}  = 
4\begin{vmatrix} 2 & 3 \\ 1 & 2 \end{vmatrix} - \begin{vmatrix} 1 & 0 \\ 0 & 1 \end{vmatrix} =
4A -I_2
$$

Une matrice $M_g$ est dans l'image de la relation $g$ ssi il existe une matrice $M \in \mathcal{M}_2(\R)$ tel que $g(M) = AM - MA = M_g$. Soit $M = \begin{vmatrix} m_{11} & m_{12} \\ m_{21} & m_{22} \end{vmatrix}$\\

$$
AM = \begin{vmatrix} 2 & 3 \\ 1 & 2 \end{vmatrix} . \begin{vmatrix} m_{11} & m_{12} \\ m_{21} & m_{22} \end{vmatrix} =
\begin{vmatrix} 2m_{11} +  3m_{21} & 2m_{12} + 3m_{22} \\ m_{11} + 2 m_{21} & m_{12} + 2m_{22} \end{vmatrix}
$$

$$
MA = \begin{vmatrix} m_{11} & m_{12} \\ m_{21} & m_{22} \end{vmatrix} . \begin{vmatrix} 2 & 3 \\ 1 & 2 \end{vmatrix}  =
\begin{vmatrix} 2m_{11} +  m_{12} & 3m_{11} + 2m_{12} \\ 2m_{21} + m_{22} & 3m_{21} + 2m_{22} \end{vmatrix}
$$

$$
M_g = AM - MA = 
\begin{vmatrix} 3m_{21} -  m_{12} & 3m_{22} - 3m_{11} \\ m_{11} - m_{22} & m_{12} - 3m_{21} \end{vmatrix} =
\begin{vmatrix} a & -3b \\ b & -a \end{vmatrix}
$$
avec $a=3m_{21} -  m_{12}$ et $b = m_{11} - m_{22}$.

QED


\subsection*{5.5}	
\subsubsection*{5.5.a}
Par d\'efinition on a $Ker(f) = \{ X \in \mathcal{M}_2(\R), f(X) = 0_{\mathcal{M}_2(\R)}\}$. Donc, calculons $f(X) = 0$ avec $X=\begin{vmatrix} x_{11} & x_{12} \\ x_{21} & x_{22} \end{vmatrix}$.\\

$$
\begin{vmatrix} 1 & 2 \\ 2 & 4 \end{vmatrix} . \begin{vmatrix} x_{11} & x_{12} \\ x_{21} & x_{22} \end{vmatrix} = \begin{vmatrix} 0 & 0 \\ 0 & 0 \end{vmatrix}
$$

$$
\begin{vmatrix} x_{11}+2x_{21} & x_{12}+2x_{22} \\ 2x_{11}+4x_{21} & 2x_{12}+4x_{22} \end{vmatrix} = \begin{vmatrix} 0 & 0 \\ 0 & 0 \end{vmatrix}
$$

$$
\left\{ 
\begin{array}{l l}
x_{11}+2x_{21} & = 0 \\
x_{12}+2x_{22} & = 0\\
2x_{11}+4x_{21}  & = 0 \\
2x_{12}+4x_{22} & = 0 \\
\end{array}
\right. 
$$

$$
\left\{ 
\begin{array}{l l}
x_{11} & = -2x_{21} \\
x_{12} & = -2x_{22}\\
\end{array}
\right. 
$$

Donc 
$$
Ker(f) = \{X \in \mathcal{M}_2(\R),\forall a,b \in  \R, X=\begin{vmatrix} -2a & -2b \\ a & b \end{vmatrix} \}
$$

Une base pour $ker(f)$ est $(-2a,-2b,a,b)$.

\subsubsection*{5.5.b}
Par d\'efinition, on a $Im(f) = \{X \in \mathcal{M}_2(\R), \forall M \in \mathcal{M}_2(\R), f(M) = X \}$. Posons, $M=\begin{vmatrix} m_{11} & m_{12} \\ m_{21} & m_{22} \end{vmatrix}$ et calculons $f(M)$.

$$
\begin{vmatrix} 1 & 2 \\ 2 & 4 \end{vmatrix} . \begin{vmatrix} m_{11} & m_{12} \\ m_{21} & m_{22} \end{vmatrix} =
\begin{vmatrix} m_{11}+2m_{21} & m_{12}+2m_{22} \\ 2m_{11}+4m_{21} & 2m_{12}+4m_{22} \end{vmatrix} 
$$

Donc 
$$
X = \begin{vmatrix} a & b \\ 2a & 2b \end{vmatrix} \text{ avec } a =  m_{11}+2m_{21} \text{ et } b = m_{12}+2m_{22}
$$

Une base pour $Im(f)$ est $(a,b,2a,2b)$ avec $a =  m_{11}+2m_{21}$ et $b = m_{12}+2m_{22}$.


\subsection*{5.6}
\subsubsection*{5.6.1}
On a l'application $f: \mathcal{M}_3(\R) \to \mathcal{M}_3(\R), f(Q) = AQ$. L'application $f$ est lin\'eaire si $f(Q_1+\lambda Q_2) = f(Q_1) + \lambda f(Q_2)$.\\

$$
f(Q_1+\lambda Q_2) = A.(Q_1+\lambda Q_2) = A.Q_1 + A.\lambda Q_2 = A.Q_1 + \lambda A.Q_2 = f(Q_1) + \lambda f(Q_2)
$$
L'application $f$ est lin\'eaire.

\subsubsection*{5.6.2}
On a $E = \{M \in \mathcal{M}_3(\R), \exists Q \in \mathcal{M}_3(\R), M = AQ\}$.\\
Le triplet $(E,+,.)$ avec la loi $+$ qui est l'addition entre matrices et la loi $.$ la multiplication d'une matrice par une constante, est un espace vectoriel si il v\'erifie les 8 propri\'et\'es suivantes:

\begin{itemize}
\item 1 - $\forall M_1, M_2 \in E, M_1 + M_2 = M_2 + M_1$
\item 2 - $\forall M_1, M_2,M_3 \in E, M_1 + (M_2 + M_3)= (M_1 + M_2) + M_3$
\item 3 - $\exists e \in E, \forall M \in E, M+e=e+M=M$
\item 4 - $\forall M_1 \in E,\exists M_2 \in E, M_1 + M_2 = M_2 + M_1 = e$
\item N - $\forall M \in E, 1.M = M$
\item A - $\forall \lambda, \mu \in \R, \forall M \in E, \lambda.(\mu M) = (\lambda \mu).M$
\item $D_1$ - $\forall \lambda, \mu \in \R, \forall M \in E, (\lambda + \mu).M = \lambda.M + \mu.M$
\item $D_2$ - $\forall \lambda \in \R, \forall M_1, M_2 \in E, \lambda.(M_1+M_2) = \lambda.M_1 + \lambda.M_2$
\end{itemize}

V\'erifions les 8 propri\'et\'es:

 \begin{itemize}
\item 1 - $\forall M_1, M_2 \in E, M_1 + M_2 = AQ_1 + AQ_2 = AQ_2 + AQ_1 = M_2 + M_1$, Vrai
\item 2 - $\forall M_1, M_2,M_3 \in E, M_1 + (M_2 + M_3)= AQ_1 + (AQ_2 + AQM_3) =  (AQ_1 + AQ_2) + AQ_3= (M_1 + M_2) + M_3$, Vrai
\item 3 - Prenons $e = 0_{\mathcal{M}_3(\R)}$. On a $e \in E$ car $0_{\mathcal{M}_3(\R)} = A0_{\mathcal{M}_3(\R)}$ et $M+e=AQ+0_{\mathcal{M}_3(\R)}= M$ et $0_{\mathcal{M}_3(\R)}+AQ= M$, Vrai 
\item 4 - $\forall M_1 \in E,\exists M_2 \in E, M_1 + M_2 = e$. Partie 1, $M_1 + M_2 = e \implies AQ_1 + M_2 = 0_{\mathcal{M}_3(\R)} \implies M_2 = e + (-AQ_1) \implies M_2 = -AQ1 \implies M2 = A(-Q_1)$. Donc $M_2$ existe, Vrai. 
\item N - $\forall M \in E, 1.M = 1.AQ = (1A)Q = AQ = M$, Vrai 
\item A - $\forall \lambda, \mu \in \R, \forall M \in E, \lambda.(\mu M) = \lambda.(\mu AQ) = \lambda \mu.AQ = (\lambda \mu).AQ = (\lambda \mu).M$, Vrai
\item $D_1$ - $\forall \lambda, \mu \in \R, \forall M \in E, (\lambda + \mu).M = \lambda.M + \mu.M$, Vrai par distribution de la loi $+$ sur la loi $.$
\item $D_2$ - $\forall \lambda \in \R, \forall M_1, M_2 \in E, \lambda.(M_1+M_2) = \lambda.M_1 + \lambda.M_2$Vrai par distribution de la loi $.$ sur la loi $+$
\end{itemize}

$(E,+,.)$ est un espace vectoriel de  $\mathcal{M}_3(\R)$.

\subsubsection*{5.6.3}
On a la matrice $A=(a_{3,3})$, $Q=(q_{3,3})$ et $X=(x_{3,1})$. \\
$Ker(f_a) = \{X,f_a(X) = AX = 0\}$.\\
$Ker(f) = \{Q \in \mathcal{M}_3(\R),f(Q) = AQ = 0 \}$.\\

On a 
$$i \in \{1..3\}, AX_{i,1} = a_{i,1}.x_{1,1} + a_{i,2}.x_{2,1} + a_{i,3}.x_{3,1}$$
Et
$$i,j \in \{1..3\}, AQ_{i,j} = a_{i,1}.q_{1,j} + a_{i,2}.q_{2,j} + a_{i,3}.q_{3,j}$$

On a 
$$Ker(f_a) = \{X, i \in \{1..3\}, AX_{i,1} = a_{i,1}.x_{1,1} + a_{i,2}.x_{2,1} + a_{i,3}.x_{3,1} = 0\}$$
Et 
$$Ker(f) = \{Q, i,j \in \{1..3\}, AQ_{i,j} = a_{i,1}.q_{1,j} + a_{i,2}.q_{2,j} + a_{i,3}.q_{3,j} = 0\}$$

Part1 $Q \in Ker(f) \to \text{ les colonnes de Q } \in Ker(f_a)$.
$$Q \in Ker(f) \to \{Q, i,j \in \{1..3\}, AQ_{i,j} = a_{i,1}.q_{1,j} + a_{i,2}.q_{2,j} + a_{i,3}.q_{3,j} = 0 \to a_{i,1}.q_{1,1} + a_{i,2}.q_{2,1} + a_{i,3}.q_{3,1} = 0, a_{i,1}.q_{1,2} + a_{i,2}.q_{2,2} + a_{i,3}.q_{3,2} = 0, a_{i,1}.q_{1,3} + a_{i,2}.q_{2,3} + a_{i,3}.q_{3,3} = 0 $$


\end{document}

