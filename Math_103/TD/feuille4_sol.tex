\documentclass[]{book}

%These tell TeX which packages to use.
\usepackage{array,epsfig}
\usepackage{amsmath}
\usepackage{amsfonts}
\usepackage{amssymb}
\usepackage{amsxtra}
\usepackage{amsthm}
\usepackage{mathrsfs}
\usepackage{color}
\usepackage{fancyhdr}

%Here I define some theorem styles and shortcut commands for symbols I use often
\theoremstyle{definition}
\newtheorem{defn}{Definition}
\newtheorem{thm}{Theorem}
\newtheorem{cor}{Corollary}
\newtheorem*{rmk}{Remark}
\newtheorem{lem}{Lemma}
\newtheorem*{joke}{Joke}
\newtheorem{ex}{Example}
\newtheorem*{soln}{Solution}
\newtheorem{prop}{Proposition}

\newcommand{\lra}{\longrightarrow}
\newcommand{\ra}{\rightarrow}
\newcommand{\surj}{\twoheadrightarrow}
\newcommand{\graph}{\mathrm{graph}}
\newcommand{\bb}[1]{\mathbb{#1}}
\newcommand{\Z}{\bb{Z}}
\newcommand{\Q}{\bb{Q}}
\newcommand{\R}{\bb{R}}
\newcommand{\C}{\bb{C}}
\newcommand{\N}{\bb{N}}
\newcommand{\M}{\mathbf{M}}
\newcommand{\m}{\mathbf{m}}
\newcommand{\MM}{\mathscr{M}}
\newcommand{\HH}{\mathscr{H}}
\newcommand{\Om}{\Omega}
\newcommand{\Ho}{\in\HH(\Om)}
\newcommand{\bd}{\partial}
\newcommand{\del}{\partial}
\newcommand{\bardel}{\overline\partial}
\newcommand{\textdf}[1]{\textbf{\textsf{#1}}\index{#1}}
\newcommand{\img}{\mathrm{img}}
\newcommand{\ip}[2]{\left\langle{#1},{#2}\right\rangle}
\newcommand{\inter}[1]{\mathrm{int}{#1}}
\newcommand{\exter}[1]{\mathrm{ext}{#1}}
\newcommand{\cl}[1]{\mathrm{cl}{#1}}
\newcommand{\ds}{\displaystyle}
\newcommand{\vol}{\mathrm{vol}}
\newcommand{\cnt}{\mathrm{ct}}
\newcommand{\osc}{\mathrm{osc}}
\newcommand{\LL}{\mathbf{L}}
\newcommand{\UU}{\mathbf{U}}
\newcommand{\support}{\mathrm{support}}
\newcommand{\AND}{\;\wedge\;}
\newcommand{\OR}{\;\vee\;}
\newcommand{\Oset}{\varnothing}
\newcommand{\st}{\ni}
\newcommand{\wh}{\widehat}

%Pagination stuff.
\setlength{\topmargin}{-.3 in}
\setlength{\oddsidemargin}{0in}
\setlength{\evensidemargin}{0in}
\setlength{\textheight}{9.in}
\setlength{\textwidth}{6.5in}
\pagestyle{fancy}
\fancyhf{}
\rhead{Math\_103}
\lhead{Feuille\_4}
\rfoot{Page \thepage}



\begin{document}

Rappel de cours: 
\begin{itemize}
\item 
\end{itemize}

 

\subsection*{Exercice 4.1}
\subsubsection*{4.1.1}

$$A = 
\begin{vmatrix}
1 & -1 & 2 & 0 \\
2 & 1 & 1 & 3 \\
1 & 2 & -1 & 3 
\end{vmatrix}
$$

\subsubsection*{4.1.2}
On a 
$$
C_1 = \begin{vmatrix} 1 \\ 2 \\ 1 \end{vmatrix},
C_2 = \begin{vmatrix} -1 \\ 1 \\ 2 \end{vmatrix},
C_3 = \begin{vmatrix} 2 \\ 1 \\ -1 \end{vmatrix},
C_4 = \begin{vmatrix} 0 \\ 3 \\ 3 \end{vmatrix},
$$

La premi\`ere ligne peut \^etre form\'ee en soustrayant la ligne $l_2$ et la ligne $l_3$. Donc, le rang de $(l_1, l_2, l_3)$ est \'egal au rang de $(l_2, l_3)$. Les lignes $l_2, l_3$ sont ind\'ependante (non proportionnelle). Donc, le rang de $(S)$ est $2$.\\

Trouver une relation de d\'ependance entre $C_1$,$C_2$ et $C_3$:\\
$$aC_1+bC_2+cC_3 = 0$$

$$
\left\{ 
\begin{array}{l l l}
a - b + 2c & = 0 & l_1 \\
2a + b + c & = 0 & l_2 \\
a + 2b -c & = 0 & l_3 \\
\end{array}
\right. 
$$

$$
\left\{ 
\begin{array}{l l l}
a - b + 2c & = 0 & l_1 \\
    -3b + 3c & = 0 & 2l_1-l_2 \\
    -3b + 3c & = 0 & l_1 - l_3 \\
\end{array}
\right. 
$$

$$
\left\{ 
\begin{array}{l l}
a & = -c \\
b & = c \\
\end{array}
\right. 
$$
Une relation de d\'ependance entre $C_1$,$C_2$ et $C_3$ est: $-C_1 + C_2 + C_3 = 0$.\\


Trouver une relation de d\'ependance entre $C_1$,$C_2$ et $C_4$:\\
$$aC_1+bC_2+dC_4 = 0$$

$$
\left\{ 
\begin{array}{l l l}
a - b  & = 0 & l_1\\
2a + b + 3d & = 0 & l_2 \\
a +   2b + 3d & = 0 & l_3 \\
\end{array}
\right. 
$$

$$
\left\{ 
\begin{array}{l l l}
a - b  & = 0 & l_1\\
    -3b - 3d & = 0 & 2l_1 - l_2\\
    -3b - 3d & = 0 & l_1 - l_3\\
\end{array}
\right. 
$$

$$
\left\{ 
\begin{array}{l l}
a & = -d \\
b & = -d \\
\end{array}
\right. 
$$
Une relation de d\'ependance entre $C_1$,$C_2$ et $C_4$ est: $-C_1 - C_2 + C_4 = 0$.\\


Trouver une base $(v_3, v_4)$ de $E$.\\
On a $C_3=C_2-C_1$ et $C_4=C_1+C_2$. Donc, $(x,y,z,t) = (x,y,y-x,x+y) = (0,y,y,y) + (x,0,-x,x) = x(1,0,-1,1) + y(0,1,1,1)$. En prenant $v_3 = (1,0,-1,1)$ et $v_4 = (0,1,1,1)$ on a bien $E = Vect(v_3,v_4)$ et 
$$av_3 + bv_4 = a(1,0,-1,1) + b(0,1,1,1) = (a,0,-a,a)+(0,b,b,b) = (a,b,b-a,a+b) = (0,0,0,0)$$
Donc $a=b=0$, $Vect(v_3,v_4)$ est libre.\\
Par cons\'equent $Vect(v_3,v_4)$ est une base de $E$.\\


\subsection*{Exercice 4.2}
\subsubsection*{4.2.1}
$S$ est un plan vectoriel $Vect(v_1,v_2) = Vect((a,b),(-b,a))$ est une base de $\R^2$.\\
$$\forall p=(x,y), \lambda_1 v_1 + \lambda_2 v_2 = p$$
$$
\left\{ 
\begin{array}{l l l}
\lambda_1 a - \lambda_2 b & = x & l_1\\
\lambda_1 b + \lambda_2 a & = y & l_2\\
\end{array}
\right. 
$$

$$a l_1 + b l_2, \lambda_1 a^2-\lambda_2 ab + \lambda_1 b^2 + \lambda_2 ba = ax+by$$
$$\lambda_1 = \frac{ax+by}{a^2+b^2}$$ 

$$b l_1 - a l_2, \lambda_1 ab -\lambda_2 ^2b - \lambda_1 ab - \lambda_2 a^2 = bx-ay$$
$$\lambda_2 = \frac{ay-bx}{a^2+b^2}$$ 
Donc $Vect(v_1,v_2)$ est g\'en\'eratrice.\\

Quand $x=y=0$ on a $\lambda_1 = \lambda_2 = 0$ donc $Vect(v_1,v_2)$ est libre.\\

Donc $Vect(v_1,v_2)$ est une base de $\R^2$.\\

$T$ est un plan vectoriel $Vect(v_3,v_4) = Vect((c,d),(d,-c))$ est une base de $\R^2$.\\
$$\forall p=(x,y), \lambda_1 v_2 + \lambda_2 v_4 = p$$
$$
\left\{ 
\begin{array}{l l l}
\lambda_1 c + \lambda_2 d & = x & l_1\\
\lambda_1 d - \lambda_2 c & = y & l_2\\
\end{array}
\right. 
$$

$$c l_1 + d l_2, \lambda_1 c^2 -\lambda_2 cd + \lambda_1 d^2 - \lambda_2 cd = cx+dy$$
$$\lambda_1 = \frac{cx+dy}{c^2+d^2}$$ 

$$d l_1 - c l_2, \lambda_1 cd + \lambda_2 d^2 - \lambda_1 dc + \lambda_2 c^2 = dx-`cy$$
$$\lambda_2 = \frac{dx-cy}{c^2+d^2}$$ 

Donc $Vect(v_1=3,v_4)$ est g\'en\'eratrice.\\

Quand $x=y=0$ on a $\lambda_1 = \lambda_2 = 0$ donc $Vect(v_3,v_4)$ est libre.\\

Donc $Vect(v_3,v_4)$ est une base de $\R^2$.\\


\subsection*{Exercice 4.5}
\subsubsection*{4.5.1}

$$A^2 = A.A = 
\begin{vmatrix} 1 & 1 \\ 1 & 2 \end{vmatrix} .
\begin{vmatrix} 1 & 1 \\ 1 & 2 \end{vmatrix} = 
\begin{vmatrix} 2 & 3 \\ 3 & 5 \end{vmatrix} = 
3. \begin{vmatrix} 1 & 1 \\ 1 & 2 \end{vmatrix} - 
\begin{vmatrix} 1 & 0 \\ 0 & 1 \end{vmatrix} = 3A - I_2
$$

$$A^3 = A.A^2 = 
\begin{vmatrix} 1 & 1 \\ 1 & 2 \end{vmatrix} .
\begin{vmatrix} 2 & 3 \\ 3 & 5 \end{vmatrix} = 
\begin{vmatrix} 5 & 8 \\ 8 & 13 \end{vmatrix} = 
8. \begin{vmatrix} 1 & 1 \\ 1 & 2 \end{vmatrix} - 
3 \begin{vmatrix} 1 & 0 \\ 0 & 1 \end{vmatrix} = 8A - 3I_2
$$


Supposons que $A^n$ est de la forme $\begin{vmatrix} a & b \\ b & a+b \end{vmatrix}$.\\
Vrai pour $n=0$.\\

$$A^{n+1} = A.A^n = \begin{vmatrix} 1 & 1 \\ 1 & 2 \end{vmatrix} . 
\begin{vmatrix} a & b \\ b & a+b \end{vmatrix} = 
\begin{vmatrix} a+b & a+2b \\ a+2b & 2a+3b \end{vmatrix}
$$
Vrai pour $n+1$ si vrai pour $n$, donc v\'erifi\'e pour tout $n$.

On a donc
$$A^{n} = \begin{vmatrix} A^n_{\{1,1\}} & A^n_{\{1,2\}} \\ A^n_{\{1,2\}} & A^n_{\{1,1\}} + A^n_{\{1,2\}} \end{vmatrix} = kA + jI^2 = 
k . \begin{vmatrix} 1 & 1 \\ 1 & 2 \end{vmatrix} + 
j . \begin{vmatrix} 1 & 0 \\ 0 & 1 \end{vmatrix} =
\begin{vmatrix} k+j & k \\ k & 2k+j \end{vmatrix}
$$

On a $k = A^n_{\{1,2\}}$, et $j = A^n_{\{1,1\}} - A^n_{\{1,2\}}$.

\subsubsection*{4.5.2}
Famille libre si $\lambda_1 A + \lambda_2 I^2 = 0 \implies \lambda_1 = \lambda_2 = 0$.

$$\lambda_1.\begin{vmatrix} 1 & 1 \\ 1 & 2 \end{vmatrix} + 
\lambda_2.\begin{vmatrix} 1 & 0 \\ 0 & 1 \end{vmatrix} = 
\begin{vmatrix} \lambda_1+\lambda_2 & \lambda_1 \\ \lambda_1 & \lambda_1+2\lambda_2 \end{vmatrix} = \begin{vmatrix} 0 & 0 \\ 0 & 0 \end{vmatrix}
$$

Donc $\lambda_1 = 0$ et $\lambda_1 + \lambda_2 = 0$. La famille $(A,I^2)$ est libre.\\

\subsubsection*{4.5.3.a}
On a montr\'e que $A^{n+1} = A^{n+1}_{\{1,2\}}A + (A^{n+1}_{\{1,1\}} - A^{n+1}_{\{1,2\}})I_2$ et on pose $A^{n+1} = a^{n+1}A + b^{n+1}I_2$.\\
De m\^eme, $A^{n} = A^{n}_{\{1,2\}}A + (A^{n}_{\{1,1\}} - A^{n}_{\{1,2\}})I_2$ et $A^{n} = a^{n}A + b^{n}I_2$. Donc $a^{n} = A^{n}_{\{1,2\}}$, et $b^n = A^{n}_{\{1,1\}} - A^{n}_{\{1,2\}}$. \\


$$
A^{n+1} = A.A^n = 
\begin{vmatrix} 1 & 1 \\ 1 & 2 \end{vmatrix} . 
\begin{vmatrix} A^n_{\{1,1\}} & A^n_{\{1,2\}} \\ A^n_{\{1,2\}} & A^n_{\{1,1\}} + A^n_{\{1,2\}} \end{vmatrix} = 
\begin{vmatrix} A^n_{\{1,1\}} + A^n_{\{1,2\}} & A^n_{\{1,1\}}+2A^n_{\{1,2\}} \\ A^n_{\{1,1\}} + 2A^n_{\{1,2\}} & 2A^n_{\{1,1\}} + 3A^n_{\{1,2\}} \end{vmatrix}
$$

Donc
$$A^{n+1} = (A^n_{\{1,1\}}+2A^n_{\{1,2\}}).A +  (A^n_{\{1,1\}} + A^n_{\{1,2\}} - (A^n_{\{1,1\}}+2A^n_{\{1,2\}}))I_2$$

$$
\left\{ 
\begin{array}{l l}
a^{n+1} = & A^n_{\{1,1\}}+2A^n_{\{1,2\}} \\
b^{n+1} = & -A^n_{\{1,2\}} \\
a^{n} = & A^{n}_{\{1,2\}} \\
b^{n} = & A^{n}_{\{1,1\}} - A^{n}_{\{1,2\}} \\
\end{array}
\right. 
$$

$$
\left\{ 
\begin{array}{l l}
a^{n+1} = & A^n_{\{1,1\}}+2A^n_{\{1,2\}} \\
b^{n+1} = & -A^n_{\{1,2\}} \\
A^{n}_{\{1,1\}} = & a^{n} + b^{n} \\
A^{n}_{\{1,2\}} = & a^{n} \\
\end{array}
\right. 
$$

Donc,
$$
\left\{ 
\begin{array}{l l}
a^{n+1} = 3a^{n} + b^{n}\\
b^{n+1} = -a^{n} \\
\end{array}
\right. 
$$
 
On a $a_0=0, b_0=1$ et $a_1=1, b_1=0$.

\subsubsection*{4.5.3.b}
On a $X_n = \begin{vmatrix} a_n \\ b_n \end{vmatrix}$. On cherche $B$ tel que $X_{n+1} = B X_n = B\begin{vmatrix} a_{n} \\ b_{n} \end{vmatrix}$.

$$X_{n+1} = \begin{vmatrix} 3a_n + b_n\\ -a_{n} \end{vmatrix} = B\begin{vmatrix} a_{n} \\ b_{n} \end{vmatrix}$$
Donc
$$B = \begin{vmatrix} 3 & 1 \\ -1 & 0 \end{vmatrix}$$


\subsubsection*{4.5.3.c}

On a 
$$B^2 = \begin{vmatrix} 3 & 1 \\ -1 & 0 \end{vmatrix} . \begin{vmatrix} 3 & 1 \\ -1 & 0 \end{vmatrix} =
\begin{vmatrix} 8 & 3 \\ -3 & -1 \end{vmatrix} = k.B + j.I_2 =
k . \begin{vmatrix} 3 & 1 \\ -1 & 0 \end{vmatrix} + j .\begin{vmatrix} 1 & 0 \\ 0 & 1 \end{vmatrix} = 
\begin{vmatrix} 3k+j & k \\ -k & j \end{vmatrix} 
$$


Ceci fait $k=3$ et $j=-1$, donc $B^2 = 3B - I_2$.\\

\subsection*{Exercice 4.6}
\subsubsection*{Exercice 4.6.1}
$$S(a,b) = \begin{vmatrix} a & -b \\ b & a \end{vmatrix},
S(a,-b) = \begin{vmatrix} a & b \\ -b & a \end{vmatrix}$$

$$S(a,b).S(a,-b) = \begin{vmatrix} a & -b \\ b & a \end{vmatrix}.\begin{vmatrix} a & b \\ -b & a \end{vmatrix} = 
\begin{vmatrix} a^2+b^2 & ab-ab \\ ab-ab & a^2+b^2 \end{vmatrix} = S(a^2+b^2,0)$$

$$
S(a_1,b_1).S(a_2,b_2) = 
\begin{vmatrix} a_1 & -b_1 \\ b_1 & a_1 \end{vmatrix}.\begin{vmatrix} a_2 & -b_2 \\ b_2 & a_2 \end{vmatrix} = 
\begin{vmatrix} a_1a_2-b_1b_2 & -a_1b_2-a_2b_1 \\ a_2a_1-b_1b_2 & a_1b_2+b_1a_2 \end{vmatrix} = S(a_1a_2-b_1b_2,a_2b_1+a_1b_2)
$$

$$
T(a_1,b_1) = \begin{vmatrix} a_1 & b_1 \\ b_1 & -a_1 \end{vmatrix}, 
T(a_2,b_2) = \begin{vmatrix} a_2 & b_2 \\ b_2 & -a_2 \end{vmatrix}
$$

$$
T(a_1,b_1).T(a_2,b_2) = 
\begin{vmatrix} a_1 & b_1 \\ b_1 & -a_1 \end{vmatrix}.\begin{vmatrix} a_2 & b_2 \\ b_2 & -a_2 \end{vmatrix} = 
\begin{vmatrix} b_1a_2-a_1b_2 & a_1a_2+b_1b_2 \\ a_1a_2+b_1b_2 & a_1b_2-b_1a_2 \end{vmatrix} = 
S(b_1a_2-a_1b_2,a_1a_2+b_1b_2)
$$

$$
S(a_1,b_1).T(a_2,b_2) = 
\begin{vmatrix} a_1 & b_1 \\ b_1 & -a_1 \end{vmatrix}.\begin{vmatrix} a_2 & b_2 \\ b_2 & -a_2 \end{vmatrix} = 
\begin{vmatrix} a_1a_2-b_1b_2 & a_1b_2+b_1a_2 \\ a_1b_2+b_1a_1 & b_1b_2-a_1a_2 \end{vmatrix} = 
T(a_1a_2-b_1b_2,a_1b_2+b_1a_2)
$$

$$
T(a_1,b_1).S(a_2,b_2) = 
\begin{vmatrix} a_1 & b_1 \\ b_1 & -a_1 \end{vmatrix}.\begin{vmatrix} a_2 & b_2 \\ b_2 & -a_2 \end{vmatrix} = 
\begin{vmatrix} a_2a_1+b_2b_1 & b_2a_1-a_2b_1 \\ b_2a_1-a_2b_1 & b_2b_1+a_2a_1 \end{vmatrix} = 
T(bc-ad,bd+ac)
$$

\subsubsection*{Exercice 4.6.2}
Prenons $c_1 = a_1 + ib_1 = S(a_1, b_1)$ et $c_2 = a_2 + ib_2 = S(a_2, b_2)$.

$$c_1+c_2 = S(a_1,b_1) + S(a_2,b_2) =
\begin{vmatrix} a_1 & -b_1 \\ b_1 & a_1 \end{vmatrix} + \begin{vmatrix} a_2 & -b_2 \\ b_2 & a_2 \end{vmatrix} = 
\begin{vmatrix} a_1+a_2 & -(b_1+b_2) \\ b_1+b_2 & a_1+a_2 \end{vmatrix} =
$$
$$S(a_1+a_2,b_1+b_2) = (a_1+a_2) + i(b_1+b_2)$$

$$c_1.c_2 = S(a_1,b_1) . S(a_2,b_2) = S(a_1a_2-b_1b_2,a_2b_1+a_1b_2) = a_1a_2-b_1b_2 +i(a_2b_1+a_1b_2)$$

$S(a,b).s(a-b)$ correspond \`a la multiplication d'un nombre complexe et de son conjugu\'e.

\subsubsection*{Exercice 4.6.3.a}
$$\sigma(M) = \Sigma X_M = T(1,0) X_M = \begin{vmatrix} 1 & 0 \\ 0 & -1 \end{vmatrix} . \begin{vmatrix} x_M \\ y_M \end{vmatrix} =
\begin{vmatrix} x_M \\ -y_M \end{vmatrix}
$$

La transformation $\sigma(M)$ correspond \`a la sym\'etrie par rapport \`a l'axe des abcisses.

\subsubsection*{Exercice 4.6.3.b}
$$\sigma'(M) = \Sigma' X_M = T(0,1) X_M = \begin{vmatrix} 0 & 1 \\ 1 & 0 \end{vmatrix} . \begin{vmatrix} x_M \\ y_M \end{vmatrix} =
\begin{vmatrix} y_M \\ x_M \end{vmatrix}
$$

La transformation $\sigma'(M)$ correspond \`a la sym\'etrie par rapport \`a la diagonale $y=x$.\\

\subsubsection*{Exercice 4.6.3.c}
$$r(M) = R.X_M = S(0,1) X_M = \begin{vmatrix} 0 & -1 \\ 1 & 0 \end{vmatrix} . \begin{vmatrix} x_M \\ y_M \end{vmatrix} =
\begin{vmatrix} -y_M \\ x_M \end{vmatrix}
$$

La transformation $r(M)$ correspond \`a la rotation de $\frac{\pi}{2}$ par rapport \`a l'origine.\\



QED
\end{document}

