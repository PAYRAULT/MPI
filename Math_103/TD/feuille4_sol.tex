\documentclass[]{book}

%These tell TeX which packages to use.
\usepackage{array,epsfig}
\usepackage{amsmath}
\usepackage{amsfonts}
\usepackage{amssymb}
\usepackage{amsxtra}
\usepackage{amsthm}
\usepackage{mathrsfs}
\usepackage{color}
\usepackage{fancyhdr}

%Here I define some theorem styles and shortcut commands for symbols I use often
\theoremstyle{definition}
\newtheorem{defn}{Definition}
\newtheorem{thm}{Theorem}
\newtheorem{cor}{Corollary}
\newtheorem*{rmk}{Remark}
\newtheorem{lem}{Lemma}
\newtheorem*{joke}{Joke}
\newtheorem{ex}{Example}
\newtheorem*{soln}{Solution}
\newtheorem{prop}{Proposition}

\newcommand{\lra}{\longrightarrow}
\newcommand{\ra}{\rightarrow}
\newcommand{\surj}{\twoheadrightarrow}
\newcommand{\graph}{\mathrm{graph}}
\newcommand{\bb}[1]{\mathbb{#1}}
\newcommand{\Z}{\bb{Z}}
\newcommand{\Q}{\bb{Q}}
\newcommand{\R}{\bb{R}}
\newcommand{\C}{\bb{C}}
\newcommand{\N}{\bb{N}}
\newcommand{\M}{\mathbf{M}}
\newcommand{\m}{\mathbf{m}}
\newcommand{\MM}{\mathscr{M}}
\newcommand{\HH}{\mathscr{H}}
\newcommand{\Om}{\Omega}
\newcommand{\Ho}{\in\HH(\Om)}
\newcommand{\bd}{\partial}
\newcommand{\del}{\partial}
\newcommand{\bardel}{\overline\partial}
\newcommand{\textdf}[1]{\textbf{\textsf{#1}}\index{#1}}
\newcommand{\img}{\mathrm{img}}
\newcommand{\ip}[2]{\left\langle{#1},{#2}\right\rangle}
\newcommand{\inter}[1]{\mathrm{int}{#1}}
\newcommand{\exter}[1]{\mathrm{ext}{#1}}
\newcommand{\cl}[1]{\mathrm{cl}{#1}}
\newcommand{\ds}{\displaystyle}
\newcommand{\vol}{\mathrm{vol}}
\newcommand{\cnt}{\mathrm{ct}}
\newcommand{\osc}{\mathrm{osc}}
\newcommand{\LL}{\mathbf{L}}
\newcommand{\UU}{\mathbf{U}}
\newcommand{\support}{\mathrm{support}}
\newcommand{\AND}{\;\wedge\;}
\newcommand{\OR}{\;\vee\;}
\newcommand{\Oset}{\varnothing}
\newcommand{\st}{\ni}
\newcommand{\wh}{\widehat}

%Pagination stuff.
\setlength{\topmargin}{-.3 in}
\setlength{\oddsidemargin}{0in}
\setlength{\evensidemargin}{0in}
\setlength{\textheight}{9.in}
\setlength{\textwidth}{6.5in}
\pagestyle{fancy}
\fancyhf{}
\rhead{Math\_103}
\lhead{Feuille\_4}
\rfoot{Page \thepage}



\begin{document}

Rappel de cours: 
\begin{itemize}
\item 
\end{itemize}

 

\subsection*{Exercice 4.1}
\subsubsection*{4.1.1}

$$A = 
\begin{vmatrix}
1 & -1 & 2 & 0 \\
2 & 1 & 1 & 3 \\
1 & 2 & -1 & 3 
\end{vmatrix}
$$

\subsubsection*{4.1.2}
On a 
$$
C_1 = \begin{vmatrix} 1 \\ 2 \\ 1 \end{vmatrix},
C_2 = \begin{vmatrix} -1 \\ 1 \\ 2 \end{vmatrix},
C_3 = \begin{vmatrix} 2 \\ 1 \\ -1 \end{vmatrix},
C_4 = \begin{vmatrix} 0 \\ 3 \\ 3 \end{vmatrix},
$$

La premi\`ere ligne peut \^etre form\'ee en soustrayant la ligne $l_2$ et la ligne $l_3$. Donc, le rang de $(l_1, l_2, l_3)$ est \'egal au rang de $(l_2, l_3)$. Les lignes $l_2, l_3$ sont ind\'ependante (non proportionnelle). Donc, le rang de $(S)$ est $2$.\\

Trouver une relation de d\'ependance entre $C_1$,$C_2$ et $C_3$:\\
$$aC_1+bC_2+cC_3 = 0$$

$$
\left\{ 
\begin{array}{l l l}
a - b + 2c & = 0 & l_1 \\
2a + b + c & = 0 & l_2 \\
a + 2b -c & = 0 & l_3 \\
\end{array}
\right. 
$$

$$
\left\{ 
\begin{array}{l l l}
a - b + 2c & = 0 & l_1 \\
    -3b + 3c & = 0 & 2l_1-l_2 \\
    -3b + 3c & = 0 & l_1 - l_3 \\
\end{array}
\right. 
$$

$$
\left\{ 
\begin{array}{l l}
a & = -c \\
b & = c \\
\end{array}
\right. 
$$
Une relation de d\'ependance entre $C_1$,$C_2$ et $C_3$ est: $-C_1 + C_2 + C_3 = 0$.\\


Trouver une relation de d\'ependance entre $C_1$,$C_2$ et $C_4$:\\
$$aC_1+bC_2+dC_4 = 0$$

$$
\left\{ 
\begin{array}{l l l}
a - b  & = 0 & l_1\\
2a + b + 3d & = 0 & l_2 \\
a +   2b + 3d & = 0 & l_3 \\
\end{array}
\right. 
$$

$$
\left\{ 
\begin{array}{l l l}
a - b  & = 0 & l_1\\
    -3b - 3d & = 0 & 2l_1 - l_2\\
    -3b - 3d & = 0 & l_1 - l_3\\
\end{array}
\right. 
$$

$$
\left\{ 
\begin{array}{l l}
a & = -d \\
b & = -d \\
\end{array}
\right. 
$$
Une relation de d\'ependance entre $C_1$,$C_2$ et $C_4$ est: $-C_1 - C_2 + C_4 = 0$.\\


Trouver une base $(v_3, v_4)$ de $E$.\\
On a $C_3=C_2-C_1$ et $C_4=C_1+C_2$. Donc, $(x,y,z,t) = (x,y,y-x,x+y) = (0,y,y,y) + (x,0,-x,x) = x(1,0,-1,1) + y(0,1,1,1)$. En prenant $v_3 = (1,0,-1,1)$ et $v_4 = (0,1,1,1)$ on a bien $E = Vect(v_3,v_4)$ et 
$$av_3 + bv_4 = a(1,0,-1,1) + b(0,1,1,1) = (a,0,-a,a)+(0,b,b,b) = (a,b,b-a,a+b) = (0,0,0,0)$$
Donc $a=b=0$, $Vect(v_3,v_4)$ est libre.\\
Par cons\'equent $Vect(v_3,v_4)$ est une base de $E$.\\


QED

\end{document}

