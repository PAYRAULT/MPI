\documentclass[]{book}

%These tell TeX which packages to use.
\usepackage{array,epsfig}
\usepackage{amsmath}
\usepackage{amsfonts}
\usepackage{amssymb}
\usepackage{amsxtra}
\usepackage{amsthm}
\usepackage{mathrsfs}
\usepackage{color}
\usepackage{fancyhdr}

%Here I define some theorem styles and shortcut commands for symbols I use often
\theoremstyle{definition}
\newtheorem{defn}{Definition}
\newtheorem{thm}{Theorem}
\newtheorem{cor}{Corollary}
\newtheorem*{rmk}{Remark}
\newtheorem{lem}{Lemma}
\newtheorem*{joke}{Joke}
\newtheorem{ex}{Example}
\newtheorem*{soln}{Solution}
\newtheorem{prop}{Proposition}

\newcommand{\lra}{\longrightarrow}
\newcommand{\ra}{\rightarrow}
\newcommand{\surj}{\twoheadrightarrow}
\newcommand{\graph}{\mathrm{graph}}
\newcommand{\bb}[1]{\mathbb{#1}}
\newcommand{\Z}{\bb{Z}}
\newcommand{\Q}{\bb{Q}}
\newcommand{\R}{\bb{R}}
\newcommand{\C}{\bb{C}}
\newcommand{\N}{\bb{N}}
\newcommand{\M}{\mathbf{M}}
\newcommand{\m}{\mathbf{m}}
\newcommand{\MM}{\mathscr{M}}
\newcommand{\HH}{\mathscr{H}}
\newcommand{\Om}{\Omega}
\newcommand{\Ho}{\in\HH(\Om)}
\newcommand{\bd}{\partial}
\newcommand{\del}{\partial}
\newcommand{\bardel}{\overline\partial}
\newcommand{\textdf}[1]{\textbf{\textsf{#1}}\index{#1}}
\newcommand{\img}{\mathrm{img}}
\newcommand{\ip}[2]{\left\langle{#1},{#2}\right\rangle}
\newcommand{\inter}[1]{\mathrm{int}{#1}}
\newcommand{\exter}[1]{\mathrm{ext}{#1}}
\newcommand{\cl}[1]{\mathrm{cl}{#1}}
\newcommand{\ds}{\displaystyle}
\newcommand{\vol}{\mathrm{vol}}
\newcommand{\cnt}{\mathrm{ct}}
\newcommand{\osc}{\mathrm{osc}}
\newcommand{\LL}{\mathbf{L}}
\newcommand{\UU}{\mathbf{U}}
\newcommand{\support}{\mathrm{support}}
\newcommand{\AND}{\;\wedge\;}
\newcommand{\OR}{\;\vee\;}
\newcommand{\Oset}{\varnothing}
\newcommand{\st}{\ni}
\newcommand{\wh}{\widehat}

%Pagination stuff.
\setlength{\topmargin}{-.3 in}
\setlength{\oddsidemargin}{0in}
\setlength{\evensidemargin}{0in}
\setlength{\textheight}{9.in}
\setlength{\textwidth}{6.5in}
\pagestyle{fancy}
\fancyhf{}
\rhead{Math\_103}
\lhead{Exam\_1}
\rfoot{Page \thepage}



\begin{document}

Rappel de cours: 
\begin{itemize}
\item 
\end{itemize}

 

\subsection*{Exercice 3.3}
\subsubsection*{Exercice 3.3.1}
Prenons $x_1$ comme inconnue sedondaire du syst\`eme d'\'equations. Donc, $(x_1,x_2,x_3) = x_1(1,2,-1)$ et

$$
\left\{ 
\begin{array}{l}
x_1 = x_1 \\
x_2 = 2x_1 \\
x_3 = -x_1 \\
\end{array}
\right. 
$$

Le syst\`eme d'\'equations est:
$$
\left\{ 
\begin{array}{l l l l}
2x_1 & -x_2 & & = 0 \\
-x_1 & & -x_3 & = 0\\
\end{array}
\right. 
$$

\subsubsection*{Exercice 3.3.2}
Le $(v,w)$ est libre si $\lambda_1v + \lambda_2w = 0 \implies \lambda_1 = \lambda_2 = 0$. Donc $\lambda_1(1,2,-3) + \lambda_2(1,-1,1) = 0$. 

Le syst\`eme d'\'equations est:
$$
\left\{ 
\begin{array}{l l l}
\lambda_1 & +\lambda_2 & = 0 \\
2\lambda_1 & -\lambda_2 & = 0 \\
-3\lambda_1 & +\lambda_2 & = 0 \\
\end{array}
\right. 
$$

De $L1$, on a $\lambda_1 = -\lambda_2$, on remplace dans $L2$, $2\lambda_1 +\lambda_1 = 0$, $3\lambda_1 = 0$. Donc $\lambda_1 = \lambda_2 = 0$. La famille est libre.\\

Le syst\`eme d'\'equations de $Vect(v,w)$ est
$$
\left\{ 
\begin{array}{l l l}
\lambda_1 & +\lambda_2 & = x_1 \\
2\lambda_1 & -\lambda_2 & = x_2 \\
-3\lambda_1 & +\lambda_2 & = x_3 \\
\end{array}
\right. 
$$

$L1+L2+L3$, $\lambda_2 = x_1+x_2+x_3$, en remplacant dans $L1$, on a $\lambda_1 = -x_2 - x_3$. Donc

$$
\left\{ 
\begin{array}{l l}
x_1 & = x_1 \\
2(-x_2 - x_3) - (x_1 + x_2 + x_3) & = x_2 \\
-3(-x_2 - x_3) + (x_1 + x_2 + x_3) & = x_3 \\
\end{array}
\right. 
$$

$$
\left\{ 
\begin{array}{l l}
-x_1 - 4x_2 - 3x_3 & = 0 \\
x_1 + 4x_2 + 3x_3 & = 0 \\
\end{array}
\right. 
$$

Les \'equations cart\'esienne de $Vect(v,w)$ est $x_1 + 4x_2 + 3x_3 =0$' L'espace vectoriel est un plan.




QED

\end{document}

