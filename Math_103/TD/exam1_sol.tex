\documentclass[]{book}

%These tell TeX which packages to use.
\usepackage{array,epsfig}
\usepackage{amsmath}
\usepackage{amsfonts}
\usepackage{amssymb}
\usepackage{amsxtra}
\usepackage{amsthm}
\usepackage{mathrsfs}
\usepackage{color}
\usepackage{pgfplots}

%Here I define some theorem styles and shortcut commands for symbols I use often
\theoremstyle{definition}
\newtheorem{defn}{Definition}
\newtheorem{thm}{Theorem}
\newtheorem{cor}{Corollary}
\newtheorem*{rmk}{Remark}
\newtheorem{lem}{Lemma}
\newtheorem*{joke}{Joke}
\newtheorem{ex}{Example}
\newtheorem*{soln}{Solution}
\newtheorem{prop}{Proposition}

\newcommand{\lra}{\longrightarrow}
\newcommand{\ra}{\rightarrow}
\newcommand{\surj}{\twoheadrightarrow}
\newcommand{\graph}{\mathrm{graph}}
\newcommand{\bb}[1]{\mathbb{#1}}
\newcommand{\Z}{\bb{Z}}
\newcommand{\Q}{\bb{Q}}
\newcommand{\R}{\bb{R}}
\newcommand{\C}{\bb{C}}
\newcommand{\N}{\bb{N}}
\newcommand{\M}{\mathbf{M}}
\newcommand{\m}{\mathbf{m}}
\newcommand{\MM}{\mathscr{M}}
\newcommand{\HH}{\mathscr{H}}
\newcommand{\Om}{\Omega}
\newcommand{\Ho}{\in\HH(\Om)}
\newcommand{\bd}{\partial}
\newcommand{\del}{\partial}
\newcommand{\bardel}{\overline\partial}
\newcommand{\textdf}[1]{\textbf{\textsf{#1}}\index{#1}}
\newcommand{\img}{\mathrm{img}}
\newcommand{\ip}[2]{\left\langle{#1},{#2}\right\rangle}
\newcommand{\inter}[1]{\mathrm{int}{#1}}
\newcommand{\exter}[1]{\mathrm{ext}{#1}}
\newcommand{\cl}[1]{\mathrm{cl}{#1}}
\newcommand{\ds}{\displaystyle}
\newcommand{\vol}{\mathrm{vol}}
\newcommand{\cnt}{\mathrm{ct}}
\newcommand{\osc}{\mathrm{osc}}
\newcommand{\LL}{\mathbf{L}}
\newcommand{\UU}{\mathbf{U}}
\newcommand{\support}{\mathrm{support}}
\newcommand{\AND}{\;\wedge\;}
\newcommand{\OR}{\;\vee\;}
\newcommand{\Oset}{\varnothing}
\newcommand{\st}{\ni}
\newcommand{\wh}{\widehat}

%Pagination stuff.
\setlength{\topmargin}{-.3 in}
\setlength{\oddsidemargin}{0in}
\setlength{\evensidemargin}{0in}
\setlength{\textheight}{9.in}
\setlength{\textwidth}{6.5in}
\pagestyle{empty}



\begin{document}

\subsection*{Rappel de cours}

\begin{itemize}
\item 
\end{itemize}


\subsection*{Exercice 1.1}
$$
\left\{ 
\begin{array}{l l l l l}
  x & +y & -z & -t & = 0\\
  x & -y & +z & -t & = 0\\
  x &    &    & -t & = 0\\
    & y  & -z &    & = 0\\
\end{array}
\right. 
$$

C'est un syst\`eme d'\'equations homog\`ene de rang 4, \`a 4 inconnues. Aucune ligne nulle. Les inconnues principales sont x et y. Les inconnues secondaires sont z et t.


\subsection*{Exercice 1.2}
$$(S_0)
\left\{ 
\begin{array}{l l l l l}
  x & -3y &     & = a_1 & [1]\\
    & 3y  & -6z & = a_2 & [2]\\
  x &     & -6z & = a_3 & [3]\\
\end{array}
\right. 
$$

Calculer [1]+[2], $x - 6z = a_1+a_2$, qui est \'egale \`a l'\'equation [3]. Donc $a_1+a_2 = a_3$.\\
Ou calculer [1]-[3], $-3y +6z = a_1-a_3$, qui est \'egale \`a la n\'egation de l'\'equation [2]. Donc $a_2 = a_3-a_1$.\\

$$(S_0)
\left\{ 
\begin{array}{l l l l l}
  x & -3y &     & = 1 & [1]\\
    & 3y  & -6z & = 1 & [2]\\
  x &     & -6z & = 2 & [3]\\
\end{array}
\right. 
$$

Lorsque $(a_1,a_2,a_3) = (1,1,2)$, le syst\`me est compatible car $2 = 1+1$. 
Une solution du syst\`eme est $(x,y,z)=(4,1,\frac{1}{3})$.\\

Lorsque $(a_1,a_2,a_3) = (0,0,0)$, le syst\`me est compatible. Une solution du syst\`eme est $(x,y,z)=(0,0,0)$.\\


\subsection*{Exercice 1.3}


\end{document}

