\documentclass[]{book}

%These tell TeX which packages to use.
\usepackage{array,epsfig}
\usepackage{amsmath}
\usepackage{amsfonts}
\usepackage{amssymb}
\usepackage{amsxtra}
\usepackage{amsthm}
\usepackage{mathrsfs}
\usepackage{color}
\usepackage{pgfplots}

%Here I define some theorem styles and shortcut commands for symbols I use often
\theoremstyle{definition}
\newtheorem{defn}{Definition}
\newtheorem{thm}{Theorem}
\newtheorem{cor}{Corollary}
\newtheorem*{rmk}{Remark}
\newtheorem{lem}{Lemma}
\newtheorem*{joke}{Joke}
\newtheorem{ex}{Example}
\newtheorem*{soln}{Solution}
\newtheorem{prop}{Proposition}

\newcommand{\lra}{\longrightarrow}
\newcommand{\ra}{\rightarrow}
\newcommand{\surj}{\twoheadrightarrow}
\newcommand{\graph}{\mathrm{graph}}
\newcommand{\bb}[1]{\mathbb{#1}}
\newcommand{\Z}{\bb{Z}}
\newcommand{\Q}{\bb{Q}}
\newcommand{\R}{\bb{R}}
\newcommand{\C}{\bb{C}}
\newcommand{\N}{\bb{N}}
\newcommand{\M}{\mathbf{M}}
\newcommand{\m}{\mathbf{m}}
\newcommand{\MM}{\mathscr{M}}
\newcommand{\HH}{\mathscr{H}}
\newcommand{\Om}{\Omega}
\newcommand{\Ho}{\in\HH(\Om)}
\newcommand{\bd}{\partial}
\newcommand{\del}{\partial}
\newcommand{\bardel}{\overline\partial}
\newcommand{\textdf}[1]{\textbf{\textsf{#1}}\index{#1}}
\newcommand{\img}{\mathrm{img}}
\newcommand{\ip}[2]{\left\langle{#1},{#2}\right\rangle}
\newcommand{\inter}[1]{\mathrm{int}{#1}}
\newcommand{\exter}[1]{\mathrm{ext}{#1}}
\newcommand{\cl}[1]{\mathrm{cl}{#1}}
\newcommand{\ds}{\displaystyle}
\newcommand{\vol}{\mathrm{vol}}
\newcommand{\cnt}{\mathrm{ct}}
\newcommand{\osc}{\mathrm{osc}}
\newcommand{\LL}{\mathbf{L}}
\newcommand{\UU}{\mathbf{U}}
\newcommand{\support}{\mathrm{support}}
\newcommand{\AND}{\;\wedge\;}
\newcommand{\OR}{\;\vee\;}
\newcommand{\Oset}{\varnothing}
\newcommand{\st}{\ni}
\newcommand{\wh}{\widehat}

%Pagination stuff.
\setlength{\topmargin}{-.3 in}
\setlength{\oddsidemargin}{0in}
\setlength{\evensidemargin}{0in}
\setlength{\textheight}{9.in}
\setlength{\textwidth}{6.5in}
\pagestyle{empty}



\begin{document}

\subsection*{Rappel de cours}

\begin{itemize}
\item 
\end{itemize}


\subsection*{Exercice 1.1}
$$
\left\{ 
\begin{array}{l l l l l}
  x & +y & -z & -t & = 0\\
  x & -y & +z & -t & = 0\\
  x &    &    & -t & = 0\\
    & y  & -z &    & = 0\\
\end{array}
\right. 
$$

C'est un syst\`eme d'\'equations homog\`ene de rang 4, \`a 4 inconnues. Aucune ligne nulle. Les inconnues principales sont x et y. Les inconnues secondaires sont z et t.


\subsection*{Exercice 1.2}
$$(S_0)
\left\{ 
\begin{array}{l l l l l}
  x & -3y &     & = a_1 & [1]\\
    & 3y  & -6z & = a_2 & [2]\\
  x &     & -6z & = a_3 & [3]\\
\end{array}
\right. 
$$

Calculer [1]+[2], $x - 6z = a_1+a_2$, qui est \'egale \`a l'\'equation [3]. Donc $a_1+a_2 = a_3$.\\
Ou calculer [1]-[3], $-3y +6z = a_1-a_3$, qui est \'egale \`a la n\'egation de l'\'equation [2]. Donc $a_2 = a_3-a_1$.\\

$$(S_0)
\left\{ 
\begin{array}{l l l l l}
  x & -3y &     & = 1 & [1]\\
    & 3y  & -6z & = 1 & [2]\\
  x &     & -6z & = 2 & [3]\\
\end{array}
\right. 
$$

Lorsque $(a_1,a_2,a_3) = (1,1,2)$, le syst\`me est compatible car $2 = 1+1$. 
La solution du syst\`eme est $(x,y,z)=(a,\frac{a-1}{3},\frac{a-2}{6})$.\\

Lorsque $(a_1,a_2,a_3) = (0,0,0)$, le syst\`me est compatible. La solution du syst\`eme est $(x,y,z)=(a,\frac{a}{3},\frac{a}{6})$.\\


\subsection*{Exercice 1.3.a}
La famille $((1, 2, 3, 0), (3, 1, 2, 0), (2, 3, 1, 0), (3, 2, 1, 0))$ engendre $\R^4$ si

$$ \forall a \in \R^4, \exists (\lambda_1,\lambda_2,\lambda_3,\lambda_4) \in \R^4,
\left\{ 
\begin{array}{l l l l l}
  1\lambda_1 & +3\lambda_2 & +2\lambda_3 & +3\lambda_4  & = a_1 \\
  1\lambda_1 & +3\lambda_2 & +2\lambda_3 & +3\lambda_4  & = a_2 \\
  1\lambda_1 & +3\lambda_2 & +2\lambda_3 & +3\lambda_4  & = a_3 \\
  0\lambda_1 & +0\lambda_2 & +0\lambda_3 & +0\lambda_4  & = a_4 \\
\end{array}
\right. 
$$

Le vecteur $(a_1, a_2, a_3, 1) \in \R^4$ mais ne peux pas \^etre g\'en\'e\'re par la famille. Donc, cette famille n'engendre pas l'espace vectoriel $\R^4$.

\subsection*{Exercice 1.3.b}
La famille $((1, 2, 3), (1, 3, 2), (2, 1, 3), (2, 3, 1))$ est libre si
$$
\left\{ 
\begin{array}{l l l l}
  1\lambda_1 & +2\lambda_2 & +3\lambda_3   & = 0 \\
  1\lambda_1 & +3\lambda_2 & +2\lambda_3   & = 0 \\
  2\lambda_1 & +1\lambda_2 & +3\lambda_3   & = 0 \\
  2\lambda_1 & +3\lambda_2 & +1\lambda_3   & = 0 \\
\end{array}
\right. 
\implies
\lambda_1=\lambda_2=\lambda_3=0
$$

$$L_2 \leftarrow L_2-L_1:\, \lambda_2 - \lambda_3 = 0$$
$$L_3 \leftarrow L_3-2L_1:\, -3\lambda_2 - 3\lambda_3 = 0$$
$$L_4 \leftarrow L_4-2L_1:\, -\lambda_2 - 5\lambda_3 = 0$$
$$L_4 \leftarrow L_4+L_1:\, -6\lambda_3 = 0$$

Donc $\lambda_3 = \lambda_2 = \lambda_1 = 0$. La famille est libre.

\subsection*{Exercice 1.4}
La famille s'\'ecrit $\mathcal{F} = ((1,1,0),(0,1,1),(0,1,-1))$.

La famille engendre-t-elle $\mathcal{P}_2$?
bd
$$ \forall a \in \mathcal{P}^2, \exists (\lambda_1,\lambda_2,\lambda_3) \in \R^3,
\left\{ 
\begin{array}{l l l l}
  \lambda_1 & +\lambda_2 &               & = a_1 \\
            & \lambda_2  & +\lambda_3   & = a_2 \\
            & \lambda_2  & -\lambda_3   & = a_3 \\
\end{array}
\right. 
$$
$$L_3 \leftarrow L_3-L_2:\, -2\lambda_3 = a_3-a_2$$
$$L_2:\, 2\lambda_2 = a_2+a_3$$
$$L_1:\, 2\lambda_1 = 2a_1-a_2-a_3$$

Donc $\lambda_1 = \frac{2a_1-a_2-a_3}{2}$, $\lambda_1 = \frac{a_2+a_3}{2}$, $\lambda_1 = \frac{a_3-a_2}{2}$. La famille $\mathcal{F}$ engendre l'espace vectoriel $\mathcal{P}_2$.\\

\medskip
La famille est-elle libre?
$$
\left\{ 
\begin{array}{l l l l}
  \lambda_1 & +\lambda_2 &               & = 0 \\
            & \lambda_2  & +\lambda_3   & = 0 \\
            & \lambda_2  & -\lambda_3   & = 0 \\
\end{array}
\right. 
$$

$$L_3 \leftarrow L_3-L_2:\, -2\lambda_3 = 0$$

Donc $\lambda_3 = \lambda_2 = \lambda_1 = 0$. La famille $\mathcal{F}$ est libre.

\subsection*{Exercice 1.5}
La famille $\mathcal{F} = (x \to \sin(x),x \to f(x))$.
$$
\forall g: x \to g(x), \exists \lambda_1, \lambda_2 \in \R,
\left\{ 
\begin{array}{l l}
  \lambda_1(x \to \sin(x)) & \lambda_2(x \to f(x)) = x \to g(x) \\
\end{array}
\right. 
$$

Non. Prenons $g: x \to \cos(x)$. Pour $x \to \infty$, on a $\lambda_1(x \to \sin(x)) = (x \to cos(x)$ Il n'existe pas de $\lambda_1$ qui engendre la fonction $x \to \cos(x)$ car $\sin(x) \neq \frac{\cos(x)}{\lambda_1}$.

\subsection*{Exercice 1.6}

La famille $((1, 0, 0, 0), (1, 2, 0, 0), (1, 2, 3, 0), (1, 2, 3, 4))$ est libre si
$$
\left\{ 
\begin{array}{l l l l l}
  \lambda_1 &  &  &   & = 0 \\
  \lambda_1 & +2\lambda_2 &  &   & = 0 \\
  \lambda_1 & +2\lambda_2 & +3\lambda_3 &  & = 0 \\
  \lambda_1 & +2\lambda_2 & +3\lambda_3 & +4\lambda_4  & = 0 \\
\end{array}
\right. 
\implies
\lambda_1=\lambda_2=\lambda_3=0
$$

$$L_1:\,  \lambda_1 = 0$$
$$L_2:\, 2\lambda_2 = 0$$
$$L_3:\, 3\lambda_3 = 0$$
$$L_4:\, 4\lambda_4 = 0$$

Donc $\lambda_3 = \lambda_2 = \lambda_1 = 0$. La famille est libre.

$$E \subset F\, ssi\, \forall f \in E,\, f \in F$$
On a $E = \lambda_{E1} u_1 + \lambda_{E2} u_2$ et $F = \lambda_{F1} u_1 + \lambda_{F2} u_2 + \lambda_{F3} u_3$\\
Soit $f \in E$, alors $f \in F$ avec $\lambda_{F1} = \lambda_{E1}, \lambda_{F2} = \lambda_{E2}, \lambda_{F3} = 0$.\\
L'inclusion est strict car lorsque $\lambda_{F3} \neq 0$ alors $f \in F$ mais $f \notin E$.\\

\medskip
Soit $f \in F$, alors $f \in \R^4$ avec $\lambda_{1} = \lambda_{F1}, \lambda_{2} = \lambda_{F2}, \lambda_{3} = \lambda_{F3}$.\\
L'inclusion est strict car $(3,3,3,3) \in \R^4$ mais $(1,2,3,6) \notin F$ car $\lnot \exists \lambda_{F1},\lambda_{F2},\lambda_{F3},\lambda_{F4}, 0\lambda_{F1} + 0\lambda_{F2} 0\lambda_{F3} 0\lambda_{F4} = 6$.

\subsection*{Exercice 1.7}
la famille $(e_1, e_2, e_3, e_4)$ est une base canonique de $\R^4$ donc
$$ \forall a \in \R^4, \exists (\lambda_1,\lambda_2,\lambda_3,\lambda_4) \in \R^4,
\left\{ 
\begin{array}{l l l l l}
  \lambda_1 e_{11} & +\lambda_2 e_{21} & +\lambda_3 e_{31} & +\lambda_4 e_{41} & = a_1 \\
  \lambda_1 e_{12} & +\lambda_2 e_{22} & +\lambda_3 e_{32} & +\lambda_4 e_{42} & = a_2 \\
  \lambda_1 e_{13} & +\lambda_2 e_{23} & +\lambda_3 e_{33} & +\lambda_4 e_{43} & = a_3 \\
  \lambda_1 e_{14} & +\lambda_2 e_{24} & +\lambda_3 e_{34} & +\lambda_4 e_{44} & = a_4 \\
\end{array}
\right. 
$$
et
$$
\left\{ 
\begin{array}{l l l l l}
  \lambda_1 e_{11} & +\lambda_2 e_{21} & +\lambda_3 e_{31} & +\lambda_4 e_{41} & = 0 \\
  \lambda_1 e_{12} & +\lambda_2 e_{22} & +\lambda_3 e_{32} & +\lambda_4 e_{42} & = 0 \\
  \lambda_1 e_{13} & +\lambda_2 e_{23} & +\lambda_3 e_{33} & +\lambda_4 e_{43} & = 0 \\
  \lambda_1 e_{14} & +\lambda_2 e_{24} & +\lambda_3 e_{34} & +\lambda_4 e_{44} & = 0 \\
\end{array}
\right. 
\implies
\lambda_1=\lambda_2=\lambda_3=\lambda_4=0
$$

Pour que $(u_1, u_2, e_2, e_4)$ soit une base canonique de $\R^4$, il faut montrer:\\
P1:
$$ \forall b \in \R^4, \exists (\beta_1,\beta_2,\beta_3,\beta_4) \in \R^4,
\left\{ 
\begin{array}{l l l l l}
  \beta_1  & +2\beta_2 & +\beta_3 e_{21} & +\beta_4 e_{41} & = b_1 \\
           &           & +\beta_3 e_{22} & +\beta_4 e_{42} & = b_2 \\
  2\beta_1 & +3\beta_2 & +\beta_3 e_{23} & +\beta_4 e_{43} & = b_3 \\
           &           & +\beta_3 e_{24} & +\beta_4 e_{44} & = b_4 \\
\end{array}
\right. 
$$
D\'emonstration de P1\\

Posons
$$
\left\{ 
\begin{array}{l}
   \beta_1 = 0 \\
   \beta_2 = 0 \\
   \beta_3 = \lambda_2 \\
   \beta_4 = \lambda_4 \\
\end{array}
\right. 
$$


et P2:
$$
\left\{ 
\begin{array}{l l l l l}
  \beta_1  & +2\beta_2 & +\beta_3 e_{21} & +\beta_4 e_{41} & = 0 \\
           &           & +\beta_3 e_{22} & +\beta_4 e_{42} & = 0 \\
  2\beta_1 & +3\beta_2 & +\beta_3 e_{23} & +\beta_4 e_{43} & = 0 \\
             &             & +\beta_3 e_{24} & +\beta_4 e_{44} & = 0 \\
\end{array}
\right. 
\implies
\beta_1=\beta_2=\beta_3=\beta_4=0
$$
D\'emonstration de P2\\


$$L4 \leftarrow e_{22}L4-e_{24}L2:\,  \beta_4 (e_{22}e_{44}-e_{24}e_{42}) = 0$$

\subsection*{Exercice 1.8}

\subsection*{Exercice 1.9}
la famille $(v_1, v_2, v_3)$ est une base canonique de $\R^3$ donc
$$ \forall a \in \R^3, \exists \lambda_1,\lambda_2,\lambda_3 \in \R,\,
\lambda_1 v_{1} +\lambda_2 v_{2} +\lambda_3 v_{3} = a 
$$
et
$$
\lambda_1 v_{1} +\lambda_2 v_{2} +\lambda_3 v_{3} = 0
\implies
\lambda_1=\lambda_2=\lambda_3=0
$$

Pour que $(u_1, u_2, u_3)$ soit une base canonique de $\R^3$, il faut montrer:\\
P1:
$$ \forall b \in \R^3, \exists \beta_1,\beta_2,\beta_3 \in \R,\,
\beta_1 v_{1} + \beta_2 (v_{1} + v_{2}) +\beta_3 (v_{1} + v_{2} + v_{3}) = b 
$$

D\'emonstration de P1.
$$ \forall b \in \R^3, \exists \beta_1,\beta_2,\beta_3 \in \R,\,
\beta_1 v_{1} + \beta_2 v_{1} + \beta_2 v_{2} +\beta_3 v_{1} + \beta_3 v_{2} + \beta_3 v_{3} = b
$$
$$ \forall b \in \R^3, \exists \beta_1,\beta_2,\beta_3 \in \R,\,
v_{1} (\beta_1 + \beta_2 + \beta_3) + v_{2} (\beta_2 + \beta_3) + \beta_3 v_{3} = b
$$
$$
\left\{ 
\begin{array}{l}
   \beta_1 + \beta_2 + \beta_3 = \lambda_1 \\
   \beta_2 + \beta_3 = \lambda_2 \\
   \beta_3 = \lambda_3 \\
\end{array}
\right. 
$$

$$
\left\{ 
\begin{array}{l}
   \beta_1 = \lambda_1 - \lambda_2 \\
   \beta_2 = \lambda_2 - \lambda_3 \\
   \beta_3 = \lambda_3 \\
\end{array}
\right. 
$$
Propri\'et\'e v\'erifi\'ee.\\


et P2:
$$
\beta_1 v_{1} +\beta_2 (v_{1} + v_{2}) + \beta_3 (v_{1} + v_{2} + v_{3}) = 0
\implies
\beta_1=\beta_2=\beta_3=0
$$

D\'emonstration de P2.
$$
v_{1} (\beta_1 + \beta_2 + \beta_3) + v_{2} (\beta_2 + \beta_3) + \beta_3 v_{3} = 0
\implies
\beta_1=\beta_2=\beta_3=0
$$

Posons
$$
\left\{ 
\begin{array}{l}
   \beta_1 = \lambda_1 - \lambda_2 \\
   \beta_2 = \lambda_2 - \lambda_3 \\
   \beta_3 = \lambda_3 \\
\end{array}
\right.
$$

$$ 
v_{1} (\lambda_1 - \lambda_2 + \lambda_2 - \lambda_3 + \lambda_3) + v_{2} (\lambda_2 - \lambda_3 + \lambda_3) + \lambda_3 v_{3} = 0
$$

Par hypoth\`ese
$$ 
v_{1} \lambda_1 + v_{2} \lambda_2 + \lambda_3 v_{3} = 0 \implies \lambda_1=\lambda_2=\lambda_3=0
$$
Donc
$$
\left\{ 
\begin{array}{l}
   \beta_1 = 0 - 0 \\
   \beta_2 = 0 - 0 \\
   \beta_3 = 0 \\
\end{array}
\right.
$$
Propri\'et\'e v\'erifi\'ee.\\

$(u_1, u_2, u_3)$ est une base canonique de $\R^3$.

Les coordonn\'ees du vecteur $v=2v_1-v_2+3v_3 = (2-(-1))u_1 + ((-1)-3)u_2 + 3u_3 = 3u_1 -4 u_2 + 3u_3$ 

\subsection*{Exercice 1.10}
Deux vecteurs sont colin\'enaires si $u_1 = \lambda u_2$
$$
\left\{ 
\begin{array}{l}
   1 = 0 \lambda\\
   -1 = \lambda \\
   1 = -\lambda \\
   -1 = \lambda \\
\end{array}
\right.
$$

De L1, il n'existe pas de $\lambda$, donc les vecteurs ne sont pas colin\'eaires.

$$
\beta_1 u_{1} +\beta_2 u_{2} + \beta_3 e_{3} + \beta_4 e_{4} = 0
\implies
\beta_1=\beta_2=\beta_3\beta_4=0
$$
$$
\left\{ 
\begin{array}{l l l l l}
  \beta_1  &          & +\beta_3 e_{31} & +\beta_4 e_{41} & = 0 \\
  -\beta_1 & +\beta_2 & +\beta_3 e_{32} & +\beta_4 e_{42} & = 0 \\
  \beta_1  & -\beta_2 & +\beta_3 e_{33} & +\beta_4 e_{43} & = 0 \\
  -\beta_1 & +\beta_2 & +\beta_3 e_{34} & +\beta_4 e_{44} & = 0 \\
\end{array}
\right. 
\implies
\beta_1=\beta_2=\beta_3=\beta_4=0
$$

$$L2 \leftarrow L2+L3:\, \beta_3(e_{32}+e_{33}) + \beta_4(e_{42}+e_{43}) = 0$$
$$L3 \leftarrow L3+L4:\, \beta_3(e_{33}+e_{34}) + \beta_4(e_{43}+e_{44}) = 0$$
$$L5 \leftarrow (e_{33}+e_{34})L2 - (e_{32}+e_{33})L3:\, \beta_4 ((e_{33}+e_{34})(e_{42}+e_{43}) - (e_{32}+e_{33})(e_{43}+e_{44})) = 0$$

Pour avoir $\beta_4 = 0$ il faut $(e_{33}+e_{34})(e_{42}+e_{43}) - (e_{32}+e_{33})(e_{43}+e_{44})\neq 0$.\\
Pour avoir $\beta_3 = 0$ lorsque $\beta_4 = 0$, il faut $(e_{32}+e_{33}) \neq 0$ et $(e_{33}+e_{34}) \neq 0$.
De $L1$, on a $\beta_1 = 0$ lorsque $\beta_3 = \beta_4 = 0$.\\
De $L2,L3,L4$, on a $\beta_2 = 0$ lorsque $\beta_1 = \beta_3 = \beta_4 = 0$.

Donc prenons $e3 = (17,1,1,1)$ et $e4=(7,1,1,2)$.

Il faut v\'erifier qu'aucune des 4 vecteurs n'est colin\'eaire: 
\begin{itemize}
\item $e_2$ avec les 3 autres, vrai car $e_{22} = 0$
\item $e_1$ et $e_3$ vrai car de $e_{11}$, il faudrait $\lambda=7$ qui est faux pour $e_{12}$
\item $e_1$ et $e_4$ vrai car de $e_{11}$, il faudrait $\lambda=17$ qui est faux pour $e_{12}$
\item $e_3$ et $e_4$ vrai car $e_{31}$ et $e_{41}$ sont premiers entre eux.
\end{itemize}

\subsection*{Exercice 1.11}
On a 
\begin{itemize}
\item $P_1(x) = (x-2)(x-3) = x^2 -5x + 6$, posons $v_1 = (1,-5,6)$
\item $P_2(x) = (x-1)(x-3) = x^2 -4x + 3$, posons $v_2 = (1,-4,3)$
\item $P_3(x) = (x-1)(x-2) = x^2 -3x + 2$, posons $v_3 = (1,-3,2)$
\end{itemize}

Montrons que
$$ \forall a \in \mathcal{P}_2, \exists (\lambda_1,\lambda_2,\lambda_3) \in \R^3,
\lambda_1 v_1 + \lambda_2 v_2 + \lambda_3 v_3 = a
$$
$$ \forall a \in \mathcal{P}_2, \exists (\lambda_1,\lambda_2,\lambda_3) \in \R^3,
\lambda_1 (x^2 -5x + 6) + \lambda_2 (x^2 -4x + 3) + \lambda_3 (x^2 -3x + 2) = \beta_1 x^2 + \beta_2 x + \beta_3
$$
$$ \forall a \in \mathcal{P}_2, \exists (\lambda_1,\lambda_2,\lambda_3) \in \R^3,
x^2 (\lambda_1 + \lambda_2 + \lambda_3) + x (-5\lambda_1 -4\lambda_2-3\lambda_3) + 6\lambda_1 + 3\lambda_2 + 2\lambda_3 = \beta_1 x^2 + \beta_2 x + \beta_3
$$
$$
\left\{ 
\begin{array}{l}
\beta_1 = \lambda_1 + \lambda_2 + \lambda_3 \\
\beta_2 = -5\lambda_1 -4\lambda_2-3\lambda_3 \\ 
\beta_3 = 6\lambda_1 + 3\lambda_2 + 2\lambda_3 \\
\end{array}
\right. 
$$
Donc
$$
\left\{ 
\begin{array}{l}
3\lambda_1 = -\beta_1 - \beta_2 - \beta_3 \\
\lambda_2 = -4\beta_1 - 2\beta_2 - \beta_3 \\
2\lambda_3 = 9\beta_1 + 3\beta_2 + 1\beta_3 \\
\end{array}
\right. 
$$
La propri\'et\'e est vraie.


Montrons que la famille $(v_1,v_2,v_3)$ est libre.
$$
\lambda_1 v_{1} +\lambda_2 v_{2} +\lambda_3 v_{3} = 0
\implies
\lambda_1=\lambda_2=\lambda_3=0
$$
$$
\left\{ 
\begin{array}{l l l l}
\lambda_1 & -5\lambda_2 & +6\lambda_3 & = 0 \\
\lambda_1 & -4\lambda_2 & +3\lambda_3 & = 0 \\
\lambda_1 & -3\lambda_2 & +2\lambda_3 & = 0 \\
\end{array}
\right. 
$$

$$L_2 \leftarrow L2-L1:\, \lambda_2 -3\lambda_3 = 0$$
$$L_3 \leftarrow L3-L1:\, 2\lambda_2 -4\lambda_3 = 0$$
$$L_4 \leftarrow L3-2L2:\, 2\lambda_3 = 0$$
Donc
$$
\lambda_1=\lambda_2=\lambda_3=0
$$

La famille $(v_1,v_2,v_3)$ est une base de l'espace $\mathcal{P}_2$.



\end{document}

