\documentclass[]{book}

%These tell TeX which packages to use.
\usepackage{array,epsfig}
\usepackage{amsmath}
\usepackage{amsfonts}
\usepackage{amssymb}
\usepackage{amsxtra}
\usepackage{amsthm}
\usepackage{mathrsfs}
\usepackage{color}
\usepackage{pgfplots}

%Here I define some theorem styles and shortcut commands for symbols I use often
\theoremstyle{definition}
\newtheorem{defn}{Definition}
\newtheorem{thm}{Theorem}
\newtheorem{cor}{Corollary}
\newtheorem*{rmk}{Remark}
\newtheorem{lem}{Lemma}
\newtheorem*{joke}{Joke}
\newtheorem{ex}{Example}
\newtheorem*{soln}{Solution}
\newtheorem{prop}{Proposition}

\newcommand{\lra}{\longrightarrow}
\newcommand{\ra}{\rightarrow}
\newcommand{\surj}{\twoheadrightarrow}
\newcommand{\graph}{\mathrm{graph}}
\newcommand{\bb}[1]{\mathbb{#1}}
\newcommand{\Z}{\bb{Z}}
\newcommand{\Q}{\bb{Q}}
\newcommand{\R}{\bb{R}}
\newcommand{\C}{\bb{C}}
\newcommand{\N}{\bb{N}}
\newcommand{\M}{\mathbf{M}}
\newcommand{\m}{\mathbf{m}}
\newcommand{\MM}{\mathscr{M}}
\newcommand{\HH}{\mathscr{H}}
\newcommand{\Om}{\Omega}
\newcommand{\Ho}{\in\HH(\Om)}
\newcommand{\bd}{\partial}
\newcommand{\del}{\partial}
\newcommand{\bardel}{\overline\partial}
\newcommand{\textdf}[1]{\textbf{\textsf{#1}}\index{#1}}
\newcommand{\img}{\mathrm{img}}
\newcommand{\ip}[2]{\left\langle{#1},{#2}\right\rangle}
\newcommand{\inter}[1]{\mathrm{int}{#1}}
\newcommand{\exter}[1]{\mathrm{ext}{#1}}
\newcommand{\cl}[1]{\mathrm{cl}{#1}}
\newcommand{\ds}{\displaystyle}
\newcommand{\vol}{\mathrm{vol}}
\newcommand{\cnt}{\mathrm{ct}}
\newcommand{\osc}{\mathrm{osc}}
\newcommand{\LL}{\mathbf{L}}
\newcommand{\UU}{\mathbf{U}}
\newcommand{\support}{\mathrm{support}}
\newcommand{\AND}{\;\wedge\;}
\newcommand{\OR}{\;\vee\;}
\newcommand{\Oset}{\varnothing}
\newcommand{\st}{\ni}
\newcommand{\wh}{\widehat}

%Pagination stuff.
\setlength{\topmargin}{-.3 in}
\setlength{\oddsidemargin}{0in}
\setlength{\evensidemargin}{0in}
\setlength{\textheight}{9.in}
\setlength{\textwidth}{6.5in}
\pagestyle{empty}



\begin{document}

\subsection*{Rappel de cours}

\begin{itemize}
\item 
\end{itemize}


\subsection*{Exercice 1.1}
$$
\left\{ 
\begin{array}{l l l l l}
  x & +y & -z & -t & = 0\\
  x & -y & +z & -t & = 0\\
  x &    &    & -t & = 0\\
    & y  & -z &    & = 0\\
\end{array}
\right. 
$$

C'est un syst\`eme d'\'equations homog\`ene de rang 4, \`a 4 inconnues. Aucune ligne nulle. Les inconnues principales sont x et y. Les inconnues secondaires sont z et t.


\subsection*{Exercice 1.2}
$$(S_0)
\left\{ 
\begin{array}{l l l l l}
  x & -3y &     & = a_1 & [1]\\
    & 3y  & -6z & = a_2 & [2]\\
  x &     & -6z & = a_3 & [3]\\
\end{array}
\right. 
$$

Calculer [1]+[2], $x - 6z = a_1+a_2$, qui est \'egale \`a l'\'equation [3]. Donc $a_1+a_2 = a_3$.\\
Ou calculer [1]-[3], $-3y +6z = a_1-a_3$, qui est \'egale \`a la n\'egation de l'\'equation [2]. Donc $a_2 = a_3-a_1$.\\

$$(S_0)
\left\{ 
\begin{array}{l l l l l}
  x & -3y &     & = 1 & [1]\\
    & 3y  & -6z & = 1 & [2]\\
  x &     & -6z & = 2 & [3]\\
\end{array}
\right. 
$$

Lorsque $(a_1,a_2,a_3) = (1,1,2)$, le syst\`me est compatible car $2 = 1+1$. 
La solution du syst\`eme est $(x,y,z)=(a,\frac{a-1}{3},\frac{a-2}{6})$.\\

Lorsque $(a_1,a_2,a_3) = (0,0,0)$, le syst\`me est compatible. La solution du syst\`eme est $(x,y,z)=(a,\frac{a}{3},\frac{a}{6})$.\\


\subsection*{Exercice 1.3.a}
La famille $((1, 2, 3, 0), (3, 1, 2, 0), (2, 3, 1, 0), (3, 2, 1, 0))$ engendre $\R^4$ si

$$ \forall a \in \R^4, \exists (\lambda_1,\lambda_2,\lambda_3,\lambda_4) \in \R^4,
\left\{ 
\begin{array}{l l l l l}
  1\lambda_1 & +3\lambda_2 & +2\lambda_3 & +3\lambda_4  & = a_1 \\
  1\lambda_1 & +3\lambda_2 & +2\lambda_3 & +3\lambda_4  & = a_2 \\
  1\lambda_1 & +3\lambda_2 & +2\lambda_3 & +3\lambda_4  & = a_3 \\
  0\lambda_1 & +0\lambda_2 & +0\lambda_3 & +0\lambda_4  & = a_4 \\
\end{array}
\right. 
$$

Le vecteur $(a_1, a_2, a_3, 1) \in \R^4$ mais ne peux pas \^etre g\'en\'e\'re par la famille. Donc, cette famille n'engendre pas l'espace vectoriel $\R^4$.

\subsection*{Exercice 1.3.b}
La famille $((1, 2, 3), (1, 3, 2), (2, 1, 3), (2, 3, 1))$ est libre si
$$
\left\{ 
\begin{array}{l l l l}
  1\lambda_1 & +2\lambda_2 & +3\lambda_3   & = 0 \\
  1\lambda_1 & +3\lambda_2 & +2\lambda_3   & = 0 \\
  2\lambda_1 & +1\lambda_2 & +3\lambda_3   & = 0 \\
  2\lambda_1 & +3\lambda_2 & +1\lambda_3   & = 0 \\
\end{array}
\right. 
\implies
\lambda_1=\lambda_2=\lambda_3=0
$$

$$L_2 \leftarrow L_2-L_1:\, \lambda_2 - \lambda_3 = 0$$
$$L_3 \leftarrow L_3-2L_1:\, -3\lambda_2 - 3\lambda_3 = 0$$
$$L_4 \leftarrow L_4-2L_1:\, -\lambda_2 - 5\lambda_3 = 0$$
$$L_4 \leftarrow L_4+L_1:\, -6\lambda_3 = 0$$

Donc $\lambda_3 = \lambda_2 = \lambda_1 = 0$. La famille est libre.

\subsection*{Exercice 1.4}
La famille s'\'ecrit $\mathcal{F} = ((1,1,0),(0,1,1),(0,1,-1))$.

La famille engendre-t-elle $\mathcal{P}_2$?
bd
$$ \forall a \in \mathcal{P}^2, \exists (\lambda_1,\lambda_2,\lambda_3) \in \R^3,
\left\{ 
\begin{array}{l l l l}
  \lambda_1 & +\lambda_2 &               & = a_1 \\
            & \lambda_2  & +\lambda_3   & = a_2 \\
            & \lambda_2  & -\lambda_3   & = a_3 \\
\end{array}
\right. 
$$
$$L_3 \leftarrow L_3-L_2:\, -2\lambda_3 = a_3-a_2$$
$$L_2:\, 2\lambda_2 = a_2+a_3$$
$$L_1:\, 2\lambda_1 = 2a_1-a_2-a_3$$

Donc $\lambda_1 = \frac{2a_1-a_2-a_3}{2}$, $\lambda_1 = \frac{a_2+a_3}{2}$, $\lambda_1 = \frac{a_3-a_2}{2}$. La famille $\mathcal{F}$ engendre l'espace vectoriel $\mathcal{P}_2$.\\

\medskip
La famille est-elle libre?
$$
\left\{ 
\begin{array}{l l l l}
  \lambda_1 & +\lambda_2 &               & = 0 \\
            & \lambda_2  & +\lambda_3   & = 0 \\
            & \lambda_2  & -\lambda_3   & = 0 \\
\end{array}
\right. 
$$

$$L_3 \leftarrow L_3-L_2:\, -2\lambda_3 = 0$$

Donc $\lambda_3 = \lambda_2 = \lambda_1 = 0$. La famille $\mathcal{F}$ est libre.

\end{document}

