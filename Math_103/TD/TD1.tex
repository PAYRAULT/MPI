\documentclass[]{book}

%These tell TeX which packages to use.
\usepackage{array,epsfig}
\usepackage{amsmath}
\usepackage{amsfonts}
\usepackage{amssymb}
\usepackage{amsxtra}
\usepackage{amsthm}
\usepackage{mathrsfs}
\usepackage{color}

%Here I define some theorem styles and shortcut commands for symbols I use often
\theoremstyle{definition}
\newtheorem{defn}{Definition}
\newtheorem{thm}{Theorem}
\newtheorem{cor}{Corollary}
\newtheorem*{rmk}{Remark}
\newtheorem{lem}{Lemma}
\newtheorem*{joke}{Joke}
\newtheorem{ex}{Example}
\newtheorem*{soln}{Solution}
\newtheorem{prop}{Proposition}

\newcommand{\lra}{\longrightarrow}
\newcommand{\ra}{\rightarrow}
\newcommand{\surj}{\twoheadrightarrow}
\newcommand{\graph}{\mathrm{graph}}
\newcommand{\bb}[1]{\mathbb{#1}}
\newcommand{\Z}{\bb{Z}}
\newcommand{\Q}{\bb{Q}}
\newcommand{\R}{\bb{R}}
\newcommand{\C}{\bb{C}}
\newcommand{\N}{\bb{N}}
\newcommand{\M}{\mathbf{M}}
\newcommand{\m}{\mathbf{m}}
\newcommand{\MM}{\mathscr{M}}
\newcommand{\HH}{\mathscr{H}}
\newcommand{\Om}{\Omega}
\newcommand{\Ho}{\in\HH(\Om)}
\newcommand{\bd}{\partial}
\newcommand{\del}{\partial}
\newcommand{\bardel}{\overline\partial}
\newcommand{\textdf}[1]{\textbf{\textsf{#1}}\index{#1}}
\newcommand{\img}{\mathrm{img}}
\newcommand{\ip}[2]{\left\langle{#1},{#2}\right\rangle}
\newcommand{\inter}[1]{\mathrm{int}{#1}}
\newcommand{\exter}[1]{\mathrm{ext}{#1}}
\newcommand{\cl}[1]{\mathrm{cl}{#1}}
\newcommand{\ds}{\displaystyle}
\newcommand{\vol}{\mathrm{vol}}
\newcommand{\cnt}{\mathrm{ct}}
\newcommand{\osc}{\mathrm{osc}}
\newcommand{\LL}{\mathbf{L}}
\newcommand{\UU}{\mathbf{U}}
\newcommand{\support}{\mathrm{support}}
\newcommand{\AND}{\;\wedge\;}
\newcommand{\OR}{\;\vee\;}
\newcommand{\Oset}{\varnothing}
\newcommand{\st}{\ni}
\newcommand{\wh}{\widehat}

%Pagination stuff.
\setlength{\topmargin}{-.3 in}
\setlength{\oddsidemargin}{0in}
\setlength{\evensidemargin}{0in}
\setlength{\textheight}{9.in}
\setlength{\textwidth}{6.5in}
\pagestyle{empty}



\begin{document}

Rappel de cours: 
\begin{itemize}
\item $\int_{a}^{b}{f(u(x))u'(x)\,dx} = \int_{x(a)}^{u(b)}{u\,du}$
\end{itemize}

 

\subsection*{Exercice 1}
\subsubsection*{Exercice 1.a}

$$\int tan(x)\,dx = \int \frac{sin(x)}{cos(x)}\,dx$$
Soit $u(x) = cos(x)$, $u'(x)=-sin(x)$ et $f(u) = -\frac{1}{u}$ donc
$$\int{\frac{sin(x)}{cos(x)}\,dx} = \int{f(cos(x))sin(x)\,dx} = \int{ -\frac{1}{u}\,du} = -ln(u) = -ln(cos(x))$$

\subsubsection*{Exercice 1.b}
$$\int{\frac{ln(x+1)}{x+1}}\,dx$$
Soit $u(x) = ln(x+1)$, $u'(x)=\frac{1}{x+1}$ et $f(u) = u$ donc
$$\int{\frac{ln(x+1)}{x+1}\,dx} = \int{f(ln(x+1)).\frac{1}{x+1}\,dx} = \int u\,du = \frac{u^2}{2} = \frac{ln^2(x+1)}{2}$$

\subsubsection*{Exercice 1.c}
$$\int{\frac{x}{\sqrt{2x^2+3}}}$$
Soit $u(x)=2x^2+3$, $u'(x)=4x$ et $f(u)=\frac{1}{4\sqrt{u}}$ donc
$$\int{\frac{x}{\sqrt{2x^2+3}}\,dx} = \int{f(2x^2+3).4x\,dx} = \int{\frac{1}{4\sqrt{u}}\,du} = \frac{1}{4}\int\frac{1}{\sqrt{u}\,du} = \frac{\sqrt{u}}{2} = \frac{\sqrt{2x^2+3}}{2}$$

\subsubsection*{Exercice 1.d}
$$\int{\frac{cos(x)}{sin^2(x)}}\,dx$$
Soit $u(x)=sin(x)$, $u'(x)=cos(x)$ et $f(u)=\frac{1}{u^2}$ donc
$$\int{\frac{cos(x)}{sin^2(x)}\,dx} = \int{f(sin(x)).cos(x)\,dx} = \int{\frac{1}{u^2}\,du} = -\frac{1}{u} = -\frac{1}{sin(x)}$$

\subsubsection*{Exercice 1.e}
$$\int {x\sqrt{x^2+1}\,dx}$$
Soit $u(x) = x^2+1$, $u'(x)=2x$ et $f(u)=\frac{\sqrt(u)}{2}$ donc
$$\int{x\sqrt{x^2+1}\,dx} = \int{f(x^2+1).2x\,dx} = \int{\frac{\sqrt(u)}{2}\,du} = \frac{1}{2}\int{\sqrt(u)\,du} = \frac{1}{2}\frac{2}{3}u^{\frac{3}{2}} = \frac{1}{3}u^{\frac{3}{2}} = \frac{(x^2+1)^{\frac{3}{2}}}{3}$$

\subsubsection*{Exercice 1.f}
$$\int{\frac{1}{xln(x)}\,dx}$$
Soit $u(x)=ln(x)$, $u'(x)=\frac{1}{x}$ et $f(u) = \frac{1}{u}$ donc
$$\int{\frac{1}{xln(x)}\,dx} = \int{f(ln(x)).\frac{1}{x}\,dx} = \int{\frac{1}{u}\,du} = ln(|u|) = ln(|ln(x)|)$$

\subsubsection*{Exercice 1.g}
$$\int{sin^2(x)cos(x)\,dx}$$
Soit $u(x)=sin(x)$, $u'(x)=cos(x)$ et $f(u)=u^2$ donc
$$\int{sin^2(x)cos(x)\,dx} = \int{f(sin(x)).cos(x)\,dx} = \int{u^2\,du} = \frac{u^3}{3} = \frac{sin^3(x)}{3}$$

\subsubsection*{Exercice 1.h}
$$\int{e^{sin(x)}cos(x)\,dx}$$
Soit $u(x)=sin(x)$, $u'(x)=cos(x)$ et $f(u)=e^u$ donc
$$\int{e^{sin(x)}cos(x)\,dx} = \int{f(sin(x)).cos(x)\,dx} = \int{e^u\,du} = e^u = e^{sin(x)}$$
 

\subsubsection*{Exercice 1.i}
$$\int{\frac{ln^2(x)}{x}\,dx}$$
Soit $u(x) = ln(x)$, $u'(x)=\frac{1}{x}$ et $f(u)=u^2$ donc
$$\int{\frac{ln^2(x)}{x}\,dx} = \int{f(ln(x)).\frac{1}{x}\,dx} = \int{u^2\,du} = \frac{u^3}{3} = \frac{ln^3(x)}{3}$$


\subsection*{Exercice 2}
\subsubsection*{Exercice 2.1}
$t = u(x) = tan(\frac{x}{2})$ donc $x = u^{-1}(t) = 2arctan(t)$ et $(u^{-1})'(t) = \frac{dx}{dt} = \frac{2}{1+t^2}$.\\
On a 
$$sin(x)=\frac{2tan(\frac{x}{2})}{1+tan^2(\frac{x}{2})}$$ 
$$1-sin(x) = 1 - \frac{2tan(\frac{x}{2})}{1+tan^2(\frac{x}{2})}
= \frac{tan^2(\frac{x}{2})-2tan^2(\frac{x}{2})+1}{1+tan^2(\frac{x}{2})} = \frac{(tan(\frac{x}{2})-1)^2}{1+tan^2(\frac{x}{2})}$$

Donc 
$$\frac{1}{1-sin(x)} = \frac{1+tan^2(\frac{x}{2})}{(tan(\frac{x}{2})-1)^2}$$

Et
$$\int_0^{\pi/3}{\frac{1}{1-sin(x)}\,dx} = \int_0^{\pi/3}{\frac{1+tan^2(\frac{x}{2})}{(tan(\frac{x}{2})-1)^2}\,dx}$$
$$\int_{tan(0)}^{tan(\pi/6)}{\frac{1+t^2}{(t-1)^2}.\frac{2}{1+t^2}\,dt} = \int_{tan(0)}^{tan(\pi/6)}{\frac{2}{(t-1)^2}\,dt}$$

Soit $u(x)=x-1$, $u'(x)=1$ et $f(u) = \frac{1}{u^2}$ donc
$$\int_{tan(0)}^{tan(\pi/6)}{\frac{2}{(t-1)^2}\,dt} = 2\int_{tan(0)}^{tan(\pi/6)}{f(t-1).1\,dt}=2\int_{tan(0)-1}^{tan(\pi/6)-1}{\frac{1}{u^2}\,du} = [\frac{2}{u}]_{tan(0)-1}^{tan(\pi/6)-1} = -2 - \frac{2}{tan(\pi/6)-1}$$


\subsubsection*{Exercice 2.2.a}
$t = u(x) = tan(\frac{x}{2})$ donc $x = u^{-1}(t) = 2arctan(t)$ et $(u^{-1})'(t) = \frac{dx}{dt} = \frac{2}{1+t^2}$.\\
On a 
$$cos(x)=\frac{1-tan^2(\frac{x}{2})}{1+tan^2(\frac{x}{2})}$$
$$5-3cos(x)=5-3.\frac{1-tan^2(\frac{x}{2})}{1+tan^2(\frac{x}{2})} = \frac{2+8tan^2(\frac{x}{2})}{1+tan^2(\frac{x}{2})}$$
Donc
$$\frac{1}{5-3cos(x)} = \frac{1+tan^2(\frac{x}{2})}{2(1+4tan^2(\frac{x}{2}))}$$

Et
$$\int_{0}^{\pi/2}{\frac{1}{5-3cos(x)}\,dx} = \int_{0}^{\pi/2}{\frac{1+tan^2(\frac{x}{2})}{2(1+4tan^2(\frac{x}{2}))}\,dx}$$
$$\int_{tan(0)}^{tan(\pi/4)}{\frac{1+t^2}{2(1+4t^2)}.\frac{2}{1+t^2}\,dt} = \int_{tan(0)}^{tan(\pi/4)}{\frac{1}{1+4t^2}\,dt}$$

Soit $u(t)=2t$, $u'(t)=2$ et $f(u)=\frac{1}{2(u^2+1)}$ donc
$$\int_{tan(0)}^{tan(\pi/4)}{\frac{1}{1+4t^2}\,dt} = \int_{tan(0)}^{tan(\pi/4)}{f(2t).2\,dt} = \int_{2tan(0)}^{2tan(\pi/4)}{\frac{1}{2(u^2+1)}\,du} = \frac{1}{2}[arctan(u)]_{2tan(0)}^{2tan(\pi/4)}$$


\subsubsection*{Exercice 2.2.b}
$t = u(x) = tan(\frac{x}{2})$ donc $x = u^{-1}(t) = 2arctan(t)$ et $(u^{-1})'(t) = \frac{dx}{dt} = \frac{2}{1+t^2}$.\\
On a 
$$cos(x)=\frac{1-tan^2(\frac{x}{2})}{1+tan^2(\frac{x}{2})}$$
$$1+cos(x)=1+\frac{1-tan^2(\frac{x}{2})}{1+tan^2(\frac{x}{2})} = \frac{2}{1+tan^2(\frac{x}{2})}$$
$$(1+cos(x))^2 = \frac{4}{(1+tan^2(\frac{x}{2}))^2}$$
Donc
$$\frac{1}{(1+cos(x))^2}=\frac{(1+tan^2(\frac{x}{2}))^2}{4}$$

Et
$$\int_{0}^{\pi/2}{\frac{1}{(1+cos(x))^2}\,dx} = \int_{0}^{\pi/2}{\frac{(1+tan^2(\frac{x}{2}))^2}{4}\,dx}$$
$$\int_{tan(0)}^{tan(\pi/4)}{\frac{(1+t^2)^2}{4}.\frac{2}{1+t^2}\,dt} = \int_{tan(0)}^{tan(\pi/4)}{\frac{1}{2}(1+t^2)\,dt} = \frac{1}{2}[t+\frac{t^3}{3}]_{tan(0)}^{tan(\pi/4)}$$

\subsubsection*{Exercice 2.3}
Trop complexe.

\subsection*{Exercice 3}
\subsubsection*{Exercice 3.a}
$$\int{\arcsin(x)\,dx}$$
Par partie avec
$$
\begin{array}{l l}
f(x) = \arcsin(x) & f'(x)=\frac{1}{\sqrt{1-x^2}} \\
g'(x)= 1 & g(x) = x \\
\end{array}
$$

Donc
$$\int{\arcsin(x)\,dx} = x\arcsin(x) - \int{\frac{x}{\sqrt{1-x^2}}\,dx}$$

Soit $u(x)=1-x^2$, $u'(x)=-2x$ et $f(u)=-\frac{1}{2\sqrt{u}}$
Donc
$$\int{\frac{x}{\sqrt{1-x^2}}\,dx} = \int{f(1-x^2).2x\,dx} = \int{-\frac{1}{2\sqrt{u}}\,du} = -\frac{1}{2}\int{\frac{1}{\sqrt{u}}\,du} = -\frac{1}{2}.2\sqrt{u} = -\sqrt{u}$$

Donc
$$\int{\arcsin(x)\,dx} = x\arcsin(x) + \sqrt{1-x^2}$$

\subsubsection*{Exercice 3.b}
$$\int{x^n\ln(x)\,dx}$$
Par partie avec
$$
\begin{array}{l l}
f(x) = \ln(x) & f'(x)=\frac{1}{x} \\
g'(x)= x^n & g(x) = \frac{x^{n+1}}{n+1} \\
\end{array}
$$

Donc
$$\int{x^n\ln(x)\,dx} = \frac{x^{n+1}\ln(x)}{n+1} - \int{\frac{1}{x}.\frac{x^{n+1}}{n+1}\,dx} = \frac{x^{n+1}\ln(x)}{n+1} - \frac{1}{n+1}\int{x^n\,dx} = \frac{x^{n+1}\ln(x)}{n+1} - \frac{x^{n+1}}{(n+1)^2}$$


\subsubsection*{Exercice 3.c}
$$\int{x\arcsin(x)\,dx}$$
Par partie avec
$$
\begin{array}{l l}
f(x) = \arcsin(x) & f'(x)=\frac{1}{\sqrt{1-x^2}} \\
g'(x)= x & g(x) = \frac{x^{2}}{2} \\
\end{array}
$$

Donc
$$\int{x\arcsin(x)\,dx} = \frac{1}{2}x^{2}.\arcsin(x) - \int{\frac{x^2}{2\sqrt{1-x^2}}\,dx}$$

Soit $x = \sin(t)$, $\frac{dx}{dt}=\cos(t)$.
Donc
$$\int{\frac{x^2}{2\sqrt{1-x^2}}\,dx} = \int{\frac{\sin^2(t)\cos(t)}{2\sqrt{1-\sin^2(t)}}\,dt} = \int{\frac{\sin^2(t)\cos(t)}{2\sqrt{\cos^2(t)}}\,dt} = \int{\frac{1}{2}\sin^2(t)\,dt}$$

\subsubsection*{Exercice 3.d}
$$\int{\ln(x^2+1)\,dx}$$

Par partie avec
$$
\begin{array}{l l}
f(x) = \ln(x^2+1) & f'(x)=\frac{2x}{1+x^2} \\
g'(x)= 1 & g(x) = x \\
\end{array}
$$

Donc
$$\int{\ln(x^2+1)\,dx} = x\ln(x^2+1) - \int{\frac{2x^2}{1+x^2}\,dx}$$
$$\int{\frac{2x^2}{1+x^2}\,dx} = 2\int{\frac{x^2+1-1}{1+x^2}\,dx} = 2\int{\frac{x^2+1}{1+x^2}-\frac{1}{1+x^2}\,dx}$$
$$=2\int{1\,dx}-2\int{\frac{1}{1+x^2}\,dx} = 2x - 2\arctan(x)$$

Enfin
$$\int{\ln(x^2+1)\,dx} = x\ln(x^2+1) - 2x + 2\arctan(x)$$


QED

\end{document}

