\documentclass[]{book}

%These tell TeX which packages to use.
\usepackage{array,epsfig}
\usepackage{amsmath}
\usepackage{amsfonts}
\usepackage{amssymb}
\usepackage{amsxtra}
\usepackage{amsthm}
\usepackage{mathrsfs}
\usepackage{color}
\usepackage{pgfplots}

%Here I define some theorem styles and shortcut commands for symbols I use often
\theoremstyle{definition}
\newtheorem{defn}{Definition}
\newtheorem{thm}{Theorem}
\newtheorem{cor}{Corollary}
\newtheorem*{rmk}{Remark}
\newtheorem{lem}{Lemma}
\newtheorem*{joke}{Joke}
\newtheorem{ex}{Example}
\newtheorem*{soln}{Solution}
\newtheorem{prop}{Proposition}

\newcommand{\lra}{\longrightarrow}
\newcommand{\ra}{\rightarrow}
\newcommand{\surj}{\twoheadrightarrow}
\newcommand{\graph}{\mathrm{graph}}
\newcommand{\bb}[1]{\mathbb{#1}}
\newcommand{\Z}{\bb{Z}}
\newcommand{\Q}{\bb{Q}}
\newcommand{\R}{\bb{R}}
\newcommand{\C}{\bb{C}}
\newcommand{\N}{\bb{N}}
\newcommand{\M}{\mathbf{M}}
\newcommand{\m}{\mathbf{m}}
\newcommand{\MM}{\mathscr{M}}
\newcommand{\HH}{\mathscr{H}}
\newcommand{\Om}{\Omega}
\newcommand{\Ho}{\in\HH(\Om)}
\newcommand{\bd}{\partial}
\newcommand{\del}{\partial}
\newcommand{\bardel}{\overline\partial}
\newcommand{\textdf}[1]{\textbf{\textsf{#1}}\index{#1}}
\newcommand{\img}{\mathrm{img}}
\newcommand{\ip}[2]{\left\langle{#1},{#2}\right\rangle}
\newcommand{\inter}[1]{\mathrm{int}{#1}}
\newcommand{\exter}[1]{\mathrm{ext}{#1}}
\newcommand{\cl}[1]{\mathrm{cl}{#1}}
\newcommand{\ds}{\displaystyle}
\newcommand{\vol}{\mathrm{vol}}
\newcommand{\cnt}{\mathrm{ct}}
\newcommand{\osc}{\mathrm{osc}}
\newcommand{\LL}{\mathbf{L}}
\newcommand{\UU}{\mathbf{U}}
\newcommand{\support}{\mathrm{support}}
\newcommand{\AND}{\;\wedge\;}
\newcommand{\OR}{\;\vee\;}
\newcommand{\Oset}{\varnothing}
\newcommand{\st}{\ni}
\newcommand{\wh}{\widehat}

%Pagination stuff.
\setlength{\topmargin}{-.3 in}
\setlength{\oddsidemargin}{0in}
\setlength{\evensidemargin}{0in}
\setlength{\textheight}{9.in}
\setlength{\textwidth}{6.5in}
\pagestyle{empty}



\begin{document}

\subsection*{Rappel de cours}

\begin{defn}
La relation $xRy$ est une relation d'\'equivalence sur l'ensemble $E$ ssi:
\begin{itemize}
\item $\forall x \in E, xRx$
\item $\forall x,y \in E, xRy \Leftrightarrow yRx$
\item $\forall x,y,z \in E, xRy \wedge yRz \Leftrightarrow xRz$
\end{itemize}
\end{defn}


\newpage
\subsection*{Exercice 1}
\subsection*{Exercice 1.1}
\begin{itemize}
\item $\forall x \in \N, x\sim x \Leftrightarrow x = 2^kx$ est vrai pour $k=0$
\item $\forall x,y \in \N, (x\sim y \Rightarrow y\sim x) \Leftrightarrow (y=2^{k_1}x  \Rightarrow x = 2^{k_2}y)$ est vrai pour $k_2 = -k_1$
\item $\forall x,y,z \in \N, (x\sim y \wedge y\sim z \Rightarrow x\sim z) \Leftrightarrow (y = 2^{k_1}x \wedge z = 2^{k_2}y \Rightarrow z = 2^{k'}x)$ est vrai pour $k'=k_1+k_2$
\end{itemize}


\subsection*{Exercice 1.2}
Admettons qu'une classe d'\'equivalence a au moins 2 nombres impairs, donc $\exists k \in \N, (2n+1) = 2^k(2m+1)$. La seule valeur de $k$ possible est $k=0$ car pour $k>0$ un cot\'e est impair et l'autre est pair et pour $k<0$, un cot\'e n'est pas un entier. Pour $k=0$ on a $(2n+1)=2^0(2m+1)$, donc $n=m$. Si il y a un nombre impair, il est unique.\\

Chaque nombre impair est dans une classe d'\'equivalence car pour tout $a=2n+1 \in E$, soit $2a \in E$, donc $a \sim 2a$, soit $2a \not\in E$ alors $a \sim a$.\\

L'ensemble $E$ contient $n$ nombres impairs, donc il y a $n$ classes d'\'equivalence.

\subsection*{Exercice 1.3}
Comme $|A| = n+1$ alors il y a au moins deux \'el\'ements de A qui sont dans la m\^eme classe d'\'equivalence (car $E$ contient $n$ classes d'\'equivalence). Si ils sont dans la m\^eme classe alors $a \sim b$ existe.\\

Si on a $a \sim b$, alors $a=2^kb$. Lorsque $k\geq 0$, $a$ est un multiple de $b$, lorsque $k<0$, $b$ est un multiple de $a$.

\subsection*{Exercice 2}
On a $pgcd(a,b) = 1$, Soit $d=pgcd(a+b,a-b)$, donc $a+b=n.d$ et $a-b=n'*d$.
$$2a = (a+b)+(a-b) = n.d+n'.d = d(n+n')$$
$$2b = (a+b)-(a-b) = n.d-n'.d = d(n-n')$$
Donc $d$ divise le $pgcd(2a, 2b)$ et $pgcd(2a, 2b)=2$ car $a$ et $b$ sont premiers entre eux. Il existe que 2 nombres qui divisent 2: 1 ou 2.
$$pgcd(a+b,a-b) = 1 \text{ ou } 2$$ 

\subsection*{Exercice 3}
\'Equations diophantiennes du premier degr\'e.
1. Trouver une solution particuli\`ere de $15x - 22y = 1$; $x=3, y=2$. Donc $15*3-22*2 = 1$. En soustraiant les 2 \'quations on a 
$$15x - 22y - (15*3-22*2) = 0, 15(x-3) - 22(y-2) = 0, 15(x-3) = 22(y-2)$$
2. Les entiers 15 et 22 sont premiers entre eux donc $22|(x-3)$, donc $x= 22k+3$.
$$15(x-3) = 22(y+2), 15(22k+3-3) = 22(y+2), y = 15k+2$$
La solution est $x=22k+3$ et $y=15k+2$.

\subsection*{Exercice 4}
$$15x+24y = 5, 3(5x+8y) = 5$$.
Pas de solution car 5 n'est pas un multiple de 3.

\subsection*{Exercice 5}
Preuve par r\'ecurrence. Soit la suite $a_{n+1} = 10a_n+1$ et $a_0 = 1$. La suite $a_n$ repr\'esente les nombres $1, 11, 111, 111111\ldots1$.
Vrai pour $a_0$ avec $n=1, m=1$. Supposons que $a_n \mod n.m = 0$, quelles sont les conditions pour que $a_{n+1} \mod n.m' = 0$?. On a $a_n \mod nm = 0$ donc $\exists k, a_n=k.n.m$.
$$a_{n+1} \mod n.m' = (10a_n + 1) \mod n.m' = (10k.n.m + 1) \mod n.m'$$
Pour que $a_{n+1} \mod n.m' = 0$? il faut $\exists k', a_{n+1}=k'.n.m'$ donc $k'.n.m' = 10k.n.m + 1$.\\

Sous quelles conditions est-ce vrai?\\
Si $n=2l$ (ie $2|n$), alors $2k'.l.m' = 20k.l.m+1$, il n'existe aucune valeur de $k,m,k',m'$ car un cot\'e est pair et l'autre impair.\\
Si $n=5l$ (ie $5|n$), alors $5k'.l.m' = 50k.l.m+1$, il n'existe aucune valeur de $k,m,k',m'$ car un cot\'e se termine par 5 ou 0 et l'autre par 1.\\


\subsection*{Exercice 6}
Il suffit de trouver tous les entiers $j,m$ qui v\'erifient $31j+12m=208$.\\
Trouver une solution particuli\'ere \`a l'\'equation $31j+12m=1$, soit $j=-5$ et $m=13$. Donc 
$$308 = 308(31.(-5) + 12.13) =  31j + 12m$$
$$31(j+308.5) = 12(13.308-m)$$
Les entiers 12 et 31 sont premiers entre eux donc $12|j+308.5$ et $12k=j+308.5$
$$31(12k-308.5+308.5) = 12(13.308-m), 31.12k=12(13.308-m), 31k = 13.308-m, m=13.308-31k$$
Il faut trouver le $k$ tel que $1\leq m\leq 12$. $k=129$, et $j=8$, $m=5$. Donc il est n\'e le 8 mai.


\subsection*{Exercice 7}
\subsubsection*{Exercice 7.1}
$1995 = 3*5*7*19$ et $2975 = 5^2*7*17$, donc $pgcd(1995,2975) = 5*7 = 35$\\

$2975=1.1995+980$, $1995 = 2.980+35$, $980 = 40.35+0$ donc $pgcd(2975,1995)=pgcd(1995,980)=pgcd(980,35)=35$

\subsubsection*{Exercice 7.2}
$n.k+8 = 2003$ et $n*k' + 27 = 3002$. donc $nk=1995$ et $nk'=2975$ et $pgcd(1975,2975) = 35$.\\
Ceci fait $n.k=35*57$ et $n.k'=35*85$. Donc, la solution est $n=35$.  	
 

\subsection*{Exercice 8}
On cherche $c$ tel que $11c+1 = x^2$. Ceci donne l'\'equation d\'eophantienne de degr\'e 2: $x^2 -11c-1 = 0$.
La solution est $x=22k+21$ et $c=44k^2+84k+40=2(22k^2+42k+20)$. 
Il n'existe pas de nombre premier $c$.

QED

\end{document}

