\documentclass[]{book}

%These tell TeX which packages to use.
\usepackage{array,epsfig}
\usepackage{amsmath}
\usepackage{amsfonts}
\usepackage{amssymb}
\usepackage{amsxtra}
\usepackage{amsthm}
\usepackage{mathrsfs}
\usepackage{color}
\usepackage{pgfplots}

%Here I define some theorem styles and shortcut commands for symbols I use often
\theoremstyle{definition}
\newtheorem{defn}{Definition}
\newtheorem{thm}{Theorem}
\newtheorem{cor}{Corollary}
\newtheorem*{rmk}{Remark}
\newtheorem{lem}{Lemma}
\newtheorem*{joke}{Joke}
\newtheorem{ex}{Example}
\newtheorem*{soln}{Solution}
\newtheorem{prop}{Proposition}

\newcommand{\lra}{\longrightarrow}
\newcommand{\ra}{\rightarrow}
\newcommand{\surj}{\twoheadrightarrow}
\newcommand{\graph}{\mathrm{graph}}
\newcommand{\bb}[1]{\mathbb{#1}}
\newcommand{\Z}{\bb{Z}}
\newcommand{\Q}{\bb{Q}}
\newcommand{\R}{\bb{R}}
\newcommand{\C}{\bb{C}}
\newcommand{\N}{\bb{N}}
\newcommand{\M}{\mathbf{M}}
\newcommand{\m}{\mathbf{m}}
\newcommand{\MM}{\mathscr{M}}
\newcommand{\HH}{\mathscr{H}}
\newcommand{\Om}{\Omega}
\newcommand{\Ho}{\in\HH(\Om)}
\newcommand{\bd}{\partial}
\newcommand{\del}{\partial}
\newcommand{\bardel}{\overline\partial}
\newcommand{\textdf}[1]{\textbf{\textsf{#1}}\index{#1}}
\newcommand{\img}{\mathrm{img}}
\newcommand{\ip}[2]{\left\langle{#1},{#2}\right\rangle}
\newcommand{\inter}[1]{\mathrm{int}{#1}}
\newcommand{\exter}[1]{\mathrm{ext}{#1}}
\newcommand{\cl}[1]{\mathrm{cl}{#1}}
\newcommand{\ds}{\displaystyle}
\newcommand{\vol}{\mathrm{vol}}
\newcommand{\cnt}{\mathrm{ct}}
\newcommand{\osc}{\mathrm{osc}}
\newcommand{\LL}{\mathbf{L}}
\newcommand{\UU}{\mathbf{U}}
\newcommand{\support}{\mathrm{support}}
\newcommand{\AND}{\;\wedge\;}
\newcommand{\OR}{\;\vee\;}
\newcommand{\Oset}{\varnothing}
\newcommand{\st}{\ni}
\newcommand{\wh}{\widehat}

%Pagination stuff.
\setlength{\topmargin}{-.3 in}
\setlength{\oddsidemargin}{0in}
\setlength{\evensidemargin}{0in}
\setlength{\textheight}{9.in}
\setlength{\textwidth}{6.5in}
\pagestyle{empty}



\begin{document}

\subsection*{Rappel de cours}


\newpage
\subsection*{Exercice 2}
\subsubsection*{Exercice 2.1.a}
Fausse, car plus de 7 nombres entre les nombres premiers 191 et 179.

\subsubsection*{Exercice 2.1.b}
Vraie, soit 7 entiers cons\'ecutifs $a, a+1, a+2, \ldots, a+6$. Soit :
\begin{itemize}
\item $a = 6k$, donc c'est un multiple de 6
\item $a = 6k + r, 0 < r < 6$, donc ce n'est pas un multiple de 6. si $r = 1$ alors $a+5 = 6(k+1)$, si $r = 2$ alors $a+4 = 6(k+1)$, \ldots,si $r = 5$ alors $a+1 = 6(k+1)$   
\end{itemize}

Donc il existe toujours un multiple de 6 parmi 7 entiers cons\'ecutifs.


\subsubsection*{Exercice 2.2.a}
Fausse. $a=7$ et $b=5$, premiers entre eux et $a+b=12$ et $a-b=2$ non premiers entre eux.

\subsubsection*{Exercice 2.2.b}
Vraie. Preuve par contracdiction.\\
Supposons que $ab$ et $a+b$ ne sont pas premiers entre eux donc $\exists d > 1, \gcd(ab, a+b) = d$. Donc $d|ab$, supposons que $d|a$, comme $d|a+b$, alors $a = k_1d$ et $a+b = k_2d$, donc $k_1d+b = k_2d$ ce qui montre aue $d|b$. On vient de trouver un $d$ qui divise $a$ et $b$, contredisant qu'ils sont premiers entre eux. Par cons\'equent, Si $a$ et $b$ sont premiers entre eux alors $ab$ et $a+b$ sont premiers entre eux.   

\subsubsection*{Exercice 2.3}
Fausse. Contre-exemple $x=27$ car $27^2 + 1 = 729+1 = 730 = 73*10$. \\
Sinon admettons qu'il existe un $x$ tel que $x^2 \equiv -1 \pmod{73}$. On sait par le petit th\'eor\`eme de Fermat que $x^{72} \equiv 1 \pmod{73}$. Donc $x^{2^{36}}	 \equiv 1 \pmod{73} \implies (-1)^{36} \equiv 1 \pmod{73}$. Ce qui est vrai car $1 \equiv 1 \pmod{73}$. Donc il existe un $x$.
Sinon admettons qu'il existe un $x$ tel que $x^2 \equiv -1 \pmod{73}$. On sait par le le petit th\'eor\`eme de Fermat que $x^72 \equiv 1 \pmod{73}$. Donc $x^{2^36} \equiv 1 \pmod{73} \implies (-1)^{36} \equiv 1 \pmod{73}$. Ce qui est vrai car $1 \equiv 1 \pmod{73}$. Donc il existe un $x$.

\subsection*{Exercice 2.4}
Vraie. Si $x^18 \equiv n \pmod{37}$ alors $x^18 = 37k+n$. Donc $x^36 = x^{18^2} = (37k+n)^2 = 37^2k^2 + 74nk + n^2 = 37(37k^2+2nk) + n^2 = n^2 \pmod{37}$. D'apr\`es le petit th\'eor\`eme de Fermat on a $x^{36} 	 \equiv 1 \pmod{37}$ donc il faut que $n^2 = 1$. Ceci implique $n=1$ ou $n=-1$ donc
$x^18 \equiv 1 \pmod{37}$ ou $x^{18} \equiv -1 \pmod{37}$.

\subsection*{Exercice 4}
\subsubsection*{Exercice 4.1}
$
\begin{array}{|l|l|}
\hline
x & x^2 \pmod{7} \\
\hline
0,7,\ldots,7k & 0 \pmod{7} = 0  \\
\hline
1,8,\ldots,7k+1 & 1 \pmod{7} = 1 \\
\hline
2,9,\ldots,7k+2 & 4 \pmod{7} = 4 \\
\hline
3,10,\ldots,7k+3 & 9 \pmod{7} = 2 \\
\hline
4,11,\ldots,7k+4 & 16 \pmod{7} = 2 \\
\hline
5,12,\ldots,7k+5 & 25 \pmod{7} = 4 \\
\hline
6,13,\ldots,7k+6 & 36 \pmod{7} = 1 \\
\hline
\end{array}
$

\subsubsection*{Exercice 4.2}
Montrons que  $a^2 + b^2  \equiv 0 \pmod{7} \implies a \equiv 0 \pmod{7} \text{ et } b \equiv 0 \pmod{7}$, Les valeurs possibles pour $x^2 \pmod{7}$ sont $\{0,1,2,4\}$, la seule combinaison qui donne $a^2 + b^2 \equiv 0 \pmod{7}$ est $a^2 \pmod{7} = 0$ et $b^2 \pmod{7} = 0$, d'apr\'es le tableau 1 est la seule valeur de $x$ qui donne $x^2 \equiv 0 \pmod{7}$ donc que 7 divise $a$ et $b$.

\subsubsection*{Exercice 4.3}
$0^2 + 0^2 = 7.0^2$ est vraie

\subsubsection*{Exercice 4.4}
On a $x = 7k_x$ et $y=7k_y$, donc $x^2 + x^2 = 49k_x^2 + 49k_y^2 = 7(7k_x^2+7k_y^2) = 7z^2$. Donc $z^2 = 7(k_x^2 + k_y^2)$. donc 7 divise $z^2$. D'apr\'es le tableau 1, la seule valeur pour $x^2 \equiv 0 \pmod{7}$ est $x=7k$. Donc $z = 7k$.

\subsubsection*{Exercice 4.5}
On a $(7a)^2 + (7b)^2 = 7(7c)^2$, donc $a^2 + b^2 = 7c^2$. C'est le triplet $(a,b,c), a^2 + b^2 = 7c^2$???

\subsubsection*{Exercice 4.6}


\subsubsection*{Exercice 4.7}


\subsection*{Exercice 5}
\subsubsection*{Exercice 5.1}
$
\begin{array}{|l|l|}
\hline
x & x^2 \pmod{7} \\
\hline
0,8,\ldots,8k & 0 \pmod{8} = 0  \\
\hline
1,9,\ldots,8k+1 & 1 \pmod{8} = 1 \\
\hline
2,10,\ldots,8k+2 & 4 \pmod{8} = 4 \\
\hline
3,11,\ldots,8k+3 & 9 \pmod{8} = 1 \\
\hline
4,12,\ldots,8k+4 & 16 \pmod{8} = 0 \\
\hline
5,13,\ldots,8k+5 & 25 \pmod{8} = 1 \\
\hline
6,14,\ldots,8k+6 & 36 \pmod{8} = 4 \\
\hline
7,15,\ldots,8k+7 & 49 \pmod{8} = 1 \\
\hline
\end{array}
$

On a $4^2 \equiv 0 \pmod{8}$, donc $\Z/8\Z$ est nilpotent.

\subsubsection*{Exercice 5.2}
$
\begin{array}{|l|l|}
\hline
x & x^2 \pmod{14} \\
\hline
14k & 0 \pmod{14} = 0 \\
\hline
14k+1 & 1 \pmod{14} = 1 \\
\hline
14k+2 & 4 \pmod{14} = 4  \\
\hline
14k+3 & 9 \pmod{14} = 9  \\
\hline
14k+4 & 16 \pmod{14} = 2 \\
\hline
14k+5 & 25 \pmod{14} = 11 \\
\hline
14k+6 & 36 \pmod{14} = 8 \\
\hline
14k+7 & 49 \pmod{14} = 7 \\
\hline
14k+8 & 64 \pmod{14} = 8 \\
\hline
14k+9 & 81 \pmod{14} = 11 \\
\hline
14k+10 & 100 \pmod{14} = 2 \\
\hline
14k+11 & 121 \pmod{14} = 9 \\
\hline
14k+12 & 144 \pmod{14} = 4 \\
\hline
14k+13 & 169 \pmod{14} = 1  \\
\hline
\end{array}
$

On a  $x^1 \not \equiv 0 \pmod{14}$ et $x^2 \not \equiv 0 \pmod{14}$. soit :
\begin{itemize}
\item $n$ est pair, $x{2n} \pmod{14} = x^{2^n}\pmod{14} \equiv (x^{2} \pmod{14})^n \not \equiv 0 \pmod{14}$ 

\item $n$ est impair,  $x^{2n+1} \pmod {14} \equiv x.x^{2^n} \pmod {14} \equiv (x.(x^2 \pmod {14}^n) \pmod {14} \not \equiv 0 \pmod{14}$.  

\end{itemize}
QED

\end{document}

