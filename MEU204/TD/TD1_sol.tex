\documentclass[]{book}

%These tell TeX which packages to use.
\usepackage{array,epsfig}
\usepackage{amsmath}
\usepackage{amsfonts}
\usepackage{amssymb}
\usepackage{amsxtra}
\usepackage{amsthm}
\usepackage{mathrsfs}
\usepackage{color}
\usepackage{pgfplots}

%Here I define some theorem styles and shortcut commands for symbols I use often
\theoremstyle{definition}
\newtheorem{defn}{Definition}
\newtheorem{thm}{Theorem}
\newtheorem{cor}{Corollary}
\newtheorem*{rmk}{Remark}
\newtheorem{lem}{Lemma}
\newtheorem*{joke}{Joke}
\newtheorem{ex}{Example}
\newtheorem*{soln}{Solution}
\newtheorem{prop}{Proposition}

\newcommand{\lra}{\longrightarrow}
\newcommand{\ra}{\rightarrow}
\newcommand{\surj}{\twoheadrightarrow}
\newcommand{\graph}{\mathrm{graph}}
\newcommand{\bb}[1]{\mathbb{#1}}
\newcommand{\Z}{\bb{Z}}
\newcommand{\Q}{\bb{Q}}
\newcommand{\R}{\bb{R}}
\newcommand{\C}{\bb{C}}
\newcommand{\N}{\bb{N}}
\newcommand{\M}{\mathbf{M}}
\newcommand{\m}{\mathbf{m}}
\newcommand{\MM}{\mathscr{M}}
\newcommand{\HH}{\mathscr{H}}
\newcommand{\Om}{\Omega}
\newcommand{\Ho}{\in\HH(\Om)}
\newcommand{\bd}{\partial}
\newcommand{\del}{\partial}
\newcommand{\bardel}{\overline\partial}
\newcommand{\textdf}[1]{\textbf{\textsf{#1}}\index{#1}}
\newcommand{\img}{\mathrm{img}}
\newcommand{\ip}[2]{\left\langle{#1},{#2}\right\rangle}
\newcommand{\inter}[1]{\mathrm{int}{#1}}
\newcommand{\exter}[1]{\mathrm{ext}{#1}}
\newcommand{\cl}[1]{\mathrm{cl}{#1}}
\newcommand{\ds}{\displaystyle}
\newcommand{\vol}{\mathrm{vol}}
\newcommand{\cnt}{\mathrm{ct}}
\newcommand{\osc}{\mathrm{osc}}
\newcommand{\LL}{\mathbf{L}}
\newcommand{\UU}{\mathbf{U}}
\newcommand{\support}{\mathrm{support}}
\newcommand{\AND}{\;\wedge\;}
\newcommand{\OR}{\;\vee\;}
\newcommand{\Oset}{\varnothing}
\newcommand{\st}{\ni}
\newcommand{\wh}{\widehat}

%Pagination stuff.
\setlength{\topmargin}{-.3 in}
\setlength{\oddsidemargin}{0in}
\setlength{\evensidemargin}{0in}
\setlength{\textheight}{9.in}
\setlength{\textwidth}{6.5in}
\pagestyle{empty}



\begin{document}

\subsection*{Rappel de cours}

\begin{defn}
La relation $xRy$ est une relation d'\'equivalence sur l'ensemble $E$ ssi:
\begin{itemize}
\item $\forall x \in E, xRx$
\item $\forall x,y \in E, xRy \Rightarrow yRx$
\item $\forall x,y,z \in E, xRy \wedge yRz \Rightarrow xRz$
\end{itemize}
\end{defn}


\begin{defn}
Les \'equations suivantes sont \'equivalentes:
\begin{itemize}
\item $a \equiv b \mod p$
\item $kp+a \equiv b \mod p$
\item $\exists k,k', a = kp + r \land b = k'p +r \land 0 \leq r < p$
\item $\exists k,k', a - kp = b - k'p$
\end{itemize}
\end{defn}

\begin{thm}
Petit Th\'eor\`eme de Fermat 1. Si $p$ est un nombre premier alors $\forall a \in \N, a^p \equiv a \mod p$. ou $\exists k, a^p-a = kp$
\end{thm}

\begin{thm}
Petit Th\'eor\`eme de Fermat 2. Si $p$ est un nombre premier alors $\forall a \in \N, p \not | a, a^{p-1} \equiv 1 \mod p$. ou $\exists k, a^{p-1}-1 = kp$
\end{thm}

\begin{defn}
Calcul du pgcd:
\begin{itemize}
\item $pcgd(a,b) = pgcd(a-b,b) \text{ quand } a > b$
\item $pcgd(a,b) = pgcd(a,b-a) \text{ quand } b > a$
\item $pcgd(a,a) = a$
\item $pcgd(a,0) = a$
\item $pcgd(a,b) = pgcd(b,a \mod b)$
\end{itemize}
\end{defn}



\newpage
\subsection*{Exercice 1}
\subsection*{Exercice 1.1}
La relation $R$ n'est pas une relation d'\'equivalence car $(4,4) \not\in R$.\\
Si on ajoute le couple $(4,4)$ \`a la relation $R$ alors $R$ est une relation d'\'equivalence car 
\begin{itemize}
\item $(1,1),(2,2),(3,3),(4,4) \in R$
\item $(1,2) \Rightarrow (2,1), (3,4) \Rightarrow (4,3)$
\item $(1,1) \wedge (1,2) \Rightarrow (1,2), (1,2) \wedge (2,1) \Rightarrow (1,1), (1,2) \wedge (2,2) \Rightarrow (1,2),\ldots $
\end{itemize}

\subsection*{Exercice 1.2}
La liste des classes d'\'equivalence est $\{(1,1),(1,2),2,1),2,2)\}, \{(3,3),(3,4),4,3),(4,4)\}$.


\subsection*{Exercice 2}
\subsection*{Exercice 2.1}
\begin{itemize}
\item $\forall a,b \in \R, (a,b)R(a,b) \Leftrightarrow a^2+b^2 = a^2 + b^2$ est vrai
\item $\forall a,b,c,d \in \R, ((a,b)R(c,d) \Rightarrow (c,d)R(a,b)) \Leftrightarrow (a^2+b^2 = c^2 + d^2 \Rightarrow c^2 + d^2 = a^2+b^2)$ est vrai car l'\'egalit\'e est symetrique
\item $\forall a,b,c,d,e,f \in \R, ((a,b)R(c,d) \wedge (c,d)R(e,f) \Rightarrow (a,b)R(e,f)) \Leftrightarrow (a^2+b^2 = c^2 + d^2 \wedge c^2 + d^2 = e^2+f^2\Rightarrow a^2 + b^2 = e^2+f^2)$ est vrai car l'\'egalit'e est transitive
\end{itemize}

\subsection*{Exercice 2.2}
La relation $R$ est l'ensemble des points du cercle de centre $(0,0)$ et de rayon $\sqrt{a^2+b^2}$.

\subsection*{Exercice 2.3}
Pas compris $|R^2\setminus\R$.


\subsection*{Exercice 3}
\subsection*{Exercice 3.1}
Prenons $x=a+i.b$, $y=c+i.d $ et $z=e+i.f$
\begin{itemize}
\item $\forall x \in \C, xRx \Leftrightarrow |x| = |x| \Leftrightarrow \sqrt{a^2+b^2} = \sqrt{a^2+b^2} $  est vrai
\item $\forall x,y \in \C, (xRy \Rightarrow yRx) \Leftrightarrow (|x| = |y| \Rightarrow |y| = |x|) \Leftrightarrow (\sqrt{a^2+b^2} = \sqrt{c^2+d^2} \Rightarrow \sqrt{c^2+d^2} = \sqrt{a^2+b^2})$ est vrai car l'\'egalit\'e est symetrique
\item $\forall x,y,z \in \C, (xRy \wedge yRz \Rightarrow xRz) \Leftrightarrow (|x| = |y| \wedge |y| = |z| \Rightarrow |x| = |z|) \Leftrightarrow (\sqrt{a^2+b^2} = \sqrt{c^2+d^2} \wedge \sqrt{c^2+d^2} = \sqrt{e^2+f^2} \Rightarrow \sqrt{a^2+b^2} = \sqrt{e^2+f^2})$ est vrai car l'\'egalit'e est transitive
\end{itemize}


\subsection*{Exercice 3.2}
\begin{itemize}
\item $\forall x \in \R, xRx \Leftrightarrow e^x = e^x$ est vrai
\item $\forall x,y \in \R, (xRy \Rightarrow yRx) \Leftrightarrow (e^x = e^y \Rightarrow e^y = e^x)$ est vrai car l'\'egalit\'e est symetrique
\item $\forall x,y,z \in \R, (xRy \wedge yRz \Rightarrow xRz) \Leftrightarrow (e^x = e^y \wedge e^y = e^z \Rightarrow e^x = e^z)$ est vrai car l'\'egalit'e est transitive
\end{itemize}

\subsection*{Exercice 4}
\subsection*{Exercice 4.1}
\begin{itemize}
\item $\forall a,b \in \R, (a,b)R(a,b) \Leftrightarrow ab = ba$ est vrai car la multiplication est commutative
\item $\forall a,b,c,d \in \R, ((a,b)R(c,d) \Rightarrow (c,d)R(a,b)) \Leftrightarrow (ad = bc \Rightarrow cb = da)$ est vrai car l'\'egalit\'e est symetrique et la multiplication est commutative
\item $\forall a,b,c,d,e,f \in \R, ((a,b)R(c,d) \wedge (c,d)R(e,f) \Rightarrow (a,b)R(e,f)) \Leftrightarrow (ad = bc \wedge cf = de \Rightarrow af = be)$ est vrai car $a = \frac{bc}{d}$, donc $af = \frac{bc}{d}f$ mais $cf=de$ donc $af = \frac{bde}{d} = be$. 
\end{itemize}

\subsection*{Exercice 4.2}
$(p,q)R(x,y) \Leftrightarrow py = qx$ avec $(p,q)$ premiers entre eux. Le seul couple est $x=np$ et $y=nq$. Donc la relation repr\'esente les couples $\forall n in \N^{*},(np,nq)$ avec $(p,q)$ premiers entre eux.


\subsection*{Exercice 5}
\subsection*{Exercice 5.1}
\begin{itemize}
\item $\forall P \in \R, PRP \Leftrightarrow P-P$ est un multiple de $X$ est vrai car 0 est un multiple de tous les nombres ($0 = 0.x$)
\item $\forall P,Q \in \R, (PRQ \Rightarrow QRP) \Leftrightarrow (P-Q = k.X \Rightarrow Q-P = k'X)?$ est vrai en prenant $k'=-k$ car $Q-P = -(P-Q) = -kX$ 
\item $\forall P,Q,S \in \R, (PRQ \wedge QRS \Rightarrow PRS) \Leftrightarrow (P-Q = kX \wedge Q-S = k'X \Rightarrow P-S = k''X)?$ est vrai $P-S = (P-Q) - (Q-S) = kX + k'X = (k+k')X$
\end{itemize}

\subsection*{Exercice 5.2}
$PRP(0) = P - P(0)$ mais $P(0)$ est le polynome de degr\'e 0, donc $P-P(0)$ est un polynome avec le degr\'e 0 \'egale \`a 0. On peut donc le factoriser par $X$. Par cons\'equent $P-P(0)=X.k$ est un multiple de $X$

\subsection*{Exercice 5.3}
En prenant par exemple, $\pi:\Z[X] \to Z[X], P \to P-X$, on a $\forall P \in \Z[X]
, P(0) = (P-X)(0)$ car $P-X$ ne change pas de degr\'e 0 du polynome $P$.


\subsection*{Exercice 6}
Prenons $a=7k+r$ et $b=7k'+r'$ avec $r,r' < 7$. On a 
$$a^2+b^2 = (7k+r)^2+(7k'+r')^2 = 49k^2+14kr+r^2+49k'^2+14k'r'+r'^2 = 7(7k^2+2kr+7k'+2k'^2r')+r^2+r'^2$$
On a $7|a^2+b^2$ donc $7|7(7k^2+2kr+7k'^2+2k'r')+r^2+r'^2$, donc $7|r^2+r'^2$ et $r,r' < 7$. On a
\begin{center}
\begin{tabular}{ l l}
 $0^2 \mod 7$ & 0 \\ 
 $1^2 \mod 7$ & 1 \\ 
 $2^2 \mod 7$ & 4 \\ 
 $3^2 \mod 7$ & 2 \\ 
 $4^2 \mod 7$ & 2 \\ 
 $5^2 \mod 7$ & 4 \\ 
 $6^2 \mod 7$ & 1 \\ 
\end{tabular}
\end{center}

La seule combinaison possible est $r=0$ et $r'=0$. Donc $7|a$ et $7|b$.

\subsection*{Exercice 7}
Preuve par r\'ecurrence. Vrai pour $n=0$ ($3^1+2^2=7$). Supposons vrai pour $n$, $3^{2n+1}+2^{n+2} = 7k$ est-ce que $3^{2(n+1)+1}+2^{(n+1)+2} = 7k'$

$$3^{2(n+1)+1}+2^{(n+1)+2} = 3^{(2n+1)+2}+2^{(n+2)+1} = 9.3^{2n+1}+2.2^{n+2}$$
On a $3^{2n+1} = 7k - 2^{n+2}$ (hypoth\`ese)
$$9.(7k - 2^{n+2}) + 2.2^{n+2} = 63k - 7.2^{n+2} = 7(9k+2^{n+2})$$

Donc vrai en prenant $k' = 9k+2^{n+2}$.

\subsection*{Exercice 8}
Preuve par r\'ecurrence. pour $n=1$, on a $2^{3+3}-7-8= 64 - 15 = 49$. Supposons vrai pour $49|2^{3n+3}-7n-8$ au rang $n$, v\'erifions que $49|2^{3(n+1)+3}-7(n+1)-8$

$$2^{3(n+1)+3}-7(n+1)-8 = 2^{3n+3+3}-7(n+1)-8 = 2^3.2^{3n+3}-7n-7-8 = (7+1)2^{3n+3}-7n-7-8 = 7.2^{3n+3}-7 + (2^{3n+3}-7n-8) $$
$$= 7(2^{3n+3}-1) + (2^{3n+3}-7n-8)$$

On a $49|(2^{3n+3}-7n-8)$ par hypoth\`ese de r\'ecurrence. Il reste \`a montrer que $49|7(2^{3n+3}-1)$ ou $7|2^{3n+3}-1$. \\

Preuve par r\'ecurrence, pour n=1, $2^6-1=64-1=63$ qui est divisible par 7. Supposons $7|2^{3n+3}-1$ et v\'erifions $7|2^{3(n+1)+3}-1$.
$$2^{3(n+1)+3}-1 = 2^{3n+3+3}-1 = 2^3.2^{3n+3}-1 = (7+1).2^{3n+3}-1 = 7.2^{3n+3} + 2^{3n+3}-1$$
On a $7|2^{3n+3}-1$ par hypoth\`ese de r\'ecurrence. et $7|7.2^{3n+3}$. donc $\forall n \in \N, 7| 2^{3n+3}-1$, et $\forall n \in \N, 49|7(2^{3n+3}-1)$, et $\forall n \in \N, 49| 7(2^{3n+3}-1) + (2^{3n+3}-7n-8)$ et $\forall n \in \N, 49| 2^{3n+3}-7n-8$.\\

La roposition est vraie.


\subsection*{Exercice 9}
\subsubsection*{Exercice 9.1}
Preuve par r\'ecurrence. Vrai pour $n=0$ ($2^3+3^1=11$). Supposons vrai pour $n$, $2^{6n+3}+3^{2n+1} = 11k$ est-ce que $2^{6(n+1)+3}+3^{2(n+1)+1} = 11k'$

$$2^{6(n+1)+3}+3^{2(n+1)+1} = 2^{(6n+3)+6}+3^{2n+1+2} = 2^6.2^{6n+1}+3^2.2^{2n+1}$$
On a $2^{6n+3} = 11k - 2^{2n+1}$ (hypoth\`ese)
$$2^6.(11k - 3^{2n+1}) + 3^2.3^{2n+1} = 2^6.11k - 2^6.3^{2n+1} + 3^2.3^{2n+1} = 11(2^6k+5.3^{2n+1})$$

Donc vrai en prenant $k' = 2^6k+5.3^{2n+1}$.

\subsubsection*{Exercice 9.2}
Preuve par r\'ecurrence. Vrai pour $n=0$ ($6|0$). Supposons vrai pour $n$, $6|5n^3+n$ est-ce que $6|5(n+1)^3+(n+1)$
$$
5(n+1)^3+(n+1) = 5(n^3+3n^2+3n+1)+n+1 = 5n^3+n +3(5n^2+5n+2)
$$
On a $6|5n^3+n$ (hypoth\`ese de r\'ecurrence). Est-ce que $6|3(5n^2+5n+2)$ ou $2|5n^2+5n+2$. 2 cas :
\begin{itemize}
\item n est pair, $n=2m$ et $n^2=4m^2$, donc $5n^2+5n+2=20m^2+10m+2=2(10m^2+5m+1)$ qui est divisible par 2
\item n est impair, $n=2m+1$ et $n^2=4m^2+2m+1$, donc $5n^2+5n+2=5(4m^2+2m+1)+5(2m+1) +2 = 20m^2+20m+12 = 2(10m^2+10m+6)$ qui est divisible par 2
\end{itemize}


\subsection*{Exercice 10}
\begin{itemize}
\item On a $p^2-1 = (p+1)(p-1)$
\item $p$ est premier donc il est impair ($p=2^n+1$). On a $(p+1)(p-1) = (2^n+2)(2^n)$. Soit $2|(p+1)$, soit $4|(p-1)$. Donc $p^2-1=2k.4k' = 8(kk')$.
\item $p$ est premier donc $p \mod 3 =1$ ou $p \mod 3 =2$. Si $p \mod 3 =1$  alors $(p+1)(p-1) = (p+1).3k = 3((p+1)k)$, et Si $p \mod 3 =2$  alors $(p+1)(p-1) = 3k.(p-1) = 3(k(p-1))$ donc $p^2-1 = 3k'$
\end{itemize}

3 et 8 sont premiers entre eux, et p est premier donc $p^2-1=3*8*k = 24k$.

QED

\end{document}

