\documentclass[]{book}

%These tell TeX which packages to use.
\usepackage{array,epsfig}
\usepackage{amsmath}
\usepackage{amsfonts}
\usepackage{amssymb}
\usepackage{amsxtra}
\usepackage{amsthm}
\usepackage{mathrsfs}
\usepackage{color}
\usepackage{pgfplots}

%Here I define some theorem styles and shortcut commands for symbols I use often
\theoremstyle{definition}
\newtheorem{defn}{Definition}
\newtheorem{thm}{Theorem}
\newtheorem{cor}{Corollary}
\newtheorem*{rmk}{Remark}
\newtheorem{lem}{Lemma}
\newtheorem*{joke}{Joke}
\newtheorem{ex}{Example}
\newtheorem*{soln}{Solution}
\newtheorem{prop}{Proposition}

\newcommand{\lra}{\longrightarrow}
\newcommand{\ra}{\rightarrow}
\newcommand{\surj}{\twoheadrightarrow}
\newcommand{\graph}{\mathrm{graph}}
\newcommand{\bb}[1]{\mathbb{#1}}
\newcommand{\Z}{\bb{Z}}
\newcommand{\Q}{\bb{Q}}
\newcommand{\R}{\bb{R}}
\newcommand{\C}{\bb{C}}
\newcommand{\N}{\bb{N}}
\newcommand{\M}{\mathbf{M}}
\newcommand{\m}{\mathbf{m}}
\newcommand{\MM}{\mathscr{M}}
\newcommand{\HH}{\mathscr{H}}
\newcommand{\Om}{\Omega}
\newcommand{\Ho}{\in\HH(\Om)}
\newcommand{\bd}{\partial}
\newcommand{\del}{\partial}
\newcommand{\bardel}{\overline\partial}
\newcommand{\textdf}[1]{\textbf{\textsf{#1}}\index{#1}}
\newcommand{\img}{\mathrm{img}}
\newcommand{\ip}[2]{\left\langle{#1},{#2}\right\rangle}
\newcommand{\inter}[1]{\mathrm{int}{#1}}
\newcommand{\exter}[1]{\mathrm{ext}{#1}}
\newcommand{\cl}[1]{\mathrm{cl}{#1}}
\newcommand{\ds}{\displaystyle}
\newcommand{\vol}{\mathrm{vol}}
\newcommand{\cnt}{\mathrm{ct}}
\newcommand{\osc}{\mathrm{osc}}
\newcommand{\LL}{\mathbf{L}}
\newcommand{\UU}{\mathbf{U}}
\newcommand{\support}{\mathrm{support}}
\newcommand{\AND}{\;\wedge\;}
\newcommand{\OR}{\;\vee\;}
\newcommand{\Oset}{\varnothing}
\newcommand{\st}{\ni}
\newcommand{\wh}{\widehat}

%Pagination stuff.
\setlength{\topmargin}{-.3 in}
\setlength{\oddsidemargin}{0in}
\setlength{\evensidemargin}{0in}
\setlength{\textheight}{9.in}
\setlength{\textwidth}{6.5in}
\pagestyle{empty}



\begin{document}

\subsection*{Rappel de cours}
\begin{defn}
Un \emph{groupe (G,*)} est un ensemble $G$ auquel est associ\'e une op\'eration $*$ (la \emph{loi de composition}) v\'erifiant les 4 propri\'et\'es suivantes:
\begin{itemize}
\item $\forall x, y \in G, x * y \in G$. $*$ est une loi de composition interne.
\item $\forall x, y, z \in G, (x * y) * z = x * (y * z)$ la loi est \emph{associative}
\item $\exists e \in G, \forall x \in G, x * e = e * x = x$. $e$ est l'\'el\'ement neutre
\item $\forall x \in G, \exists x' \in G, x * x' = e$. $x'$ est l'inverse de $x$ et est not\'e $x^{-1}$. 
\end{itemize}
\end{defn}

\newpage
\subsection*{Exercice 1}
Pour que $\R$, muni de la multiplication soit un groupe, il faut qu'il v\'efifie les 4 propri\'et\'es d'un groupe. La multiplication est une loi de composition interne pour $\R$. La multiplication est associative dans $\R$. 1 est l'\'el\'ement neutre pour la multiplication dans $\R$. V\'erifions si tout les \'el'\'ements de $\R$ ont un inverse dans $\R$. 
0, n'a pas d'inverse dans $\R$, donc $(\R,*)$ n'est pas un groupe.

\subsection*{Exercice 2}
On a $G=\{a,b,e\}$, $(G,.)$ est un groupe et $e$ l\'el\'ement neutre du groupe $(G,.)$. Donc $\forall x \in G, \exists x' \in G, x . x' = e$ et $\forall x, y, \in G, x.y \in G = \{a,b,e\}$. \\
Donc $a.b \in {a,b,e}$. Plusieurs cas possibles:
\begin{itemize}
\item $b$ est l'inverse de $a$ dans le groupe. Donc, $a.b = e$
\item $b$ n'est pas l'inverse de $a$ dans le groupe. donc $a.b \in {a,b}$. Soit $a.b = a$, as possible car $a.e=a$ et $b \neq e$, ou $a.b=b$ pas possible car $(a.b).b \neq a.(b.b)$.
\end{itemize}
Donc $a.b = e$

\subsection*{Exercice 3}
Non. $a$ et $b$ premiers entre eux donc $\gcd(a,b) = 1$ et $b$ et $c$ premiers entre eux donc $\gcd(b,c) = 1$. Prenons, $a=3, b=5, c=9$, on a $gcd(3,5) = 1$ et $\gcd(5,9) = 1$ mais $gcd(3,9) = 3$. Donc $a$ et $c$ ne sont pas premiers entre eux.

\subsection*{Exercice 4}


\subsection*{Exercice 5}
\subsubsection*{Exercice 5.1}
$p^2 -1 = (p+1)(p-1)$, comme $p$ est un nombre premier sup\'erieur \`a 5, $p$ est impair. Donc $p^2 - 1 = (2k+1-1)(2k+1+1) = 2k(2k+2) = 4k(k+1)$.

\subsubsection*{Exercice 5.2}
$8 | p^2-1$ si $\exists n, p^2-1 = 8n$. 
\begin{itemize}
\item $k$ est pair donc $k=2k'$ et $4k(k+1) = 8k'(2k'+1)$ donc $n = k'(2k'+1)$
\item $k$ est impair donc $k=2k'+1$ et $4k(k+1) = 4(2k'+1)(2k'+1+1) = 4(2k'+1)(2k'+2) = 8(2k'+1)(k'+1)$ donc $n=(2k'+1)(k'+1)$.
\end{itemize}
$n$ existe, donc $8 | p^2 - 1$.\\

$16 | p^4-1$ si $\exists n, p^4-1 = 16n$. 
$p^4-1 = (p^2-1)(p^2+1)$ et $8 | p^2 -1$ mais $p$ est impair donc $p^2$ est impair et $p^2+1$ est pair. Par cons\'equent $2 | p^2 + 1$. Par cons\'equent, $(p^2-1)(p^2+1) = 8n.2n' = 16nn'$ donc $16 | p^4-1$.

\subsubsection*{Exercice 5.3}

 




QED

\end{document}

