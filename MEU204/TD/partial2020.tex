\documentclass[]{book}

%These tell TeX which packages to use.
\usepackage{array,epsfig}
\usepackage{amsmath}
\usepackage{amsfonts}
\usepackage{amssymb}
\usepackage{amsxtra}
\usepackage{amsthm}
\usepackage{mathrsfs}
\usepackage{color}
\usepackage{pgfplots}

%Here I define some theorem styles and shortcut commands for symbols I use often
\theoremstyle{definition}
\newtheorem{defn}{Definition}
\newtheorem{thm}{Theorem}
\newtheorem{cor}{Corollary}
\newtheorem*{rmk}{Remark}
\newtheorem{lem}{Lemma}
\newtheorem*{joke}{Joke}
\newtheorem{ex}{Example}
\newtheorem*{soln}{Solution}
\newtheorem{prop}{Proposition}

\newcommand{\lra}{\longrightarrow}
\newcommand{\ra}{\rightarrow}
\newcommand{\surj}{\twoheadrightarrow}
\newcommand{\graph}{\mathrm{graph}}
\newcommand{\bb}[1]{\mathbb{#1}}
\newcommand{\Z}{\bb{Z}}
\newcommand{\Q}{\bb{Q}}
\newcommand{\R}{\bb{R}}
\newcommand{\C}{\bb{C}}
\newcommand{\N}{\bb{N}}
\newcommand{\M}{\mathbf{M}}
\newcommand{\m}{\mathbf{m}}
\newcommand{\MM}{\mathscr{M}}
\newcommand{\HH}{\mathscr{H}}
\newcommand{\Om}{\Omega}
\newcommand{\Ho}{\in\HH(\Om)}
\newcommand{\bd}{\partial}
\newcommand{\del}{\partial}
\newcommand{\bardel}{\overline\partial}
\newcommand{\textdf}[1]{\textbf{\textsf{#1}}\index{#1}}
\newcommand{\img}{\mathrm{img}}
\newcommand{\ip}[2]{\left\langle{#1},{#2}\right\rangle}
\newcommand{\inter}[1]{\mathrm{int}{#1}}
\newcommand{\exter}[1]{\mathrm{ext}{#1}}
\newcommand{\cl}[1]{\mathrm{cl}{#1}}
\newcommand{\ds}{\displaystyle}
\newcommand{\vol}{\mathrm{vol}}
\newcommand{\cnt}{\mathrm{ct}}
\newcommand{\osc}{\mathrm{osc}}
\newcommand{\LL}{\mathbf{L}}
\newcommand{\UU}{\mathbf{U}}
\newcommand{\support}{\mathrm{support}}
\newcommand{\AND}{\;\wedge\;}
\newcommand{\OR}{\;\vee\;}
\newcommand{\Oset}{\varnothing}
\newcommand{\st}{\ni}
\newcommand{\wh}{\widehat}

%Pagination stuff.
\setlength{\topmargin}{-.3 in}
\setlength{\oddsidemargin}{0in}
\setlength{\evensidemargin}{0in}
\setlength{\textheight}{9.in}
\setlength{\textwidth}{6.5in}
\pagestyle{empty}



\begin{document}

\subsection*{Rappel de cours}
\begin{defn}
Un \emph{groupe (G,*)} est un ensemble $G$ auquel est associ\'e une op\'eration $*$ (la \emph{loi de composition}) v\'erifiant les 4 propri\'et\'es suivantes:
\begin{itemize}
\item $\forall x, y \in G, x * y \in G$. $*$ est une loi de composition interne.
\item $\forall x, y, z \in G, (x * y) * z = x * (y * z)$ la loi est \emph{associative}
\item $\exists e \in G, \forall x \in G, x * e = e * x = x$. $e$ est l'\'el\'ement neutre
\item $\forall x \in G, \exists x' \in G, x * x' = e$. $x'$ est l'inverse de $x$ et est not\'e $x^{-1}$. 
\end{itemize}
\end{defn}

\newpage
\subsection*{Exercice 1}
Pour que $\R$, muni de la multiplication soit un groupe, il faut qu'il v\'efifie les 4 propri\'et\'es d'un groupe. La multiplication est une loi de composition interne pour $\R$. La multiplication est associative dans $\R$. 1 est l'\'el\'ement neutre pour la multiplication dans $\R$. V\'erifions si tout les \'el'\'ements de $\R$ ont un inverse dans $\R$. 
0, n'a pas d'inverse dans $\R$, donc $(\R,*)$ n'est pas un groupe.

\subsection*{Exercice 2}
On a $G=\{a,b,e\}$, $(G,.)$ est un groupe et $e$ l\'el\'ement neutre du groupe $(G,.)$. Donc $\forall x \in G, \exists x' \in G, x . x' = e$ et $\forall x, y, \in G, x.y \in G = \{a,b,e\}$. \\
Donc $a.b \in {a,b,e}$. Plusieurs cas possibles:
\begin{itemize}
\item $b$ est l'inverse de $a$ dans le groupe. Donc, $a.b = e$
\item $b$ n'est pas l'inverse de $a$ dans le groupe. donc $a.b \in {a,b}$. Soit $a.b = a$, as possible car $a.e=a$ et $b \neq e$, ou $a.b=b$ pas possible car $(a.b).b \neq a.(b.b)$.
\end{itemize}
Donc $a.b = e$

\subsection*{Exercice 3}
Non. $a$ et $b$ premiers entre eux donc $\gcd(a,b) = 1$ et $b$ et $c$ premiers entre eux donc $\gcd(b,c) = 1$. Prenons, $a=3, b=5, c=9$, on a $gcd(3,5) = 1$ et $\gcd(5,9) = 1$ mais $gcd(3,9) = 3$. Donc $a$ et $c$ ne sont pas premiers entre eux.

\subsection*{Exercice 4}
Preuve par r\'ecurence. Suppusons que $7| 3^{2n+1} + 2^{n+2}$, montrons que $7| 3^{2(n+1)+1} + 2^{(n+1)+2}$.
$$
3^{2n+1} + 2^{n+2} = 3.3^{2n} + 42^{n} = 3.9^n + 4.2^n
$$
cqlculons 
$$
3.9^n + 4.2^n [7] = 3.2^n + 4.2^n [7] = 2^n(3+4) [7] = 7.2^n [7] = 0
$$
donc $7 | 3^{2n+1} + 2^{n+2}$.

\subsection*{Exercice 5}
\subsubsection*{Exercice 5.1}
$p^2 -1 = (p+1)(p-1)$, comme $p$ est un nombre premier sup\'erieur \`a 5, $p$ est impair. Donc $p^2 - 1 = (2k+1-1)(2k+1+1) = 2k(2k+2) = 4k(k+1)$.

\subsubsection*{Exercice 5.2}
$8 | p^2-1$ si $\exists n, p^2-1 = 8n$. 
\begin{itemize}
\item $k$ est pair donc $k=2k'$ et $4k(k+1) = 8k'(2k'+1)$ donc $n = k'(2k'+1)$
\item $k$ est impair donc $k=2k'+1$ et $4k(k+1) = 4(2k'+1)(2k'+1+1) = 4(2k'+1)(2k'+2) = 8(2k'+1)(k'+1)$ donc $n=(2k'+1)(k'+1)$.
\end{itemize}
$n$ existe, donc $8 | p^2 - 1$.\\

$16 | p^4-1$ si $\exists n, p^4-1 = 16n$. 
$p^4-1 = (p^2-1)(p^2+1)$ et $8 | p^2 -1$ mais $p$ est impair donc $p^2$ est impair et $p^2+1$ est pair. Par cons\'equent $2 | p^2 + 1$. Par cons\'equent, $(p^2-1)(p^2+1) = 8n.2n' = 16nn'$ donc $16 | p^4-1$.

\subsubsection*{Exercice 5.3}
$3 | p^2-1$ si $\exists n, p^2-1 = 3n$. 
Chaque entier $n$ peut s'\'ecrire $3k$, $3k+1$ ou $3k+2$. Comme $p$ est un nombre premier il ne peut pas \^etre \'egal a $3n$. Donc il reste 2 cas:
\begin{itemize}
\item $p = 3n+1$, donc $p^2-1 = (3n+1)^2 -1 = 9n^2+6n+1 -1 = 3(3n^2+2n)$
\item $p = 3n+2$, donc $p^2-1 = (3n+2)^2 -1 = 9n^2+12n+4 -1 = 3(3n^2+6n+1)$
\end{itemize}
Donc $3|p^2 -1$ pour tout nombre premier $p$.

\subsubsection*{Exercice 5.4}
$5 | p^2-1$ si $\exists n, p^2-1 = 5n$. 
$p^4 -1 = (p^2-1)(p^2+1)$
Chaque entier $n$ peut s'\'ecrire $5k$, $5k+1$, $5k+2$, $5k+3$ ou $5k+4$. Comme $p$ est un nombre premier il ne peut pas \^etre \'egal a $5n$. Donc il reste 4 cas:
\begin{itemize}
\item $p = 5n+1$, donc $p^4-1 = ((5n+1)^2-1)((5n+1)^2+1) = (25n^2+10n+1-1)((5n+1)^2-1) = 5(n^2+2n)((5n+1)^2-1)$
\item $p = 5n+2$, donc $p^4-1 = ((5n+2)^2-1)((5n+2)^2+1) = ((5n+2)^2-1)(25n^2+10n+4+1) = 5((5n+2)^2-1)(5n^2+2n+1)$
\item $p = 5n+3$, donc $p^4-1 = ((5n+3)^2-1)((5n+3)^2+1) = ((5n+2)^2-1)(25n^2+30n+9+1) = 5((5n+2)^2-1)(5n^2+6n+2)$
\item $p = 5n+4$, donc $p^4-1 = ((5n+4)^2-1)((5n+4)^2+1) = (25n^2+40n+16-1)((5n+2)^2+1) = 5(5n^2+8n+3)((5n+2)^2-1)$
\end{itemize}
Donc $5|p^4 -1$ pour tout nombre premier $p$.

\subsubsection*{Exercice 5.5}
$a | c$ donc $c = k_1a$ et $b | c$ donc $c = k_2b$. il faut montrer que $ab | c$ ou $c = kab$. En partant de l'identit\'e de Bezout on a $\gcd(a,b) = ax+ by$ donc $ax + by =1$, en multipliant par $c$ on a $cax+cby = c$, cela fait $k_2bax + k_1aby = c$ et $ab(k_2x+k_1y) = c$. Donc $ab| c$.


\subsubsection*{Exercice 5.6}
On a $16| p^4-1$ et $5|p^4-1$, et $\gcd(16, 5) = 1$ donc d'apr\`es la question 5 on a $80| p^4-1$.\\
On a $p^4-1 = (p^2-1)(p^2+1)$ et $3|p^2-1)$ donc $p^2-1 = 3n$ et $p^4-1 = 3n(p^2+1)$ donc $3|p^4-1$., $gcd(3,80) = 1$ donc d'apr\`es la question 5, on a $240|p^4-1$.

\subsection*{Exercice 6}
\subsubsection*{Exercice 6.1}
Non, car $N = 4u_1u_2u_3\ldots u_n -1$ avec $\gcd(2,u_1, u_2, \ldots, u_n) = 1$ (car tous $u_n$ premiers et $u_1=3$.). Prenons un $u_i$, on a $N = u_i(4u_1u_2u_3\ldots u_n) -1$ avec $4u_1u_2u_3\ldots u_n$ non divisible par $u_i$. donc le reste de la division $\frac{N}{u_i} = u_i-1$ qui est diff\'erent de 0.

\subsubsection*{Exercice 6.2}

\subsubsection*{Exercice 6.3}


\subsection*{Exercice 7}
\subsubsection*{Exercice 7.1}
$$\gcd(171,160), 171 = 160*1 + 11$$
$$\gcd(11,160), 160 = 11*14 + 6$$
$$\gcd(11, 6), 11 = 6*1 + 5$$
$$\gcd(5, 6), 6 = 5*1 + 1$$
$$\gcd(5, 1), 5 = 1*5 + 0$$
$$\gcd(0, 1) = 1$$

\subsubsection*{Exercice 7.2}
On a $\gcd(171,160) = 1$, donc $\exists x, y, 171x+160y = 1$.
$$1 = 6 - 1*5 = 6 - (11-6*1) = 2*6-11 = 2*(160-11*14) - 11 = 2*160 - 29*11 = 2*160 -29*(171-160) = 31*160 -29*171$$
Identit\'e de Bezout: $31*160 - 29*171 = 1$.

\subsubsection*{Exercice 7.3}
$(x,y) = (x_0 + nb/d, y_0 - na/d)$ avec $\gcd(x,y)=d$ et $a,b$ une solution de $ax+by=d$. Donc, $x_0=-29$, $y_0=31$, $a=171$, $b=160$ et $d=1$.
$$
(x,y) = (-29 + 160n, 31 - 171n)
$$



\subsection*{Exercice}
\subsubsection{Ex 1}
Montrer que tout nombre impair peut se mettre sous la forme $4k+1$ ou $4k+3$.\\
\subsubsection{Ex 2}
Montrer que tout nombre impair peut se mettre sous la forme $4k+1$ ou $4k-1$.\\
\subsubsection{Ex 3}
Montrer que pour tout nombre impair $p$ on a $p^2 \equiv 1 \pmod{8}$.\\

Preuve, un nombre impair p est de la forme $p = 2x+1$, $x$ est soit pair soit impair. cas $x$ est pair donc $p = 4l +1$, cas $x$ est impair donc $p=4k+3 = 4(k+1) -1$.\\


QED

\end{document}

