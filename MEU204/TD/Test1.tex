\documentclass[]{book}

%These tell TeX which packages to use.
\usepackage{array,epsfig}
\usepackage{amsmath}
\usepackage{amsfonts}
\usepackage{amssymb}
\usepackage{amsxtra}
\usepackage{amsthm}
\usepackage{mathrsfs}
\usepackage{color}
\usepackage{pgfplots}

%Here I define some theorem styles and shortcut commands for symbols I use often
\theoremstyle{definition}
\newtheorem{defn}{Definition}
\newtheorem{thm}{Theorem}
\newtheorem{cor}{Corollary}
\newtheorem*{rmk}{Remark}
\newtheorem{lem}{Lemma}
\newtheorem*{joke}{Joke}
\newtheorem{ex}{Example}
\newtheorem*{soln}{Solution}
\newtheorem{prop}{Proposition}

\newcommand{\lra}{\longrightarrow}
\newcommand{\ra}{\rightarrow}
\newcommand{\surj}{\twoheadrightarrow}
\newcommand{\graph}{\mathrm{graph}}
\newcommand{\bb}[1]{\mathbb{#1}}
\newcommand{\Z}{\bb{Z}}
\newcommand{\Q}{\bb{Q}}
\newcommand{\R}{\bb{R}}
\newcommand{\C}{\bb{C}}
\newcommand{\N}{\bb{N}}
\newcommand{\M}{\mathbf{M}}
\newcommand{\m}{\mathbf{m}}
\newcommand{\MM}{\mathscr{M}}
\newcommand{\HH}{\mathscr{H}}
\newcommand{\Om}{\Omega}
\newcommand{\Ho}{\in\HH(\Om)}
\newcommand{\bd}{\partial}
\newcommand{\del}{\partial}
\newcommand{\bardel}{\overline\partial}
\newcommand{\textdf}[1]{\textbf{\textsf{#1}}\index{#1}}
\newcommand{\img}{\mathrm{img}}
\newcommand{\ip}[2]{\left\langle{#1},{#2}\right\rangle}
\newcommand{\inter}[1]{\mathrm{int}{#1}}
\newcommand{\exter}[1]{\mathrm{ext}{#1}}
\newcommand{\cl}[1]{\mathrm{cl}{#1}}
\newcommand{\ds}{\displaystyle}
\newcommand{\vol}{\mathrm{vol}}
\newcommand{\cnt}{\mathrm{ct}}
\newcommand{\osc}{\mathrm{osc}}
\newcommand{\LL}{\mathbf{L}}
\newcommand{\UU}{\mathbf{U}}
\newcommand{\support}{\mathrm{support}}
\newcommand{\AND}{\;\wedge\;}
\newcommand{\OR}{\;\vee\;}
\newcommand{\Oset}{\varnothing}
\newcommand{\st}{\ni}
\newcommand{\wh}{\widehat}

%Pagination stuff.
\setlength{\topmargin}{-.3 in}
\setlength{\oddsidemargin}{0in}
\setlength{\evensidemargin}{0in}
\setlength{\textheight}{9.in}
\setlength{\textwidth}{6.5in}
\pagestyle{empty}



\begin{document}

\subsection*{Rappel de cours}


\newpage
\subsection*{Exercice 1}
\subsubsection*{Exercice 1.1}
$a = 5n+1$ et $b=2n+1$. On a $d|a \Leftrightarrow kd=a=5n+1$, donc $n=\frac{kd-1}{5}$ et $d|b \Leftrightarrow k'd=b=2n+1$.
$n=\frac{k'd-1}{2}$.
$$\frac{kd-1}{5} = \frac{k'd-1}{2}$$
$$2(kd-1) = 5(k'd-1)$$
$$d(2k-5k') = 3$$

Donc $d|3$.

\subsection*{Exercice 1.2}
\begin{itemize}
\item Prenons $n \equiv 0 \mod 3$ donc $n = 3m$
$$pgcd(a,b) = pgcd(5(3m)+1, 2(3m)+1) = pgcd(15m+1, 6m+1) = pgcd(9m, 6m +1) $$
$$= pgcd(3m-1, 6m+1) = pgcd(3m-1, 3m+2) = pgcd(3m-1, 3) = 1$$
\item Prenons $n \equiv 1 \mod 3$ donc $n = 3m+1$
$$pgcd(a,b) = pgcd(5(3m+1)+1, 2(3m+1)+1) = pgcd(15m+6, 6m+3) = pgcd(9m+3, 6m +3)$$
$$= pgcd(3m, 6m+3) = pgcd(3m, 3m+3) = pgcd(3m, 3) = 3$$
\item Prenons $n \equiv 2 \mod 3$ donc $n = 3m+2$
$$pgcd(a,b) = pgcd(5(3m+2)+1, 2(3m+2)+1) = pgcd(15m+11, 6m+5) = pgcd(9m+6, 6m +3)$$
$$= pgcd(3m+3, 6m+3) = pgcd(3m+3, 3m) = pgcd(3m, 3) = 3$$
\end{itemize}


\subsection*{Exercice 2}
on a $8k+7 \equiv 7 \mod 8$. et on cherche si il existe $a,b,c$ tel que $a^2+b^2+c^2=8k+7$ ou $a^2+b^2+c^2 \equiv 7 \mod 8$. Tout entier peut s'\'ecrire sous la forme $8k+i$ avec $0 \geq i \geq 7$. Et on a 
\begin{center}
\begin{tabular}{ l l}
 $(8k)^2 \mod 8$ & 0 \\ 
 $(8k+1)^2 \mod 8$ & 1 \\ 
 $(8k+2)^2 \mod 8$ & 4 \\ 
 $(8k+3)^2 \mod 8$ & 1 \\ 
 $(8k+4)^2 \mod 8$ & 0 \\ 
 $(8k+5)^2 \mod 8$ & 1 \\ 
 $(8k+6)^2 \mod 8$ & 4 \\ 
 $(8k+7)^2 \mod 8$ & 1 \\ 
\end{tabular}
\end{center}

Si $a_1 \equiv a_2 \mod n$ et $b_1 \equiv b_2 \mod n$ alors $a_1+b_1 \equiv a_2+b_2 \mod n$. En ce basant sur la table pr\'ec\'edente, il n'existe pas de combinaison possible pour avoir $a^2+b^2+c^2 \equiv 7 \mod 8$


\subsection*{Exercice 3}
$$666^999 \mod 13 = (666 \mod 13)^999 = 1^999 = 1$$

\subsection*{Exercice 4}
Soit $a=a_1^2+a_2^2 = (a_1+ia_2)(a_1-ia_2)$ et $b=b_1^2+b_2^2=(b_1+ib_2)(b_1-ib_2)$, on a .
$$ab = (a_1+ia_2)(a_1-ia_2)(b_1+ib_2)(b_1-ib_2) = (a_1+ia_2)(b_1+ib_2)(a_1-ia_2)(b_1-ib_2)$$
$$=(a_1b_1-a_2b_2 + i(a_2b_1+a_1b_2))(a_1b_1-a_2b_2 - i(a_2b_1+a_1b_2)) = (a_1b_1-a_2b_2)^2+(a_2b_1+a_1b_2)^2$$

QED

\end{document}

