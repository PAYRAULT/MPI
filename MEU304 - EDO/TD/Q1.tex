\documentclass[]{book}

%These tell TeX which packages to use.
\usepackage{array,epsfig}
\usepackage{amsmath}
\usepackage{amsfonts}
\usepackage{amssymb}
\usepackage{amsxtra}
\usepackage{amsthm}
\usepackage{mathrsfs}
\usepackage{color}
\usepackage[margin=2cm,top=2.5cm,headheight=16pt,headsep=0.1in,heightrounded]{geometry}
\usepackage{fancyhdr}
\pagestyle{fancy}
\usepackage{tikz}


%Here I define some theorem styles and shortcut commands for symbols I use often
\theoremstyle{definition}
\newtheorem{defn}{Definition}
\newtheorem{thm}{Theorem}
\newtheorem{cor}{Corollary}
\newtheorem*{rmk}{Remark}
\newtheorem{lem}{Lemma}
\newtheorem*{joke}{Joke}
\newtheorem{ex}{Example}
\newtheorem*{soln}{Solution}
\newtheorem{prop}{Proposition}

\newcommand{\lra}{\longrightarrow}
\newcommand{\ra}{\rightarrow}
\newcommand{\surj}{\twoheadrightarrow}
\newcommand{\graph}{\mathrm{graph}}
\newcommand{\bb}[1]{\mathbb{#1}}
\newcommand{\Z}{\bb{Z}}
\newcommand{\Q}{\bb{Q}}
\newcommand{\R}{\bb{R}}
\newcommand{\C}{\bb{C}}
\newcommand{\N}{\bb{N}}
\newcommand{\M}{\mathbf{M}}
\newcommand{\m}{\mathbf{m}}
\newcommand{\MM}{\mathscr{M}}
\newcommand{\HH}{\mathscr{H}}
\newcommand{\Om}{\Omega}
\newcommand{\Ho}{\in\HH(\Om)}
\newcommand{\bd}{\partial}
\newcommand{\del}{\partial}
\newcommand{\bardel}{\overline\partial}
\newcommand{\textdf}[1]{\textbf{\textsf{#1}}\index{#1}}
\newcommand{\img}{\mathrm{img}}
\newcommand{\ip}[2]{\left\langle{#1},{#2}\right\rangle}
\newcommand{\inter}[1]{\mathrm{int}{#1}}
\newcommand{\exter}[1]{\mathrm{ext}{#1}}
\newcommand{\cl}[1]{\mathrm{cl}{#1}}
\newcommand{\ds}{\displaystyle}
\newcommand{\vol}{\mathrm{vol}}
\newcommand{\cnt}{\mathrm{ct}}
\newcommand{\osc}{\mathrm{osc}}
\newcommand{\LL}{\mathbf{L}}
\newcommand{\UU}{\mathbf{U}}
\newcommand{\support}{\mathrm{support}}
\newcommand{\AND}{\;\wedge\;}
\newcommand{\OR}{\;\vee\;} 
\newcommand{\Oset}{\varnothing}
\newcommand{\st}{\ni}
\newcommand{\wh}{\widehat}
\newcommand{\vect}[1]{\overrightarrow{#1}}

%Pagination stuff.
\setlength{\topmargin}{-.3 in}
%\setlength{\oddsidemargin}{0in}
%\setlength{\evensidemargin}{0in}
\setlength{\textheight}{9.in}
\setlength{\textwidth}{6.5in}
\cfoot{page \thepage}
\lhead{MEU303 - Alg\`ebre}
\rhead{Cours 1}
\pagestyle{fancy}


\begin{document}

\subsection*{Rappel de cours}
\begin{defn}
Bla bla
\end{defn}



\newpage
\subsection*{II.1 Exercice 1}

Comme $x$ est une solution de $E$ on a :
$$
x'= -\cos(t)x -e^{t+\sin(t)}x^2
$$
on prend $y=1/x$. Donc
$$
y'=-\frac{x'}{x^2} = -\frac{-\cos(t)x -e^{t+\sin(t)}x^2}{x^{2}} = \frac{\cos(t)}{x} + e^{t+\sin(t)} = \cos(t)y + e^{t+\sin(t)}
$$
L'\'equation diff\'erentielle sous forme lin\'eaire est $y'- \cos(t)y = e^{t+\sin(t)} $

1- Trouver la solution g\'en\'erale $y_c$ de l'\'equation homog\`ene: $y' = \cos(t)y$. Donc $y_c(t) = Ce^{\sin(t)}$. 

2 - Trouver une solution particuli\`ere $y_0$ de $y'- \cos(t)y = e^{t+\sin(t)}$. Prenons $y(t)=e^{t-\sin(t)}$, on a $y'(t)=(1+\cos(t))e^{1-\sin(t)}$. 
$$
y'- \cos(t)y = e^{t+\sin(t)}
$$
$$
(1+\cos(t))e^{1-\sin(t)}- \cos(t)e^{t-\sin(t)} = e^{t+\sin(t)}
$$
Vrai.

3 - La solution g\'en\'erale est $y(t) = Ce^{\sin(t)} + e^{t+\sin(t)}$

La solution de l'\'equation $E$ est $x=1/y$.

\end{document}

