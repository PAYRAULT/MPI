\documentclass[]{book}

%These tell TeX which packages to use.
\usepackage{array,epsfig}
\usepackage{amsmath}
\usepackage{amsfonts}
\usepackage{amssymb}
\usepackage{amsxtra}
\usepackage{amsthm}
\usepackage{mathrsfs}
\usepackage{color}
\usepackage[margin=2cm,top=2.5cm,headheight=16pt,headsep=0.1in,heightrounded]{geometry}
\usepackage{fancyhdr}
\pagestyle{fancy}
\usepackage{tikz}


%Here I define some theorem styles and shortcut commands for symbols I use often
\theoremstyle{definition}
\newtheorem{defn}{Definition}
\newtheorem{thm}{Theorem}
\newtheorem{cor}{Corollary}
\newtheorem*{rmk}{Remark}
\newtheorem{lem}{Lemma}
\newtheorem*{joke}{Joke}
\newtheorem{ex}{Example}
\newtheorem*{soln}{Solution}
\newtheorem{prop}{Proposition}

\newcommand{\lra}{\longrightarrow}
\newcommand{\ra}{\rightarrow}
\newcommand{\surj}{\twoheadrightarrow}
\newcommand{\graph}{\mathrm{graph}}
\newcommand{\bb}[1]{\mathbb{#1}}
\newcommand{\Z}{\bb{Z}}
\newcommand{\Q}{\bb{Q}}
\newcommand{\R}{\bb{R}}
\newcommand{\C}{\bb{C}}
\newcommand{\N}{\bb{N}}
\newcommand{\M}{\mathbf{M}}
\newcommand{\m}{\mathbf{m}}
\newcommand{\MM}{\mathscr{M}}
\newcommand{\HH}{\mathscr{H}}
\newcommand{\Om}{\Omega}
\newcommand{\Ho}{\in\HH(\Om)}
\newcommand{\bd}{\partial}
\newcommand{\del}{\partial}
\newcommand{\bardel}{\overline\partial}
\newcommand{\textdf}[1]{\textbf{\textsf{#1}}\index{#1}}
\newcommand{\img}{\mathrm{img}}
\newcommand{\ip}[2]{\left\langle{#1},{#2}\right\rangle}
\newcommand{\inter}[1]{\mathrm{int}{#1}}
\newcommand{\exter}[1]{\mathrm{ext}{#1}}
\newcommand{\cl}[1]{\mathrm{cl}{#1}}
\newcommand{\ds}{\displaystyle}
\newcommand{\vol}{\mathrm{vol}}
\newcommand{\cnt}{\mathrm{ct}}
\newcommand{\osc}{\mathrm{osc}}
\newcommand{\LL}{\mathbf{L}}
\newcommand{\UU}{\mathbf{U}}
\newcommand{\support}{\mathrm{support}}
\newcommand{\AND}{\;\wedge\;}
\newcommand{\OR}{\;\vee\;} 
\newcommand{\Oset}{\varnothing}
\newcommand{\st}{\ni}
\newcommand{\wh}{\widehat}
\newcommand{\vect}[1]{\overrightarrow{#1}}

%Pagination stuff.
\setlength{\topmargin}{-.3 in}
%\setlength{\oddsidemargin}{0in}
%\setlength{\evensidemargin}{0in}
\setlength{\textheight}{9.in}
\setlength{\textwidth}{6.5in}
\cfoot{page \thepage}
\lhead{MEU303 - Alg\`ebre}
\rhead{Cours 1}
\pagestyle{fancy}


\begin{document}

\subsection*{Rappel de cours}
\begin{defn}
Bla bla
\end{defn}



\newpage
\subsection*{Question 1}
On a $Y = (H, P)$ et $Y' = F(Y)$ avec 
$$
\left\{
\begin{array}{l}
F_1(x_1, x_2) = a x_1 - b x_1 x_2 \\
F_2(x_1, x_2) = -c x_2 + dx_1x_2 \\
\end{array}
\right.
$$

\subsection*{Question 2}
$$
\left\{
\begin{array}{l}
F_1(H_{eq}, P_{eq}) = 0 \\
F_2(H_{eq}, P_{eq}) = 0 \\
\end{array}
\right.
$$

$$
\left\{
\begin{array}{l}
a H_{eq} - b H_{eq} P_{eq} = 0 \\
-cP_{eq} + d H_{eq} P_{eq} = 0 \\
\end{array}
\right.
$$
Premi\`ere solution triviale $H_{eq} = P_{eq} = 0$.

Avec $H_{eq} \neq 0$ et $P_{eq} \neq 0$
$$
\left\{
\begin{array}{l}
a  - b  P_{eq} = 0 \\
-c + d H_{eq}  = 0 \\
\end{array}
\right.
$$

Seconde solution: $P_{eq} = \frac{a}{b}$ et $H_{eq} = \frac{c}{d}$

Si on choisit $(H_{eq}, P_{eq})$ comme condition initiale on a $H'(t) = 0$ et $P'(t) = 0$ donc les populations restent constante dans le temps.

\subsection*{Question 3}
Regardons comment evolue $H'$ et $P'$ lorsque autour du point $(H_{eq}, P_{eq}$.

On a 
$$
\left\{
\begin{array}{l}
a (H_{eq}+\epsilon_1) - b (H_{eq}++\epsilon_1)(P_{eq}+\epsilon_2) = 0 \\
-c(P_{eq}+\epsilon_2) + d (H_{eq}++\epsilon_1)(P_{eq}+\epsilon_2) = 0 \\
\end{array}
\right.
$$



\end{document}

