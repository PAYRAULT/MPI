\documentclass[]{book}

%These tell TeX which packages to use.
\usepackage{array,epsfig}
\usepackage{amsmath}
\usepackage{amsfonts}
\usepackage{amssymb}
\usepackage{amsxtra}
\usepackage{amsthm}
\usepackage{mathrsfs}
\usepackage{color}

%Here I define some theorem styles and shortcut commands for symbols I use often
\theoremstyle{definition}
\newtheorem{defn}{Definition}
\newtheorem{thm}{Theorem}
\newtheorem{cor}{Corollary}
\newtheorem*{rmk}{Remark}
\newtheorem{lem}{Lemma}
\newtheorem*{joke}{Joke}
\newtheorem{ex}{Example}
\newtheorem*{soln}{Solution}
\newtheorem{prop}{Proposition}

\newcommand{\lra}{\longrightarrow}
\newcommand{\ra}{\rightarrow}
\newcommand{\surj}{\twoheadrightarrow}
\newcommand{\graph}{\mathrm{graph}}
\newcommand{\bb}[1]{\mathbb{#1}}
\newcommand{\Z}{\bb{Z}}
\newcommand{\D}{\bb{D}}
\newcommand{\Q}{\bb{Q}}
\newcommand{\R}{\bb{R}}
\newcommand{\C}{\bb{C}}
\newcommand{\N}{\bb{N}}
\newcommand{\M}{\mathbf{M}}
\newcommand{\m}{\mathbf{m}}
\newcommand{\MM}{\mathscr{M}}
\newcommand{\HH}{\mathscr{H}}
\newcommand{\Om}{\Omega}
\newcommand{\Ho}{\in\HH(\Om)}
\newcommand{\bd}{\partial}
\newcommand{\del}{\partial}
\newcommand{\bardel}{\overline\partial}
\newcommand{\textdf}[1]{\textbf{\textsf{#1}}\index{#1}}
\newcommand{\img}{\mathrm{img}}
\newcommand{\ip}[2]{\left\langle{#1},{#2}\right\rangle}
\newcommand{\inter}[1]{\mathrm{int}{#1}}
\newcommand{\exter}[1]{\mathrm{ext}{#1}}
\newcommand{\cl}[1]{\mathrm{cl}{#1}}
\newcommand{\ds}{\displaystyle}
\newcommand{\vol}{\mathrm{vol}}
\newcommand{\cnt}{\mathrm{ct}}
\newcommand{\osc}{\mathrm{osc}}
\newcommand{\LL}{\mathbf{L}}
\newcommand{\UU}{\mathbf{U}}
\newcommand{\support}{\mathrm{support}}
\newcommand{\AND}{\;\wedge\;}
\newcommand{\OR}{\;\vee\;}
\newcommand{\Oset}{\varnothing}
\newcommand{\st}{\ni}
\newcommand{\wh}{\widehat}

%Pagination stuff.
\setlength{\topmargin}{-.3 in}
\setlength{\oddsidemargin}{0in}
\setlength{\evensidemargin}{0in}
\setlength{\textheight}{9.in}
\setlength{\textwidth}{6.5in}
\pagestyle{empty}



\begin{document}

Rappel de cours :
\begin{itemize}
\item Si le point $I$ est le milieu d'un segment $AB$ alors $k\overrightarrow{IA}+k\overrightarrow{IB} = \overrightarrow{0}$, donc $I=Bari((A,k),(B,k))$.
\item Si $Bari((A,m_a), (A,n_a), ...) = Bari((A,m_a+n_a), ...)$.
\item Baricentre partiel. Si $F = Bari((B_j, n_j))$ alors $Bari((A_i,m_i),(B_j, n_j)) = Bari((A_i,m_i),(F,\sum n_j))$
\end{itemize}


\subsection*{Question 1.a}
Les distances $AB$ et $A'B'$ sont identiques, montrons qu'il existe une isom\'etrie $\phi(z) =  az+ b$ qui transforme $A' = \phi(A)$ et $B' = \phi(B)$ 
\begin{itemize}
\item soit $a=1$, donc la transformation $\phi$ est la translation $\overrightarrow{AA'}$.
\item soit $a \neq 1$. Il existe un angle $\theta$ tel que $a=e^{i\theta}$. Pour que $\phi$ soit une rotation alors $\phi(z) =  c + d(z - c) = dz +c(1-d)$. Prenons, $d = a = e^{i\theta}$ et $b=c(1-d)=c(1-a)$. Alors $\phi$ est la rotation de centre $c$ et d'angle $\theta$. 
\end{itemize}

Les valeurs de $a$ et $b$ sont uniques donc la transformation $\phi$ est unique.

\subsection*{Question 1.b}
La transformation $\phi$ est une translation lorsque $a=1=e^{i\theta}$. Donc $\theta = 0$, par cons\'equent les droites $AB$ et $A'B'$ sont parall\'eles. 

\subsection*{Question 1.c}
??


\subsection*{Question 1.d}
??

\subsection*{Question 1.e}
$\phi$ est une isom\'etrie donc $AB = A'B'$, Soit $D = \phi(D)$, on a $AD = AC$ et $BD = BC$ car $\phi$ est une isom\'etrie.
De m\^eme, l'angle $(AB, AC) = (A'B', A'D)$. $D$ est le point tel que $AD=AC$ et  $(AB, AC) = (A'B', A'D)$. Donc $\phi(C)=D=C'$.

\subsection*{Question 2.a}
La transformation est une translation.

\subsection*{Question 2.b}

\subsection*{Question 3}
Soit $s$ la r\'eflexion d'axe $\D$.
La rotation $\phi$ de centre $O$ et d'angle $-\theta$ de la droite $\D$ est l'axe des abscisses. La r\'eflexion sur l'axe des abscisses, $\phi_a$  d'un point A est le conjugu\'e du point A, $\phi_a(A) = \bar(A)$. Donc en utilisant le principe de conjugaison on a 
$$\phi_a = \phi \bullet s \bullet \phi^{-1}$$
$$\phi^{-1} \bullet \phi_a \bullet \phi = s$$
$$s = e^{i\theta}.\bar{(e^{-i\theta}z0}$$

\subsection*{Question 4}
 





QED

\end{document}

