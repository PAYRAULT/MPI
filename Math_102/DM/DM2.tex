\documentclass[]{book}

%These tell TeX which packages to use.
\usepackage{array,epsfig}
\usepackage{amsmath}
\usepackage{amsfonts}
\usepackage{amssymb}
\usepackage{amsxtra}
\usepackage{amsthm}
\usepackage{mathrsfs}
\usepackage{color}

%Here I define some theorem styles and shortcut commands for symbols I use often
\theoremstyle{definition}
\newtheorem{defn}{Definition}
\newtheorem{thm}{Theorem}
\newtheorem{cor}{Corollary}
\newtheorem*{rmk}{Remark}
\newtheorem{lem}{Lemma}
\newtheorem*{joke}{Joke}
\newtheorem{ex}{Example}
\newtheorem*{soln}{Solution}
\newtheorem{prop}{Proposition}

\newcommand{\lra}{\longrightarrow}
\newcommand{\ra}{\rightarrow}
\newcommand{\surj}{\twoheadrightarrow}
\newcommand{\graph}{\mathrm{graph}}
\newcommand{\bb}[1]{\mathbb{#1}}
\newcommand{\Z}{\bb{Z}}
\newcommand{\Q}{\bb{Q}}
\newcommand{\R}{\bb{R}}
\newcommand{\C}{\bb{C}}
\newcommand{\N}{\bb{N}}
\newcommand{\M}{\mathbf{M}}
\newcommand{\m}{\mathbf{m}}
\newcommand{\MM}{\mathscr{M}}
\newcommand{\HH}{\mathscr{H}}
\newcommand{\Om}{\Omega}
\newcommand{\Ho}{\in\HH(\Om)}
\newcommand{\bd}{\partial}
\newcommand{\del}{\partial}
\newcommand{\bardel}{\overline\partial}
\newcommand{\textdf}[1]{\textbf{\textsf{#1}}\index{#1}}
\newcommand{\img}{\mathrm{img}}
\newcommand{\ip}[2]{\left\langle{#1},{#2}\right\rangle}
\newcommand{\inter}[1]{\mathrm{int}{#1}}
\newcommand{\exter}[1]{\mathrm{ext}{#1}}
\newcommand{\cl}[1]{\mathrm{cl}{#1}}
\newcommand{\ds}{\displaystyle}
\newcommand{\vol}{\mathrm{vol}}
\newcommand{\cnt}{\mathrm{ct}}
\newcommand{\osc}{\mathrm{osc}}
\newcommand{\LL}{\mathbf{L}}
\newcommand{\UU}{\mathbf{U}}
\newcommand{\support}{\mathrm{support}}
\newcommand{\AND}{\;\wedge\;}
\newcommand{\OR}{\;\vee\;}
\newcommand{\Oset}{\varnothing}
\newcommand{\st}{\ni}
\newcommand{\wh}{\widehat}

%Pagination stuff.
\setlength{\topmargin}{-.3 in}
\setlength{\oddsidemargin}{0in}
\setlength{\evensidemargin}{0in}
\setlength{\textheight}{9.in}
\setlength{\textwidth}{6.5in}
\pagestyle{empty}



\begin{document}

\subsection*{Question 3}
Prenons le rep\`ere de centre $O$, avec l'axe $Ox$ align\'e avec la droite $OA$. Dans ce rep\`ere, le point $A$ a l'affixe $a+i0$, le point $B = ae^{i\frac{\pi}{3}}$ car le triangle direct $ABC$ est un triangle isoc\`ele et $O$ est le centre du triangle donc $\Vert OA \Vert = \Vert OB \Vert = \Vert OC \Vert$. Les droites $OA$ et $OB$ sont \`a $\frac{\pi}{3}$. L'affixe du vecteur unitaire $\overrightarrow{u}$ dirigeant la droite $OB$ est $\frac{a.e^{\frac{i\pi}{3}}}{\Vert OB \Vert} = \frac{\Vert OA \Vert.e^{\frac{i\pi}{3}}}{\Vert OB \Vert} = e^{\frac{i\pi}{3}}$. Les droites $OA$ et $OC$ sont \`a $\frac{2\pi}{3}$. L'affixe du vecteur unitaire $\overrightarrow{u}$ dirigeant la droite $OC$ est $\frac{a.e^{\frac{2i\pi}{3}}}{\Vert OC \Vert} = \frac{\Vert OA \Vert.e^{\frac{2i\pi}{3}}}{\Vert OC \Vert} = e^{\frac{2i\pi}{3}}$.\\

Dans ce rep\`ere, la rotation de centre $O$ et d'angle $\frac{2\pi}{3}$ est $r(z) = e^{\frac{2\pi}{3}}z$.\\

Dans ce rep\`ere, on a $\sigma_1(z) = \overline{z}$ (ie r\'eflexion sur l'axe $Ox$).\\

Dans ce rep\`ere, on a  $\sigma_2(z) = (e^{\frac{i\pi}{3}})^2.\overline{z} = e^{\frac{2i\pi}{3}}.\overline{z}$ (r\'eflexion de droite $OB$). \\

Dans ce rep\`ere, on a  $\sigma_3(z) = (e^{\frac{i2\pi}{3}})^2.\overline{z} = e^{\frac{i4\pi}{3}}.\overline{z} $ (r\'eflexion de droite $OC$). \\

Donc $\sigma_1 \circ \sigma_2(z) = \overline{e^{\frac{i2\pi}{3}}.\overline{z}} = e^{\frac{-i2\pi}{3}}.z = e^{\frac{i4\pi}{3}}.z = r \circ r(z)$.\\

Donc $\sigma_1 \circ \sigma_3(z) = \overline{e^{\frac{i4\pi}{3}}.\overline{z}} = e^{\frac{-i4\pi}{3}}.z = e^{\frac{i2\pi}{3}}.z = r(z)$.\\


\subsection*{Question 4}
On a $\sigma_1 \circ \sigma_1 = Id$ car une r\'eflexion est une involution.\\
On a
$$\sigma_1 \circ \sigma_2 = r \circ r$$
$$\sigma_1 \circ \sigma_1 \circ \sigma_2 = \sigma_1 \circ r \circ r$$
$$\sigma_2 = \sigma_1 \circ r \circ r$$
$$\sigma_2 = s \circ r \circ r$$
$$\sigma_2 = s \circ r^2$$

On a
$$\sigma_1 \circ \sigma_3 = r$$
$$\sigma_1 \circ \sigma_1 \circ \sigma_3 = \sigma_1 \circ r$$
$$\sigma_3 = \sigma_1 \circ r$$
$$\sigma_3 = s \circ r$$


\subsection*{Question 5}
On a
$$ (s \circ r) \circ (s \circ r) = \sigma_3 \circ \sigma_3 = Id$$
 

QED

\end{document}

