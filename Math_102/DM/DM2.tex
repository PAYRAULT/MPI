\documentclass[]{book}

%These tell TeX which packages to use.
\usepackage{array,epsfig}
\usepackage{amsmath}
\usepackage{amsfonts}
\usepackage{amssymb}
\usepackage{amsxtra}
\usepackage{amsthm}
\usepackage{mathrsfs}
\usepackage{color}

%Here I define some theorem styles and shortcut commands for symbols I use often
\theoremstyle{definition}
\newtheorem{defn}{Definition}
\newtheorem{thm}{Theorem}
\newtheorem{cor}{Corollary}
\newtheorem*{rmk}{Remark}
\newtheorem{lem}{Lemma}
\newtheorem*{joke}{Joke}
\newtheorem{ex}{Example}
\newtheorem*{soln}{Solution}
\newtheorem{prop}{Proposition}

\newcommand{\lra}{\longrightarrow}
\newcommand{\ra}{\rightarrow}
\newcommand{\surj}{\twoheadrightarrow}
\newcommand{\graph}{\mathrm{graph}}
\newcommand{\bb}[1]{\mathbb{#1}}
\newcommand{\Z}{\bb{Z}}
\newcommand{\Q}{\bb{Q}}
\newcommand{\R}{\bb{R}}
\newcommand{\C}{\bb{C}}
\newcommand{\N}{\bb{N}}
\newcommand{\M}{\mathbf{M}}
\newcommand{\m}{\mathbf{m}}
\newcommand{\MM}{\mathscr{M}}
\newcommand{\HH}{\mathscr{H}}
\newcommand{\Om}{\Omega}
\newcommand{\Ho}{\in\HH(\Om)}
\newcommand{\bd}{\partial}
\newcommand{\del}{\partial}
\newcommand{\bardel}{\overline\partial}
\newcommand{\textdf}[1]{\textbf{\textsf{#1}}\index{#1}}
\newcommand{\img}{\mathrm{img}}
\newcommand{\ip}[2]{\left\langle{#1},{#2}\right\rangle}
\newcommand{\inter}[1]{\mathrm{int}{#1}}
\newcommand{\exter}[1]{\mathrm{ext}{#1}}
\newcommand{\cl}[1]{\mathrm{cl}{#1}}
\newcommand{\ds}{\displaystyle}
\newcommand{\vol}{\mathrm{vol}}
\newcommand{\cnt}{\mathrm{ct}}
\newcommand{\osc}{\mathrm{osc}}
\newcommand{\LL}{\mathbf{L}}
\newcommand{\UU}{\mathbf{U}}
\newcommand{\support}{\mathrm{support}}
\newcommand{\AND}{\;\wedge\;}
\newcommand{\OR}{\;\vee\;}
\newcommand{\Oset}{\varnothing}
\newcommand{\st}{\ni}
\newcommand{\wh}{\widehat}

%Pagination stuff.
\setlength{\topmargin}{-.3 in}
\setlength{\oddsidemargin}{0in}
\setlength{\evensidemargin}{0in}
\setlength{\textheight}{9.in}
\setlength{\textwidth}{6.5in}
\pagestyle{empty}



\begin{document}

\subsection*{Question 3}
Prenons le rep\`ere de centre $O$, avec l'axe $Ox$ align\'e avec la droite $OA$. Dans ce rep\`ere, le point $A$ a l'affixe $a+i0$, le point $B = ae^{i\frac{\pi}{3}}$ car le triangle direct $ABC$ est un triangle isoc\`ele et $O$ est le centre du triangle donc $\Vert OA \Vert = \Vert OB \Vert = \Vert OC \Vert$. Les droites $OA$ et $OB$ sont \`a $\frac{2\pi}{3}$. L'affixe du vecteur unitaire $\overrightarrow{u}$ dirigeant la droite $OB$ est $\frac{a.e^{\frac{i2\pi}{3}}}{\Vert OB \Vert} = \frac{\Vert OA \Vert.e^{\frac{i2\pi}{3}}}{\Vert OB \Vert} = e^{2\frac{i\pi}{3}}$. Les droites $OA$ et $OC$ sont \`a $\frac{\pi}{3}$. L'affixe du vecteur unitaire $\overrightarrow{u}$ dirigeant la droite $OC$ est $\frac{a.e^{\frac{i\pi}{3}}}{\Vert OC \Vert} = \frac{\Vert OA \Vert.e^{\frac{i\pi}{3}}}{\Vert OC \Vert} = e^{\frac{i\pi}{3}}$.\\

Dans ce rep\`ere, la rotation de centre $O$ et d'angle $\frac{2\pi}{3}$ est $r(z) = e^{\frac{2\pi}{3}}z$.\\

Dans ce rep\`ere, on a $\sigma_1(z) = \overline{z}$ (ie r\'eflexion sur l'axe $Ox$).\\

Dans ce rep\`ere, on a  $\sigma_2(z) = (e^{\frac{i2\pi}{3}})^2.\overline{z} = e^{\frac{i4\pi}{3}}.\overline{z}$ (r\'eflexion de droite $OB$). \\

Dans ce rep\`ere, on a  $\sigma_3(z) = (e^{\frac{i\pi}{3}})^2.\overline{z} = e^{\frac{i2\pi}{3}}.\overline{z} $ (r\'eflexion de droite $OC$). \\

Donc $\sigma_1 \circ \sigma_2(z) = \overline{e^{\frac{i4\pi}{3}}.\overline{z}} = e^{\frac{-i4\pi}{3}}.z = e^{\frac{i2\pi}{3}}.z =  r(z)$.\\

Donc $\sigma_1 \circ \sigma_3(z) = \overline{e^{\frac{i2\pi}{3}}.\overline{z}} = e^{\frac{-i2\pi}{3}}.z = e^{\frac{i4\pi}{3}}.z = r \circ r(z)$.\\


\subsection*{Question 4}
On a $\sigma_1 \circ \sigma_1 = Id$ car une r\'eflexion est une involution.\\
On a
$$\sigma_1 \circ \sigma_2 = r$$
$$\sigma_1 \circ \sigma_1 \circ \sigma_2 = \sigma_1 \circ r$$
$$Id \circ \sigma_2 = \sigma_1 \circ r$$
$$\sigma_2 = s \circ r $$

On a
$$\sigma_1 \circ \sigma_3 = r \circ r$$
$$\sigma_1 \circ \sigma_1 \circ \sigma_3 = \sigma_1 \circ r \circ r$$
$$Id \circ \sigma_3 = \sigma_1 \circ r \circ r$$
$$\sigma_3 = s \circ r \circ r = s \circ r^2$$


\subsection*{Question 5}
On a $\sigma_2 \circ \sigma_2 = Id$ car une r\'eflexion est une involution. $s=s^{-1} \circ Id$ \\

On a
$$ (s \circ r) \circ (s \circ r) = \sigma_2 \circ \sigma_2 = Id$$

On a
$$ (s \circ r) \circ (s \circ r) = Id$$
$$ s \circ r \circ s \circ r = Id$$
$$ s^{-1} \circ s \circ r \circ s \circ r = s^{-1} \circ Id$$
$$ s^{-1} \circ s \circ r \circ s \circ r \circ r{-1} = s^{-1} \circ Id \circ r^{-1}$$
$$ Id \circ r \circ s \circ Id = s \circ r^{-1}$$
$$ r \circ s = s \circ r^{-1}$$

\subsection*{Question 6}
Montrons d'abord que $\sigma_3 = s \circ r^{-1}$. On a $r \circ r \circ r = e^{\frac{2\pi}{3}} \circ e^{\frac{2\pi}{3}} e^{\frac{2\pi}{3}} = e^{\frac{6\pi}{3}} = e^{2\pi} = Id$ :
$$\sigma_3 = s \circ r \circ r$$
$$\sigma_3 \circ r = s \circ r \circ r \circ r$$
$$\sigma_3 \circ r = s \circ Id$$
$$\sigma_3 \circ r \circ r^{-1} = s \circ Id  \circ r^{-1}$$
$$\sigma_3 = s \circ r^{-1} = r \circ s$$

Donc 
$$E = \sigma_3(D') = \sigma_3(\sigma_1(D)) = \sigma_3 \circ \sigma_1(D)$$
$$\sigma_3 \circ \sigma_1(D) = r \circ s \circ s(D) = r(D)$$

Et
$$E' = \sigma_2(E) = \sigma_2(\sigma_3(D')) = \sigma_2 \circ \sigma_3(D')$$
de la question 4, $\sigma_2 = s \circ r$ et $\sigma_2 = s \circ r \circ r$,
$$\sigma_2 \circ \sigma_3(D') = s \circ r \circ s \circ r \circ r(D') = s \circ s \circ r^{-1} \circ r \circ r(D') = r(D')$$

Et 
$$F = \sigma_1(E') = \sigma_1(\sigma_2(E)) = \sigma_1 \circ \sigma_2(E)$$
de la question 4, $\sigma_2 = s \circ r$,
$$\sigma_1 \circ \sigma_2(E) = s \circ s \circ r(E) = r(E)$$

Et 
$$F' = \sigma_3(F) = \sigma_3(\sigma_1(E')) = \sigma_3 \circ \sigma_1(E')$$
$$\sigma_3 \circ \sigma_1(E') = r \circ s \circ s(E') = r(E')$$

\subsection*{Question 7}
On a $E = r(D)$ et $F = r(E)$. Donc $\Vert \overrightarrow{OD} \Vert = \Vert \overrightarrow{OE} \Vert = \Vert \overrightarrow{OF} \Vert$. Donc les triangles $EOD$, $EOF$, $FOD$ sont isoc\`eles.\\

Et $\widehat{DOE} = \widehat{EOF} = \frac{2\pi}{3}$ donc $\widehat{FOD} = \frac{2\pi}{3}$ car $r \circ r \circ r = Id$.\\

Donc $\widehat{ODE} = \widehat{OED} = \frac{\pi}{6}$, $\widehat{OEF} = \widehat{OFE} = \frac{\pi}{6}$, $\widehat{OFD} = \widehat{ODF} = \frac{\pi}{6}$ 

Donc $\widehat{DEF} = \widehat{DEO} + \widehat{OEF} = \frac{\pi}{6} + \frac{\pi}{6} = \frac{\pi}{3}$ et 
$\widehat{DEF} = \widehat{DEO} + \widehat{OEF} = \frac{\pi}{6} + \frac{\pi}{6} = \frac{\pi}{3}$ et 
$\widehat{EFD} = \widehat{EFO} + \widehat{OFD} = \frac{\pi}{6} + \frac{\pi}{6} = \frac{\pi}{3}$ et
$\widehat{FDE} = \widehat{FDO} + \widehat{ODE} = \frac{\pi}{6} + \frac{\pi}{6} = \frac{\pi}{3}$.\\

Donc, le triangle $DEF$ est \'equilat\'eral.
Le triangle est direct car $O$ est le centre du cercle circonscrit du triangle $DEF$ et les angles  $\widehat{DOE} = \widehat{EOF} = \frac{2\pi}{3}$.

D\'emonstration identique pour le triangle $D'E'F'$.\\


\subsection*{Question 8}
On a $E=r(D)$, donc $D = r^{-1}(E)$, donc l'angle orient\'e $(\overrightarrow{OE},\overrightarrow{OD})$ est $-\frac{2\pi}{3}$.\\

On a $D'D \parallel BC$ car $D'=\sigma_1(D)$ et $B=\sigma_1(C)$ (la droite $OA$ est la hauteur du cot\'e $BC$). On a $D'E \parallel AB$ pour la m\^eme raison. Et l'angle orient'e $(\overrightarrow{AB},\overrightarrow{AC}) = \frac{\pi}{3}$ car le triangle $ABC$ est direct.

Il y a deux cas:
\begin{itemize}
\item   
\item  
\end{itemize}

\subsection*{Question 9}


QED

\end{document}

