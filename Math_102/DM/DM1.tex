\documentclass[]{book}

%These tell TeX which packages to use.
\usepackage{array,epsfig}
\usepackage{amsmath}
\usepackage{amsfonts}
\usepackage{amssymb}
\usepackage{amsxtra}
\usepackage{amsthm}
\usepackage{mathrsfs}
\usepackage{color}

%Here I define some theorem styles and shortcut commands for symbols I use often
\theoremstyle{definition}
\newtheorem{defn}{Definition}
\newtheorem{thm}{Theorem}
\newtheorem{cor}{Corollary}
\newtheorem*{rmk}{Remark}
\newtheorem{lem}{Lemma}
\newtheorem*{joke}{Joke}
\newtheorem{ex}{Example}
\newtheorem*{soln}{Solution}
\newtheorem{prop}{Proposition}

\newcommand{\lra}{\longrightarrow}
\newcommand{\ra}{\rightarrow}
\newcommand{\surj}{\twoheadrightarrow}
\newcommand{\graph}{\mathrm{graph}}
\newcommand{\bb}[1]{\mathbb{#1}}
\newcommand{\Z}{\bb{Z}}
\newcommand{\Q}{\bb{Q}}
\newcommand{\R}{\bb{R}}
\newcommand{\C}{\bb{C}}
\newcommand{\N}{\bb{N}}
\newcommand{\M}{\mathbf{M}}
\newcommand{\m}{\mathbf{m}}
\newcommand{\MM}{\mathscr{M}}
\newcommand{\HH}{\mathscr{H}}
\newcommand{\Om}{\Omega}
\newcommand{\Ho}{\in\HH(\Om)}
\newcommand{\bd}{\partial}
\newcommand{\del}{\partial}
\newcommand{\bardel}{\overline\partial}
\newcommand{\textdf}[1]{\textbf{\textsf{#1}}\index{#1}}
\newcommand{\img}{\mathrm{img}}
\newcommand{\ip}[2]{\left\langle{#1},{#2}\right\rangle}
\newcommand{\inter}[1]{\mathrm{int}{#1}}
\newcommand{\exter}[1]{\mathrm{ext}{#1}}
\newcommand{\cl}[1]{\mathrm{cl}{#1}}
\newcommand{\ds}{\displaystyle}
\newcommand{\vol}{\mathrm{vol}}
\newcommand{\cnt}{\mathrm{ct}}
\newcommand{\osc}{\mathrm{osc}}
\newcommand{\LL}{\mathbf{L}}
\newcommand{\UU}{\mathbf{U}}
\newcommand{\support}{\mathrm{support}}
\newcommand{\AND}{\;\wedge\;}
\newcommand{\OR}{\;\vee\;}
\newcommand{\Oset}{\varnothing}
\newcommand{\st}{\ni}
\newcommand{\wh}{\widehat}

%Pagination stuff.
\setlength{\topmargin}{-.3 in}
\setlength{\oddsidemargin}{0in}
\setlength{\evensidemargin}{0in}
\setlength{\textheight}{9.in}
\setlength{\textwidth}{6.5in}
\pagestyle{empty}



\begin{document}

\subsection*{Question 7}
On prend la vecteur $\overrightarrow{AB}/|AB|$ comme vecteur unit\'e support de l'axe des abcisses et le point $A$ comme origine du rep\`ere orthonormal. Dans ce rep\`ere les coordonne\'es des points $A$ et $B$ sont $(0,0)$ et $(r,0)$. 


\subsection*{Question 8}

Rappel de cours: Soit un point $M$ de coordonn\'es $(x_M,y_M)$. Dans le plan complexe, le point $M$ correspond \`a l'\'equation $x_m + iy_m$. 
\begin{itemize}
\item Le point sym\'etrique par rapport \`a l'abcisse est $M_1$ avec les coordonn\'ees $(x_{M1},-y_{M1}) =(x_M,-y_M)$ soit $x_m - iy_m$. Dans le plan complexe, la transformation $s$ d'un point $M$ correspond au conjugu\'e du point $M$.
\item La rotation de centre $C$ et d'angle $\theta$ d'un point $M$ est \'egale \`a $C+e^{i\theta}(M - C)$    
\item La translation d'un point $M$ par rapport \`a un vecteur $V=(V_x, V_y)$ est $M1 = (x_M+V_x) + i(y_M+V_y)$.
\end{itemize}


SOit $M$ un point du plan complexe de coordonn\'ee $(x_M, y_M)$.\\

\begin{itemize}
\item L'expression complexe de $s(M)$ est le conjugu\'e du point $M$. Donc, $s(M) = x_M - iy_M$.
\item L'expression complexe de $r^{-}_A$ correspond \`a la rotation $-\frac{\pi}{2}$ par rapport au point $A$ du plan complexe. Donc, $r^{-}_A(M) = 0 + e^{-i\frac{\pi}{2}}(M-0) = (cos(-\frac{\pi}{2})+i.sin(-\frac{\pi}{2}))M = -i.M = y_M - ix_M$.
\item L'expression complexe de $r^{+}_B$ correspond \`a la rotation $\frac{\pi}{2}$ par rapport au point $B$ du plan complexe. Donc, $r^{+}_B(M) = r + e^{i\frac{\pi}{2}}(M-r) = r + (cos(\frac{\pi}{2})+i.sin(\frac{\pi}{2}))(M-r) = r + i(M-r) = r-y_M + i(x_M-r)$. 
\end{itemize}


\subsection*{Question 9}
Soit $M$ le point d'affixe $z=x_M + i y_M)$. 
\begin{itemize}
\item $M1 = s(M)$, et $z_1 = x_M -i y_M$. 
\item $M2 = r_{A}^{-}(M1)$, et $z_2 = -i(x_M -i y_M) = -y_M - i x_M$. 
\item $M3 = s(M2)$, et $z_3 = = -y_M + i x_M$.
\item $M4 = r_{B}^{+}(M3)$, et $z_4 = = r-x_M + i(-y_M-r)$.  
\end{itemize}
 
\subsection*{Question 10}
$\varphi(M) = r_{B}^{+} \circ s \circ r_{A}^{-} \circ s(M)) = r_{B}^{+}(s(r_{A}^{-}(s(M)))) = M4 = (r-x_M) + i(-y_M-r)$.


\subsection*{Question 11}
$\varphi$ est une sym\'etrie centrale si $\exists C,\, t.q.\, C\,est\,au\,milieu\,du\,segment\,[M,\varphi(M)]$. Calculons les coordonn\'ees du point $C$. $C = \frac{x_M + (-x_M + r)}{2} + ib-c\frac{y_M + (-y_M - r)}{2} = \frac{r}{2} + i\frac{-r}{2}$.\\
Le point $C$ existe car les coordonn\'ees du point $C$ sont fixes (ie. ind\'ependantes du point de d\'epart $M$). Donc, la transformation $\varphi$ est une sym\'etrie centrale.

\subsection*{Question 12}
Les coordonn\'ees de $A$, $B$ et $C$ sont $0+i0$, $r+i0$, $\frac{r}{2} + i\frac{r}{2}$.
$$\frac{a-c}{b-c} = \frac{(0-r/2)+i(0-r/2)}{(r-r/2)+i(0-r/2)} = \frac{-(r/2+ir/2)}{r/2-ir/2} = -\frac{(r/2+ir/2)^2}{(r/2-ir/2)(r/2+ir/2)} = -\frac{2ir^2/4}{2r^2/4} = -i$$


$a-c$ correspond au vecteur $\overrightarrow{CA}$ et $b-c$ correspond au vecteur $\overrightarrow{CB}$. Et $|\overrightarrow{CA}| = \sqrt{(-r/2)^2+(-r/2)^2} = \sqrt{2}\frac{r}{2}$ et $|\overrightarrow{CB}| = \sqrt{(r/2)^2+(-r/2)^2} = \sqrt{2}\frac{r}{2}$. Donc $AC=BC$.\\

L'angle orient\'e $(\overrightarrow{CB}, \overrightarrow{CA}) = arg(\frac{a-c}{b-c}) = arg(-i) = -\frac{\pi}{2}$.


\subsection*{Question 13}
Le point $C$ est le centre de la sym\'etrie centrale $\varphi$, par d\'efinition, c'est le milieu du segment $[M,\varphi(M)]$. Le point $I$ est le centre du segment $[M, M4]$. Les points $\varphi(M)$ et $M4$ sont identiques, donc $I=C$. Les triangles $CAB$ et $IAB$ sont identiques. De la questions 12, on a montr\'e que l'angle $\widehat{ACB} = \pi - (\pi/4+\pi/4) = \pi/2$. Donc le triangle $IAC$ est rectangle en $I$. Comme $|IA|=|IB|$, le triangle est aussi isoc\`ele. 

\subsection*{Question 14}
Des questions pr\'ec\'edentes on a montr\'e que la transformation $\varphi$ est une symetrie centrale de centre $C = (r/2, r/2)$. Le centre de sym\'etrie de la tranformation est le m\^eme quelque soit les points $A$ et $B$. Il suffit ce calculer les coordonn\'ees du triangles isoc\`ele $ACB$, rectangle en $C$. Il y a 2 triangles possibles (un de chaque c\^ot\'e de la droite $AB$). Il faut prendre le triangle avec l'angle orient\'e $(\overrightarrow{CB}, \overrightarrow{CA})$ qui est n\'egatif.\\

Soit, $D$, le centre du segment $[AB]$. Ses coordonn\'ees sont $(\frac{1+2}{2}, \frac{1}{2})$. Le point $C$ est la rotation de $-\frac{\pi}{2}$ du point $A$ par rapport \`a $D$. Le vecteur $\overrightarrow{DA} = -1/2 -i/2$. La rotation de $-\frac{\pi}{2}$ du point $A$ par rapport \`a $D$ est $i(-1/2 -i/2) = i/2 - i/2$ qui est \'egale au vecteur $\overrightarrow{DC}$. Donc les coordonn\'ees du point $C$ sont $3/2-1/2+i(1/2-(-1/2)) = 1 + i$.
\\

QED

\end{document}

