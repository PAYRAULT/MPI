\documentclass[]{book}

%These tell TeX which packages to use.
\usepackage{array,epsfig}
\usepackage{amsmath}
\usepackage{amsfonts}
\usepackage{amssymb}
\usepackage{amsxtra}
\usepackage{amsthm}
\usepackage{mathrsfs}
\usepackage{color}
\usepackage[margin=2cm,top=2.5cm,headheight=16pt,headsep=0.1in,heightrounded]{geometry}
\usepackage{fancyhdr}
\pagestyle{fancy}
\usepackage{tikz}


%Here I define some theorem styles and shortcut commands for symbols I use often
\theoremstyle{definition}
\newtheorem{defn}{Definition}
\newtheorem{thm}{Theorem}
\newtheorem{cor}{Corollary}
\newtheorem*{rmk}{Remark}
\newtheorem{lem}{Lemma}
\newtheorem*{joke}{Joke}
\newtheorem{ex}{Example}
\newtheorem*{soln}{Solution}
\newtheorem{prop}{Proposition}

\newcommand{\lra}{\longrightarrow}
\newcommand{\ra}{\rightarrow}
\newcommand{\surj}{\twoheadrightarrow}
\newcommand{\graph}{\mathrm{graph}}
\newcommand{\bb}[1]{\mathbb{#1}}
\newcommand{\Z}{\bb{Z}}
\newcommand{\Q}{\bb{Q}}
\newcommand{\R}{\bb{R}}
\newcommand{\E}{\bb{E}}
\newcommand{\C}{\bb{C}}
\newcommand{\N}{\bb{N}}
\newcommand{\Pe}{\bb{P}}
\newcommand{\M}{\mathbf{M}}
\newcommand{\m}{\mathbf{m}}
\newcommand{\MM}{\mathscr{M}}
\newcommand{\HH}{\mathscr{H}}
\newcommand{\Om}{\Omega}
\newcommand{\Ho}{\in\HH(\Om)}
\newcommand{\bd}{\partial}
\newcommand{\del}{\partial}
\newcommand{\bardel}{\overline\partial}
\newcommand{\textdf}[1]{\textbf{\textsf{#1}}\index{#1}}
\newcommand{\img}{\mathrm{img}}
\newcommand{\ip}[2]{\left\langle{#1},{#2}\right\rangle}
\newcommand{\inter}[1]{\mathrm{int}{#1}}
\newcommand{\exter}[1]{\mathrm{ext}{#1}}
\newcommand{\cl}[1]{\mathrm{cl}{#1}}
\newcommand{\ds}{\displaystyle}
\newcommand{\vol}{\mathrm{vol}}
\newcommand{\cnt}{\mathrm{ct}}
\newcommand{\osc}{\mathrm{osc}}
\newcommand{\LL}{\mathbf{L}}
\newcommand{\UU}{\mathbf{U}}
\newcommand{\support}{\mathrm{support}}
\newcommand{\AND}{\;\wedge\;}
\newcommand{\OR}{\;\vee\;} 
\newcommand{\Oset}{\varnothing}
\newcommand{\st}{\ni}
\newcommand{\wh}{\widehat}
\newcommand{\vect}[1]{\overrightarrow{#1}}

%Pagination stuff.
%\setlength{\oddsidemargin}{0in}
%\setlength{\evensidemargin}{0in}
\setlength{\textheight}{9.in}
\setlength{\textwidth}{6.5in}
\cfoot{page \thepage}
\lhead{MEU302 - Alg\`ebre}
\rhead{TD2}
\pagestyle{fancy}


\begin{document}

\subsection*{Rappel de cours}
\begin{defn}
Bla bla
\end{defn}



\newpage
\subsection*{Exercice 2}
\subsection*{Exercice 2.1}
On a par d\'efinition 
$$
F_X(x) = \int_{-\infty}^{x}{\lambda e^{-\lambda x} 1_{[0,\infty]}} = \int_{0}^{x}{\lambda e^{-\lambda x}} = \left[-e^{-\lambda x}\right]_{0}^{x} = 1 - e^{-\lambda x}
$$

\subsection*{Exercice 2.2}
Calculons 
$$
F_{X}(F_X^{-1}(x)) = 1 - e^{-\lambda F_X^{-1}(x)} = x
$$

$$
e^{-\lambda F_X^{-1}(x)} = 1-x
$$

$$
-\lambda F_X^{-1}(x) = \ln(1-x)
$$

$$
F_X^{-1}(x) = -\frac{1}{\lambda} \ln(1-x)
$$


\subsection*{Exercice 2.4}
$$
\Pe(X > m_{\lambda}) = 1 - \Pe(X \leq m_{\lambda}) = 1 - F_X(m_{\lambda}) = 0.05
$$
Donc
$$
F_X(m_{\lambda}) = 1 - 0.05 = .95
$$
et
$$
m_{\lambda} = F_X^{-1}(.95) = -\frac{1}{\lambda} \ln(1-0.95) = -\frac{\ln(0.05)}{\lambda}
$$


\subsection*{Exercice 3}
\subsection*{Exercice 3.1}
On a par d\'efinition:
$$
F_X(x) = \int_{-\infty}^{x}{\pi^{-1}\frac{1}{1+y^2}dy} = \frac{1}{\pi}[arctan(x)]_{-\infty}^{x} = \frac{1}{\pi}(\arctan(x)- \arctan(-\infty)) = \frac{1}{\pi}\left(\arctan(x)+\frac{\pi}{2}\right) = \frac{\arctan(x)}{\pi}+\frac{1}{2}
$$

\subsection*{Exercice 3.2}
Calculons 
$$
F_{X}(F_X^{-1}(x)) = \frac{\arctan(F_X^{-1}(x))}{\pi}+\frac{1}{2} = x
$$

$$
\arctan(F_X^{-1}(x)) = \pi(x-\frac{1}{2})
$$

$$
F_X^{-1}(x) = \tan\left(\pi\left(x-\frac{1}{2}\right)\right)
$$


\subsection*{Exercice 3.4}
$$
\Pe(|X| > m) = 1 - \Pe(|X| \leq m) = 1 - F_X(m) = 0.05
$$
Donc
$$
F_X(m) = 1 - 0.05 = .95
$$
et
$$
m_ = F_X^{-1}(.95) = \tan\left(\pi\left(.95-\frac{1}{2}\right)\right) = 6.31
$$


\end{document}

