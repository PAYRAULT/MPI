\documentclass[]{book}

%These tell TeX which packages to use.
\usepackage{array,epsfig}
\usepackage{amsmath}
\usepackage{amsfonts}
\usepackage{amssymb}
\usepackage{amsxtra}
\usepackage{amsthm}
\usepackage{mathrsfs}
\usepackage{color}
\usepackage[margin=2cm,top=2.5cm,headheight=16pt,headsep=0.1in,heightrounded]{geometry}
\usepackage{fancyhdr}
\pagestyle{fancy}
\usepackage{tikz}


%Here I define some theorem styles and shortcut commands for symbols I use often
\theoremstyle{definition}
\newtheorem{defn}{Definition}
\newtheorem{thm}{Theorem}
\newtheorem{cor}{Corollary}
\newtheorem*{rmk}{Remark}
\newtheorem{lem}{Lemma}
\newtheorem*{joke}{Joke}
\newtheorem{ex}{Example}
\newtheorem*{soln}{Solution}
\newtheorem{prop}{Proposition}

\newcommand{\lra}{\longrightarrow}
\newcommand{\ra}{\rightarrow}
\newcommand{\surj}{\twoheadrightarrow}
\newcommand{\graph}{\mathrm{graph}}
\newcommand{\bb}[1]{\mathbb{#1}}
\newcommand{\Z}{\bb{Z}}
\newcommand{\Q}{\bb{Q}}
\newcommand{\R}{\bb{R}}
\newcommand{\E}{\bb{E}}
\newcommand{\C}{\bb{C}}
\newcommand{\N}{\bb{N}}
\newcommand{\M}{\mathbf{M}}
\newcommand{\m}{\mathbf{m}}
\newcommand{\MM}{\mathscr{M}}
\newcommand{\HH}{\mathscr{H}}
\newcommand{\Om}{\Omega}
\newcommand{\Ho}{\in\HH(\Om)}
\newcommand{\bd}{\partial}
\newcommand{\del}{\partial}
\newcommand{\bardel}{\overline\partial}
\newcommand{\textdf}[1]{\textbf{\textsf{#1}}\index{#1}}
\newcommand{\img}{\mathrm{img}}
\newcommand{\ip}[2]{\left\langle{#1},{#2}\right\rangle}
\newcommand{\inter}[1]{\mathrm{int}{#1}}
\newcommand{\exter}[1]{\mathrm{ext}{#1}}
\newcommand{\cl}[1]{\mathrm{cl}{#1}}
\newcommand{\ds}{\displaystyle}
\newcommand{\vol}{\mathrm{vol}}
\newcommand{\cnt}{\mathrm{ct}}
\newcommand{\osc}{\mathrm{osc}}
\newcommand{\LL}{\mathbf{L}}
\newcommand{\UU}{\mathbf{U}}
\newcommand{\support}{\mathrm{support}}
\newcommand{\AND}{\;\wedge\;}
\newcommand{\OR}{\;\vee\;} 
\newcommand{\Oset}{\varnothing}
\newcommand{\st}{\ni}
\newcommand{\wh}{\widehat}
\newcommand{\vect}[1]{\overrightarrow{#1}}

%Pagination stuff.
%\setlength{\oddsidemargin}{0in}
%\setlength{\evensidemargin}{0in}
\setlength{\textheight}{9.in}
\setlength{\textwidth}{6.5in}
\cfoot{page \thepage}
\lhead{MEU302 - Alg\`ebre}
\rhead{TD2}
\pagestyle{fancy}


\begin{document}

\subsection*{Rappel de cours}
\begin{defn}
Bla bla
\end{defn}



\newpage
\subsection*{Exercice 1}
\subsection*{Exercice 1.1}
Il faut trouver un espace de probabilit\'e $(\Omega, \mathcal{F}, P)$ avec $\Omega$ un univers, $\mathcal{F}$ un espace d'\'ev\'enements et $P$ un espace de probabilit\'e de $\mathcal{F} \to [0,1]$.
Pour $i \in {1,2,3}$, on a $P(X=i) = P(\{\omega \in \Omega \text{ tq } X^{-1}(i) = \omega\}) = 1/3$.  

Prenons l'espace de probabilit\'e $(\Omega, \mathcal{P}(\Omega), P)$ avec $P(\{\omega \in \Omega \text{ tq } X^{-1}(i) = \omega\}) = 1/3$.


\subsection*{Exercice 1.2}
L'ensemble des sous-parties de $\Omega$ est une tribu d'un espace de probabilit\'e $(\Omega, F,P)$. On a $(X=i) = \{\omega \in \Omega \text{ tq } X^{-1}(i) = \omega\}$. L'ensemble $\Omega$ contient au moins 3 \'el\'ements, donc $card(\Omega) \geq 3$ car il faut au moins une valeur de $\Omega$ pour chaque valeur de $X$. Donc, on a $card(\mathcal{P}(\Omega)) \geq 3^2 = 8$.




\subsection*{Exercice 2}
\subsection*{Exercice 2.1}
Soit $E$ l'\'ev\'enement sur lequel $X=\pi X$. Si $\omega \in E$ alors 
$$
\lim_{n \to \infty}{\cos(X(\omega))^{n}+\cos(2X(\omega))^{2n}} = \lim_{n \to \infty}{1^n+1^{2n} = 2}
$$

Si $\omega \notin E$ alors 
$$
\lim_{n \to \infty}{\cos(X(\omega))^{n}+\cos(2X(\omega))^{2n}} = \lim_{n \to \infty}{[0,1[^n+[0,1[^{2n} = 0}
$$ 

Donc $\lim_{n \to \infty}{\cos(X)^{n}+\cos(2X)^{2n}} = 2.1_{E}$. 

On a $\exists Z \text{ tq } \forall n, |X_n| \leq Z$ $\implies Z = 2$ et $\exists X \text{ tq } X = \lim{n \to \infty}{X_n}$ $\implies X = 2.1_E$. Donc on peut utiliser le th\'eor\`eme de convergence domin\'ee.
$$
\lim_{n \to \infty} \E[X_n] = \E[X] = \E[2.1_E] = 2P(E) = 2P(X \in \pi \Z) = 0
$$

\subsection*{Exercice 2.2}
m\^eme raisonnement
$$
\lim_{n \to \infty} \E[X_n] = \E[X] = \E[2.1_E] = 2P(E) = 2P(X \in \pi \Z) = 2p_1
$$


\end{document}

