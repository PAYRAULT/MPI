\documentclass[]{book}

%These tell TeX which packages to use.
\usepackage{array,epsfig}
\usepackage{amsmath}
\usepackage{amsfonts}
\usepackage{amssymb}
\usepackage{amsxtra}
\usepackage{amsthm}
\usepackage{mathrsfs}
\usepackage{color}
\usepackage[margin=2cm,top=2.5cm,headheight=16pt,headsep=0.1in,heightrounded]{geometry}
\usepackage{fancyhdr}
\pagestyle{fancy}
%\usepackage{tikz}
\usepackage{pgfplots}

%Here I define some theorem styles and shortcut commands for symbols I use often
\theoremstyle{definition}
\newtheorem{defn}{Definition}
\newtheorem{thm}{Theorem}
\newtheorem{cor}{Corollary}
\newtheorem*{rmk}{Remark}
\newtheorem{lem}{Lemma}
\newtheorem*{joke}{Joke}
\newtheorem{ex}{Example}
\newtheorem*{soln}{Solution}
\newtheorem{prop}{Proposition}

\newcommand{\lra}{\longrightarrow}
\newcommand{\ra}{\rightarrow}
\newcommand{\surj}{\twoheadrightarrow}
\newcommand{\graph}{\mathrm{graph}}
\newcommand{\bb}[1]{\mathbb{#1}}
\newcommand{\Z}{\bb{Z}}
\newcommand{\Q}{\bb{Q}}
\newcommand{\R}{\bb{R}}
\newcommand{\E}{\bb{E}}
\newcommand{\C}{\bb{C}}
\newcommand{\N}{\bb{N}}
\newcommand{\M}{\mathbf{M}}
\newcommand{\m}{\mathbf{m}}
\newcommand{\MM}{\mathscr{M}}
\newcommand{\HH}{\mathscr{H}}
\newcommand{\Om}{\Omega}
\newcommand{\Ho}{\in\HH(\Om)}
\newcommand{\bd}{\partial}
\newcommand{\del}{\partial}
\newcommand{\bardel}{\overline\partial}
\newcommand{\textdf}[1]{\textbf{\textsf{#1}}\index{#1}}
\newcommand{\img}{\mathrm{img}}
\newcommand{\ip}[2]{\left\langle{#1},{#2}\right\rangle}
\newcommand{\inter}[1]{\mathrm{int}{#1}}
\newcommand{\exter}[1]{\mathrm{ext}{#1}}
\newcommand{\cl}[1]{\mathrm{cl}{#1}}
\newcommand{\ds}{\displaystyle}
\newcommand{\vol}{\mathrm{vol}}
\newcommand{\cnt}{\mathrm{ct}}
\newcommand{\osc}{\mathrm{osc}}
\newcommand{\LL}{\mathbf{L}}
\newcommand{\UU}{\mathbf{U}}
\newcommand{\support}{\mathrm{support}}
\newcommand{\AND}{\;\wedge\;}
\newcommand{\OR}{\;\vee\;} 
\newcommand{\Oset}{\varnothing}
\newcommand{\st}{\ni}
\newcommand{\wh}{\widehat}
\newcommand{\vect}[1]{\overrightarrow{#1}}

%Pagination stuff.
%\setlength{\oddsidemargin}{0in}
%\setlength{\evensidemargin}{0in}
\setlength{\textheight}{9.in}
\setlength{\textwidth}{6.5in}
\cfoot{page \thepage}
\lhead{MEU302 - Alg\`ebre}
\rhead{TD2}
\pagestyle{fancy}


\begin{document}

\subsection*{Exercice 1}
\subsection*{Question 1.1}
Supposons que la proposition est vraie, on a 

$$
\frac{\sqrt{n}(V^2_n - \sigma^2)}{\sqrt{\bar{\mu_4}-\sigma^4}} = \mathscr{N}(0,1)
$$
$$
(V^2_n - \sigma^2) = \frac{\sqrt{\bar{\mu_4}-\sigma^4}}{\sqrt{n}}\mathscr{N}(0,1)
$$
$$
(V^2_n - \sigma^2) = \mathscr{N}(0,\frac{\bar{\mu_4}-\sigma^4}{n})
$$
$$
V^2_n  = \sigma^2+ \mathscr{N}(0,\frac{\bar{\mu_4}-\sigma^4}{n})
$$
$$
V^2_n  = \mathscr{N}(\sigma^2,\frac{\bar{\mu_4}-\sigma^4}{n})
$$
Donc la proposition est vraie si $V^2_n$ suit une loi normale. Il faut montrer que $E(V_n^2) = \sigma^2$ et $V(V_n^2) = \frac{\bar{\mu_4}-\sigma^4}{n}$.

$$
E(V_n^2) = E\left( \frac{1}{n} \sum_{i=1}^{n}(X_i-m)^2 \right) = \frac{1}{n}\sum_{i=1}^{n}E\left((X_i-m)^2\right) = \frac{1}{n}\sum_{i=1}^{n}E(X_i^2-2X_im + m^2) = \frac{1}{n}\sum_{i=1}^{n}E(X_i^2)-2E(X_im) + E(m^2)
$$
$$
= \frac{1}{n}\sum_{i=1}^{n}E(X_i^2) - E(X_i)^2 = \frac{1}{n}\sum_{i=1}^{n}V(X_i) = \frac{1}{n}\sum_{i=1}^{n}\sigma^2 = \sigma^2
$$
et
$$
V(V_n^2) = V\left(  \frac{1}{n} \sum_{i=1}^{n}(X_i-m)^2 \right) = \frac{1}{n^2}\sum_{i=1}^{n}V((X_i-m)^2) = 
$$
????

\subsection*{Question 1.2}
$$
\hat{\sigma}_n^2 = V_n^2 - (\bar{X}_n - m)^2
$$


\subsection*{Question 1.3}


\subsection*{Question 1.4}


\subsection*{Exercice 2}
\subsection*{Question 2.1}
Si une variable al\'eatoire $X_i$ suit une loi normale $\mathscr{N}(\mu, \sigma^2)$ alors sa loi chi deux de degr\'es $n$ est \'egale \`a $\sum_{i=1}^{n}{\left(\frac{X_i - \mu}{\sigma}\right)^2}$. Comme on a  la variable al\'eatoire $X_i$ suit une loi normale $\mathscr{N}(5, \sigma^2)$, on a $\sum_{i=1}^{n}{\left(\frac{X_i - 5}{\sigma}\right)^2}$ qui suit la loi de chi deux de degr\`es $n$.  

\subsection*{Question 2.2}
Calculons $E(V_n^2)$, si c'est \'egal \`a $\sigma^2$, c'est que l'estimateur est non biais\'e.
D'abord on simplifie:
$$
V_n^2 = \frac{1}{n}\sum_{i=1}^{n}{(X_i -5)^2} = \frac{1}{n}\sum_{i=1}^{n}{(X_i^2 -10X_i + 25)} = \frac{1}{n}\sum_{i=1}^{n}{X_i^2} -10\frac{1}{n}\sum_{i=1}^{n}{X_i} + \frac{1}{n}\sum_{i=1}^{n}{25}
$$
$$
= \frac{1}{n}\sum_{i=1}^{n}{X_i^2} -50 + 25 = \frac{1}{n}\sum_{i=1}^{n}{X_i^2} -25
$$
Petit rappel $E(X_i) = \frac{1}{n}\sum_{i=1}^{n}{X_i} = 5$ la moyenne.

Donc
$$
E(V_n^2) = E\left(\frac{1}{n}\sum_{i=1}^{n}{(X_i -5)^2}\right) =  E\left(\frac{1}{n}\sum_{i=1}^{n}{X_i^2} -25\right) = \frac{1}{n} \sum_{i=1}^{n}{E(X_i^2)} -25 = 
$$
Mais on a $\sigma^2 = V(X_i) = E(X_i^2) - E^2(X_i) $ donc $E(X_i^2) = \sigma^2 + E^2(X_i)$. Dans notre cas $E(X_i) = 5$ donc $E(X_i^2) = \sigma^2 + 25$.
$$
\frac{1}{n} \sum_{i=1}^{n}{E(X_i^2)} -25 = \frac{1}{n} \sum_{i=1}^{n}{\sigma^2 + 25} -25 = \sigma^2 + 25-25 = \sigma^2  
$$

$V_n^2$ est un estimateur non biais\'e donc $B(V_n^2) = 0$.


Le risque quadratique est d\'efini par
$$
R(V_n^2) = E((V_n^2 - 5)^2) = B^2(V_n^2) + V(V_n^2) = V(V_n^2) = V \left( \frac{1}{n}\sum_{i=1}^{n}{(X_i -5)^2} \right) = \frac{1}{n^2} \sum_{i=1}^{n}{V((X_i -5)^2)}
$$
$$
= \frac{1}{n^2} \sum_{i=1}^{n}{E((X_i -5)^2-5)^2} = \frac{1}{n^2} \sum_{i=1}^{n}{E(X_i^2 - 10 X_i +25 -5)^2}
$$
????

\subsection*{Question 2.3}


\end{document}

