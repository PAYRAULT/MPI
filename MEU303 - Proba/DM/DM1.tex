\documentclass[]{book}

%These tell TeX which packages to use.
\usepackage{array,epsfig}
\usepackage{amsmath}
\usepackage{amsfonts}
\usepackage{amssymb}
\usepackage{amsxtra}
\usepackage{amsthm}
\usepackage{mathrsfs}
\usepackage{color}
\usepackage[margin=2cm,top=2.5cm,headheight=16pt,headsep=0.1in,heightrounded]{geometry}
\usepackage{fancyhdr}
\pagestyle{fancy}
%\usepackage{tikz}
\usepackage{pgfplots}

%Here I define some theorem styles and shortcut commands for symbols I use often
\theoremstyle{definition}
\newtheorem{defn}{Definition}
\newtheorem{thm}{Theorem}
\newtheorem{cor}{Corollary}
\newtheorem*{rmk}{Remark}
\newtheorem{lem}{Lemma}
\newtheorem*{joke}{Joke}
\newtheorem{ex}{Example}
\newtheorem*{soln}{Solution}
\newtheorem{prop}{Proposition}

\newcommand{\lra}{\longrightarrow}
\newcommand{\ra}{\rightarrow}
\newcommand{\surj}{\twoheadrightarrow}
\newcommand{\graph}{\mathrm{graph}}
\newcommand{\bb}[1]{\mathbb{#1}}
\newcommand{\Z}{\bb{Z}}
\newcommand{\Q}{\bb{Q}}
\newcommand{\R}{\bb{R}}
\newcommand{\E}{\bb{E}}
\newcommand{\C}{\bb{C}}
\newcommand{\N}{\bb{N}}
\newcommand{\M}{\mathbf{M}}
\newcommand{\m}{\mathbf{m}}
\newcommand{\MM}{\mathscr{M}}
\newcommand{\HH}{\mathscr{H}}
\newcommand{\Om}{\Omega}
\newcommand{\Ho}{\in\HH(\Om)}
\newcommand{\bd}{\partial}
\newcommand{\del}{\partial}
\newcommand{\bardel}{\overline\partial}
\newcommand{\textdf}[1]{\textbf{\textsf{#1}}\index{#1}}
\newcommand{\img}{\mathrm{img}}
\newcommand{\ip}[2]{\left\langle{#1},{#2}\right\rangle}
\newcommand{\inter}[1]{\mathrm{int}{#1}}
\newcommand{\exter}[1]{\mathrm{ext}{#1}}
\newcommand{\cl}[1]{\mathrm{cl}{#1}}
\newcommand{\ds}{\displaystyle}
\newcommand{\vol}{\mathrm{vol}}
\newcommand{\cnt}{\mathrm{ct}}
\newcommand{\osc}{\mathrm{osc}}
\newcommand{\LL}{\mathbf{L}}
\newcommand{\UU}{\mathbf{U}}
\newcommand{\support}{\mathrm{support}}
\newcommand{\AND}{\;\wedge\;}
\newcommand{\OR}{\;\vee\;} 
\newcommand{\Oset}{\varnothing}
\newcommand{\st}{\ni}
\newcommand{\wh}{\widehat}
\newcommand{\vect}[1]{\overrightarrow{#1}}

%Pagination stuff.
%\setlength{\oddsidemargin}{0in}
%\setlength{\evensidemargin}{0in}
\setlength{\textheight}{9.in}
\setlength{\textwidth}{6.5in}
\cfoot{page \thepage}
\lhead{MEU302 - Alg\`ebre}
\rhead{TD2}
\pagestyle{fancy}


\begin{document}

\subsection*{Exercice 1}
\subsection*{Question 1.1(a)}
La fonction de r\'epartition $F(x)$ est croissante, continue \`a droite, limite \`a gauche et born\'ee par 0 quand $x \to -\infty$ et par 1 quand $x \to +\infty$. 

Prenons $y$ tel que $0<y<1$, il existe un $x$ tel que $F(x) \geq y$. car $F(+\infty) = 1$ et $y < 1$. La fonction $F(x)$ est croissante donc $\inf\{ F(x) \geq y\}$ est unique. Supposons $x_1$ le plus petit $x$ telle que $F(x_1) \geq y$. Montrons que $x_2$ ne peut pas \^etre inf\'erieure \`a $x_1$. $x_2$ est inf\'erieure \`a $x_1$ alors $F(x_2) \leq F(x_1)$ car la fonction de r\'epartition $F(x)$ est croissante. Soit $F(x_2) = F(x_1)$, mais par hypoth\`ese $x_1$ est la plus petite valeur de $x$ tel que $F(x_1) \geq y$, Soit $F(x_2) < F(x_1)$ ce qui contredit \'egalement l'hypoth\`ese. Donc $x_2$ n'existe pas. 


\subsection*{Question 1.1(b)}
La fonction de r\'epartion $F(x)$ est croissante. Donc $\forall x \in \R, \forall \epsilon > 0, F(x + \epsilon) \geq F(x)$. Comme la fonction $Q_F(y)$ est bien d\'efinie on a $\forall y \in ]0,1[, \forall \epsilon > 0, F(Q_F(y) + \epsilon) \geq F(Q_F(y))$. On a $Q_F(y)$ est le plus petit $x$ tel que $F(x) \geq y$ et la fonction $F(x)$ est croissante, donc $F(Q_F(y)) \geq y$. Par cons\'equent $F(Q_F(y) + \epsilon) \geq F(Q_F(y)) \geq y$,   


\subsection*{Question 2}
Si la fonction $F$ est continue et strictement croissante, alors il existe exactement un seul $x$ tel que $F(x) = y$ et la fonction $F(x)$ est inversible. Donc on a $Q_F(y) = F^{-1}(x)$ car $\inf\{x \in \R, F(x) \geq y\}$ est le $x$ tel que $F(x) = y$. ???

\subsection*{Question 3(a)}
\begin{tikzpicture}
    \begin{axis}[]
    \addplot [domain=-2:1/3] {0};
    \addplot [domain=1/3:2/3] {0.5};
    \addplot [domain=2/3:1] {3/2*x - 1/2};
    \addplot [domain=1:3] {1};   
    \end{axis}
\end{tikzpicture}

\subsection*{Question 3(b)}
$$Q_F(1/4) = \inf\{x \in \R, F(x) \geq 1/4\}= \inf \{ x \in [1/3, \infty[\} = 1/3$$
et
$$Q_F(3/4) = \inf\{x \in \R, F(x) \geq 3/4\}= \inf \{ x \in R, F(x) = 3/4\} = (3/4+1/2)*2/3 = 5/6$$
et
$$Q_F(F(1/2)) = Q_F(1/2) = 1/3$$
et
Pour $x \in ]2/3,1[$, on a $F(x) \in ]1/2,1[$ donc $Q_F(F(x)) \in ]1/3,1[$.

\subsection*{Question 4}
Non. On a sur l'exemple pr\'ec\'edent $F(2/3) = 1/2$ et $Q_F(1/2) = 1/3 \neq 2/3$ et de m\^eme $Q_F(1/4) = 1/3$ et $F(1/3) = 1/2 \neq 1/4$. 

\subsection*{Question 5}
Pour tout $x \in \R$ on a $\forall x \in \R, Q_F(F(x)) \leq x$ par d\'efinition de la borne inf\'erieure de $Q_F$ et de la non d\'ecroissance de $F$. De m\^eme, on a $\forall y \in ]0,1[, F(Q_F(y)) \geq y$. 

\begin{itemize}
\item $Q(y) \leq x \implies y \leq F(x)$. On suppose $Q(y) \leq x$, comme $F$ est non d\'ecroissante on a $F(Q_F(y)) \leq F(x)$. Mais $y \leq F(Q_F(y))$, donc $y \leq F(x)$
\item $y \leq F(x) \implies Q(y) \leq x$. pour $y_1, y_2 \in ]0,1[, \text{ avec } y_1 \leq y_2$ on a $\{x \in \R, F(x) \geq y_2\} \subseteq \{x \in \R, F(x) \geq y_1\}$. 
Donc $\inf \{x \in \R, F(x) \geq y_1\} \leq  \inf \{x \in \R, F(x) \geq y_2\}$ donc $Q_F(y_1) \leq Q_F(y_2)$. Donc la fonction $Q_F$ est croissante. On suppose $y \leq F(x)$, on a $Q_F(y) \leq Q_F(F(x))$, mais $Q_F(F(x)) \leq x$ donc $Q_F(y) \leq x$.
\end{itemize}

Donc $Q(y) \leq x \text{ ssi } y \leq F(x)$

\subsection*{Question 6}
???

\subsection*{Question 7(a)}
Supposons $x_1 \leq x_2$, on a $\{x_i \leq x_1\} \subseteq \{x_i \leq x_2\}$, donc $\sum_{n=1}^n {1_{\{x_i \leq x_1\}}} \leq \sum_{n=1}^n {1_{\{x_i \leq x_2\}}}$ donc $F_n(x_1) \leq F_n(x_2)$. Par cons\'equent, $F_n$ est croissante.

Soit $x$ une s\'erie num\'erique de $\R^n$, classons ces \'el\'ement en une s\'equence d\'ecroissante $x_1 > x_2 > \ldots > x_n$. Prenons $y < x_n$, on a $F_n(y) = 0$ car $\{ x_i < y \} = \emptyset$. Donc $\lim_{x \to -\infty}F_n(x) = 0$.

Soit $x$ une s\'erie num\'erique de $\R^n$, classons ces \'el\'ement en une s\'equence croissante $x_1 < x_2 < \ldots < x_n$. Prenons $y > x_n$, on a $F_n(y) = 1$ car $card \{ x_i < y \} = n$. Donc $\lim_{x \to \infty}F_n(x) = 1$.

\subsection*{Question 7(b)}
Soit une s\'erie num\'erique sur $\R^n$, prenons la sous-s\'erie $x_m$ qui tend \`a droite vers un point $x$ lorsque $m \to +\infty$. On a $\{ x_i < x\} \subseteq \{x_i < x_m\}$ car $\forall m, x leq x_m$ et les ensembles $\{x_i < x_m\}$ sont de plus en plus petits lorsque $n$ croit. On a donc $\lim_{n \to \infty} \{ x_i < x_n\} = \{x_i < x\}$. Par cons\'equent $\lim_{m \to infty, x_m > x} F_n(x_m) = F_n(x)$ qui est la d\'efinition de la continuit\'e \`a droite.

La fonction $F_n$ est croissante et est continue \`a droite en tous points $x$, donc elle a $F(x)$ comme limite \`a droite. La fonction $F_n$ est comprise entre 0 et 1. Donc pour tous point $x$ on a $0 \leq \lim_{m \to \infty, x_m < x} \leq F(x)$.  Donc il existe une limite \`a gauche en tous point $x$.

\subsection*{Question 8(a)}
La fonction de r\'epartition de $X_n$ est $F(x) = \sum_{i=1}^{n} p_i 1_{x_i < x}$. Comme la variable al\'eatoire $X_n$ est uniforme, on a tous les $p_i = 1/n$. Donc $F(x) = \sum_{i=1}^{n} 1/n 1_{x_i < x} = F(x) = 1/n \sum_{i=1}^{n} 1_{x_i < x} = F_n()$.

\subsection*{Question 8(b)}
L'esp\'erance dune variable al\'eatoire discrete $X_n = \{x_1, x_2, \ldots, x_n\}$ est $E(X_n) = \frac{x_1+x_2+\ldots + x_n}{n}$ et la variance $V(x_n) = \frac{x_1^2+x_2^2+\ldots + x_n^2}{n} - E^2(X-n)$.

\subsection*{Question 9(a)}
\begin{tikzpicture}
    \begin{axis}[]
    \addplot [domain=-3:-1] {0};
    \addplot [domain=-1:1/2] {0.25};
    \addplot [domain=1/2:2] {0.5};
    \addplot [domain=2:5] {0.75};   
    \addplot [domain=5:7] {1};   
    \end{axis}
\end{tikzpicture}

\subsection*{Question 9(b)}
On a $Q_F(1/4) = -1$, $Q_F(2/4) = 1/2$, $Q_F(3/4) = 2$, $Q_F(0.05) = -1$, $Q_F(0.95) = 5$ 

\end{document}

