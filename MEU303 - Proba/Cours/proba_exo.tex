\documentclass[]{book}

%These tell TeX which packages to use.
\usepackage{array,epsfig}
\usepackage{amsmath}
\usepackage{amsfonts}
\usepackage{amssymb}
\usepackage{amsxtra}
\usepackage{amsthm}
\usepackage{mathrsfs}
\usepackage{color}

%Here I define some theorem styles and shortcut commands for symbols I use often
\theoremstyle{definition}
\newtheorem{defn}{Definition}
\newtheorem{thm}{Theorem}
\newtheorem{cor}{Corollary}
\newtheorem*{rmk}{Remark}
\newtheorem{lem}{Lemma}
\newtheorem*{joke}{Joke}
\newtheorem{ex}{Exercice}
\newtheorem*{soln}{Solution}
\newtheorem{prop}{Proposition}

\newcommand{\lra}{\longrightarrow}
\newcommand{\ra}{\rightarrow}
\newcommand{\surj}{\twoheadrightarrow}
\newcommand{\graph}{\mathrm{graph}}
\newcommand{\bb}[1]{\mathbb{#1}}
\newcommand{\Z}{\bb{Z}}
\newcommand{\Q}{\bb{Q}}
\newcommand{\R}{\bb{R}}
\newcommand{\C}{\bb{C}}
\newcommand{\N}{\bb{N}}
\newcommand{\M}{\mathbf{M}}
\newcommand{\m}{\mathbf{m}}
\newcommand{\MM}{\mathscr{M}}
\newcommand{\HH}{\mathscr{H}}
\newcommand{\Om}{\Omega}
\newcommand{\Ho}{\in\HH(\Om)}
\newcommand{\bd}{\partial}
\newcommand{\del}{\partial}
\newcommand{\bardel}{\overline\partial}
\newcommand{\textdf}[1]{\textbf{\textsf{#1}}\index{#1}}
\newcommand{\img}{\mathrm{img}}
\newcommand{\ip}[2]{\left\langle{#1},{#2}\right\rangle}
\newcommand{\inter}[1]{\mathrm{int}{#1}}
\newcommand{\exter}[1]{\mathrm{ext}{#1}}
\newcommand{\cl}[1]{\mathrm{cl}{#1}}
\newcommand{\ds}{\displaystyle}
\newcommand{\vol}{\mathrm{vol}}
\newcommand{\cnt}{\mathrm{ct}}
\newcommand{\osc}{\mathrm{osc}}
\newcommand{\LL}{\mathbf{L}}
\newcommand{\UU}{\mathbf{U}}
\newcommand{\support}{\mathrm{support}}
\newcommand{\AND}{\;\wedge\;}
\newcommand{\OR}{\;\vee\;}
\newcommand{\Oset}{\varnothing}
\newcommand{\st}{\ni}
\newcommand{\wh}{\widehat}

%Pagination stuff.
\setlength{\topmargin}{-.3 in}
\setlength{\oddsidemargin}{0in}
\setlength{\evensidemargin}{0in}
\setlength{\textheight}{9.in}
\setlength{\textwidth}{6.5in}
\pagestyle{empty}



\begin{document}

\subsection*{Exercice 1}
Un jeu de carte a 52 cartes. Trouver la probabilit\'e d'avoir\ :
\begin{enumerate}
\item Un As
\item Un Valet de Coeur
\item Un 3 de Tr\`efle ou un 6 de Carreau
\item Une carte de Coeur
\item Une carte de couleur autre que Carreau
\item Un 10 ou un As
\item Ni un 4 ni une carte de couleur Pique
\end{enumerate}

\subsection*{Exercice 2}
Un sac contient 6 balles rouges, 4 balles blanches et 5 balles bleues. Calculer la probabilit\'e de tirer\ :
\begin{enumerate}
\item une balle bleue
\item une balle blanche
\item Pas une balle rouge
\item Une balle rouge ou bleue
\end{enumerate}

\subsection*{Exercice 3}
Un d\'e est lanc\'e deux fois. Trouver la probabilit\'e d'avoir un 4, 5, ou 6 au premier tirage et un 1, 2, 3 ou 4 au second tirage.

\subsection*{Exercice 4}
Deux d\'es sont lanc\'es simultan\'ement. Trouver la probabilit\'e de ne pas avoir un total de 7 ou 11.

\subsection*{Exercice 5}
2 cartes sont tir\'ees au hasard d'un jeu de 52 cartes. Trouver la probabilit\'e d'avoir 2 Valets quand\ :
\begin{enumerate}
\item la premi\`ere carte est replac\'ee dans le jeu
\item la premi\`ere carte n'est pas replac\'ee dans le jeu
\end{enumerate}

\subsection*{Exercice 6}
Trois balles sont tir\'ees du sac de l'exercice 2, trouver la probabilit\'e de tirer dans l'ordre Bleu, Blanc et Rouge quand\ : 
\begin{enumerate}
\item les balles sont replac\'ees dans le sac
\item les balles ne sont pas replac\'ees dans le sac
\end{enumerate}

\subsection*{Exercice 7}
Un d\'e est lanc\'e deux fois de suite. Trouver la probabilit\'e d'avoir au moins un 4.


QED

\end{document}

