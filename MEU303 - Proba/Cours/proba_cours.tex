\documentclass[]{book}

%These tell TeX which packages to use.
\usepackage{array,epsfig}
\usepackage{amsmath}
\usepackage{amsfonts}
\usepackage{amssymb}
\usepackage{amsxtra}
\usepackage{amsthm}
\usepackage{mathrsfs}
\usepackage{color}

%Here I define some theorem styles and shortcut commands for symbols I use often
\theoremstyle{definition}
\newtheorem{defn}{Definition}
\newtheorem{thm}{Theorem}
\newtheorem{cor}{Corollary}
\newtheorem*{rmk}{Remark}
\newtheorem{lem}{Lemma}
\newtheorem*{joke}{Joke}
\newtheorem{ex}{Example}
\newtheorem*{soln}{Solution}
\newtheorem{prop}{Proposition}

\newcommand{\lra}{\longrightarrow}
\newcommand{\ra}{\rightarrow}
\newcommand{\surj}{\twoheadrightarrow}
\newcommand{\graph}{\mathrm{graph}}
\newcommand{\bb}[1]{\mathbb{#1}}
\newcommand{\Z}{\bb{Z}}
\newcommand{\Q}{\bb{Q}}
\newcommand{\R}{\bb{R}}
\newcommand{\C}{\bb{C}}
\newcommand{\N}{\bb{N}}
\newcommand{\M}{\mathbf{M}}
\newcommand{\m}{\mathbf{m}}
\newcommand{\MM}{\mathscr{M}}
\newcommand{\HH}{\mathscr{H}}
\newcommand{\Om}{\Omega}
\newcommand{\Ho}{\in\HH(\Om)}
\newcommand{\bd}{\partial}
\newcommand{\del}{\partial}
\newcommand{\bardel}{\overline\partial}
\newcommand{\textdf}[1]{\textbf{\textsf{#1}}\index{#1}}
\newcommand{\img}{\mathrm{img}}
\newcommand{\ip}[2]{\left\langle{#1},{#2}\right\rangle}
\newcommand{\inter}[1]{\mathrm{int}{#1}}
\newcommand{\exter}[1]{\mathrm{ext}{#1}}
\newcommand{\cl}[1]{\mathrm{cl}{#1}}
\newcommand{\ds}{\displaystyle}
\newcommand{\vol}{\mathrm{vol}}
\newcommand{\cnt}{\mathrm{ct}}
\newcommand{\osc}{\mathrm{osc}}
\newcommand{\LL}{\mathbf{L}}
\newcommand{\UU}{\mathbf{U}}
\newcommand{\support}{\mathrm{support}}
\newcommand{\AND}{\;\wedge\;}
\newcommand{\OR}{\;\vee\;}
\newcommand{\Oset}{\varnothing}
\newcommand{\st}{\ni}
\newcommand{\wh}{\widehat}

%Pagination stuff.
\setlength{\topmargin}{-.3 in}
\setlength{\oddsidemargin}{0in}
\setlength{\evensidemargin}{0in}
\setlength{\textheight}{9.in}
\setlength{\textwidth}{6.5in}
\pagestyle{empty}



\begin{document}

\begin{defn}
On d\'efinit un \emph{espace de probibilit\'e} par un triplet $(\Omega, \mathscr{F}, P)$ avec:
\begin{itemize}
\item $\Omega$ est un ensemble de r\'esultats
\item $\mathscr{F}$ une collection de sous-ensembles de $\Omega$ qui satisfait:
	\begin{itemize}
	\item $\Omega \in \mathscr{F}$
	\item $\forall A \in \mathscr{F} \implies ^{c}A \in \mathscr{F}$
	\item $A_1, A_2 \ldots A_n \in \mathscr{F} \implies \cup_{m = 1\ldots n}A_m \in \mathscr{F}$ 
	\end{itemize}
 \item $P$, une application de $\mathscr{F} \to [0,1]$
\end{itemize}
\end{defn}

Par exemple: $\Omega=\{1,2,3\}, \mathscr{F} = \{\emptyset, \{1\}, \{2,3\}, \{1,2,3\} \}$. \\

Un autre exemple $\bb{P}(\{1,2,3\}) = \{\emptyset, \{1\},  \{2\}, \{3\}, \{1,2\}, \{1,3\}, \{2,3\}, \{1,2,3\}\}$ est appel\'e l'ensemble des sous-ensembles.

\begin{defn}
L'application $P$ est d\'efinie par les 3 r\`egles suivantes:
\begin{itemize}
\item $P(\Omega) = 1$
\item $\forall A \in \mathscr{F}, P( {}^{c}A) = 1 - P(A)$
\item Toute suite d\'enombrable d'\'ev\'enements disjoints $A_1, A_2, \ldots, A_n \in \mathscr{F}, P(\cup_{m=1..n}A_m) = \sum_{m=1}^{n} P(A_m) $
\end{itemize}
\end{defn}


On peut d\'eduire tous les th\'eor\`emes suivants:
\begin{itemize}
\item $P(\emptyset) = 0$. $\Omega$ et $\emptyset$ sont disjoints et $\Omega \cup \emptyset = \Omega$ donc
$$P(\Omega) = P(\Omega \cup \emptyset) = P(\Omega) + P(\emptyset)$$ 
donc $P(\emptyset) = 0$
\item si $A_1 \subset A_2 \implies P(A_1) \leq P(A_2)$. On a $A_2 = A_1 \cup (A_2-A_1)$ car $A_1 \subset A_2$. Et $A_1$ et $(A_2 - A_1)$ sont disjoints donc
$$P(A_2) = P(A_1) + P(A_2-A_1)$$
$$P(A_2-A_1) = P(A_2) - P(A_1)$$
Comme $P(X)$ est positif, on a $P(A_2) - P(A_1) \geq 0$, donc $P(A_2) \geq P(A_1)$
\item $\forall X \in \mathscr{F}, P(X) \leq 1$. On a $\forall X \in \mathscr{F}, X \subset \Omega$ part d\'efinition, donc $P(X) \leq P(\Omega) = 1$
\end{itemize}



QED

\end{document}

